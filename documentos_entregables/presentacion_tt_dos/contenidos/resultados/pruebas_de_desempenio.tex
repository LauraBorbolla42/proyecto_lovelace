%
% Presentación de TT2.
% Proyecto Lovelace.
%
% Resultados.
% Pruebas de desempeño.
%

\subsection{Pruebas de desempeño}

\begin{frame}{Pruebas de desempeño}

  \begin{figure}
    \begin{center}
      \includegraphics[width=0.9\linewidth]
        {../../../diagramas_comunes/implementacion_api_web/%
        peticiones_tokenizacion_grafica}
    \end{center}
%    \begin{table}
%      \centering
%      \resizebox{0.7\textwidth}{!}{%
%        \begin{tabular}{c|c|c|c|c|c|}
%          \cline{2-6}
%          & FFX & BPS & TKR & AHR & DRBG \\
%          \hline
%          \multicolumn{1}{|c|}{Tokenización}
%            & 206.526s & 219.036s & 824.969s & 436.053s & 437.406s \\\hline
%          \multicolumn{1}{|c|}{Detokenización}
%            & 208.204s & 219.448s & 103.854s & 132.297s & 124.064s \\\hline
%        \end{tabular}
%      }
%    \end{table}
    \caption{Comparación de tiempos de 10K operaciones.}
  \end{figure}

  \note{
    Como se quería ver el desempeño del servicio de tokenización, se decidió
    medir los tiempos de respuesta de las peticiones de tokenización y
    detokenización.

    En la gráfica se observan los tiempos que se obtuvieron al realizar 10K
    peticiones con los 5 métodos disponibles.

    Bueno, como se observa los tiempos de tokenización para los 2 primeros
    métodos que son los irreversibles son considerablemente mas pequeños
    que los tiempos de los siguientes 3 métodos los cuales son irreversibles.
    Pero en contra parte, los tiempos de detokenización de estos métodos son
    mejores, dado que detokenización solo una consulta a la base de datos.

    Con esto se puede ver que los métodos irreversibles mostrarían un mejor
    desempeño cuando un sistema necesita hacer mas operaciones de
    detokenización que de tokenización, y en caso contrario los métodos
    reversibles tendrían mejores resultados
  }

\end{frame}

