%
% Presentación de TT2.
% Proyecto Lovelace.
%
% Conclusiones.
%

\section{Conclusiones}

\begin{frame}{Sobre los objetivos}
  Los objetivos planteados al inicio de este trabajo terminal fueron cumplidos:
  \begin{itemize}
    \item Implementar algoritmos generadores de tokens.
    \item Diseñar e implementar un servicio web que proporcione la
      generación de tokens a, al menos, una tienda en línea.
    \item Implementar una tienda en línea que utilice el servicio de
      generación de tokens.
  \end{itemize}

  \note{
  \begin{itemize}
    \item Resaltar que, para implementar los algoritmos tokenizadores,
      primero tuvimos que revisar otros conceptos criptográficos. También
      que Sandra ayudó bastante para encontrar información sobre algoritmos
      tokenizadores (¿recuerdan que estuvimos buscando algoritmos y no
      encontramos muchos que digamos?).
    \item Recuerden que, aunque sí proveemos lo de la generación de tokens
      y la detokenización en el  servicio, en realidad faltan requerimientos
      por cumplir (como auditorías, tener certificaciones, etcétera) que
      quedaron fuera del alcance el proyecto (ya sea por factores de tiempo
      o económicos). También: Ricardo se esforzó mil para que tengamos un
      certificado y el servicio pueda usar el HTTPS.
    \item Hacer énfasis en que solo se implementaron los casos de uso que
      eran necesarios para probar la interacción con el sistema tokenizador,
      EVIDENTEMENTE, el sistema de la librería no está completo. También que,
      aunque estuvimos buscando en varios sitios que podrían utilizar
      servicios de tokenización (Ticketmaster, Gandhi, Porrúa, El Sótano,
      IKEA, El fondo de cultura económica, El Péndulo, MercadoLibre, Amazon,
      Uber, Spotify), no encontramos ninguno que especificara que utiliza
      un servicio de tokenización; empero, encontramos MUCHOS que hacían uso
      de una API de PayPal que nos hace pensar que tiene un vil MONOPOLIO.
  \end{itemize}}
\end{frame}

% \begin{frame}{Sobre los algoritmos}
%   \begin{itemize}
%     \item La \emph{detokenización} es más rápida que la generación de
%       tokens para los algoritmos irreversibles (TKR, AHR, DRBG).
%     \item Los algoritmos reversibles (FFX y BPS) son más veloces que los
%       irreversibles al \emph{tokenizar}.
%     \item Con la llave de cifrado, un atacante puede...
%       \begin{itemize}
%         \item Tokenizar \emph{y} detokenizar \textbf{si} se utilizó un
%           algoritmo reversible.
%         \item \emph{Solo} tokenizar \textbf{si} se utilizó un algoritmo
%           irreversible.
%       \end{itemize}
%   \end{itemize}
%
%   \note{
%     \begin{itemize}
%       \item Recordar que la tokenización en los algoritmos irreversibles
%         involucra consultas en la base de datos y hace uso de entropía (y que
%         puede tomar su tiempo el obtenerla). También recalcar que las
%         detokenizaciones aquí son una vil consulta a la base de datos. Y que
%         super depende del gestor de la base de datos y del número de datos
%         que se tienen (cuando hacíamos pruebas, si no reiniciabas la base de
%         datos entre cada algoritmo, los últimos eran los que peor se
%         desempeñaban {porque la base tenía TODOS los tokens de los algoritmos
%         anteriores}). Probablemente este útlimo punto es un super factor
%         a considerar, los irreversibles podrían ser muy útiles cuando no vas
%         a realizar muchas tokenizaciones y no piensas tener muuuchos tokens.
%       \item Resaltar aquí que, de buenas a primeras, está medio extraño que
%         los reversibles sean más rápidos que los irreversibles, pues los
%         primeros realizan más operaciones al tokenizar. Explicar que, de hecho,
%         cuando no se toma en cuenta el tiempo de acceso a la base de datos,
%         los algoritmos irreversibles son MUCHO MÁS rápidos que los reversibles,
%         pero pues que no sirve de mucho, porque en la vida real, en un caso
%         práctico, está medio en chino quitar (o reducir) ese tiempo de acceso.
%       \item Dejar MUY en claro que no por esto los algoritmos reversibles son
%         inseguros (o sea, están basados en redes Feistel), solo que sirven
%         más para realizar tokenizaciones rápidamente, en especial cuando
%         se maneja una gran cantidad de tokens.
%     \end{itemize}}
% \end{frame}

\begin{frame}{Sobre la clasificación}
  Uno de los resultados más importantes, es la clasificación propuesta por
  los autores para los algoritmos tokenizadores en respuesta a la clasificación
  del PCI.
  \begin{itemize}
    \item La denominación de \emph{no criptográficos} es engañosa.

    \item La denominación \emph{irreversibles} no es muy útil para las
      aplicaciones que tokenizan números de tarjetas.
  \end{itemize}

  \note{
    \begin{itemize}
      \item Este nombre hace pensar que los algoritmos que están aquí NO
        utilizan criptografía para nada; cuando en realidad, aquí caen varios
        algoritmos que la usan en gran medida Los que sí podían ir aquí, son
        los generadores verdaderamente aleatorios, pero son pocos y no comunes;
        de hecho, esta clasificación puede ser, sino uno de los orígenes, una
        de las principales fuentes de la creencia de que la tokenización y la
        criptografía no se llevan y el de que la tokenización siga rodeada
        de una aurora de misterio e intriga (bien dramática la conclusión).
      \item Primero: recordar que el PCI es EL concilio de tarjetas ¿para
        pagos?, uno pensaría que todas sus aplicaciones están orientadas a este
        fin, luego, ¿para que tomarse tanto tiempo ¿para generar tokens (y
        gastar tiempo y recursos) si luego será imposible utilizarlos? Deberían
        dar mejores ejemplos del uso de estos tokens.
    \end{itemize}}
\end{frame}

\begin{frame}{Sobre el trabajo desarrollado}
  \begin{itemize}
    \item Se escribió un artículo con los resultados de este trabajo y fue
      presentado en la \textit{Reunión de Ciberseguridad para la Industria 4.0}
      (RCI4.0 2018); será publicado en las memorias del evento.
    \item Las implementaciones de los algoritmos son públicas.
      \begin{itemize}
        \item \url{https://github.com/RQF7/proyecto_lovelace}
        \item \url{https://ricardo-quezada.159.65.96.59.xip.io/sistema_tokenizador/estaticos/doxygen/index.html}
      \end{itemize}
    \item Tanto el servicio tokenizador como la tienda en línea pueden ser
      consultadas en las siguientes direcciones:
      \begin{itemize}
        \item \url{https://ricardo-quezada.159.65.96.59.xip.io/sistema_tokenizador}
        \item \url{https://ricardo-quezada.159.65.96.59.xip.io/libreria}
      \end{itemize}
  \end{itemize}
\end{frame}
