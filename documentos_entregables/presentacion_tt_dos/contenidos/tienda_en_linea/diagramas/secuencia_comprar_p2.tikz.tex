%
% Diagrama de secuencia para realizar una compra con la tienda en línea
% Proyecto Lovelace.
%

\resizebox{\textwidth}{!}{%

  \begin{tikzpicture}
    \begin {umlseqdiag}

      % Líneas de vida %%%%%%%%%%%%%%%%%%%%%%%%%%%%%%%%%%%%%%%%%%%%%%%%%%%%%%%%

      \umlcontrol [x=00, fill=gray!20, class = Servicio]{servicioGeneral}
      \umlcontrol [x=06, fill=gray!20, class = Servicio]{servicioTokenizador}
      \umlentity  [x=12, fill=gray!20, class = Negocio] {negocioGeneral}
      \umldatabase[x=18, fill=gray!20, no ddots]        {Modelo}

      % Finalizar compra  %%%%%%%%%%%%%%%%%%%%%%%%%%%%%%%%%%%%%%%%%%%%%%%%%%%%%

      \begin{umlcall}[fill=gray!15, padding=4, dt=10,%
        op = {obtenerToken(compra.ultimosDigitos)}, %
        return = token]{servicioGeneral}{Modelo}
      \end{umlcall}

      \begin{umlcall}[fill=gray!15, padding=4, dt=5,%
        op = {POST api/detokenizar/\{token\}}, %
        return = tarjeta]{servicioGeneral}{servicioTokenizador}
      \end{umlcall}

      \begin{umlcall}[fill=gray!15, padding=4, dt=5,%
        op = {registrarCompra(compra)}, %
        return = true]{servicioGeneral}{negocioGeneral}
        \begin{umlcall}[fill=gray!15, padding=4, dt=5,%
          op = {compra.guardar()}, %
          return = true]{negocioGeneral}{Modelo}
        \end{umlcall}
      \end{umlcall}

      %%%%%%%%%%%%%%%%%%%%%%%%%%%%%%%%%%%%%%%%%%%%%%%%%%%%%%%%%%%%%%%%%%%%%%%%%

    \end{umlseqdiag}
  \end{tikzpicture}
}
