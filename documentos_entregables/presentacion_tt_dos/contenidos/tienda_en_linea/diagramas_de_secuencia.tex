%
% Presentación de TT2.
% Proyecto Lovelace.
%
% Prototipo de tienda en línea (caso de prueba).
% Diagramas de secuencia.
%

\subsection{Diagramas de secuencia}

% NOTA: Aun no tenemos el diagrama de registro de pago, si nos falta
%       tiempo lo podemos hacer y ponerlo aquí

%\begin{frame}{Diagramas de secuencia}{CU de registro de pago}
%
%  Info
%
%  \note{
%    notas
%  }
%
%\end{frame}

\begin{frame}{Diagramas de secuencia}{CU de compra}

  \begin{figure}
    \begin{center}
      \subimport{diagramas/}{secuencia_comprar.tikz.tex}
      \caption{Diagrama de secuencia completo del caso de uso de compra.}
    \end{center}
  \end{figure}

  \note{
    Para dejar más claro el comportamiento de un compra, y ver integración
    del servicio web con la tienda, vamos a describir el diagrama de
    secuencias de una compra.
    Este es el diagrama completo, pero para que se puede ver mejor, se
    partió en dos.
  }

\end{frame}

\begin{frame}{Diagramas de secuencia}{CU de compra, obtención de datos}

  \begin{figure}
    \begin{center}
      \subimport{diagramas/}{secuencia_comprar_p1.tikz.tex}
      \caption{Parte de la obtención de datos del diagrama de secuencia
        del CU de compra.}
    \end{center}
  \end{figure}

  \note{
    Primero, una ves que un cliente ya puso libros en el carrito de compras
    va a esta sección y finaliza la compra.
    Entonces el navegador le solicita las formas de pago al servicio de la
    tienda, que las obtiene de la base de datos en las que están almacenas.
    El usuario selecciona la forma de pago que desea, y se hace el mismo
    proceso, pero ahora con las direcciones que se tienen para la entrega
    de los libros.
  }

\end{frame}

\begin{frame}{Diagramas de secuencia}{CU de compra, finalización de compra}

  \begin{figure}
    \begin{center}
      \subimport{diagramas/}{secuencia_comprar_p2.tikz.tex}
      \caption{Parte de la finalización de a compra del diagrama de
        secuencia del CU de compra.}
    \end{center}
  \end{figure}

  \note{
    Posteriormente, ya selecciona la forma de pago y la dirección de entrega,
    el usuario acepta la compra, lo que hace que el servicio de la tienda le
    solicite la detokenización el valor que se estableció como forma de pago,
    para proseguir con la transacción bancario (que no hacemos por cuestión
    de alcance) y el registro de la compra en la base de datos del sistema
    de la tienda.
  }

\end{frame}
