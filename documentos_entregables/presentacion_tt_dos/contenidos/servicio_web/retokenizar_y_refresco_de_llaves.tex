%
% Presentación de TT2.
% Proyecto Lovelace.
%
% Prototipo de servicio web.
% CU de retokenizar y refresco de llaves.
%

\subsection{Refresco de llaves y retokenización}

\begin{frame}{Refresco de llaves y retokenización}{Refresco de llaves}
  \begin{itemize}
    \item En el estándar del NIST sobre la administración de las llaves
    criptográficas (NIST SP-800-57~\cite{nist_llaves}) se habla sobre los
    criptoperiodos y se hace énfasis en cambiar las llaves cada que estos se
    cumplan.
    \item El PCI DSS especifica que debe existir un mecanismo que permita a los
    usuarios reemplazar sus llaves y actualizar sus tokens.

  \end{itemize}

\end{frame}

\begin{frame}{Refresco de llaves y retokenización}{Refresco de llaves}
  El servicio web logra esto mediante un mecanismo llamado \textbf{refresco de
  llaves} existente en la aplicación web.
  \begin{itemize}
    \item Cambia el estado del cliente a \emph{en cambio de llaves}.
    \item Se crean nuevas llaves.
    \item Las tokenizaciones se realizan con las llaves nuevas.
    \item Las detokenizaciones se pueden realizar con ambas llaves.
  \end{itemize}
  Cuando el cliente termina el \textbf{refresco de llaves}, se le regresa a su
  estado anterior y se eliminan los tokens que no fueron actualizados y las
  llaves anteriores.

  \note{
    \begin{itemize}
      \item Especificar que, aunque el sistema le manda unas alertas, el
        cliente tiene la responsabilidad de hacer el cambio de llaves.
      \item Recordar que se pueden realizar detokenizaciones con ambas
        llaves y, que si no se especifica, se van a utilizar las viejas.
      \item También que se consideraron otras alternativas, como enviar el
        token nuevo cuando se realizase una detokenización, pero esto obligaba
        al cliente a estar escuchando siempre. También está la de cambiarlas
        nosotros solos, pero solo funcionaba con los algoritmos irreversibles,
        pues si había tokens reversibles que se atrasasen dos periodos, sería
        imposible recuperar el PAN.
      \item Recalcar que se le avisa al usuario que no ha terminado de
        retokenizar sus tokens, porque una vez que los pierde, lo cargó el
        payaso.
    \end{itemize}}
\end{frame}

\begin{frame}{Refresco de llaves y retokenización}{Retokenización}
  La retokenización permite obtener la versión actual de los tokens:
  \begin{enumerate}
    \item El cliente envía el token anterior.
    \item El servicio detokeniza el token recibido.
    \item El servicio tokeniza el PAN obtenido en el paso anterior.
    \item El servicio envía el token actualizado al cliente.
  \end{enumerate}

  La retokenización está permitida solamente durante un refresco de llaves.

  \note{
    Notar que este es el proceso general, falta anotar que se cambia el estado
    de los tokens cuando son irreversibles y que el contador de malas acciones
    aumenta cuando se intenta retokenizar sin estar en el estado correcto o se
    mandan tokens que no son tokens.}
\end{frame}
