%
% Presentación de TT2.
% Proyecto Lovelace.
%
% Recapitulación.
% Planteamiento del problema.
%

\subsection{Planteamiento del problema}

\begin{frame}{Planteamiento del problema}
  {La protección de datos bancarios}
  El crecimiento del comercio en línea, aunado a sistemas débilmente
  protegidos, propició un incremento en los robos de datos bancarios.

  \begin{figure}
    \centering
    \includegraphics[width=0.85\linewidth]
      {diagramas_comunes/estadisticas_fraudes/%
        perdidas_fraude_2002_2012_color.png}
    \caption{Pérdidas debidas al fraude en línea (2001-2012)~\cite{wallethub}.}
  \end{figure}
  \note{
    \begin{itemize}
      \item En la década de los 80 y 90 nadie previó el impacto que tendría el
        internet, no estaban preparados y mientras más se utilizaba el
        internet, más se acrecentaron los fraudes de tarjetas.
      \item Recalcar que las pérdidas mostradas en la gráfica son solo de
        fraudes en EE. UU. y Canadá.
    \end{itemize}
  }
\end{frame}

\begin{frame}{Planteamiento del problema}
  {La protección de datos bancarios}
  \begin{figure}
    \centering
    \includegraphics[width=1\linewidth]
       {diagramas_comunes/estadisticas_fraudes/%
        perdidas_fraude_2011_2018_color.png}
    \caption{Pérdidas debidas al fraude con tarjetas no presentes (CNP) en
              EE.~UU. (2011-2018)~\cite{creditcards}.}
  \end{figure}
  \note{
    \begin{itemize}
      \item Recordar que, a partir del 2014, los datos son estimados.
      \item Parece que las cifras no coinciden con la gráfica pasada, pero es
        porque aquí solo se están considerando los fraudes debido a tarjetas
        no presentes (la gráfica anterior contaba las pérdidas por tarjetas
        robadas/clonadas).
      \item Se proyecta un repunte de las pérdidas debido al desarrollo de un
        cambio en los chips de las tarjetas que hacen más difíciles los fraudes
        por tarjetas presentes.
    \end{itemize}}
\end{frame}

\begin{frame}{Planteamiento del problema}
  {Publicaciones del PCI}
  \begin{itemize}
    \item En 2004, se publicó el PCI DSS\footnotemark \cite{pci_dss}.
    \item Hasta este momento, el enfoque era proteger la información en
      donde sea que se encuentre.
    \item En 2011, el PCI SSC\footnotemark{} publicó las primeras guías
      para los procesos de tokenización~\cite{pci_tokens}.
  \end{itemize}

  \addtocounter{footnote}{-2}
  \stepcounter{footnote}\footnotetext{
    \textit{Payment Card Industry, Data Security Standard}}
  \stepcounter{footnote}\footnotetext{
    \textit{Payment Card Industry, Security Standards Council}}

  \note{
    \begin{itemize}
      \item Remarcar que antes del estándar unificado, cada quién intentó
        sacar sus buenas prácticas, pero, pues si para una PyME estaba medio
        en chino satisfacer uno, ahora satisfacer 3 o 4, pues nein. Oh,
        también mencionar quiées componen el concilio: Mastercard, VISA,
        AmericanExpress, Discovery...
        También recordar que solo es (era) obligatorio cumplir con él cuando
        se realizan más de 20K transacciones al año. Y que es MUY difícil de
        satisfacer, pues aunque tiene menos de 15 requerimientos, los
        subrequerimientos y subsubrequerimientos complican todo.
      \item Dar el típico ejemplo de esto: había que proteger la información de
        un cliente en el depto. de ventas, entregas, usuarios, ect. y la
        comunicación entre todos estos canales. Luego mencionar que el
        paradigma cambia y ahora ponen TODA la información valiosa en un mismo
        lugar, así solo hay que proteger este lugar. Recordar que un adversario
        no puede hacer mucho si solo tiene los tokens.
      \item Aunque indica lo que debe satisfacer el sistema tokenizador,
        no dice cómo generar los tokens.
    \end{itemize}}
\end{frame}
