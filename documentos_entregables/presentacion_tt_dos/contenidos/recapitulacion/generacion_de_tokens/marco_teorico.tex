%
% Presentación de TT2.
% Proyecto Lovelace.
%
% Recapitulación.
% Prototipo de generación de tokens.
% Marco teórico.
%

\subsubsection{Marco teórico}

\begin{frame}{Generación de tokens}
{Clasificación del PCI}

  A continuación, la clasificación propuesta por el PCI SSC~\cite{pci_tokens}.

  \begin{itemize}
    \item Reversibles
    \begin{itemize}
      \item Criptográficos (FFX, BPS)
      \item No criptográficos (TKR, AHR, DRBG)
    \end{itemize}
    \item Irreversibles
    \begin{itemize}
      \item Autenticables
      \item No autenticables
    \end{itemize}
  \end{itemize}

  \note{
    \begin{itemize}
      \item \textbf{Irreversibles.}
        Bajo NINGUNA circunstancia se puede recuperar el PAN partiendo del
        token.
        \begin{itemize}
          \item \textbf{Autenticables.}
            Permiten validar que un token dado pertenece a un PAN.
          \item \textbf{No autenticables.}
            Sabrá Dios para qué quieren estos tokens: no puedes volver al PAN
            con el token ni puedes saber si un PAN fue utilizado para generar
            X token.
        \end{itemize}

      \item \textbf{Reversibles.}
      \begin{itemize}
        \item \textbf{Criptográficos.}
          Utilizan criptografía \emph{fuerte}; además, el token no se almacena,
          solo la llave con la que fue cifrado.
        \item \textbf{No criptográficos.}
          Se guarda la relación PAN-token en una base de datos; además, la
          base de datos debe estar cifrada.
      \end{itemize}
    \end{itemize}
  }

\end{frame}

\begin{frame}{Prototipo de generación de tokens}
{Clasificación propuesta}

  A continuación, la clasificación propuesta.

  \begin{itemize}
    \item Criptográficos
    \begin{itemize}
      \item Reversibles (FFX, BPS)
      \item No reversibles (TKR, AHR, DRBG)
    \end{itemize}
    \item No criptográficos
  \end{itemize}

  \note{
    \begin{itemize}
      \item Primero se separan los que usan criptografía y los que ño.
      \item Luego, con unos nombres más transparentes, le siguen, los
        reversibles e irreversibles.
      \item Los reversibles, como su nombre lo dice, tienen la posibilidad de
        regresar, partiendo del token, al PAN. Es el equivalente a los
        \emph{criptográficos} en la clasificación del PCI, pues se guarda
        solo la llave con la que fueron creados.
      \item Los irreversibles no pueden hacer esto. Es menester guardar la
        relación entre un PAN y su token; si se pierde la relación, es
        imposible regresar, partiendo del token, al PAN.
      \item El ejemplo por antonomasia de los no criprográficos es el generador
        de bits VERDADERAMENTE aleatorio; pero son costosos y medio raros.
    \end{itemize}
  }

\end{frame}
