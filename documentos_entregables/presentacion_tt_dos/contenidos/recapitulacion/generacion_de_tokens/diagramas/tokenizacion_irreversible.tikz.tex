%
% Presentación de TT2.
% Proyecto Lovelace.
%
% Recapitulación.
% Generación de tokens.
% Diagrama de tokenización con métodos irreversibles.
%

\begin{tikzpicture}[
  entorno/.style={
    rectangle,
    draw = black,
    thin,
    inner sep = 1mm,
    text width = 15mm,
    align = center,
    minimum height = 1.0cm},
  etiqueta/.style={
    align = center
  },
  database/.style={
    scale=0.6,
    cylinder,
    cylinder uses custom fill,
    cylinder body fill=white!50,
    cylinder end fill=white!50,
    shape border rotate=90,
    aspect=0.25,
    node distance=0.8cm,
    minimum width=3.5cm,
    minimum height=1.2cm,
    font=\footnotesize,
    draw
  }]

  \node[entorno]
    (irreversible)
    {Algoritmo \\ tokenizador};

  \node[database, right = 0.5in of irreversible, yshift = -0.31in]
    (base_down) [text=white] {Base de datos};
  \node[database, above of=base_down]
    (base) [text=black] {Base de datos};
  \node[database, above of=base]
    (base_top) [text=white] {Base de datos};

  \draw[-Stealth]
    ($(irreversible.west) + (-0.5in, 0) + (0, 3mm)$)
    node[etiqueta, anchor = east]
      {Número de \\ tarjeta}
    --
    ($(irreversible.west) + (0, 3mm)$);

  \draw[-Stealth]
    ($(irreversible.west) + (-0.5in, 0) - (0, 3mm)$)
    node[etiqueta, anchor = east]
      {Llave}
    --
    ($(irreversible.west) - (0, 3mm)$);

  \draw[-Stealth]
    (irreversible.east)
    --
    (base.west);

  \draw[-Stealth]
    (base.east)
    --
    ($(base.east) + (0.5in, 0)$)
    node[etiqueta, anchor = west]
      {Token};

\end{tikzpicture}
