%
% Presentación de TT2.
% Proyecto Lovelace.
%
% Recapitulación.
% Prototipo de generación de tokens.
% Algoritmos implementados.
%

\subsubsection{Algoritmos implementados}

\begin{frame}{Generación de tokens}
{Algoritmos implementados (reversibles)}

  Entre los algoritmos reversibles, se encuentran FFX y BPS.
  \begin{itemize}
    \item Métodos que utilizan cifrados que preservan el formato.
    \item Cifran la tarjeta y descifran el token.
    \item Se volvieron estándares en 2016 y fueron renombrados por el NIST
      a FF1 y FF3 respectivamente~\cite{nist_fpe}.
    \item Están basados en redes Feistel.
  \end{itemize}

  \note{
    Estos algoritmos utilizan AES y CBC-MAC. Recordar también que la seguridad
    de estos algoritmos tokenizadores SÍ ha sido probada criptográficamente,
    pues están basados en cifrados que preservan el formato.}

\end{frame}

\begin{frame}{Generación de tokens}
  {Algoritmos implementados (reversibles)}
  \begin{figure}
    \centering
    \subimport{diagramas/}{tokenizacion_reversible.tikz.tex}
    \caption{Proceso de tokenización, método reversible.}
  \end{figure}

  \begin{figure}
    \centering
    \subimport{diagramas/}{detokenizacion_reversible.tikz.tex}
    \caption{Proceso de detokenización, método reversible.}
  \end{figure}

\end{frame}

\begin{frame}{Generación de tokens}
{Algoritmos implementados (irreversibles)}

  Entre los algoritmos irreversibles, se encuentran TKR, AHR y DRBG.
  \begin{itemize}
    \item Utilizan varias primitivas criptográficas (cifrados por bloque,
      funciones hash, generadores pseudoaleatorios).
    \item Requieren guardar la relación tarjeta-token.
    \item Su desempeño está ligado al desempeño de la base de datos.
  \end{itemize}

  \note{
    Hacer énfasis en que estos algoritmos pertenecen a los
    \emph{no criptográficos} en la clasificación del PCI, cuando en realidad
    utilizan cifrados por bloque y funciones hash.}

\end{frame}

\begin{frame}{Generación de tokens}
  {Algoritmos implementados (irreversibles)}
  \begin{figure}
    \centering
    \subimport{diagramas/}{tokenizacion_irreversible.tikz.tex}
    \caption{Proceso de tokenización, método irreversible.}
  \end{figure}

  \begin{figure}
    \centering
    \subimport{diagramas/}{detokenizacion_irreversible.tikz.tex}
    \caption{Proceso de detokenización, método irreversible.}
  \end{figure}

\end{frame}
