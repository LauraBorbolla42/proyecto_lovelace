%
% Clasificación de FPE, presentación de cifrados que preservan el formato.
% Proyecto Lovelace.
%

\subsection{Clasificación}

\begin{frame}{Clasificación de FPE}

  En \cite{sinopsis_rogaway}, Phillip Rogaway clasifica a los algoritmos que
  preservan el formato según el tamaño del espacio de mensajes ($ N = |X| $).

  \begin{itemize}
    \item Espacios minúsculos.
    \item Espacios pequeños.
    \item Espacios grandes.
  \end{itemize}

  \only<2>
  {
    \textbf{Espacios minúsculos:} el espacio es tan pequeño que es aceptable
    gastar $ O(N) $ en el proceso de cifrado.

    Por ejemplo, se puede inicializar una tabla de $ N $ elementos, y realizar
    las operaciones de cifrado y descifrado con consultas. Para esto se pueden
    ocupar métodos como el \textit{Knuth shuffle} o un cifrado con prefijo.
  }

  \note<2>
  {
    \textit{Knuth shuffle} (también \textit{Fisher-Yates shuffle}) genera una
    permutación pseudoaleatorioa de una secuencia finita.

    También había un \textit{permutation numbering}; no lo entiendo.

    El cifrado con prefijo utiliza un cifrador estándar para cifrar todos los
    elementos y después utiliza un ordenamiento para determinar la permutación.
  }

  \only<3>
  {
    \textbf{Espacios pequeños:} el espacio no es más grande que $ 2^w $, en
    donde $ w $ es el tamaño de bloque del cifrado subyacente. Para AES, en
    donde $ w = 128 $, $ N = 2^{128} \approx 10^{38} $.

    En este esquema, el mensaje se ve como una cadena de $ n $ elementos
    pertenecientes a un alfabeto de cardinalidad $ m $ (i. e. $ N = m^n $).

    Por ejemplo, para números de tarjetas de crédito, $ n \approx 16 $ y
    $ m = 10 $, por lo que $ N = 10^{16} $ (diez mil trillones); lo cual es
    aproximadamente $ 2.93 \times 10^{-21} \% $ de $ 2^{128} $.
  }

  \note<3>
  {
    \textit{Pequeño} en comparación con el tamaño del bloque, no con
    estándares humanos (p. ej. el universo solo lleva $ \approx 2^{86}$
    nanosegundos de existencia).

    Las técnicas para los espacios pequeños pueden ser usadas en los
    espacios minúsculos, aunque con menos garantías de seguridad que con
    las técnicas propias.
  }

  \only<4>
  {
    \textbf{Espacios grandes:} el espacio es más grande que $ 2^w $.

    Para estos casos, el mensaje se ve como una cadena binaria. Las técnicas
    utilizadas incluyen cualquier cifrado cuya salida sea \textit{de la misma}
    longitud que la entrada (p. ej. los TES: CMC, EME, HCH, etc.).
  }

  \note<4>
  {
    Los ejemplos más clásicos que se usan aquí son los cifradores para sectores
    de discos duros.
  }

\end{frame}
