%
% Propuesta de parámetros A10, presentación de FPE.
% Proyecto Lovelace.
%

\subsection{Colección FFX-A10}

\begin{frame}{FFX-A10}

  En~\cite{ffx_2} se proponenen dos colecciones de parámetros para instanciar
  a FFX: FFX-A2 y FFX-A10. La primera está pensada para cadenas binarias y la
  segunda para cadenas de dígitos.

  Aquí presentamos FFX-A10 dado que es la que se adapta mejor para la
  tokenización de números de tarjetas de crédito.

\end{frame}

\begin{frame}{FFX-A10}{Parámetros}

  \only<1>
  {
    \begin{block}{\textbf{1. \textit{Radix:} 10}}
      Lo que implica que el alfabeto es $ C = \{ 0, 1, \dots, 9 \} $,
    \end{block}

    \begin{block}{\textbf{2. Longitudes: $ [ 4, 36 ] $}}
      Las longitudes de cadenas permitidas.
    \end{block}

    \begin{block}{\textbf{3. Llaves: $ \{ 0, 1\}^{128} $}}
      Tamaño de llave para AES.
    \end{block}
  }

  \only<2>
  {
    \begin{block}{\textbf{4. \textit{Tweaks}: $ \text{BYTE}^{\leq 2^{64} - 1} $}}
      El \textit{tweak} es una cadena de bytes arbitraria.
    \end{block}

    \begin{block}{\textbf{5. Suma: por bloque}}
      Combinación de números mediante sumas a nivel de bloque.
    \end{block}

    \begin{block}{\textbf{6. Método: redes alternantes}}
    \end{block}
  }

  \only<3>
  {
    \begin{block}{\textbf{7. \textit{Split}: $ \lfloor \frac{n}{2} \rfloor $ }}
      Lo más cerca del centro posible.
    \end{block}

    \begin{block}{\textbf{8. Rondas: dependen de $ n $}}
      12, si $ 10 \leq n \leq 36 $; \par
      18, si $ 6 \leq n \leq 9 $; \par
      24, si $ 4 \leq n \leq 5 $
    \end{block}
  }

\end{frame}

\begin{frame}{FFX-A10}{La función de ronda}

  \only<1>
  {
    La función de ronda de FFX-A10 ocupa a CBC MAC con AES: la en ntrada es una
    concatenación de todos los parámetros (una representación como cadena
    binaria), junto con el \textit{tweak} y el mensaje de entrada.

    El MAC ($ Y $) es tratado de la siguiente manera para poder regresar un
    número de la longitud necesaria.
    \begin{align*}
      Y\prime &= Y[1 \dots 64] \quad Y\prime\prime = Y[65 \dots 128] \\
      y\prime &= \text{NUM}_{2}(Y\prime) \quad y\prime\prime
        = \text{NUM}_{2}(Y\prime\prime)
    \end{align*}
  }

  \only<2>
  {
    Ahora, considerando a $ m $ como el lugar de corte del \textit{split} en
    la ronda actual, si este es menor a 9:
    \begin{align*}
      z = y\prime\prime \mod 10^{m}
    \end{align*}

    Y si es mayor a 9:
    \begin{align*}
      z = (y\prime \mod 10^{m - 9}) \cdot 10^{9} + (y\prime\prime \mod 10^{m})
    \end{align*}

    Lo que regresa la función es una representación decimal de $ z $.
  }

\end{frame}