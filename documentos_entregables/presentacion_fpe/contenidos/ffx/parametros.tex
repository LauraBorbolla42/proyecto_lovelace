%
% Parámetros de FFX, presentación de FPE.
% Proyecto Lovelace.
%

\subsection{Definición de parámetros}

\begin{frame}{FFX}{Parámetros}

  Los siguientes 9 parámetros hacen de FFX un esquema muy general, que
  puede ser utilizado para cifrar cadenas de \textit{cualquier} longitud.

  \only<1>
  {
    \begin{block}{\textbf{1. \textit{Radix}}}
      Número que determina el alfabeto usado.
      $ C = \{ 0, 1, \dots, \text{radix} - 1\} $. Tanto el texto en claro como
      el texto cifrado pertenecen a este alfabeto.
    \end{block}

    \begin{block}{\textbf{2. Longitudes}}
      El rango permitido para longitudes de mensaje.
    \end{block}

    \begin{block}{\textbf{3. Llaves}}
      El conjunto que representa al espacio de llaves.
    \end{block}
  }

  \note<1>
  {
    Los parámetros se condicionan unos a otros: el rango permitido depende
    de la función de ronda utilizada.
  }

  \only<2>
  {
    \begin{block}{\textbf{4. \textit{Tweaks}}}
      El conjunto que representa al espacio de \textit{tweaks}.
    \end{block}

    \begin{block}{\textbf{5. Suma}}
      El operador utilizado en la red Feistel para combinar la parte izquierda
      con la salida de la función de ronda.
    \end{block}

    \begin{block}{\textbf{6. Método}}
      El tipo de red Feistel a ocupar: desbalanceada o alternante.
    \end{block}
  }

  \note<2>
  {
    Explicar la diferencia entre una suma a nivel de caracter o una suma a
    nivel de bloque.

    Recordar que las redes balanceadas son caso específico de cualquiera de
    los dos métodos.
  }

  \only<3>
  {
    \begin{block}{\textbf{7. \textit{Split}}}
      El grado de desbalanceo de la red Feistel.
    \end{block}

    \begin{block}{\textbf{8. Rondas}}
      El número de rondas de la red Feistel.
    \end{block}

    \begin{block}{\textbf{9. F}}
      La función de ronda. Recibe la llave, el \textit{tweak}, el número de
      ronda y un mensaje; regresa una cadena del alfabeto de la longitud
      apropiada.
    \end{block}
  }

  \note<3>
  {
    El número de rondas necesario depende de cada caso en particular.
    Por ejemplo, en FF2 van de 12 a 36 y en FF10, de 12 a 24.
  }

\end{frame}
