\documentclass{beamer}

\mode<presentation>{
\usetheme{Dresden}
\setbeamercovered{transparent}
\usecolortheme{lsc}
}

\mode<handout>{
  % tema simples para ser impresso
  \usepackage[bar]{beamerthemetree}
  % Colocando um fundo cinza quando for gerar transparências para serem impressas
  % mais de uma transparência por página
  \beamertemplatesolidbackgroundcolor{black!5}
}

\usepackage{amsmath,amssymb}
\usepackage[brazil]{varioref}
\usepackage[english,brazil]{babel}
\usepackage[utf8]{inputenc}
%\usepackage[latin1]{inputenc}
\usepackage{graphicx}
\usepackage{listings}
\usepackage{url}
\usepackage{colortbl}

\beamertemplatetransparentcovereddynamic

\title[Título no rodapé]{Título}
\author[Autor no rodapé]{%
  Autor\inst{1} \\
  Orientador: \inst{1}}
  \institute[UFSM]{
  \inst{1}%
     Laboratório de Sistemas de Computação\\
     Universidade Federal de Santa Maria}

% Se comentar a linha abaixo, irá aparecer a data quando foi compilada a apresentação
\date{dia de mês de ano}

\pgfdeclareimage[height=1cm]{inf}{figs/CienciaDaComputacao.png}

% pode-se colocar o LOGO assim
\logo{\pgfuseimage{inf}}

\AtBeginSection[]{
  \begin{frame}<beamer>
    \frametitle{Roteiro}
    \tableofcontents[currentsection,currentsubsection]
  \end{frame}
}

\begin{document}

\begin{frame}
\titlepage
\end{frame}

\begin{frame}
\frametitle{Roteiro}
\tableofcontents
\end{frame}


\section{Introdução}
\frame{
    \frametitle{Introdução}
      \begin{itemize}
         \item Contexto
         \item Use ilustrações para ajudar

      \end{itemize}
}

\frame{
    \frametitle{Objetivo e Justificativas}
    \begin{itemize}
       \item Objetivo
       \begin{itemize}
           \item Isso também faz parte da introdução
       \end{itemize}

       \item Justificativas
       \begin{itemize}
       	   \item não exagere na quantidade de texto
       	   \item não exagere na quantidade de texto
       	   \item não exagere na quantidade de texto
       \end{itemize}
   \end{itemize}
}


\section{Fundamentação}


\pgfdeclareimage[height=3cm]{YARNArq}{figs/Figura06-YarnArch.png}%
\pgfdeclareimage[height=3cm]{HDFSArq}{figs/Figura07-HDFS.png}%
\frame{
	\begin{columns}
    \frametitle{Assuntos Importantes}
        \column{3cm}
            Primeiro assunto
            \begin{itemize}
            		\item \textit{NameNode}
	            \item \textit{DataNode}
            \end{itemize}
            \pgfuseimage{HDFSArq}
            \begin{tiny}

            (HADOOP, 2013)
            \end{tiny}
		\column{5cm}
			Segundo assunto
			\begin{itemize}
				\item \textit{ResourceManager}
				\item \textit{NodeManager}
			\end{itemize}
			\pgfuseimage{YARNArq}
			\begin{tiny}

            (HADOOP, 2013)
            \end{tiny}
    \end{columns}
}

\frame{
	\frametitle{Outro Assunto Importante}
	\begin{itemize}
		\item Item
		\item Item
		\begin{itemize}
			\item Item
			\item Item
			\item Item
		\end{itemize}

	\end{itemize}
}

\logo{\pgfuseimage{inf}}
\frame{
	\frametitle{Trabalhos relacionados}
	\begin{columns}
		\column{5cm}
		\begin{itemize}
			\item (Kumar et al., 2012).
			\item (Rasooli; Down, 2012).
			\item (Chen et al., 2010).
			\item (Xie et al., 2010).
			\item (Tian et al., 2009).
			\item (Isard et al., 2009).
			\item (Zaharia et al., 2008).
		\end{itemize}

		\column{7,5cm}
			\begin{itemize}
				\item Item
				\begin{itemize}
					\item Item
					\item Item
				\end{itemize}

				\item Item
				\begin{itemize}
					\item Item
					\item Item
				\end{itemize}
			\end{itemize}
	\end{columns}
}
\section{Desenvolvimento}
\logo{}
\frame{
    \frametitle{Arquitetura}
    \pgfdeclareimage[height=6.5cm]{RM}{figs/Figura14-RMHorton.png}
    %TODO fig RM

    \pgfuseimage{RM}
    \begin{tiny}
	{(HortonWorks, 2014)}
    \end{tiny}
}



\logo{}
\frame{
    \frametitle{Diagrama de classes}
    \center
    \pgfdeclareimage[height=5.5cm]{CDCS}{figs/Figura01-ClassDiagram.png}
    \pgfuseimage{CDCS}
}




\section{Resultados}
\logo{\pgfuseimage{inf}}
\frame{
	\frametitle{Ambiente de experimentação}

	\begin{itemize}
		\item \textit{Hardware}: 2 CPU AMD@1.7Ghz, 12 cores/CPU e 47GB RAM.
		\item \textit{Software}: Ubuntu x64 12.04, Hadoop 2.2.0, Sun JDK 1.7.
	\end{itemize}
}

\logo{\pgfuseimage{inf}}
\frame{
	\frametitle{Configuração dos experimentos}
	Texto texto texto texto texto texto texto texto
	\begin{table}
		\renewcommand{\figurename}{Table}
		\centering
		\begin{tabular}{|l|c|c|}
			\hline
			  & Coluna1 & Coluna2 \\
			\hline
			\textit{Linha1} & 1 & 2 \\
			\hline
			\textit{Linha2} Vcores & 3 & 4 \\
			\hline
		\end{tabular}
	\end{table}
}


\logo{}
\frame{
	\frametitle{Resultados de desempenho}

	\pgfdeclareimage[height=6cm]{totalVcores}{figs/Figura12-totalCores.png}
    %TODO fig colectInt
    \pgfuseimage{totalVcores}
}



\section{Conclusão e Trabalhos Futuros}
\logo{\pgfuseimage{inf}}
\frame{
   \frametitle{Conclusão}
   	\begin{itemize}
   		\item Item
        \item Item
        \item Item
   	\end{itemize}
}

\logo{\pgfuseimage{inf}}
\frame{
   \frametitle{Trabalhos Futuros}
   	\begin{itemize}
   		\item Item
        \item Item
        \item Item
   	\end{itemize}
}


\section{Referências}
\logo{}
\frame{
    \frametitle{Referências}
    \begin{itemize}
	\begin{tiny}
    	   \item DEY, A. K. Understanding and Using Context. Personal Ubiquitous Comput., London, UK, UK, v.5, n.1, p.4-7, Jan. 2001.
       \item MAAMAR, Z.; BENSLIMANE, D.; NARENDRA, N. C. What can context do for web services? Commun. ACM, New York, NY, USA, v.49, n.12, p.98-103, Dec. 2006.
       \item KUMAR, K. A. et al. CASH: context aware scheduler for hadoop. In: INTERNATIONAL CONFERENCE ON ADVANCES IN COMPUTING, COMMUNICATIONS AND INFORMATICS, New York, NY, USA. Proceedings. . . ACM, 2012. p.52-61. (ICACCI '12).
       \item ZAHARIA, M. et al. Improving MapReduce performance in heterogeneous environments. In: USENIX CONFERENCE ON OPERATING SYSTEMS DESIGN AND IMPLEMENTATION, 8., Berkeley, CA, USA. Proceedings. . . USENIX Association, 2008. p.29-42. (OSDI'08).
       \item TIAN, C. et al. A Dynamic MapReduce Scheduler for Heterogeneous Workloads. In: EIGHTH INTERNATIONAL CONFERENCE ON GRID AND COOPERATIVE COMPUTING, 2009., Washington, DC, USA. Proceedings. . . IEEE Computer Society, 2009. p.218-224. (GCC '09).
        \item CHEN, Q. et al. SAMR: a self-adaptive mapreduce scheduling algorithm in heterogeneous environment. In: IEEE INTERNATIONAL CONFERENCE ON COMPUTER AND INFORMATION TECHNOLOGY, 2010., Washington, DC, USA. Proceedings. . . IEEE Computer Society, 2010. p.2736-2743. (CIT '10).

	\end{tiny}
    \end{itemize}
}

\logo{}
\frame{
    \frametitle{Referências (Continuação)}
    \begin{itemize}
    \begin{tiny}
       \item RASOOLI, A.; DOWN, D. G. Coshh: a classification and optimization based scheduler for heterogeneous hadoop systems. In: SC COMPANION: HIGH PERFORMANCE COMPUTING, NETWORKING STORAGE AND ANALYSIS, 2012., Washington, DC, USA. Proceedings. . . IEEE Computer Society, 2012. p.1284-1291. (SCC '12).
       \item ISARD, M. et al. Quincy: fair scheduling for distributed computing clusters. In: ACM SIGOPS 22ND SYMPOSIUM ON OPERATING SYSTEMS PRINCIPLES, New York, NY, USA. Proceedings. . . ACM, 2009. p.261-276. (SOSP '09).
       \item XIE, J. et al. Improving MapReduce performance through data placement in heterogeneous Hadoop clusters. In: PARALLEL AND DISTRIBUTED PROCESSING, WORKSHOPS AND PHD FORUM (IPDPSW). Anais. . . IEEE International Symposium, 2010.
       \item HADOOP, A. Arquitetura do HDFS. http://hadoop.apache.org/docs/current/hadoop-project-dist/hadoop-hdfs/Federation.html, Acesso em novembro de 2013.
       \item HADOOP, A. Arquitetura do YARN. http://hadoop.apache.org/docs/current/hadoop-yarn/hadoop-yarn-site/YARN.html, Acesso em novembro de 2013.

	\end{tiny}
    \end{itemize}
}





\begin{frame}
\titlepage
\end{frame}

\end{document}
