%
% Sección de generadores pseudoaleatorios,
% Presentación de TT uno.
%
% Proyecto Lovelace.
%

\subsection{Generación de números aleatorios}

\begin{frame}{Generación de números aleatorios}

  Existen dos maneras de generar números aleatorios:

  \begin{itemize}

    \item<2-> Mediante un generador no determinístico (\textit{Non-deterministic
      Random Bit Generator}, NRBG).

      \begin{itemize}
        \item Están ligados a un proceso físico impredecible.
        \item Son difíciles de implementar y no existe ningún estándar
          aprobado.
      \end{itemize}

    \note<2>
    {
      Como con los volados: si tuvieramos manera de medir con precisión todos
      los subprocesos involucrados (el tamaño y peso de la moneda, la forma de
      mis dedos, la posición de la moneda con respecto a mis dedos, el punto de
      impacto, la fuerza del impacto, la resistencia del aire, la fuerza de
      gravedad en el punto de la tierra en cuetión, la presión atmosférica, etc)
      los volados dejarían de ser aleatorios.

      Estos generadores buscan este tipo de procesos para generar números. Por
      ejemplo, la temperatura del procesador en un momento dado, los tiempos de
      acceso a memoria, los tiempos de respuesta a través de una comunicación
      por internet, la posición del mouse, etcétera.
    
      Dada su naturaleza dependiente del hardware, son difíciles de implementar
      y, por lo tanto, aún no existe ningún estándar al respecto.
    }

    \item<3-> Mediante un generador determinístico (\textit{Deterministic
      Random Bit Generator}, DRBG).

      \begin{itemize}
        \item Utiliza un mecanismo claramente definido para producir
          secuencias de bits a partir de un valor inicial.
        \item Dada la naturaleza determinística, se dice que los números son
          \textit{pseudoaleatorios}.
      \end{itemize}

    \note<3>
    {
      Este valor es conocido como semilla.
      Dada la naturaleza determinística, alguien con el mismo valor inicial (con
      la misma semilla) puede generar los mismos números pseudoaleatorios. En un
      criptográfico se busca evitar esto, por lo que el valor de la semilla se
      debe mantener en secreto.

      Generalmente el valor inicial se obtiene a través de una fuente de
      aleatoriedad (o de entropia). Estas, al igual que los NRBG, están ligadas
      a un proceso físico, pero a diferencia de estos, con un valor pequeño
      pueden generar muchos números pseudoaleatorios.
    }

  \end{itemize}

\end{frame}

