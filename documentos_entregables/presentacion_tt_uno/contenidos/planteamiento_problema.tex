%
% Presentación de TT-I.
% Planteamiento del problema.
%
% Proyecto Lovelace.
%

% NOTA: los títulos de las diapositivas son, aún, temporales.

\begin{frame}{Un inicio tormentoso}
  \begin{itemize}
    \item En la década de los 80 y 90, el comercio en línea comenzó a crecer y
      tomar importancia.
    \item Las empresas no estaban preparadas para el impacto que tuvieron y
      los fraudes relacionados con el comercio electrónico aumentaron
      rápidamente~\cite{search_security}.
      \begin{itemize}
        \item Visa y Mastercard reportaron, entre 1988 y 1998, pérdidas de 750
          millones de dólares.
        \item En 2001, se reportaron pérdidas de 1.7 miles de millones de
          dólares. y  2.1 miles de millones de dólares al año siguiente.
      \end{itemize}
  \end{itemize}
\end{frame}

\begin{frame}{Un estándar para gobernarlos a todos}
  % One ring to rule them all.
  \begin{itemize}
    \item A inicios del 2000, las grandes compañías (¿emisoras de tarjetas?)
      comenzaron a publicar, individualmente, \textit{buenas prácticas} de
      seguridad.
    \item Las empresas intentaron adoptar las prácticas, pero era tremendamente
      complicado y costoso.
    \item Se aliaron las compañías y, en 2004, publicaron un estándar unificado:
      PCI-DSS\footnotemark~\cite{pci_dss}.
      \begin{itemize}
        \item Se hizo obligatorio para quienes realizasen más de 20K
          transaccciones al año.
        \item Tiene un gran número de requerimientos (y subrequerimientos), por
          lo que es difícil de satisfacer.
      \end{itemize}
  \end{itemize}
  \footnotetext{
    Payment Card Industry - Data Security Standard
  }
\end{frame}

\begin{frame}{Cambio de estrategia}
  \begin{itemize}
    \item Hasta ahora, el enfoque era proteger los datos sensibles donde sea
      que se encuentren y por donde sea que transiten.
    \item Surge un nuevo enfoque: cambiar la información valiosa, por
      \textit{valores representativos} (tokens); es decir, la tokenización
      de la información.
    \item En 2011, el PCI-SSC\footnotemark publicó las primeras guías para los
      procesos de tokenización~\cite{pci_tokens}.
      \begin{itemize}
        \item Aunque indica lo que debe satisfacer el sistema tokenizador,
          no dice cómo generar los tokens.
      \end{itemize}
  \end{itemize}
  \footnotetext{
    Payment Card Industry - Security Standards Council
  }
\end{frame}

\begin{frame}{Pero ¿por qué?}
  A pesar de ser una práctica extendida, la tokenización sigue estando
  rodeada de desinformación y desconfianza.
  \begin{itemize}
    \item Se busca combatir la desinformación al estudiar e implementar cinco
      algoritmos tokenizadores, compararlos y mostrar los resultados.
    \item Hacer notar que la criptografía y la tokenización no están peleadas;
      pues la tokenización puede verse como una aplicación de la criptografía.
  \end{itemize}
\end{frame}
