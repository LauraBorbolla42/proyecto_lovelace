%
% Archivo principal de artículo.
% Proyecto Lovelace.
%

\documentclass[11pt]{llncs}

% http://mirrors.ctan.org/macros/latex/base/inputenc.pdf
% Soporte para distintas codificaciones.
% \usepackage[utf8]{inputenc}

% http://mirrors.ctan.org/macros/latex/required/babel/base/babel.pdf
% Soporte para lenguajes distintos del inglés.
\RequirePackage[spanish, mexico, es-noindentfirst]{babel}

% http://ctan.math.washington.edu/tex-archive/macros/latex/contrib/
% import/import.pdf
% Establece inputs relativos con respecto a una carpeta.
\RequirePackage{import}

% http://mirrors.ctan.org/macros/latex/required/graphics/grfguide.pdf
\RequirePackage{graphicx}

% http://mirrors.ctan.org/macros/latex/required/amsmath/amsmath.pdf
\RequirePackage{amsmath}

% http://ctan.math.illinois.edu/fonts/amsfonts/doc/amssymb.pdf
% Simbología extra.
\RequirePackage{amssymb}

\begin{document}

  \title{Estudio y comparación de métodos de tokenización}
  \author{Daniel Ayala Zamorano,
     Laura Natalia Borbolla Palacios, \\
     Ricardo Quezada Figueroa}
  \institute{Escuela Superior de Cómputo,
    Instituto Politécnico Nacional\\
    \email{daniel@ejemplo.com,
      laura@ejemplo.com,
      qf7.ricardo@gmail.com}}
  % :O ¿Por qué el email va dentro de la institución?
  \maketitle

  \begin{abstract}
    La tokenización consiste en el reemplazo de información sensible por valores
    sustitutos, llamados tokens, en donde el camino de regreso, del  token a la
    información sensible, no es factible. En los últimos años este proceso se ha
    vuelto muy popular entre los comercios en línea, pues les permite descargar
    parte de las responsabilidades de seguridad adquiridas al manejar números de
    tarjetas de crédito en un tercero, proveedor de servicios de tokenización.
    Lamentablemente, existe una gran cantidad de desinformación alrededor de
    cómo generar los tokens, principalmente producida por las estrategias
    publicitarias de las empresas tokenizadoras, en donde cada una intenta
    convencer al comprador de que su sistema es el mejor, sin explicar realmente
    qué es lo que hacen para generar tokens. Uno de los mensajes más comunes
    entre la publicidad es que la criptografía y la tokenización son cosas
    distintas, y la segunda es mucho más segura. En este trabajo se explica a
    detalle en qué consiste la tokenización y cuál es su relación con la
    criptografía; se revisan y comparan los desempeños de los métodos más comunes
    para tokenizar; para terminar se concluye con una discusión alrededor de las
    ventajas y desventajas de cada uno.
  \end{abstract}

  \import{articulo/contenidos/}{introduccion}
  \import{articulo/contenidos/}{preliminares_y_notacion}
  \import{articulo/contenidos/}{algoritmos_tokenizadores}
  %\import{articulo/contenidos/}{resultados}
  %\import{articulo/contenidos/}{conclusiones}

%  \section{Clasificación de los algoritmos tokenizadores}
%
%  \section{Métodos reversibles}
%
%  \subsection{FFX (\textit{Format-preserving Feistel-based Encryption})}
%
%  \subsection{BPS (Brier, Peyrin, Stern)}
%
%  \section{Métodos irreversibles}
%
%  \subsection{TKR}
%
%  \subsection{RHA (\textit{Reversile Hybrid Algorithm})}
%  % Creo que tenemos un problema con el nombre de este.
%  % ¿reversible o irreversible?
%
%  \subsection{UTO (\textit{Updatable Tokenization})}
%
%  \subsection{Basados en DRBG (\textit{Deterministic Random Bit Generator})}

  \bibliographystyle{abbrv}
  \bibliography{referencias}

\end{document}
