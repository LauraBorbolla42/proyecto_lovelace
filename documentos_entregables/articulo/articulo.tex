%
% Archivo principal de artículo.
% Proyecto Lovelace.
%

% TODO: Revisar apariciones de siglas, esto es, que siempre aparezcan
% de forma explícita en la primera ocasión.

\documentclass[conference]{IEEEtran}

% Propios
\usepackage[spanish, mexico, es-noindentfirst]{babel}
\usepackage{import}

% Agregados a las traducciones de babel.
% Técnicamente esto tenría que venir desde INAOE, para
% estandarizar.
\addto\captionsspanish
{
  \renewcommand*\IEEEkeywordsname{Palabras clave}
}

% Parte de plantilla de IEEE.
\usepackage{cite}
\usepackage{amsmath,amssymb,amsfonts}
\usepackage{algorithmic}
\usepackage{graphicx}
\usepackage{textcomp}
\usepackage{xcolor}
\def\BibTeX{{\rm B\kern-.05em{\sc i\kern-.025em b}\kern-.08em
    T\kern-.1667em\lower.7ex\hbox{E}\kern-.125emX}}

\begin{document}

  \title{Estudio y comparación de métodos de tokenización}

  % Modo compacto.
  %\author{%
  %  \IEEEauthorblockN{%
  %    1\textsuperscript{o} Ricardo Quezada Figueroa}
  %  \IEEEauthorblockA{%
  %    \textit{ESCOM IPN}\\
  %    Ciudad de México, México \\
  %    qf7.ricardo@gmail.com}
  %  \and
  %  \IEEEauthorblockN{%
  %    2\textsuperscript{o} Laura Natalia Borbolla Palacios}
  %  \IEEEauthorblockA{%
  %    \textit{ESCOM IPN}\\
  %    Ciudad de México, México \\
  %    ln.borbolla.42@gmail.com}
  %  \and
  %  \IEEEauthorblockN{%
  %    3\textsuperscript{o} Daniel Ayala Zamorano}
  %  \IEEEauthorblockA{%
  %    \textit{ESCOM IPN}\\
  %    Ciudad de México, México \\
  %    daz23ayala@gmail.com}}

  % Modo extendido.
  \author{%
    \IEEEauthorblockN{%
      Ricardo Quezada Figueroa\IEEEauthorrefmark{1},
      Laura Natalia Borbolla Palacios\IEEEauthorrefmark{1},
      Daniel Ayala Zamorano\IEEEauthorrefmark{1}}
    \IEEEauthorblockA{%
      \IEEEauthorrefmark{1} Escuela Superior de Cómputo,
      Instituto Politécnico Nacional\\
      Ciudad de México, México\\
      qf7.ricardo@gmail.com, ln.borbolla.42@gmail.com, daz23ayala@gmail.com}}

  % En el manual recomiendan el modo extendido solo cuando son más de tres
  % autores o la longitud del bloque toma maś de dos filas. Tendrían que
  % agregar el caso de cuando hay instituciones en común (como en este caso).
  % Si se deja el modo compacto resulta repetitivo que los tres tengan
  % la misma institución.

  \maketitle

  \begin{abstract}
    La tokenización consiste en el reemplazo de información sensible por valores
    sustitutos, llamados tokens, en donde el camino de regreso, del  token a la
    información sensible, no es factible. En los últimos años este proceso se ha
    vuelto muy popular entre los comercios en línea, pues les permite descargar
    parte de las responsabilidades de seguridad adquiridas al manejar números de
    tarjetas de crédito en un tercero, proveedor de servicios de tokenización.
    Lamentablemente, existe una gran cantidad de desinformación alrededor de
    cómo generar los tokens, principalmente producida por las estrategias
    publicitarias de las empresas tokenizadoras, en donde cada una intenta
    convencer al comprador de que su sistema es el mejor, sin explicar realmente
    qué es lo que hacen para generar tokens. Uno de los mensajes más comunes
    entre la publicidad es que la criptografía y la tokenización son cosas
    distintas, y la segunda es mucho más segura. En este trabajo se explica a
    detalle en qué consiste la tokenización y cuál es su relación con la
    criptografía; se revisan y comparan los desempeños de los métodos más comunes
    para tokenizar; para terminar se concluye con una discusión alrededor de las
    ventajas y desventajas de cada uno.
  \end{abstract}

  % ¿Ciberseguridad? ¿De dónde surgió esto? No le veo mucha relación con lo que
  % se supone que es la cibernética...
  \begin{IEEEkeywords}
    tokenización, criptografía, seguridad web
  \end{IEEEkeywords}

  \import{articulo/contenidos/}{introduccion}
  \import{articulo/contenidos/}{preliminares_y_notacion}
  \import{articulo/contenidos/algoritmos_tokenizadores/}
    {algoritmos_tokenizadores}
  \import{articulo/contenidos/}{resultados}
  \import{articulo/contenidos/}{conclusiones}

  \bibliographystyle{abbrv}
  \bibliography{referencias}

\end{document}
