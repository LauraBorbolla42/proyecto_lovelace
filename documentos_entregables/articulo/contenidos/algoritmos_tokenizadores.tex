%
% Sección de algoritmos tokenizadores.
% Artículo.
% Proyecto Lovelace.
%

\section{Algoritmos tokenizadores}

Como el enfoque de este artículo es ver a la tokenización como un servicio, la
interfaz para los procesos de tokenización y detokenización, desde el punto de
vista de los usuarios del servicio, es sumamente simple: el proceso de
tokenización es una función $ E: \mathcal{X} \rightarrow \mathcal{Y} $ y el de
detokenización es  simplemente la función inversa $ D: \mathcal{Y} \rightarrow
\mathcal{X} $, en donde $ \mathcal{X} $ y $ \mathcal{Y} $ son los espacios de
números de tarjetas y tokens, respectivamente. Ambos conjuntos son cadenas de
dígitos de entre 12 y 19 caracteres. Los números de tarjeta cuentan con un
dígito verificador que hace que \texttt{algoritmoDeLuhn(X) = 0}; los tokens
cuentan con un dígito verificador que hace que \texttt{algoritmoDeLuhn(Y) = 1}.
El último punto es con el propósito de que sea posible distinguir entre un
número de tarjeta y un token.

El PCI SSC (\textit{Payment Card Industry Security Standard Council}) establece
en sus guías de tokenización la siguiente clasificación para los algoritmos
tokenizadores\cite{pci_tokens}:

% TODO: explicar cada una.

\begin{itemize}
  \item Métodos reversibles:
  \begin{itemize}
    \item Criptográficos.
    \item No criptográficos.
  \end{itemize}
  \item Métodos irreversibles:
  \begin{itemize}
    \item Autenticable.
    \item No autenticable.
  \end{itemize}
\end{itemize}

La denominación \textit{no criptográficos} resulta totalmente confusa, pues en
realidad todos los métodos conocidos que caen en las categorías de arriba ocupan
primitivas criptográficas. % TODO: decir que los irreversibles son inútiles.
Por lo anterior, en este trabajo se propone una clasificación distinta:

% TODO: explicar cada una.

\begin{itemize}
  \item Métodos criptográficos:
  \begin{itemize}
    \item Reversibles.
    \item Irreversibles.
  \end{itemize}
  \item Métodos no criptográficos:
\end{itemize}

\subsection{FFX (\textit{Format-preserving Feistel-based Encryption})}
