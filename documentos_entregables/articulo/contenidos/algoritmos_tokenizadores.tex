%
% Sección de algoritmos tokenizadores.
% Artículo.
% Proyecto Lovelace.
%

\section{Algoritmos tokenizadores}

Como el enfoque de este artículo es ver a la tokenización como un servicio, la
interfaz para los procesos de tokenización y detokenización, desde el punto de
vista de los usuarios del servicio, es sumamente simple: el proceso de
tokenización es una función $ E: \mathcal{X} \rightarrow \mathcal{Y} $ y el de
detokenización es  simplemente la función inversa $ D: \mathcal{Y} \rightarrow
\mathcal{X} $, en donde $ \mathcal{X} $ y $ \mathcal{Y} $ son los espacios de
números de tarjetas y tokens, respectivamente. Ambos conjuntos son cadenas de
dígitos de entre 12 y 19 caracteres. Los números de tarjeta cuentan con un
dígito verificador que hace que \texttt{algoritmoDeLuhn(X) = 0}; los tokens
cuentan con un dígito verificador que hace que \texttt{algoritmoDeLuhn(Y) = 1}.
El último punto es con el propósito de que sea posible distinguir entre un
número de tarjeta y un token.

El PCI SSC (\textit{Payment Card Industry Security Standard Council}) establece
en sus guías de tokenización la siguiente clasificación para los algoritmos
tokenizadores\cite{pci_tokens}:

% TODO: explicar cada una.

\begin{itemize}
  \item Métodos reversibles:
  \begin{itemize}
    \item Criptográficos.
    \item No criptográficos.
  \end{itemize}
  \item Métodos irreversibles:
  \begin{itemize}
    \item Autenticable.
    \item No autenticable.
  \end{itemize}
\end{itemize}

La denominación \textit{no criptográficos} resulta totalmente confusa, pues en
realidad todos los métodos conocidos que caen en las categorías de arriba ocupan
primitivas criptográficas. % TODO: decir que los irreversibles son inútiles.
Por lo anterior, en este trabajo se propone una clasificación distinta:

% TODO: explicar cada una.

\begin{itemize}
  \item Métodos criptográficos:
  \begin{itemize}
    \item Reversibles.
    \item Irreversibles.
  \end{itemize}
  \item Métodos no criptográficos:
\end{itemize}

\subsection{FFX (\textit{Format-preserving Feistel-based Encryption})}

Cifrado que preserva el formato presentado en \cite{ffx_1} por Mihir Bellare,
Phillip Rogaway y Terence Spies.  En su forma más general, el algoritmo se
compone de 9 parámetros que permiten cifrar cadenas de cualquier longitud en
cualquier alfabeto; los autores también proponen dos formas más específicas (dos
colecciones de parámetros) para alfabetos binarios y alfabetos decimales: A2 y
A10, respectivamente. De aquí en adelante se hablará solamente de la colección
A10.

% TODO: Poner significado de AES CBC MAC, ¿en los preliminares?

FFX ocupa una red Feistel alternante junto con una adaptación de AES-CBC-MAC
(usada como función de ronda) para lograr preservar el formato. La operación
general del algoritmo se describe completamente por la operación de una red
alternante:
\begin{equation}
  \begin{split}
    L_{i} &=
    \begin{cases}
      F_k(R_{i - 1}) \oplus L_{i - 1},  & \text{si } i \text{ es par} \\
      L_{i - 1},                          & \text{si } i \text{ es impar}
    \end{cases}
    \\
    R_{i} &=
    \begin{cases}
      R_{i - 1},                          & \text{si } i \text{ es par} \\
      F_k(L_{i - 1}) \oplus R_{i - 1},  & \text{si } i \text{ es impar}
    \end{cases}
  \end{split}
\end{equation}

En la figura \ref{ffx_ronda} se describe a la función de ronda. La idea general
consiste en interpretar la salida de AES CBC MAC de forma que tenga el formato
deseado. El valor de $ m $ corresponde al \textit{split} en la ronda actual;
esto es la longitud de la cadena de entrada.

% Esto está terrible ¿Para qué son las plantillas de Springer si se termina con
% este tipo de cosas?

% IMPORTANTE. Para darle un margen en el costado derecho a la minipágina:
% compilan una primera vez, identifican la línea más larga, le agregan cuatro
% espacios al final (\ \ \ \). En este caso es la línea 8.

% TODO: ¿Pongo toda la concatenación de parámetros inútiles?
% Así se ve más limpio todo, claro que estrictamente hablando, no es FFX A10.

\begin{figure}
  \begin{center}
    \begin{tabular}{|l|}
      \hline
      \begin{minipage}{220pt}
        \begin{tabbing}
          \ \ \ \ \ \=\ \ \ \ \=\ \ \ \ \=\ \ \ \ \=\ \ \ \ \=\ \ \ \ \=\ \ \
          \ \kill \\
          \ \ \ \ {\bf Algoritmo} FFX-AES-CBC-MAC($ x $, $ k $, $ t $)\\
          \> 1. \> $ a \gets x \ || \ t $ \\
          \> 2. \> $ b \gets $ \texttt{aes\_cbc\_mac($ a $, $ k $)} \\
          \> 3. \> $ y\prime \gets $ a[1 ... 64] \\
          \> 4. \> $ y\prime\prime \gets $ a[65 ... 128] \\
          \> 5. \> {\bf si} m $ \leq 9 $ {\bf entonces}: \\
          \> 6. \> \> c $ \gets y\prime\prime \mod 10^m $ \\
          \> 7. \> {\bf sino}: \\
          \> 8. \> \> c $ \gets (y\prime \mod 10^{m - 9})
                          \times 10^9 + (y\prime\prime \mod 10^m)$ \ \ \ \  \\
          \> 9. \> {\bf regresar} c \\
        \end{tabbing}
        \end{minipage}\\
        \hline
      \end{tabular}
    \end{center}
    \caption{\label{ffx_ronda} Función de ronda de FFX A10.}
\end{figure}

% Esto suena a broma, solo cambiamos el orden.

Con la clasificación del NIST, este método cae en los reversibles
criptográficos. Con la clasificación propuesta en este trabajo, se trata de un
criptográfico reversible.

\subsection{TKR}

%\subsection{BPS}
%
%\subsection{AHR}
