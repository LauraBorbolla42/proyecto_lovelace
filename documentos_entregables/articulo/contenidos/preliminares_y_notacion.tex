%
% Sección de preliminares.
% Artículo.
% Proyecto Lovelace.
%

\section{Preliminares}

% TODO: ¿Qué más va en los preliminares? Candidatos:
% AES
% Redes Feistel
% CBC MAC

\subsection{Notación}

% TODO: Revisar todas las fórmulas y pseudocódigos y poner las cuestiones
% relacionadas de notación en este párrafo.

Se denotarán a todas las cadenas de bits de longitud $ n $ como $ \{ 0, 1 \}^n $.
Para cadenas $ x $ y $ y $ de símbolos sobre alfabetos arbitrarios, con
$ | x | $ se denotará la longitud de la cadena y con $ x \ || \ y $ la
concatenación de ambas cadenas. El operador $ \oplus $ simboliza la combinación
de los operandos, ya sea mediante sumas sin acarreo (equibalente del operador
\texttt{xor} en cadenas binarias) o mediante suma global modular.

\subsection{Primitivas criptográficas}

Un cifrado por bloques es un cifrado simétrico que se define por la función $ E:
\mathcal{M} \times \mathcal{K} \rightarrow \mathcal{C} $ en donde $ \mathcal{M} $
es el espacio de textos en claro, $ \mathcal{K} $ es el espacio de llaves y $
\mathcal{C} $ es el espacio de mensajes cifrados. Tanto los mensajes en claro
como los cifrados tienen una misma longitud $ n $, que representa el tamaño del
bloque \cite{menezes}.

% Los cifrados por bloque son un elemento de construcción fundamental para otras
% primitivas criptográficas. Muchos de los algoritmos tokenizadores que se
% presentan en este trabajo los ocupan de alguna forma. Las definiciones de los
% algoritmos son flexibles en el sentido de que permiten instanciar cada
% implementación con el cifrado por bloques que se quiera; en el caso de las
% implementaciones hechas para este trabajo se ocupó AES (\textit{Advanced
% Encryption Standard}) en la mayoría de los casos.

% Coregir letra del contradominio de la función de FPE e investigar qué
% demonios es.
% Pues no entiendo por el contexto qué significa; de momento la borré.

% ¿Tweak va en itálicas?
% http://www.rae.es/consultas/los-extranjerismos-y-
% latinismos-crudos-no-adaptados-deben-escribirse-en-cursiva
% Pues, a diferencia de token, me parece que con la «tw» sí hacemos uso de
% la pronunciación en inglés; cuando menos ahora no se me ocurre una palabra
% en español que tenga «tw».
% Sí, debe de ir en cursivas, o cambiar por <<ajuste>>.

Un cifrado que preserva el formato (en inglés \textit{Format-preserving
Encryption}, FPE) puede ser visto como un cifrado simétrico en donde el mensaje
en claro y el mensaje cifrado mantienen un formato en común. Formalmente, de
acuerdo a lo definido en \cite{DBLP:conf/sacrypt/BellareRRS09}, se trata de una
función $ E: \mathcal{K} \times \mathcal{N} \times \mathcal{T} \times
\mathcal{X} \rightarrow \mathcal{X} $, en donde los conjuntos $ \mathcal{K} $, $
\mathcal{N} $, $ \mathcal{T} $, $ \mathcal{X} $ corresponden al espacio de
llaves, espacio de formatos, espacio de \textit{tweaks} y el dominio,
respectivamente. El proceso de cifrado de un elemento del dominio con respecto a
una llave $ K $, un formato $ N $ y un \textit{tweak} $ T $ se escribe como  $
E_K^{N,T}(X) $. El proceso inverso es también una función $ D: \mathcal{K}
\times \mathcal{N} \times \mathcal{T} \times \mathcal{X} \rightarrow \mathcal{X}
$, en donde $ D_K^{N,T}\big( E_K^{N,T}(X) \big) = X $.

% ¿Hacer aquí aclaración sobre la diferencia en los dígitos verificadores
% en los espacios de orígen y destino?
% Creo que en realidad eso le aplica a todos los esquemas, por lo que mejor lo
% ponemos en el propio desarrollo.

% Para lo que a este trabajo respecta, el formato usado es el de las tarjetas de
% crédito: una cadena de entre 12 y 19 dígitos decimales. Esto es $ N = \{0, 1,
% \dots, 9\}^n $ en donde $ 12 \leq n \leq 19 $.

% TODO: Hmmmm... técnicamente al FPE solo le damos el identificador de la
% persona, por lo que los dígitos son menos.
% ¿Pero no en un cifradro de esos utilizan la otra parte de la tarjeta como
% tweak o algo así?

% En marzo de 2016 el NIST (\textit{National Institute of Standards and
% Technology}) publicó un estándar referente a los cifrados que preservan el
% formato\cite{nist_fpe}. En él se definen dos posibles métodos: FF1 (lo que en
% este trabajo es FFX) y FF3 (lo que en este trabajo es BPS).

% TODO: ¿Hablar también sobre la clasificación que hace Rogaway?

% Existen dos maneras de generar bits aleatorios: la primera es producir bits de
% manera no determinística, donde el estado de cada uno (uno o cero) está
% determinado por un proceso físico impredecible. Este tipo de generadores se
% conocen como \textit{no determinísticos} (NRBG,
% \textit{Non-deterministic Random
% Bit Generator}) o \textit{realmente aleatorios} (TRBG, \textit{Truly Random Bit
% Generator}). La segunda manera, que es la que se explicará a detalle en esta
% sección, es calcular los bits de forma determinística mediante un algoritmo.
% Este tipo de generadores se conocen como \textit{determinísticos} (DRBG,
% \textit{Deterministic Random Bit Generator}), y por lo tanto, los números que
% genera se denominan \textit{pseudoaleatorios}.

Un generador determinístico de números aleatorios (DRBG,
\textit{Deterministic Random Bit Generator}) utiliza un mecanismo interno,
generalmente una función hash o un cifrador por bloques, y un valor inicial
llamado semilla, para producir bits de aspecto aleatorio. Dada la naturaleza
determinísitca del proceso, a los números generados se les conoce como
\textit{pseudoaletorios}. El método que utiliza como mecanismo interno a una
función hash consiste en ir concatenando de forma consecutiva los valores hash
derivados de la semilla e ir incrementando el valor de esta. El método basado en
un cifrador por bloques utiliza el modo de operación de contador,
en donde la semilla juega el papel de vector de inicialización. En ambos casos,
la seguridad se basa en que la semilla sea un valor secreto
\cite{nist_aleatorios}.
