%
% Sección de Introducción.
% Artículo.
% Proyecto Lovelace.
%

\section{Introducción}

% Contexto histórico: cómo llegó a ser popular la tokenización.

Cuando el comercio a través de Internet comenzó a popularizarse, los fraudes de
tarjetas bancarias se volvieron un problema alarmante: según~\cite{wallethub},
en 2001 se tuvieron pérdidas de 1.7 miles de millones de dólares y para 2002
aumentaron a 2.1. Como una medida de protección, las principales compañías de
tarjetas de crédito publicaron un estándar obligatorio para todos aquellos que
procesaran más de 20 000 transacciones anuales: el PCI DSS (en inglés,
\textit{Payment Card Industry Data Security Standard} \cite{pci_dss}); este
estándar cuenta con una gran cantidad de requerimientos \cite{search_security,
uk_association}, por lo que resulta muy difícil satisfacerlo,
especialmente para los negocios pequeños y medianos. A pesar de la publicación
del estándar en 2004, las grandes filtraciones de datos no han cesado:
\textit{TJX} en 2006, \textit{Hannaford Bros.} en 2008, \textit{Target} en
2013 y \textit{Home Depot} en 2014, por mencionar algunos ejemplos.
%% <<... para negocios medianos y pequeños, resulta muy difícil obtener un
%% resultado positivo.>> ¿qué es un resultado positivo?, si es un estándar,
%% lo cumples o no, ¿no?

% Introducción a los sistemas tokenizadores
% ¿Qué es la tokenización y cómo se usa normalmente (arquitectura).

Es en este contexto en el que surge la tokenización como alternativa, pues hasta
ese momento, la idea era proteger la información sensible en donde
quiera que se encontrase: si los números de tarjetas de los usuarios se
encontraban dispersos en diversas partes de un sistema, había que proteger todas
esas partes. La tokenización, en cambio, consiste en concentrar la información
sensible en un solo lugar para hacer la tarea de protección más sencilla; así,
cuando se ingresa un nuevo valor sensible, por ejemplo, el número de tarjeta
de un usuario, se genera un token ligado a esa información; el token se usa en
todo el sistema y la información sensible se protege en un solo lugar. Un
posible adversario con acceso a los tokens no debe poder obtener la información
sensible a partir de estos.

Una de las ventajas de la tokenización es que puede verse como un sistema
autónomo, independiente al sistema principal; de esta manera se establece una
separación de responsabilidades: el sistema principal se ocupa de la operación
del negocio (por ejemplo, una tienda en línea) y el sistema tokenizador se
dedica a la protección de la información sensible. Hoy en día, varias compañías
ofrecen servicios de tokenización que permiten que los comerciantes se libren
casi por completo de cumplir con el PCI DSS. En la Figura
\ref{figura:arquitectura_tokenizacion}, se muestra una distribución bastante
común para un comercio en línea: el sistema tokenizador guarda la información
sensible en su base de datos y se encarga de realizar las transacciones
bancarias.

% Muestra del problema: ejemplos de discursos publicitarios comunes.

La tokenización se ha visto rodeada por una nube de confusión y 
desinformación desde sus inicios; la falta de una definición formal permitió
que las campañas publicitarias de las empresas tokenizadoras esparcieran
mensajes llenos de imprescisiones que, generalmente, dan a entender que la
tokenización y la criptografía son dos cosas completamente aisladas la una de la
otra, o que la primera es una alternativa (y no una aplicación) de la segunda.
Por ejemplo, lo único que \textit{Shift4} dice con claridad sobre sus tokens,
es que se trata de valores aleatorios, únicos y alfanuméricos \cite{shif4_uno};
para \textit{Braintree}, la única manera de generar tokens es por métodos
aleatorios \cite{braintree_uno}; finalmente, para \textit{Securosis} los tokens
son valores aleatorios que nada tienen que ver con la criptografía
\cite{securosis}; es fácil observar que la mayoría de las soluciones trata a
sus métodos como secretos de compañía, además, esperan que el trato entre
cliente y proveedor esté basado en la confianza y no en la comparación de los
propios métodos tokenizadores.

% Establecer relación entre criptografía y tokenización.

Como se verá a continuación por la exposición de algunos métodos tokenizadores,
la tokenización es una aplicación de la criptografía y, como tal, debe ser
analizada con la misma formalidad que las demás herramientas criptográficas; uno
de los errores más comunes entre las empresas tokenizadoras es considerar que
la generación de números aleatorios no utiliza algoritmos criptográficos.
% ¿Por qué <<métodos>> y no <<algoritmos>>?

% Estructura del artículo.

En la Sección \ref{sec:preliminares} de este trabajo se revisan temas y
nociones preliminares necesarias para entender mejor el trabajo; posteriormente,
en la Sección \ref{sec:algoritmos}, se exponen brevemente los métodos de
tokenización analizados. Finalmente, en la Sección \ref{sec:conclusiones}, se
presentan los resultados de las comparaciones de desempeño realizadas y se
concluye con una discusión alrededor del tema.

% TODO: cambiar con qué se concluye.

% \begin{figure}
%   \centering
%   \includegraphics[width=0.8\linewidth]
%     {algoritmos_tokenizadores/diagramas/sistema_tokenizador.png}
%   \caption{Arquitectura típica de un sistema tokenizador.}
%   \label{figura:arquitectura_tokenizacion}
% \end{figure}

\begin{figure}
  \centering
  \begin{tikzpicture}[
    entorno/.style={
      rectangle,
      draw = black,
      thin,
      inner sep = 4mm,
      text width = 20mm,
      align = center,
      minimum height = 1.5cm},
    etiqueta/.style={
      text width = 15mm,
      align = center
    }]

    % Entornos
    \node[entorno]
      (tokens)
      {Sistema tokenizador};
    \node[entorno]
      (tienda)
      [right = 2.5cm of tokens]
      {Entorno de comerciante};
    \node[entorno]
      (banco)
      [below = 1.5cm of tokens]
      {Entorno de banco};
    \node[entorno]
      (usuario)
      [below = 1.5cm of tienda]
      {Cliente};

    % Usuario
    % TODO: hacer un posicionamiento entermanete realativo (ese 13.2cm no
    % debe de estar).
    % \begin{scope}[
    %   scale=0.2,
    %   shift={($(tienda.south)-(0cm, 13.3cm)$)},
    %   local bounding box = usuario]
    %   \draw[black, thick] (-1, 0) -- (0, 2);     % Pierna derecha
    %   \draw[black, thick] (0, 2) -- (1, 0);      % Pierna izquierda
    %   \draw[black, thick] (0, 2) -- (0, 4);      % Torso
    %   \draw[black, thick] (0, 4) -- (-1, 3);     % Brazo derecho
    %   \draw[black, thick] (0, 4) -- (1, 3);      % Brazo izquierdo
    %   \draw[black, thick] (0, 4.6) circle (0.6); % Cabeza
    %   \node[etiqueta] at (0, -1) {Usuario};      % Etiqueta
    %   \node[etiqueta] at (0, 5.2) {};            % Margen superior
    % \end{scope}

    % Comunicación
    \draw[-stealth',
      dashed,
      transform canvas = {
        yshift=6pt}]
      (tienda.west)
      --
      node[
        etiqueta,
        above]
        {Número de tarjeta}
      (tokens.east);
    \draw[-stealth',
      transform canvas = {
        yshift=-6pt}]
      (tokens.east)
      --
      node[
        etiqueta,
        below]
        {Token}
      (tienda.west);
    \draw[-stealth',
      dashed]
      (tokens.south)
      --
      node[
        etiqueta,
        left]
        {Número de tarjeta}
      (banco.north);
    \draw[-stealth',
      dashed]
      (usuario.north)
      --
      node[
        etiqueta,
        right]
        {Número de tarjeta}
      (tienda.south);

    % Acotaciones
    \begin{scope}[shift={($(banco.south)-(1.5cm, 0.5cm)$)}]
      \draw[dashed]
        (0, 0)
        --
        (1, 0);
      \node at (2.1, 0) {Canal seguro};
    \end{scope}

  \end{tikzpicture}
  \caption{Arquitectura típica de un sistema tokenizador.}
  \label{figura:arquitectura_tokenizacion}
\end{figure}
