%
% Sección de DRBG.
% Artículo.
% Proyecto Lovelace.
%

\subsection{Algoritmo basado en generador pseudoaleatorio}

Probablemente este método es el más directo para generar tokens: la idea es
producir una cadena binaria aleatoria con un DRBG e interpretarla para que
tenga el formato de un token. El funcionamiento general es el mismo que en TKR
(figura \ref{tkr_metodos}): la operación de tokenización primero verifica en la
base de datos que el número de tarjeta no se encuentre ya asociado a un token;
de ser este el caso, se regresa el token asociado previamente; en caso contrario
se genera un nuevo token aleatorio, se guarda en la base de datos y se regresa.
La detokenización es simplemente una consulta en la base de datos.  En la figura
\ref{drbg_generacion} se muestra uno de los posibles métodos para generar un
token a partir de una cadena binaria ($ n $ es la longitud que debe tener el
token generado).

\begin{figure}
  \begin{center}
    \begin{tabular}{|l|}
      \hline
      \begin{minipage}{220pt}
        {\scriptsize\begin{tabbing}
          \ \ \ \ \ \=\ \ \ \ \=\ \ \ \ \=\ \ \ \ \=\ \ \ \ \=\ \ \ \ \=\ \ \
          \ \kill \\
          \ \ \ \ {\bf Algoritmo} DRBG-tokenización($ n $) \\
          \> 1. \> \texttt{token} $ \gets $ \texttt{""} \\
          \> 2. \> \texttt{cadena\_aleatoria} $ \gets $
                   \texttt{drbg.generar($ n $)} \\
          \> 3. \> {\bf para} $ i = 0 $ {\bf hasta} $ n - 1 $: \\
          \> 4. \> \> \texttt{token[$ i $]} $ \gets $
                      \texttt{cadena\_aleatoria[$ i $]} $ \mod 10$ \ \ \ \ \\
          \> 5. \> {\bf regresar} token \\
        \end{tabbing}}
      \end{minipage}\\
      \hline
    \end{tabular}
  \end{center}
  \caption{\label{drbg_generacion} Generación de tokens a partir
    de cadena binaria aleatoria.}
\end{figure}

% TODO: ¿decir aquí que, probablemente, este método es el que ocupan todos los
% locos que te dicen que no ocupes criptgrafía porque es insegura?

Según la clasificación del PCI, este método tendría que ser reversible no
criptográfico; con la clasificación de este trabajo, se trata de un
criptográfico irreversible. Al igual que con TKR, la denominación bajo la
clasificación del PCI resulta contradictoria: los generadores pseudoaleatorios
son posibles debido a herramientas criptográficas.
