%
% Sección de TKR.
% Artículo.
% Proyecto Lovelace.
%

\subsection{TKR}

En \cite{doc_sandra} se analiza formalmente el problema de la generación de
tokens y se propone un algoritmo que no está basado en cifrados que preservan el
formato. Hasta antes de la publicación de este documento, los únicos métodos
para generar tokens cuya seguridad estaba formalmente demostrada eran los
basados en cifrados que preservan el formato.

El algoritmo propuesto usa un cifrado por bloques para generar tokens
pseudoaleatorios y almacena en una base de datos la relación original de estos
con los números de tarjetas. A continuación se muestra el proceso de
tokenización.

\begin{algorithm}
  \caption{\label{tkr_tokenizacion} Tokenización de TKR}
  \begin{algorithmic}[1]
    \Function{TKR-tokenización}{$ x, k $}
      \State $ q \gets $ \Call{buscarTarjeta}{$ x $}
      \If{$ q = 0 $}
        \State $ y \gets $ \Call{RN}{$ k $}
        \State \Call{insertar}{$ x, y $}
      \Else
        \State $ y \gets q $
      \EndIf
      \State \Return{$ y $}
    \EndFunction
  \end{algorithmic}
\end{algorithm}

Las funciones \textsc{buscarTarjeta} e \textsc{insertar} sirven para interactuar
con la base de datos. El proceso de detokenización es simplemente una consulta
sobre la base de datos. A continuación se describe el funcionamiento de la
función \textsc{RN}.

\begin{algorithm}
  \caption{\label{tkr_rn} Generación de tokens pseudoaleatorios en TKR}
  \begin{algorithmic}[1]
    \Function{TKR-RN}{$ k $}
      \State $ x \gets $ \Call{f}{$ k, cnt $}
      \State $ x_1, x_2, \dots, x_m \gets $
        \Call{cortar}{$ x $, $ \lambda $}
      \State $ t \gets $ "", $ i \gets 0 $
      \While{$ |t| \neq \mu $}
        \If{\Call{entero}{$ X_i $} $ \leq l $}
          \State $ t \gets t \ || $ \Call{entero}{$ X_i $}
        \EndIf
        \State $ i \gets i + 1 $
      \EndWhile
      \State $ cnt \gets cnt + 1 $
    \EndFunction
  \end{algorithmic}
\end{algorithm}

Existen varios candidatos viables para la función $ f $: un cifrado de flujo,
pues el flujo de llave de estos produce cadenas de aspecto aleatorio, o un
cifrado por bloques con un modo de operación de contador. En la implementación
de este trabajo se ocupa esta última opción.

Con la clasificación del PCI, este método cae, contradictoriamente, en los
reversibles no criptográficos. Con la clasificación propuesta en este trabajo se
encuentra dentro de los criptográficos irreversibles.
