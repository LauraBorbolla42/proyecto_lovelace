%
% Sección de TKR.
% Artículo.
% Proyecto Lovelace.
%

\subsection{TKR}

En \cite{doc_sandra} se analiza formalmente el problema de la generación de
tokens y se propone un algoritmo que no está basado en cifrados que preservan el
formato. Hasta antes de la publicación de este documento, los únicos métodos
para generar tokens cuya seguridad estaba formalmente demostrada eran los
basados en cifrados que preservan el formato.

El algoritmo propuesto usa un cifrado por bloques para generar tokens
pseudoaleatorios y almacena en una base de datos la relación original de estos
con los números de tarjetas. En la figura \ref{tkr_metodos} se muestra el
proceso de tokenización y detokenización.

\begin{figure}
  \begin{center}
    \begin{tabular}{|l|}
      \hline
      \begin{minipage}{.5\textwidth}
        \begin{tabbing}
          \ \ \ \ \ \=\ \ \ \ \=\ \ \ \ \=\ \ \ \ \=\ \ \ \ \=\ \ \ \ \=\ \ \
          \ \kill \\
          \ \ \ \ {\bf Algoritmo} TKR-tokenización($ x $, $ k $) \ \ \ \ \\
          \> 1. \> $ q \gets $ \texttt{buscarTarjeta($ x $)} \\
          \> 2. \> {\bf si} $ q = 0 $ {\bf entonces}: \\
          \> 3. \> \> $ y \gets $ \texttt{RN($ k $)} \\
          \> 4. \> \> \texttt{insertar($ x $, $ y $)} \\
          \> 5. \> {\bf sino:} \\
          \> 6. \> \> $ y \gets q $ \\
          \> 7. \> {\bf regresar} $ y $ \\
        \end{tabbing}
      \end{minipage}\\
      \hline
      \begin{minipage}{.5\textwidth}
        \begin{tabbing}
          \ \ \ \ \ \=\ \ \ \ \=\ \ \ \ \=\ \ \ \ \=\ \ \ \ \=\ \ \ \ \=\ \ \
          \ \kill \\
          \ \ \ \ {\bf Algoritmo} TKR-detokenización($ y $, $ k $) \ \ \ \ \\
          \> 1. \> $ q \gets $ \texttt{buscarToken($ t $)} \\
          \> 2. \> {\bf si} $ q = 0 $ {\bf entonces}: \\
          \> 3. \> \> {\bf regresar error} \\
          \> 4. \> {\bf sino:} \\
          \> 5. \> \> {\bf regresar} $ q $ \\
        \end{tabbing}
      \end{minipage}\\
      \hline
    \end{tabular}
  \end{center}
  \caption{\label{tkr_metodos} Tokenización y detokenización de TKR}
\end{figure}

Las funciones \texttt{buscarTarjeta}, \texttt{buscarToken} e \texttt{insertar}
sirven para interactuar con la base de datos. Lo único que queda por esclarecer
es el el contenido de la función generadora de tokens pseudoaleatorios, la
función \texttt{RN()}. El algoritmo de esta función se muestra en la figura
\ref{tkr_rn}. Idealmente, esta función debe regresar un elemento uniformemente
aleatorio del espacio de tokens. La variable \textit{contador} mantiene un
estado del algoritmo (mantiene su valor a lo largo de las distintas llamadas);
el espacio de tokens contiene cadenas de longitud fija $ \mu $ de un alfabeto
$ AL $ cuya cardinalidad es $ l $; el número de bits necesarios para enumerar a
todo el alfabeto se guardan en $ \lambda = \lceil \log_2 l \rceil$.

\begin{figure}
  \begin{center}
    \begin{tabular}{|l|}
      \hline
      \begin{minipage}{220pt}
        \begin{tabbing}
          \ \ \ \ \ \=\ \ \ \ \=\ \ \ \ \=\ \ \ \ \=\ \ \ \ \=\ \ \ \ \=\ \ \
          \ \kill \\
          \ \ \ \ {\bf Algoritmo} TKR-RN($ k $) \\
          \> 1. \> $ x \gets $ \texttt{f($ k $, contador)} \\
          \> 2. \> $ x_1, x_2, \dots, x_m \gets $
                    \texttt{cortar($ x $, $ \lambda $)} \ \ \ \ \\
          \> 3. \> $ t \gets $ "", $ i \gets 0 $ \\
          \> 4. \> {\bf mientras} $ |t| \neq \mu $: \\
          \> 5. \> \> {\bf si} \texttt{entero($ x_i $)} {\bf entonces}: \\
          \> 6. \> \> \> $ t \gets t + $ \texttt{entero($ X_i $)} \\
          \> 7. \> \> $ i \gets i + 1 $ \\
          \> 8. \> \texttt{contador $\gets$ contador $ + 1 $} \\
          \> 9. \> {\bf regresar} $ t $ \\
        \end{tabbing}
      \end{minipage}\\
      \hline
    \end{tabular}
  \end{center}
  \caption{\label{tkr_rn} Generación de tokens pseudoaleatorios en TKR}
\end{figure}

Existen varios candidatos viables para la función $ f $: un cifrado de flujo,
pues el flujo de llave de estos produce cadenas de aspecto aleatorio, o un
cifrado por bloques con un modo de operación de contador. En la implementación
de este trabajo se ocupa esta última opción.

Con la clasificación del PCI, este método cae, contradictoriamente, en los
reversibles no criptográficos. Con la clasificación propuesta en este trabajo se
encuentra dentro de los criptográficos irreversibles.
