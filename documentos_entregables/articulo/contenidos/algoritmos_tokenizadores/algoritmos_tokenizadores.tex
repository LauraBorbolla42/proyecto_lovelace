%
% Sección de algoritmos tokenizadores.
% Artículo.
% Proyecto Lovelace.
%

\section{Algoritmos tokenizadores}
\label{sec:algoritmos}

Como el enfoque de este artículo es ver a la tokenización como un servicio
(Figura \ref{figura:arquitectura_tokenizacion}), la interfaz para los procesos
de tokenización y detokenización, desde el punto de vista de los usuarios del
servicio, es sumamente simple: el proceso de tokenización es una función $ E:
\mathcal{X} \rightarrow \mathcal{Y} $ y el de detokenización es  simplemente la
función inversa $ D: \mathcal{Y} \rightarrow \mathcal{X} $, en donde $
\mathcal{X} $ y $ \mathcal{Y} $ son los espacios de números de tarjetas y
tokens, respectivamente.

Los números de tarjetas bancarias cuentan con entre 12 y 19 dígitos, y se
encuentran normados por el estándar ISO/IEC-7812~\cite{iso_7812}. El último
dígito se trata de un código de verificación calculado mediante el algoritmo de
Luhn; este dígito hace que \textsc{algoritmoDeLuhn($ x $) = 0}. Para poder
diferenciar a los números de tarjeta válidos de los tokens se establece que,
para los tokens, el dígito verificador haga que
\textsc{algoritmoDeLuhn($ x $) = 1}~\cite{doc_sandra}.

% Hay una corrección diciendo que falta una cita en la última frase. La que
% pongo ahorita es del artículo de Sandra: dentro de uno de los dos requisitos
% se establece que los tokens tienen un dígito de verificación que hace que el
% algoritmo de Luhn dé 1. La corrección también se podría referir a que citemos
% en donde dice que las tarjetas deben cumplir con el algoritmo de Luhn, en cuyo
% caso es la cita de arriba (la del iso); o referir a en donde se define el
% algoritmo de Luhn, que también está en el ISO. La patente original de Luhn
% está enfocada a un aespecie de calculadora o hardware (eran los años 50)
% para calcular el dígito.

\subsection{Clasificación del PCI SSC}

El PCI SSC (\textit{Payment Card Industry Security Standard Council}) establece
en sus guías de tokenización la siguiente clasificación para los algoritmos
tokenizadores~\cite{pci_tokens}:

\begin{itemize}
  \item Métodos reversibles. Aquellos para los cuales es posible obtener el
    número de tarjeta a partir del token.
    \begin{itemize}
      \item Criptográficos. Ocupan un esquema de cifrado simétrico: el número
        de tarjeta y una llave entran al mecanismo de tokenización para obtener
        un token; el token y la misma llave entran al mecanismo de
        detokenización para obtener el número de tarjeta original.
      \item No criptográficos. Ocupan una base de datos para guardar las
        relaciones entre números de tarjetas y tokens; el proceso de
        detokenización simplemente es una consulta a la base de datos.
    \end{itemize}
  \item Métodos irreversibles. Aquellos en los que no es posible regresar al
    número de tarjeta original a partir del token.
    \begin{itemize}
      \item Autenticable. Permiten validar cuando un token dado corresponde a
        un número de tarjeta dado.
      \item No autenticable. No permiten hacer la validación anterior.
    \end{itemize}
\end{itemize}

\subimport{/}{tkr}
\subimport{/}{ffx}
\subimport{/}{bps}
\subimport{/}{ahr}
%\subimport{/}{drbg}
