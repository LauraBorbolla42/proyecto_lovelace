%
% Sección de BPS.
% Artículo.
% Proyecto Lovelace.
%

\subsection{BPS (Brier, Peyrin y Stern)}

Cifrado que preserva el formato diseñado por Eric Brier, Thomas Peyrin y
Jacques Stern~\cite{bps}. En \cite{nist_fpe} el NIST estandariza a FFX y a
BPS, denominándolos FF1 y FF3, respectivamente. BPS se conforma de 2 partes: un
cifrado interno $BC$ que se encarga de cifrar bloques de longitud fija; y un modo
de operación especial, encargado de extender la funcionalidad de $BC$ y permitir
cifrar cadenas de mayor longitud.

El cifrado interno utiliza una red Feistel alternante y se define como
$BC_{F,s,b,w}(X,K,T)$, donde $F$ es un cifrado por bloques con $f$ bits de
salida; $s$ es la cardinalidad del alfabeto de la cadena a cifrar, $b$ es su
longitud, $w$ es el número de rondas de la red Feistel, $X$ es la cadena, $K$ es
una llave acorde al cifrado $F$, y $T$ es un \textit{tweak} de 64 bits. A
continuación se describe su funcionamiento.

\begin{algorithm}
  \caption{\label{proceso_bc} Cifrado interno BC.}
  \begin{algorithmic}[1]
    \Function{Cifrado $ BC_{F,s,b,w} $}{$ X, K, T$}
      \State $ T_R \gets T \mod 2^{32} $,
        $ T_L \gets (T - T_R) / 2^{32} $
      \State $ l \gets \lceil b/2 \rceil $,
        $ r \gets \lfloor b/2 \rfloor $
      \State $ L_0 \gets \sum_{j=0}^{l-1} X[j] \cdot s^j $,
        $ R_0 \gets \sum_{j=0}^{r-1} X[j+l] \cdot s^j $
      \For{$ i = 0 \dots w - 1 $}
        \If{$ i \% 2 = 0 $}
          \State $ L_{i+1} \gets L_i \boxplus F_K((T_R \oplus i)
            \cdot 2^{f-32} + R_i)\quad (\mod s^l) $
          \State $ R_{i+1} \gets R_i $
        \Else
          \State $ R_{i+1} \gets R_i \boxplus F_K((T_L \oplus i)
            \cdot 2^{f-32} + L_i)\quad (\mod s^r) $
          \State $ L_{i+1} \gets L_i $
        \EndIf
      \EndFor
%      \For{$ i = 0 \dots l - 1 $}
%        \State $ Y_L[i] \gets L_w \mod s $
%        \State $ L_w \gets (L_w - Y_L[i])/s $
%      \EndFor
%      \For{$ i = l \dots r - 1$}
%        \State $ Y_R[i] \gets R_w \mod s $
%        \State $ R_w \gets (R_w - Y_R[i])/s $
%      \EndFor
      \State \Return{$ L_w \parallel R_w $}
    \EndFunction
  \end{algorithmic}
\end{algorithm}

% Lo que está comentado es simplemente la conversión de número a
% cadena. No es demasiado relevante a nivel de pseudocódigo. Sería más
% rápido todo (esto ya es crítica sobre el propio algoritmo) si siempre
% se pensara como cadena, y no tener que estar cambiando de cadena a número
% constantemente.

% TODO: acortar este algoritmo.

% Para cada bloque a cifrar, el cifrado $BC$ debe instanciarse con una longitud
% de $max_b = 2 \cdot log_s(2^{f-32})$ caracteres, y cuando la longitud total
% del mensaje a cifrar no sea múltiplo de este valor, en el último bloque $BC$
% se tendrá que instanciar con una longitud igual a la de ese bloque.

% El modo de operación de BPS es un variación de CBC, con la diferencia de que
% usa sumas modulares carácter por carácter en lugar de aplicar operaciones
% \textit{xor}, además de que no emplea un vector de inicialización.
%
% Otra característica de este modo de operación es que utiliza un contador $u$
% para aplicar un \textit{xor} a los 16 bits más significativos de cada mitad
% del \textit{tweak} $T$ que utiliza BPS, por lo cual este se puede ver como
% una función $u(n) = n \cdot (2^{16} + 2^{48})$.
%
% El funcionamiento del modo de operación se describe en la figura \ref{modo_bps}.
%
% \begin{figure}
%   \begin{center}
%     \includegraphics[width=0.85\linewidth]
%     {../../../diagramas_comunes/bps/modo_de_operacion_bps}
%     \caption{Modo de operación de BPS.}
%     \label{modo_bps}
%    \end{center}
% \end{figure}

El PCI clasifica a este algoritmo como reversibles criptográficos y según la
clasificación de este trabajo se trata de un criptográfico reversible.
