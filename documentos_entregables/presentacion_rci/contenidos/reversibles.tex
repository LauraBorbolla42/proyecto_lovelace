%
% Algoritmos reveribles,
% presentación en RCI.
% Proyecto Lovelace.
%

\section{Métodos reversibles: FFX y BPS}

\begin{frame}{Métodos reversibles: FFX y BPS}

  \begin{itemize}
    \item Métodos que utilizan cifrados que preservan el formato.
    \item Cifran la tarjeta y descifran el token.
    \item Se volvieron estándares en 2016 y fueron renombrados por el NIST
      a FF1 y FF3 respectivamente~\cite{nist_fpe}.
    \item Están basados en redes Feistel.
  \end{itemize}

  \begin{figure}
    \centering
    \subfloat[Proceso de tokenización]{
      \subimport{diagramas/}{tokenizacion_reversible.tikz.tex}}
    \\\vspace{0.1in}
    \subfloat[Proceso de detokenización]{
      \subimport{diagramas/}{detokenizacion_reversible.tikz.tex}}
  \end{figure}
\end{frame}

\begin{frame}{Comparativa: FFX y BPS}

  \begin{table}
    \begin{tabular}{|m{0.35\textwidth}|m{0.26\textwidth}|m{0.25\textwidth}|}
      \hline
      \parbox{0.35\textwidth}{\textbf{Características}} &
      \parbox{0.26\textwidth}{\textbf{FFX}} &
      \parbox{0.25\textwidth}{\textbf{BPS}}
      \\\hline
      \parbox{0.35\textwidth}{%
        \strut Longitud de cadena \\
        {\scriptsize(en caracteres)} \strut} &
      \parbox{0.26\textwidth}{$4 \ - \ 36$} &
      \parbox{0.25\textwidth}{$0 \ - \ 1.9x10^{40}$}
      \\\hline
      \parbox{0.35\textwidth}{\strut Primitivas \\ criptográficas\strut} &
      \parbox{0.26\textwidth}{AES CBC-MAC} &
      \parbox{0.25\textwidth}{AES}
      \\\hline
      \parbox{0.35\textwidth}{Tamaño de llave} &
      \parbox{0.26\textwidth}{128 bits} &
      \parbox{0.25\textwidth}{128 bits}
      \\\hline
      \parbox{0.35\textwidth}{Tamaño de \textit{tweak}} &
      \parbox{0.26\textwidth}{menor a $2^{64}$ bits} &
      \parbox{0.25\textwidth}{64 bits}
      \\\hline
      \parbox{0.35\textwidth}{Número de rondas} &
      \parbox{0.26\textwidth}{12, 24 o 28} &
      \parbox{0.25\textwidth}{\strut mínimo 8 recomendadas \strut}
      \\\hline
    \end{tabular}
    \caption{Características de los algoritmos tokenizadores
      reveribles~\cite{ffx_1, bps}.}
  \end{table}

\end{frame}
