%
% Archivo principal de documento final.
% Proyecto Lovelace.
%

\documentclass[11pt, letterpaper, twosides]{book}
\usepackage{formato}
\newcommand{\Titulo}{Generación de \textit{tokens} para proteger
                     los datos de tarjetas bancarias}
\newcommand{\Fecha}{Enero de 2018}
\usepackage{reporte_tecnico/formato}

\begin{document}

  \frontmatter
  \Portada

  \pdfbookmark[1]{\contentsname}{Contenido}
  \setcounter{page}{3}
  \setcounter{tocdepth}{5}
  \tableofcontents
  \newpage
  \import{reporte_tecnico/contenidos/}{simbologia.tex}

  \mainmatter
  \import{reporte_tecnico/contenidos/intro/}{intro.tex}
  \import{reporte_tecnico/contenidos/antecedentes/}{antecedentes.tex}
  \import{reporte_tecnico/contenidos/analisis_y_disenio/}
    {analisis_y_disenio.tex}

  \clearpage
  \phantomsection
  \addcontentsline{toc}{chapter}{Bibliografía}
  \printbibliography[
    heading=subbibliography,
    title={Bibliografía}
  ]

  \setglossarystyle{altlist}
  \printglossary[
    type=main,
    entrycounter=true
  ]

  \section*{Siglas y acrónimos}
  \phantomsection
  \addcontentsline{toc}{chapter}{Siglas y acrónimos}
  \setglossarystyle{list}
  \renewcommand*{\glsgroupskip}{}
  \setglossarysection{subsection}

  \printglossary[
    type=siglas_cripto,
    title={Criptográficos}
  ]

  \printglossary[
    type=siglas_computacionales,
    title={Computacionales}
  ]

  \printglossary[
    type=siglas_bancarias,
    title={Bancarios}
  ]

  \printglossary[
    type=siglas_instituciones,
    title={De instituciones y asociaciones}
  ]

  \printglossary[
    type=\acronymtype,
    title={Misceláneos}
  ]

  \newpage
  \phantomsection
  \addcontentsline{toc}{chapter}{\listfigurename}
  \listoffigures

  \newpage
  \phantomsection
  \addcontentsline{toc}{chapter}{\listtablename}
  \listoftables

  \newpage
  \phantomsection
  \addcontentsline{toc}{chapter}{\lstlistlistingname}
  \lstlistoflistings

\end{document}
