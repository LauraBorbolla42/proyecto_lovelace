%
% Capítulo de conclusiones,
% Reporte técnico.
%
% Proyecto Lovelace.

\capitulo{Conclusiones}{sec:conclusiones}
{
  \epigrafe
  {%
    Reserve your right to think, for even to think wrongly is better than not to
    think at all.%
  }
  {%
    \textsc{Hypatia}.%
  }
}

\noindent
Como se planteó al inicio, este trabajo fue desarrollado en tres grandes
módulos:
\begin{description}
  \item [Programa generador de tokens] Aquí se analizaron e implementaron
    cinco algitomtos tokenizadores: \acrshort{gl:ffx}, \acrshort{gl:bps}, TKR,
    AHR y \acrshort{gl:drbg}.
  \item [Servicio de tokenización] Se encarga de proveer una interfaz de
    comunicación entre el programa generador de tokens y los posibles clientes;
    además, incluye una aplicación web que permite gestionar a los usuarios.
  \item [Caso de prueba: tienda de libros en línea] Se implementaron,
    principalmente, las partes que involucran comunicación con el servicio de
    tokenización (agregar método de pago y simulación de una compra).
\end{description}

Sobre el desempeño de los algoritmos, se concluye que los reversibles
(\acrshort{gl:ffx} y \acrshort{gl:bps}) son considerablemente más rápidos que
los tres irreversibles (TKR, AHR y \acrshort{gl:drbg}). También es posible
observar que, para los métodos irreversibles, el proceso de detokenización es
mucho más rápido que la generación de tokens.  Estos dos resultados dejan ver
un poco la carga de las operaciones en los métodos irreversibles: la
tokenización involucra una consulta a la base de datos, la generación del
\gls{gl:token} y una inserción; la detokenización solamente es una consulta.

El que \acrshort{gl:ffx} y \acrshort{gl:bps} sean más rápidos puede resultar
un poco contraintuitivo, pues la generación de tokens reversibles involucra
más operaciones; cuando en la medición de tiempos se excluye el acceso a  la
base de datos, el algoritmo más veloz es \acrshort{gl:drbg}, seguido de cerca
por TKR y AHR; los dos reversibles terminan siendo los más lentos.

Además de los tiempos de ejecución, también es importante señalar que los
irreversibles, al operar como funciones de un solo sentido, son más seguros
que los reversibles: un atacante con acceso a la llave de cifrado puede
obtener el número de tarjeta correspondiente si se trata de un método
reversible, mientras que con un método irreversible necesita también acceso a
la base de datos.

También es posible concluir que la denominación \textit{no criptográficos}, de
la clasificación del PCI DSS resulta totalmente confusa, pues en realidad todos
los métodos conocidos que caen en esa categoría utilizan primitivas
criptográficas. La segunda categoría (irreversibles) carece de utilidad para
aplicaciones que procesan pagos con tarjetas de crédito, pues la capacidad de
regresar al número de tarjeta a partir de su token es uno de los requerimientos
principales para los sistemas tokenizadores.

Respecto a los objetivos planteados en la sección~\ref{sec:objetivos}, podemos
concluir que se satisficieron los tres objetivos específicos: revisar diversos
algoritmos para la generación de \glspl{gl:token} (corresponde al módulo de
\textit{Programa generador de tokens}); diseñar e implementar un servicio web
que proporcione el servicio de generación de tokens (corresponde al módulo de
\textit{Servicio de tokenización}) y, finalmente, implementar una tienda que
use el servicio de tokenización (corresponde al módulo de \textit{Caso de
prueba}). También se cumplió el objetivo general, pues se implementaron
algoritmos criptográficos y no criptográficos\footnote{Tomando en cuenta la
clasificación propuesta por el \gls{gl:pci} \gls{gl:ssc}, descrita en la
sección~\ref{sec:pci_dss}.}.
