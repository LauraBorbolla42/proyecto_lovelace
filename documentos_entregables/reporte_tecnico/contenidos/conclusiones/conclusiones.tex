%
% Capítulo de conclusiones,
% Reporte técnico.
%
% Proyecto Lovelace.

\capitulo{Conclusiones}{sec:conclusiones}
{
  \epigrafe
  {%
    Reserve your right to think, for even to think wrongly is better than not to
    think at all.%
  }
  {%
    \textsc{Hypatia}.%
  }
}

\noindent
El principal resultado de este Trabajo Terminal I es el desarrollo del primer
módulo del proyecto (descrito en~\ref{sec:objetivos}): un programa generador de
\glspl{gl:token} para proveer confidencialidad a los datos de las tarjetas
bancarias. Para lograrlo, primero se estudiaron, como se expone en el capítulo
\ref{sec:marco_teorico}, aspectos de la criptografía relacionados con la
generación de \glspl{gl:token}; posteriormente, como se observa en el capítulo
\ref{sec:generacion_de_tokens}, se revisaron los estándares y las
recomendaciones asociadas al tema; en el mismo capítulo se describieron los
algoritmos generadores de \glspl{gl:token} implementados.

Además de la implementación de los algoritmos, que cubre el primer objetivo
específico, se realizaron pruebas de comparación de desempeño en aras de
disminuir la desinformación que rodea al tema de la tokenización.

Otra parte importante del trabajo fue la implementación de un generador de
números aleatorios capaz de satisfacer las pruebas estadísticas de aleatoriedad
establecidas por el \gls{gl:nist}\footnotemark.

\footnotetext{
  Intituto norteamericano encargado de la publicación de estándares relacionados
  con tecnología y computación.
}

La siguiente parte del proyecto, que será desarrollada en Trabajo Terminal II,
consiste en la elaboración de los dos módulos siguientes: una interfaz en red
que permita comunicarse con el programa tokenizador y una tienda en línea que la
utilice. Esto permitirá que lo desarrollado en el primer módulo llegue hasta los
usuarios finales, esto es, hasta los clientes de las tiendas en línea.
De esta forma se cubrirán los otros dos objetivos específicos del trabajo.
