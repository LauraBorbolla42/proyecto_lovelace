%
% Código de módulo de tienda en línea,
% Reporte técnico.
% Proyecto Lovelace.
%

\section{Aspectos relevantes de la implementación}
\label{sec:codigo_sistema_tokenizador}

En esta sección se muestran, a modo de ejemplo, algunos de los aspectos más
relevantes de la implementación del programa descrito en el capítulo anterior.

\subsection{Ejemplos del programa servidor}

El programa está dividido en dos grandes módulos: lo referente al programa
tokenizador, y lo referente a las operaciones web (diagrama de
clases~\ref{fig:clases_aplicacion_web}). Cada módulo tiene una sección de
direcciones, una de operaciones de negocio, una de operaciones web y una de
clases de modelos; exite también una sección de utilidades generales. A
continuación se muestran ejemplos de cada una de estas secciones.

El código fuente~\ref{codigo:paginador_clientes_aprobados} muestra una
operación web del módulo general: la petición de paginador para los clientes
\textbf{aprobados}. Existen funciones similares para los clientes en estado de
\textbf{lista negra} y los clientes \textbf{en espera}. Estas funciones se
utilizan para armar e interactuar con los paginadores de la interfaz de un
administrador (pantalla \hipervinculo{iu:administracion}).

\codigoFuente[codigo:paginador_clientes_aprobados]{474}{491}{python}{%
  sistema_tokenizador/general/funciones_web.py}{%
  Paginador de clientes aprobados.}

En el código fuente~\ref{codigo:funcion_de_autentificacion} se muestra una
operación de negocio del módulo general: la autentificación de los usuarios.
Esta función se utiliza al momento de un inicio de sesión. Este código
representa el cumplemiento del requerimiento
\hipervinculo{rqnf_api:alamcenamiento_de_contrasenias}, ya que para realizar el
proceso de autentificación se calcula el hash de la entrada del usuario y se
compara contra lo que hay guardado en la base de datos.

\codigoFuente[codigo:funcion_de_autentificacion]{19}{31}{python}{%
  sistema_tokenizador/general/negocio.py}{%
  Operación de autentificación de usuarios.}

Como ejemplo de operación web del módulo tokenizador, los códigos
fuente~\ref{codigo:tokenizar_uno} y~\ref{codigo:tokenizar_dos} muestran la
operación de tokenización. Esta función representa casi la totalidad del caso
de uso \hipervinculo{cu:tokenizar_tarjeta}. En esta función se puede apreciar
mejor que en los códigos anteriores la interacción entre las distintas capas:
las validaciones se hacen en la capa de negocio (línea~128); el módulo de
configuraciones almacena constantes importantes (línea~144), en este caso, se
trata de la ruta del ejecutable al programa generador de tokens. Todos los
códigos de error y mensajes que regresa está función son los descritos en la
página de documentación
(\url{https://ricardo-quezada.159.56.43.6/lovelace/documentacion}).

\codigoFuente[codigo:tokenizar_uno]{107}{151}{python}{%
  sistema_tokenizador/programa_tokenizador/funciones_web.py}{%
  Operación de tokenización (primera parte).}

\codigoFuente[codigo:tokenizar_dos]{153}{184}{python}{%
  sistema_tokenizador/programa_tokenizador/funciones_web.py}{%
  Operación de tokenización (segunda parte).}

En el código fuente~\ref{codigo:contador_malas_acciones} se muestra un ejemplo
de función de negocio de módulo generador de tokens. Se trata de la gestión del
contador de malas acciones (regla de negocio \hipervinculo{rn:malas_acciones}).
Al inicio del código se muestra también cómo se gestionan las constantes de
incrementos y límites. De esta forma se establecen en el código todas las
constantes definidas en las reglas de negocio (esto solamente debe de ir en los
archivos de capa de negocio).

\codigoFuente[codigo:contador_malas_acciones]{23}{42}{python}{%
  sistema_tokenizador/programa_tokenizador/negocio.py}{%
  Gestión de contador de malas acciones.}

La realización de las capas de modelos de ambos módulos es a través de la
\gls{gl:api} de \textit{Django} (descrita de modo general
en~\cite{django_modelos}). En el código fuente~\ref{codigo:modelo_de_token} se
muestra la clase de modelo de un token. Django toma esta descripción para
producir las funciones básicas de un \gls{gl:crud}. Ya se ha visto en los
códigos anteriores cómo es la interacción con esta \gls{gl:api} (métodos como
\texttt{save}, \texttt{filter}, \texttt{get} o \texttt{delete}).

\codigoFuente[codigo:modelo_de_token]{1}{45}{python}{%
  sistema_tokenizador/programa_tokenizador/models/token.py}{%
  Modelo de token.}

Por último, como ejemplo de la capa general de utilidades, en el
código~\ref{codigo:privilegios} se muestra una de las funciones más usadas en
las implementaciones de las funciones web: la validación de los privilegios del
usuario en sesión. Por ejemplo, esta función se utiliza en la línea~33 del
código fuente~\ref{codigo:paginador_clientes_aprobados} para validar que
solamente un usuario de tipo \textbf{administrador} pueda acceder al paginador
de los clientes aprobados. La función \texttt{privilegiosRequeridos} funciona
como una fábrica de decoradores: toma como argumento el nombre del tipo de
usuario y regresa una función decorador para ese tipo de usuario. Los
decoradores en \textit{python} son funciones que toman como argumentos a la
función decorada: intercalan el nuevo código (en este caso la validación de las
credenciales) con la llamada a la función de entrada~\cite{python_decoradores}.

Con la función de validación de privilegios se ocupan dos patrones de diseño:
una fábrica y un decorador. Aunque su realización es distinta de lo que se
esperaría en un lenguaje completamente orientado a objetos y, por lo tanto, de
la descrita en~\cite{patrones_de_disenio}, la funcionalidad obtenida en un
último término corresponde a la obtenida por los patrones de diseño originales.
La fábrica permite la creación de objetos (en este caso, métodos) con una
estructura determinada en tiempo de ejecución; aquí, la estructura se encuentra
representada por el tipo de usuario al que se le desean dar privilegios. Los
decoradores permiten extender o alterar la funcionalidad de un método:
el decorador agrega la validación de privilegios justo antes de ejecutar
la propia función decorada.

\codigoFuente[codigo:privilegios]{33}{79}{python}{%
  sistema_tokenizador/utilidades.py}{%
  Fábrica de decoradores para validación de privilegios.}

\subsection{Ejemplos del programa cliente}

En el código fuente~\ref{codigo:ventana_clientes} se muestra el inicio del
\gls{gl:html} de la ventana para operar a un cliente (dependiendo del contexto,
se puede tratar de la interfaz \hipervinculo{iu:registrar_usuario} o de
\hipervinculo{iu:actualizar_cliente}). Como se puede observar al inicio del
archivo, hay reglas que permiten modificar el tamaño de las ventanas dependiendo
del tamaño de la pantalla del usuario.

\codigoFuente[codigo:ventana_clientes]{1}{45}{html}{%
  sistema_tokenizador/archivos_web/html/ventanas/operar_cliente.ventana.html}{%
  Ventana emergente para operaciones de un cliente.}

Cada \gls{gl:html} tiene un controlador de \textit{javascript} asociado. En el
caso de la ventana de operación de un cliente
(código~\ref{codigo:ventana_clientes}), un fragmento de su controlador se
muestra en el código fuente~\ref{codigo:controlador_clientes}. \textit{Angular}
sigue un patrón de inyección de dependencias~\cite{inyeccion_de_dependencias}
para la construcción de controladores: después del nombre dado al controlador,
recibe un arreglo con cadenas que representan los nombres de las dependencias
que el inyector tiene que buscar para construir la función (el último elemento
del arreglo recibido)~\cite{angular_inyector}.

% TODO: La ruta del controlador es demasiado larga: el comando de codigoFuente
% debería de partirla, o de tomar solo lo último.
% Los marcadores de grunt están resaltados con rojo.

\codigoFuente[codigo:controlador_clientes]{1}{40}{javascript}{%
  sistema_tokenizador/archivos_web/js/controladores/secundarios/%
  operar_cliente.controlador.js}{%
  Controlador de ventana emergente para operaciones de un cliente.}

Los dos códigos anteriores, la ventana y su controlador, dejan ver cómo es la
gestión de los mensajes y las expresiones regulares en el código del cliente.
En los archivos \gls{gl:html} hay marcadores con formato \verb|<!-- @@ -->|
(línea~42 del código~\ref{codigo:ventana_clientes}); en los archivos
\textit{javascript} hay marcadores con formato \verb|@@( )| (línea~26 del
código~\ref{codigo:controlador_clientes}). Estos marcadores son sustituidos en
tiempo de compilación por un componente de \textit{Grunt}
(\textit{includereplace}): el marcador indica un archivo de texto plano en
donde se encuentra definido el mensaje o la regla de negocios en cuestión. Estos
archivos se producen en una etapa de análisis y son los mismos que utiliza este
documento en la sección de reglas de negocio~(\ref{sec:reglas_de_negocio}) o la
sección del catálogo de mensajes~(\ref{sec:catalogo_de_mensajes}).

El punto anterior busca satisfacer una buena práctica de programación: las
definiciones redundantes son malas; la información debe de estar en un solo
lugar y de ahí debe de ser usada por los códigos dependientes. Por ejemplo,
en un supuesto de que se quiera cambiar el contenido de un mensaje, o modificar
una expresión regular, solamente hay que modificar el archivo de texto
correspondiente y ambas dependencias, programa y documentación, se adecuarán
al cambio. Este principio es conocido como \gls{gl:dry} y es uno de los
principios de diseño de las aplicaciones hechas en
\textit{Django}~\cite{django_principios_de_disenio}.

La comunicación con el programa servidor se define en una capa separada de los
propios controladores. En el código fuente~\ref{codigo:api_javascript} se
muestra un fragmento de esta capa. Este es un modo de separar las direcciones y
fuentes de datos que se obtienen del servidor, del propio funcionamiento de
cada controlador.

% TODO: hablar aquí sobre los métodos de http y de la tesis de Roy Fielding.

\codigoFuente[codigo:controlador_clientes]{1}{46}{javascript}{%
  sistema_tokenizador/archivos_web/js/servicios/api.servicio.js}{%
  Definición de comunicación con programa servidor.}
