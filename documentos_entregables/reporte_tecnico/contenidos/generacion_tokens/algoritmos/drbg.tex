%
% Generador pseudoaleatorio como método tokenizador,
% creación mediante funciones hash y cifrado por bloques.
% Capítulo de análisis y diseño, reporte técnico.
%
% Proyecto Lovelace.
%

\subsection{Tokenización mediante generador de números pseudoaleatorios}
\label{sec:drbg_lista}

En la sección \ref{sec:generadores_pseudoaleatorios} se habló sobre el
estándar del \gls{gl:nist} para la generación de números aleatorios. En esta
mostramos cómo estos métodos pueden ser utilizadas para tokenizar y detokenizar
números de tarjetas. Primero se muestran dos posibles instanciaciones
con \glspl{gl:primitiva_criptografica} del método general descrito en
\ref{sec:generadores_pseudoaleatorios}: una basada en funciones hash
(sección \ref{sec:hash}) y otra basada en cifrados por bloque (sección
\ref{sec:bloques}); ambas presentadas en \cite{nist_aleatorios}.

\subsubsection{DRBG basado en funciones hash}

El método genérico define un solo valor crítico (un valor que se debe mantener
en secreto, pues la seguridad en generador depende de esto): la semilla, tanto
al inicio, como su valor a lo largo de su vida útil. Este método introduce
un segundo valor crítico: una constante $ C $ cuyo valor depende también de
la entrada de entropía. Por tanto, las funciones de instanciación,
cambio de semilla y desinstanciación se deben modificar para incluir
a este segundo valor.

En el pseudocódigo \ref{drbg_hash} se muestra la función de generación de bits
pseudoaleatorios basada en una función hash. \textit{tamanioDeHash} regresa el
tamaño de los \textit{digest} retornados por la función hash usada.

\begin{pseudocodigo}[%
    caption={Generación de bits pseudoaleatorios mediante función hash},
    label={drbg_hash}%
  ]
    entrada: número $ n $ de bytes deseados.
    salida:  arreglo de $ n $ bytes.
    inicio
      longitud $ \gets $ tamanioDeHash()
      numeroDeBloques $ \gets $ $ \lceil \text{longitud} / n \rceil $
      datos $ \gets $ semilla
      para_todo i en numeroDeBloques:
        resultado ||= hash(datos)
        datos += 1
      fin
      regresar $ n $ bytes de resultado
    fin
\end{pseudocodigo}

\subsubsection{DRBG basado en cifrado por bloque}

Al igual que el generador basado en funciones hash, este método también
introduce un segundo valor crítico: el valor de la llave utilizada por el
cifrado por bloque subyacente. Esta llave se genera a partir de la
entrada de entropía y debe ser tratada de igual forma que la semilla por
las funciones de instanciación, cambio de semilla y desinstanciación.

En el pseudocódigo \ref{ctr_drbg} se muestra la función de generación de
bits psudoaleatorios basada en un cifrado por bloques.

\begin{pseudocodigo}[%
    caption={Generación de bits pseudoaleatorios mediante cifrado por bloques},
    label={ctr_drbg}%
  ]
    entrada: número $ n $ de bytes deseados.
    salida:  arreglo de $ n $ bytes.
    inicio
      mientras longitud(resultado) $ \le n $:
        semilla += 1
        resultado ||= cifrar(semilla)
      fin
      regresar $ n $ bytes de resultado
    fin
\end{pseudocodigo}

\subsubsection{Uso como algoritmo tokenizador}

Un \gls{gl:drbg} puede ser utilizado de manera bastante similar a la
función RN del algoritmo TKR (sección \ref{sec:tkr}): en lugar de utlizar
la función $ f $ como método para obtener bits pseudoaleatorios, se ocupa
el generador pseudoaleatorio. El pseudocódigo para esto se muestra en
\ref{tkr_drbg}; la función \textit{drbg} es una llamada a la operación
de generación de bits pseudoaleatorios de cualquiera de los dos
generadores presentados anteriormente.

Al igual que con TKR, ambas operaciones (tokenización y detokenización)
deben tener el soporte de una base de datos: al generar un nuevo \gls{gl:token}
este se guarda en la base de datos, y la operación de detokenización es
solamente una consulta en la base.

\begin{pseudocodigo}[%
    caption={Generación de \textit{tokens} mediante \gls{gl:drbg}},
    label={tkr_drbg}%
  ]
    entrada: llave k
    salida:  token
    inicio
      $X$ $\gets$ drbg()
      $X_1, X_2, \dots, X_m$ $\gets$ cortar($X$, $\lambda$)
      t $\gets$ ""
      i $\gets$ 0
      mientras $|t| \neq \mu$:
        si entero(X_i) $\le$ $l$:
          t $\gets$ t + entero(X_i)
        fin
        i $\gets$ i  + 1
      fin
      contador $\gets$ contador + 1
      regresar t
    fin
\end{pseudocodigo}
