%
% Explicación sobre FFX.
% Proyecto Lovelace.
%

\subsection{Algoritmo FFX}
\label{sec:ffx}

\Gls{gl:ffx} es un cifrado que preserva el formato presentado en \cite{ffx_1}
por Mihir Bellare, Phillip Rogaway y Terence Spies.  En su forma más general,
el algoritmo se compone de 9 parámetros que permiten cifrar cadenas de
cualquier longitud en cualquier alfabeto; los autores también proponen dos
formas más específicas (dos colecciones de parámetros) para alfabetos binarios
y alfabetos decimales: A2 y A10, respectivamente. El \gls{gl:nist} llama a este
método FF1 en su estándar sobre los cifrados que preservan el
formato\cite{nist_fpe}.

\Gls{gl:ffx} ocupa redes Feistel (ver sección \ref{sec:red_feistel}) junto con
\glspl{gl:primitiva_criptografica} adaptadas (usadas como función de ronda) para
lograr preservar el formato. La idea general para las posibles funciones de
ronda es interpretar la salida binaria de una primitiva (p. ej. un cifrado por
bloques, una función hash, un cifrado de flujo) de forma que tenga el formato
deseado; esto es, que tenga la misma longitud y se encuentre en el mismo
alfabeto que la entrada. Es importante notar que esta función no tiene por qué
ser invertible: las redes Feistel ocupan la misma operación tanto para cifrar
como para descifrar.

La operación general del algoritmo es la misma que la operación de una red
Feistel. Para redes desbalanceadas:
\begin{equation}
  \begin{split}
    L_{i} &= R_{i - 1} \\
    R_{i} &= F_k(R_{i - 1}) \oplus L_{i - 1}
  \end{split}
\end{equation}

Y para redes alternantes:
\begin{equation}
  \begin{split}
    L_{i} &=
    \begin{cases}
      F^1_k(R_{i - 1}) \oplus L_{i - 1},  & \text{si } i \text{ es par} \\
      L_{i - 1},                          & \text{si } i \text{ es impar}
    \end{cases}
    \\
    R_{i} &=
    \begin{cases}
      R_{i - 1},                          & \text{si } i \text{ es par} \\
      F^2_k(L_{i - 1}) \oplus R_{i - 1},  & \text{si } i \text{ es impar}
    \end{cases}
  \end{split}
\end{equation}

\subsubsection{Especificación de parámetros}

A continuación se espifican los 9 parámetros de \gls{gl:ffx}.

\begin{enumerate}

  \item \textbf{\textit{Radix}}.
    Número que determina el alfabeto usado.
    $ C = \{ 0, 1, \dots, \text{radix} - 1\} $. Tanto el texto en claro como
    el texto cifrado pertenecen a este alfabeto.

  \item \textbf{Longitudes}.
    El rango permitido para longitudes de mensaje.

  \item \textbf{Llaves}.
    El conjunto que representa al espacio de llaves.

  \item \textbf{\textit{Tweaks}}.
    El conjunto que representa al espacio de \textit{tweaks}.

  \item \textbf{Suma}.
    El operador utilizado en la red Feistel para combinar la parte izquierda
    con la salida de la función de ronda.

  \item \textbf{Método}.
    El tipo de red Feistel a ocupar: desbalanceada o alternante.

  \item \textbf{\textit{Split}}.
    El grado de desbalanceo de la red Feistel.

  \item \textbf{Rondas}.
    El número de rondas de la red Feistel.

  \item \textbf{F}.
    La función de ronda. Recibe la llave, el \textit{tweak}, el número de
    ronda y un mensaje; regresa una cadena del alfabeto de la longitud
    apropiada.

\end{enumerate}

\subsubsection{FFX A10}

De las dos colecciones de parámetros presentadas en \cite{ffx_2}, la que se
adapta mejor al dominio de los números de tarjetas de crétdito es A10, por lo
que es la única que se presenta aquí. En la tabla \ref{ffx_a10} se muestran los
valores con los que la colección \gls{gl:ffx} A10 inicializa FFX para crear un
cifrado en el dominio decimal.

\begin{table}
  \centering
  \begin{tabular}{| c | c | l |}
    \hline
    \textbf{Parámetro} &
    \textbf{Valor} &
    \textbf{Comentario} \\
    \hline

    \textit{radix}  & 10 & alfabeto de dígitos decimales \\
    \hline

    longitudes & $ [ 4, 36 ] $ & rango de cadenas aceptadas \\
    \hline

    llaves & $ \{ 0, 1\}^{128} $ & misma longitud que para \gls{gl:aes} \\
    \hline

    \textit{tweaks} & $ \text{BYTE}^{\leq 2^{64} - 1} $ & cadena de longitud
    arbitraria \\
    \hline

    suma & por bloque & combinación por operaciones a nivel de bloque \\
    \hline

    método & alternante & redes Feistel alternantes \\
    \hline

    \textit{split} & 0 & lo más cercano al centro posible \\
    \hline

    rondas & 12, 18 o 24 & depende de longitud de mensaje \\
    \hline

  \end{tabular}
  \caption{Colección de parámetros \gls{gl:ffx} A10.}
  \label{ffx_a10}
\end{table}

La dependencia entre el número de rondas y la longitud del mensaje está dada
por la siguiente relación ($ n $ es la longitud del mensaje):
\begin{equation}
  \text{rondas } =
  \begin{cases}
    12 & \text{ si } 10 \leq n \leq 36 \\
    18 & \text{ si } 6  \leq n \leq 9 \\
    24 & \text{ si } 4  \leq n \leq 5
  \end{cases}
\end{equation}

En el pseudocódigo \ref{pc:ffx_a10} se describe a la función de ronda.
Esta concatena el \textit{tweak}, el mensaje de entrada y los
demás parámetros usados por la red en una sola cadena; esta cadena se cifra con
\gls{gl:aes} \gls{gl:cbc} \gls{gl:mac}; la salida se parte en dos (mitad derecha
y mitad izquierda) y es operada para que el número resultado tenga el mismo
número de dígitos que la cadena de entrada. Esta última operación se describe
con la sguiente ecuación ($y\prime$ es la parte izquierda y $ y\prime\prime $
la derecha):
\begin{equation}
  z =
  \begin{cases}
    y\prime\prime \mod 10^{m}, & \text{ si } m \leq 9 \\
    (y\prime \mod 10^{m - 9}) \cdot 10^{9} + (y\prime\prime \mod 10^{m}),
    & \text{ en cualquier otro caso }
  \end{cases}
\end{equation}

El valor de $ m $ corresponde al \textit{split} en la ronda actual; esto es
la longitud de la cadena de entrada.

\begin{pseudocodigo}[%
    caption={\textit{FFX A10}, función de ronda},
    label={pc:ffx_a10}%
  ]
    entrada: Arreglo número; llave k; tweak t; información adicional i
    salida:  Arreglo salida
    inicio
      entradaPrimitiva $ \gets $ número || t || i
      mac $ \gets $ CBC_MAC_AES(entradaPrimitiva, k)
      $ y\prime \gets $ mac[1 ... 64]
      $ y\prime\prime \gets $ mac[65 ... 128]
      si m $ \leq 9 $ entonces:
        z $ \gets y\prime\prime \mod 10^m $
      sino:
        z $ \gets (y\prime \mod 10^{m - 9}) \times 10^9 + (y\prime\prime \mod 10^m)$
      fin
      regresar z
    fin
\end{pseudocodigo}
