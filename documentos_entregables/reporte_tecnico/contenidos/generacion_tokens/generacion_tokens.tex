%
% Capítulo de estándares y publicaciones,
% Reporte técnico.
%
% Proyecto Lovelace.
%

\capitulo{Estándares y publicaciones sobre la generación de \textit{tokens}}
{sec:generacion_de_tokens}
{
  \epigrafe
  {%
    Anyone who attempts to generate random numbers by arithmetical methods is,
    of course, living in a state of sin.%
  }
  {%
    Various techniques used in connection with random digits, \\
    \textsc{John von Neunmann}.%
  }
}

\noindent
Este capítulo inicia con un resumen sobre la composición de los números de
tarjeta, posteriormente, se incluye una serie de resúmenes sobre los estándares
publicados por el \gls{gl:pci} \gls{gl:ssc} y por el \gls{gl:nist} necesarios
para las implementaciones realizadas; por ejemplo, las recomendaciones para la
generación y administración de llaves y la generación de bits pseudoaleatorios.
Dado que generadores de números pseudoaleatorios son utilizados frecuentemente
en los algoritmos tokenizadores, también se incluye un resumen de las pruebas a
las que deben someterse. Finalmente, se clasifican los algoritmos que serán
implementados y, para cada uno, se agrega una subsección donde se explica su
notación, composición y funcionamiento.

\subimport{tarjetas/}{tarjetas}
\subimport{nist_pci/}{nist_pci}
\subimport{algoritmos/}{algoritmos}
