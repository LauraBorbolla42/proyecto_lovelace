%
% Recomendaciones del NIST para la generación de bits pseudoaleatorios,
% capítulo de análisis y diseño para la generación de tokens.
% Proyecto Lovelace.
%

\subsection{Generación de bits pseudoaleatorios}
\label{sec:generadores_pseudoaleatorios}

Existen dos maneras de generar bits aleatorios: la primera es producir bits
de manera no determinística, donde el estado de cada uno (uno o cero) está
determinado por un proceso físico impredecible. Estos generadores de bits
aleatorios (\gls{gl:rbg}) son conocidos como generadores no determinísticos, o
\gls{gl:nrbg}. La otra manera, que será explorada a continuación, es calcular
determinísticamente los bits mediante un algoritmo; estos generadores
determinísticos son conocidos como \gls{gl:drbg}.

Un \gls{gl:drbg} tiene un mecanismo que utiliza un algoritmo que produce una
secuencia de bits partiendo de un valor inicial que es determinado por una
\gls{gl:semilla} que, a su vez, está determinada por la salida de la fuente de
aleatoriedad. Una vez que se tiene la \gls{gl:semilla} y se determina el valor
inicial, el \gls{gl:drbg} es instanciado y puede producir valores. Dado a su
naturaleza determinística, se dice que los valores producidos por el
\gls{gl:drbg} son pseudoaleatorios y no aleatorios; si la \gls{gl:semilla} es
mantenida oculta y el algoritmo fue bien diseñado, los bits de salida del
\gls{gl:drbg} serán impredecibles.

%% MODELO DE UN DRBG

La entrada de \gls{gl:entropia} es provista a un mecanismo \gls{gl:drbg} para
obtener una \gls{gl:semilla} utillizando una fuente de aleatoriedad. La entrada
de \gls{gl:entropia} y la \gls{gl:semilla} deben mantenerse secretas; que estos
valores permanezcan secretos es una de las bases de la seguridad del
\gls{gl:drbg}. Otras entradas, como un \gls{gl:nonce} o una cadena de
personalización pueden ser utilizadas como entradas; estas pueden o no requerir
ser mantenidas secretas también, y ser utilizadas para crear la \gls{gl:semilla}
inicial para el \gls{gl:drbg}.

El estado interno es la memoria del \gls{gl:drbg} y consiste en todos los
valores que requiere el mecanismo (parámetros, variables, etcétera).

El mecanismo \gls{gl:drbg} requiere cinco funciones; estas son explicadas con
más detalle abajo:
\begin{enumerate}
  \item Instanciación (\textit{instantiate function}): obtiene la entrada de
    \gls{gl:entropia} para crear una \gls{gl:semilla} con la cual será
    creado un nuevo estado interno. La entrada puede ser combinada con
    una \gls{gl:nonce} o una cadena de personalización.
  \item Generación (\textit{generate function}): genera bits pseudoaleatorios
    utilizando el estado interno actual; también tiene como salida un nuevo
    estado interno que es utilizado para el siguiente pedido.
  \item Cambio de \gls{gl:semilla} (\textit{reseed function}): obtiene una
    nueva entrada de \gls{gl:entropia} y la combina con el estado interno
    actual para crear una nueva \gls{gl:semilla} y un nuevo estado interno.
  \item Desinstanciación (\textit{uninstantiate function}): elimina el estado
    interno actual.
  \item Prueba de salud (\textit{healt htest function}): determina que el
    mecanismo \gls{gl:drbg} siga funcionando correctamente.
\end{enumerate}

Cuando a un \gls{gl:drbg} se le aplica la función de cambio de \gls{gl:semilla},
es imperativo que la \gls{gl:semilla} sea distinta a la que se utilizó en la
función de instanciación. Cada \gls{gl:semilla} define un nuevo
periodo de \gls{gl:semilla} (\textit{seed period}) para la instanciación del
\gls{gl:drbg}. Una instanciación consiste en uno o más periodos de
\gls{gl:semilla}, estos comienzan cuando se obtiene una nueva \gls{gl:semilla} y
terminan cuando la siguiente \gls{gl:semilla} es obtenida o el \gls{gl:drbg}
deja de utilizarse.

El estado interno deriva de la \gls{gl:semilla}; este incluye el estado de
trabajo (uno o más valores derivados de la \gls{gl:semilla} que deben
permanecer secretos, y la cuenta con el número de salidas que se han producido
con esa semilla) y la información administrativa (el nivel de seguridad, etc).
Es menester proteger el estado interno del \gls{gl:drbg}. La implementación del
mecanismo \gls{gl:drbg} puede haber sido diseñado para tener múltiples
instancias; en este caso, cada instancia debe tener su propio estado interno y
el estado interno de una instancia \gls{gl:drbg} jamás debe ser utilizado como
estado interno para una instancia distinta. El estado interno no debe ser
accesible a funciones distintas a las cinco del \gls{gl:drbg}, ni a otras
instancias del \gls{gl:drbg} o a otros \gls{gl:drbg}.

Los mecanismos especificados en~\cite{nist_aleatorios} soportan cuatro niveles
de seguridad: 112, 128, 192 y 256 bits. Este es uno de los parámetros que
se necesitan para instanciar un \gls{gl:drbg}; además, dependiendo de su diseño,
cada mecanismo  \gls{gl:drbg} tiene sus restricciones de nivel de seguridad.
El nivel de seguridad depende de la implementación del \gls{gl:drbg} y la
cantidad de \gls{gl:entropia} que se da como entrada a la función de
instanciación.

Los bits pseudoaleatorios obtenidos mediante un \gls{gl:drbg} no deben ser
utilizados por una aplicación que requiera un nivel mayor de seguridad que con
el que fue instanciado el \gls{gl:drbg}. La concatenación de dos salidas del
\gls{gl:drbg} tampoco proveen un nivel de seguridad más alto que del que fueron
instanciados (por ejemplo, dos cadenas concatenadas de 128 bits no dan como
resultado una cadena de 256 bits con el nivel de seguridad de 256 bits).

\subimport{drbg/}{semillas}
\subimport{drbg/}{funciones}
\subimport{drbg/}{mec_hash}
\subimport{drbg/}{mec_bloques}
\subimport{drbg/}{garantias}
