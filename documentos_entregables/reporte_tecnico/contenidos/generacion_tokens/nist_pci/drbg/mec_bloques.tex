%
% Recomendaciones del NIST para la generación de bits pseudoaleatorios,
% capítulo de análisis y diseño para la generación de tokens,
% Mecanismos con bloques
% Proyecto Lovelace.
%
\subsubsection{Mecanismos basados en cifrados por bloque}

Un cifrado por bloques \gls{gl:drbg} está basado en un algoritmo de cifrado por
bloques. La seguridad que puede alcanzar el \gls{gl:drbg} depende del cifrado
por bloques y el tamaño de llave utilizado. Se tiene al mecanismo CTR\_DRBG,
que utiliza un cifrado por bloques aprobado con el modo de operación de
contador (véase sección~\ref{sec:modos}); debe utilizarse el mismo algoritmo
de cifrado y tamaño de llave para todas las operaciones de cifrado por bloques
en el \gls{gl:drbg}. El CTR\_DRBG tiene una función actualizadora que es llamada
por los algoritmos de instanciación, generación y cambio de semilla para ajustar
el estado interno cuando hay nueva \gls{gl:entropia}, se le dan entradas
adicionales o cuando se actualiza el estado interno después de generar bits
pseudoaleatorios.
