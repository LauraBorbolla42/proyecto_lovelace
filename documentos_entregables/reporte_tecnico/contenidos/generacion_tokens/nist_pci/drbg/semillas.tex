%
% Recomendaciones del NIST para la generación de bits pseudoaleatorios,
% capítulo de análisis y diseño para la generación de tokens,
% Semillas
% Proyecto Lovelace.
%
\subsubsection{Semillas}

Las \glspl{gl:semilla} deben ser obtenidas antes de generar bits
pseudoaleatorios en el \gls{gl:drbg}, pues esta es utilizada para instanciar al
\gls{gl:drbg} y determinar el estado inicial interno del mecanismo.

Cambiar la \gls{gl:semilla} restaura el secreto de la salida del \gls{gl:drbg}
si el estado interno o la \gls{gl:semilla} son conocidos. Hacer este cambio
periódicamente es una buena manera de mantener a raya el peligro de que valores
como la entrada de \gls{gl:entropia}, la \gls{gl:semilla} o el estado interno
de trabajo; hayan sido comprometidos.

Los ingredientes para determinar una nueva \gls{gl:semilla} para la función de
instanciación son la entrada de entropía de una fuente aleatoria, un
\gls{gl:nonce} y una cadena de personalización (recomendada, pero no
obligatoria). Para hacer un cambio de semilla, se necesita el estado interno
actual, la entrada de \gls{gl:entropia} y una entrada adicional opcional.

La longitud de la \gls{gl:semilla} depende del mecanismo \gls{gl:drbg} y el
nivel de seguridad requerido; sin embargo, siempre debe ser de, al menos, el
mismo número de bits de \gls{gl:entropia} requerida.

La entrada de \gls{gl:entropia} y la \gls{gl:semilla} resultante deben de ser
protegidas con el mismo cuidado con el que se protege la salida del
\gls{gl:drbg}; por ejemplo, si el mecanismo es utilizado para generar llaves,
estos valores deben protegerse con la misma seguridad como son protegidas las
llaves generadas. Además, la seguridad del \gls{gl:drbg} depende de mantener
en secreto la entrada de \gls{gl:entropia}, por lo que esa entrada debe ser
tratada como un parámetro crítico de seguridad (\gls{gl:csp}) y ser obtenido
desde un módulo criptográfico que contenga la función necesaria o ser
transmitido desde un canal seguro.

Cuando se requiera un \gls{gl:nonce} para la construcción de la
\gls{gl:semilla}, esta debe cumplir con una de las siguientes dos condiciones:
tener $nivel\_seguridad/2$ bits de \gls{gl:entropia} o un valor que se espera
no se repita más de lo que se repetiría una cadena aleatoria de
$nivel\_seguridad/2$ bits. Aunque no debe ser mantenido en secreto, cada
\gls{gl:nonce} debe ser considerado como un \gls{gl:csp} y debe ser único en
el módulo criptográfico en donde se realiza la instanciación. El \gls{gl:nonce}
puede estar compuesto por uno o más de los siguientes valores:
\begin{enumerate}
  \item Valor aleatorio generado para cada \gls{gl:nonce} por un generador
    de bits aleatorios aprobado.
  \item Una marca de tiempo con la resolución suficiente para que sea distinto
    cada vez que sea utilizado.
  \item Un número de secuencia que se incremente constantemente.
  \item Una combinación de una marca de tiempo y un número de secuencia que se
    incremente constantemente; tal que el número de secuencia regrese a su
    valor inicial solo cuando la marca de tiempo cambie.
\end{enumerate}

Generar demasiadas salidas partiendo de una misma \gls{gl:semilla} puede proveer
suficiente información para ser capaz de predecir las salidas futuras; por lo
que el cambio de \glspl{gl:semilla} reduce riesgos de seguridad. Las
\glspl{gl:semilla} tienen una vida finita que depende el mecanismo \gls{gl:drbg}
utilizado. Es imperativo que las implementaciones respeten el límite de la vida
de las \glspl{gl:semilla} especificado para el mecanismo; y, cuando se alcance
el límite de la vida de una \gls{gl:semilla}, el \gls{gl:drbg} no debe generar
salidas hasta que se haya cambiado la \gls{gl:semilla} o se cree una nueva
instancia del \gls{gl:drbg} (aunque se prefiere que se cambie la
\gls{gl:semilla}). Una \gls{gl:semilla} jamás debe ser utilizada para
inicializar o cambiar la semilla de otra instancia del \gls{gl:drbg} o la suya.

La cadena de personalización (que es opcional pero recomendada) es utilizada
para derivar la \gls{gl:semilla}, puede ser obtenida dentro o fuera del
módulo criptográfico y hasta puede ser una cadena vacía, pues el \gls{gl:drbg}
no depende de esta cadena para obtener \gls{gl:entropia}. De hecho, que
el adversario conozca la cadena de personalización no disminuye el nivel de
seguridad de una instancia de \gls{gl:drbg} siempre y cuando la entrada de
\gls{gl:entropia} se mantenga desconocida. Esta cadena no es considerada un
\gls{gl:csp}; puede introducir datos adicionales al \gls{gl:drbg}, tales como
identificadores de usuario, aplicación, versiones, protocolos, marcas de tiempo,
direcciones de red, números de serie, etcétera.
