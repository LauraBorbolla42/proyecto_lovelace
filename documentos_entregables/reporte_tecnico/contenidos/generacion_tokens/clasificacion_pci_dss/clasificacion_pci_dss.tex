\section{Clasificación del \texorpdfstring{\acrshort{gl:pci}}{PCI} %
\texorpdfstring{\acrshort{gl:ssc}}{SSC}}
\label{sec:pci_dss}

En~\cite{pci_tokens}, el \gls{gl:pci} \gls{gl:ssc} divide a los
\glspl{gl:token} en reversibles e irreversibles. A su vez, los reversibles se
dividen en criptográficos y en no criptográficos; mientras que los
irreversibles se dividen en autenticables y no autenticables (figura
\ref{fig:division_tokens}).

Los \glspl{gl:token} irreversibles no pueden, bajo ninguna circunstancia, ser
reconvertidos al \gls{gl:pan} original. Esta restricción aplica tanto para
cualquier entidad en el entorno del negocio (comerciante, proveedor de
\glspl{gl:token}, banco) como para cualquier posible atacante. Dados un
\gls{gl:pan} y un \gls{gl:token}, los identificables permiten validar cuando el
primero fue utilizado para la creación del segundo, mientras que los no
identificables, no.

La clasificación del \gls{gl:pci} \gls{gl:ssc} con respecto a los reversibles
resulta un poco confusa: establece que los criptográficos son generados
utilizando \gls{gl:criptografia_fuerte}, el \gls{gl:pan} nunca se almacena,
solamente se guarda una llave; los no criptográficos guardan la relación entre
\glspl{gl:token} y \gls{gl:pan} en una base de datos. El problema está en que
no se menciona \textit{cómo} generar los no criptográficos. A pesar del nombre,
los métodos más comunes para esta categoría ocupan
\glspl{gl:primitiva_criptografica} (p. ej. generadores pseudoaleatorios);
además de que, en una implementación real, para poder cumplir con el
\gls{gl:pci} \gls{gl:dss}, la propia base de datos debe de estar cifrada
\cite{pci_dss}.

\begin{figure}
  \begin{center}
    \subimport{diagramas/}{clasificacion_pci.tikz.tex}
    \caption{Clasificación de los \glspl{gl:token}
      según \gls{gl:pci} \gls{gl:ssc}.}
    \label{fig:division_tokens}
  \end{center}
\end{figure}
