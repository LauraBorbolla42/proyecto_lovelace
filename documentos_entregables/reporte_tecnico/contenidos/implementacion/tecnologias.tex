%
% Selección de tecnologías.
% Análisis y diseño de programa tokenizador, reporte técnico.
%
% Proyecto Lovelace.
%

\section{Tecnologías}
\label{sec:tecnologias}

Las primeras decisiones tomadas con respecto a la implementación y al diseño
fueron sobre el paradigma de programación y el lenguaje a utilizar. Para
tomar estas decisiones se tomaron en cuenta varias consideraciones: primero,
el paradigma orientado a objetos ofrece considerables ventajas con respecto
a la programación estructurada, por lo que, dentro de lo posible, se buscaría
un lenguaje con soporte a este paradigma; segudo, las implementaciones
criptográficas tienen altos requerimientos de rendimiento, por lo que, de los
posibles lenguajes, se necesitaba elegir uno que, aunque no el más rápido,
sí se encontrara entre los de mayor velocidad.

A lo largo de la carrera se ha tenido contacto con bastantes lenguajes que,
si bien no se dominan, sí se posee una base firme como para ser usada: C, C++,
Java, Python, Javascript y \acrshort{gl:php}. Por la cuestión del rendimiento,
todos los lenguajes interpretados (Python, Javascript, \acrshort{gl:php})
quedaron fuera de consideración; la decisión (dentro de los lenguajes con
soporte al paradigma orientado a objetos) quedó entre Java y C++: Java (aún en
las últimas versiones, en donde se han hecho considerables progresos) es mucho
más lento que C++, dado que el trabajo de la máquina virtual en tiempo de
ejecución es bastante considerable; por lo tanto, si de un lenguaje orientado a
objetos se trataba, sería C++. La última decisión se dió entre C y C++:
orientación a objetos contra rendimiento. Al final se optó por la orientación a
objetos: aún cuando C es más rápido que C++, la diferencia no es tan grande,
mientras que un programa con un buen diseño orientado a objetos sí puede ser
mucho más mantenible que uno con un enfoque estructurado.

\subsection{Dependencias}

A continuación se enlistan las principales dependencias del proyecto. Para
cada una se da una breve argumentación sobre su uso, se especifíca, en donde
aplica, la licencia que tiene, y se coloca la versión ocupada.

% Formato de comando:
% 1. Nombre de categoría.
% 2. Nombre de dependencia (con versión).
% 3. Licencia.
% 4. URL de página oficial.
% 5. Descripción.

\dependencia{Compilador}{\acrshort{gl:gcc} v7.3}{\acrshort{gl:gpl}
  v3}{https://gcc.gnu.org/}
{%
  La razón principal de su uso es porque se trata del compilador por defecto de
  los sistemas operativos \acrshort{gl:gnu}/Linux. La verisón de C++ que
  se utiliza es C++14 (opción \texttt{-std=c++14} del compilador). Por razones
  de compatibilidad con algunas otras dependencias (en particular, el conector
  de la base de datos) no es posible utilizar la última versión al momento,
  C++17.
}

\dependencia{Gestor de bases de datos}{MariaDB v10.1}{\acrshort{gl:gpl}
  v2}{https://mariadb.org/}
{%
  El gestor de base de datos que más se ha utilizado en la carrera es MySQL,
  por lo que es el seleccionado para el programa tokenizador; en realidad,
  se trata de una bifurcación de MySQL: MariaDB, cuyo uso ha sido ampliamente
  difundido en varias distribuciones de \acrshort{gl:gnu}/Linux.
}

\dependencia{Librería criptográfica}{Crypto++ v7}{\acrshort{gl:bsl}
  v1}{https://cryptopp.com/}
{%
  Otra decisión importante a tomar antes de hacer las implementaciones de los
  algoritmos tokenizadores fue la librería de funciones criptográficas que
  se utilizaría. De todas las posibles librerías que están validadas
  por el \gls{gl:nist} y que cuentan con una licencia para el uso público, al
  final se hicieron pruebas con dos: Openssl y Crypto++. Openssl está escrita
  en C y cuenta con un historial que le da mucha reputación (es la librería
  que utiliza Opengpg y Openssh). Crypto++ cuenta con una extensa red de
  colaboradores, implementa una gran cantidad de algoritmos y está escrita sólo
  en C++. De ambas, la que se encontró más fácil de usar fue Crypto++, por lo
  que es la que se utiliza en las implementaciones.
}

\subsection{Dependencias de desarrollo}

A continuación se enlistan las dependencias de desarrollo.
Aunque estas no son necesarias para el producto final, sí lo son para
el cambio y mantenimiento del programa. Muchas de estas dependencias
se usan como servicios (v. gr. github o travis). Las elecciones
de estas dependencias no siguieron siempre un proceso de eliminación como las
anteriores, sino que fueron seleccionadas por cuestión de facilidad
de uso (de los autores).

\dependencia{Control de versiones}{Git v2.17}{\acrshort{gl:gpl}
  v2}{https://git-scm.com/}
{%
  Es el \gls{gl:vcs} más común en entornos \acrshort{gl:gnu}/Linux.
  Es muy importante su integración con Github.
}

\dependencia{Servicio de \textit{hosting} para el \acrshort{gl:vcs}}{Github}{No
  aplica (servicio)}{https://github.com}
{%
  Aunque a la fecha soporta otros tipos de \gls{gl:vcs}, originalmente
  era exclusivo a Git. Su función principal es mantener una copia
  remota del código, desde la cuál un desarrollador baja la última versión
  y en donde actualiza sus cambios. También se llegó a utilizar algunas
  funcionalidades adicionales, como el reporte de problemas, los comentarios
  en código y la integración con Travis.
}

\dependencia{Herramienta de integración continua}{Travis}{No
  aplica (servicio)}{https://travis-ci.org/}
{%
  Entre sus principales tareas están compilar desde cero el proyecto, ejecutar
  todas las pruebas, generar y desplegar la documentación y el reporte de
  vuelta a github, ejecutar las pruebas de desempeño y guardar los resultados
  en la base de datos.
}

\dependencia{Documentación de código}{Doxygen}{\acrshort{gl:gpl}
  v2}{https://www.stack.nl/~dimitri/doxygen/}
{%
  Herramienta para generar documentación del código. Permite mantener la
  documentación en los mismos archivos fuente, ya que la salida es generada
  a partir de comentarios. Se cuentan con dos versiones, una en
  \acrshort{gl:html} (\url{https://rqf7.github.io/proyecto_lovelace/documentacion_doxygen/html/index.html})
  y la otra como \acrshort{gl:pdf} (\url{https://rqf7.github.io/proyecto_lovelace/documentacion_doxygen/latex/refman.pdf}).
}

\subsection{Instrucciones de ensamblador}

Las implementaciones de los algoritmos tokenizadores dependen de un par de
conjuntos de instrucciones desarrollados por Intel: el primero es AES-NI y el
segundo RD-SEED.

AES-NI son siete instrucciones que implementan directamente, a nivel de
hardware, algunos de los procesos del algoritmo de AES (sección \ref{sec:aes}).
Estas instrucciones fueron sacadas al mercado en en septiembre de 2010, con la
versión tal del procesador tal. Su objetivo es acelerar la operación de AES,
ya que es una de las primitivas criptográficas más utilizada \cite{aesni_wp}.

RD-SEED se publicó junto con otra instrucción, RD_RAND. RD-SEED es un generador
de números aleatorios que obtiene entropía de procesos físicos relacionados con
el ptrocesador; cumple con los estándares definidos por el NIST en \cite{}
(apéndice tal). Su función es servir como fuente de entropía a generadores
pseudoaleatorios. RD-RAND es un generador de números pseudoaleatorios que
cumple con el estándar del NIST tal. Ambas instrucciones estuvieron disponibles
a partir del mes tal del año tal, con la versión tal dep procesador tal. Para
los fienes de este proyecto, la instrucción que ocupamos es RD-SEDD; RD-RAND es
equivalente a nuestra implementación del generador pseudoaleatorio (sección
tal). También es importante recalcar que RD-RAND, por su naturaleza de
algoritmo, se encuentra disponible en casi todos los procesadores de Intel y
AMD, mientras que RD-SEED implica funcionalidades en hardware que solo soportan
algunas versiones de procesadores Intel.

Para ambas funcionalidades, el enfoque de implementación busca ser lo más
flexible posible: si la arquitectura en la corre el programa soporta las
instrucciones en cuestión, perfecto, sino, se buscan alternativas que de todas
formas permitan ejecutar el programa. El el caso de AES-NI, la alternativa es
utilizar la implementación de AES de Cryptopp. Para RD-SEED, se ocupa el
generador del kernel de linux (\textit{/dev/random} en modo bloqueante y
\textit{/dev/urandom} en modo no bloqueante).
