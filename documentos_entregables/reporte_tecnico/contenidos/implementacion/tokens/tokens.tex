%
% Implementación de programa de tokens,
% Reporte técnico.
%
% Proyecto Lovelace.
%

\section{Programa para generar \textit{tokens}}

En la figura \ref{arbol:distribucion_general} se muestra la estructura general
del programa junto con los contenidos de los módulos de utilidades. En
las siguientes secciones se explicarán los aspectos más relevantes del
código de los módulos principales: \texttt{ahr}, \texttt{bps}, \texttt{drbg},
\texttt{ffx} y \texttt{tkr}.

\begin{figure}
  \begin{center}
    \begin{varwidth}{\textwidth}
      \dirtree{%
      .0 /.
      .1 implementaciones.
      .2 acceso\_a\_datos.
      .2 aes\_ensamblador.
      .2 ahr.
      .2 bps.
      .2 drbg.
      .2 ffx.
      .2 redes\_feistel.
      .2 tkr.
      .2 utilidades.
      .1 utilidades.
      .2 algoritmo\_tokenizador.
      .2 algoritmo\_tokenizador\_reversible.
      .2 algoritmo\_tokenizador\_irreversible.
      .2 utilidades\_criptograficas.
      .2 utilidades\_tarjetas.
      .2 interfaces\_comunes.
      .3 funcion.
      .3 funcion\_con\_inverso.
      .3 funcion\_con\_inverso\_simetrico.
      .2 arreglo.
      .2 arreglo\_de\_digitos.
      .2 codificador.
      .2 conjunto\_de\_pruebas.
      .2 error.
      .2 intermediario\_de\_arreglo.
      .2 intermediario\_de\_arreglo\_de\_digitos.
      .2 prueba.
      .2 utilidades\_matematicas.
      }
    \end{varwidth}
    \caption{Distribución general del código}
    \label{arbol:distribucion_general}
  \end{center}
\end{figure}

\subimport{/}{ffx}
