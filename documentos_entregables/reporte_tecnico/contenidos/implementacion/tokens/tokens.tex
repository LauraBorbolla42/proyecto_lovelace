%
% Implementación de programa de tokens,
% Reporte técnico.
%
% Proyecto Lovelace.
%

\section{Programa para generar \textit{tokens}}

A continuación se explican las partes más importantes del programa:
las implementaciones de los algoritmos tokenizadores mostrados en la
sección \ref{sec:algoritmos}.

En un intento por no complicar demasiado la exposición del programa, se  redujo
al máximo la aparición de código en el documento. Aquí solo se muestran las
implementaciones de los algoritmos tokenizadores, y aún en estos casos,
solamente se muestran las fracciones más relevantes. Al día de hoy, el programa
cuenta con alrededor de 16000 líneas de código (contando comentarios y líneas
en blanco), por lo que cualquier intento de incluir aquí a todo el programa
resultaría en un documento de dimensiones estratosféricas. Si lo que se busca
es una implementación en específico que no aparezca aquí, o mayor detalle en el
código, se puede consultar la documentación en línea:
\url{https://rqf7.github.io/proyecto_lovelace/documentacion_doxygen/html/index.html}.
Durante el desarrollo del programa se han hecho muchos esfuerzos para
mantenerla completa.

Antes de iniciar con la explicación sobre las implementaciones sería
comveniente aclarar un par de puntos sobre las reglas que sigue el formato del
código, para facilitar la lectura: todos los identificadores se encuentran en
español (el conjunto permitido por los caracteres \gls{gl:ascii}); las clases y
espacios de nombres siguen \texttt{EsteFormato} mientras que las variables y
funciones siguen \texttt{esteOtroFormato}; las variables miembro de una clase
llevan simepre en el nombre una «m» como prefijo. Se pueden encontrar mayores
detalles en:
\url{https://rqf7.github.io/proyecto_lovelace/reglas_de_estilo.html}.

%\begin{figure}
%  \begin{center}
%    \begin{varwidth}{\textwidth}
%      \dirtree{%
%      .0 /.
%      .1 implementaciones.
%      .2 acceso\_a\_datos.
%      .2 aes\_ensamblador.
%      .2 ahr.
%      .2 bps.
%      .2 drbg.
%      .2 ffx.
%      .2 redes\_feistel.
%      .2 tkr.
%      .2 utilidades.
%      .1 utilidades.
%      .2 algoritmo\_tokenizador.
%      .2 algoritmo\_tokenizador\_reversible.
%      .2 algoritmo\_tokenizador\_irreversible.
%      .2 utilidades\_criptograficas.
%      .2 utilidades\_tarjetas.
%      .2 interfaces\_comunes.
%      .3 funcion.
%      .3 funcion\_con\_inverso.
%      .3 funcion\_con\_inverso\_simetrico.
%      .2 arreglo.
%      .2 arreglo\_de\_digitos.
%      .2 codificador.
%      .2 conjunto\_de\_pruebas.
%      .2 error.
%      .2 intermediario\_de\_arreglo.
%      .2 intermediario\_de\_arreglo\_de\_digitos.
%      .2 prueba.
%      .2 utilidades\_matematicas.
%      }
%    \end{varwidth}
%    \caption{Distribución general del código}
%    \label{arbol:distribucion_general}
%  \end{center}
%\end{figure}

\subimport{/}{ffx}
\subimport{/}{tkr}
\subimport{/}{drbg}
\subimport{/}{ahr}
