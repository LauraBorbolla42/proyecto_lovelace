%
% Implementación de programa de tokens,
% Reporte técnico.
%
% Proyecto Lovelace.
%

\section{Programa para generar \textit{tokens}}

A continuación se explican las partes más importantes del programa:
las implementaciones de los algoritmos tokenizadores mostrados en la
sección \ref{sec:algoritmos}.

%\begin{figure}
%  \begin{center}
%    \begin{varwidth}{\textwidth}
%      \dirtree{%
%      .0 /.
%      .1 implementaciones.
%      .2 acceso\_a\_datos.
%      .2 aes\_ensamblador.
%      .2 ahr.
%      .2 bps.
%      .2 drbg.
%      .2 ffx.
%      .2 redes\_feistel.
%      .2 tkr.
%      .2 utilidades.
%      .1 utilidades.
%      .2 algoritmo\_tokenizador.
%      .2 algoritmo\_tokenizador\_reversible.
%      .2 algoritmo\_tokenizador\_irreversible.
%      .2 utilidades\_criptograficas.
%      .2 utilidades\_tarjetas.
%      .2 interfaces\_comunes.
%      .3 funcion.
%      .3 funcion\_con\_inverso.
%      .3 funcion\_con\_inverso\_simetrico.
%      .2 arreglo.
%      .2 arreglo\_de\_digitos.
%      .2 codificador.
%      .2 conjunto\_de\_pruebas.
%      .2 error.
%      .2 intermediario\_de\_arreglo.
%      .2 intermediario\_de\_arreglo\_de\_digitos.
%      .2 prueba.
%      .2 utilidades\_matematicas.
%      }
%    \end{varwidth}
%    \caption{Distribución general del código}
%    \label{arbol:distribucion_general}
%  \end{center}
%\end{figure}

\subimport{/}{ffx}
