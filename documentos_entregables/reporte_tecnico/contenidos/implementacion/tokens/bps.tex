%
% Explicación de código de BPS, reporte técnico.
%
% Proyecto Lovelace.
%

%
\subsection{Módulo de BPS}

El funcionamiento de BPS se describe de forma mas amplia en la sección
\ref{sec:bps}.

A pesar de que no de que se puede ver a BPS como una versión más especifica de
\gls{gl:ffx}, esta implementación se realizo de manera distinta.

Viendo de forma general, BPS consta de los 2 métodos principales, el de cifrar
y el de descifrar, los cuales se usas en modo de operación utilizado por BPS
y que se describe en la sección \ref{sec:bps_modo_operacion}.

\begin{listing}
  \inputminted[firstline=77, lastline=156]
    {c++}{../implementaciones/bps/cifrador_BPS.cpp}
  \caption{Función de cifrado de BPS}
  \label{codigo:bps_cifrado}
\end{listing}

\begin{listing}
  \inputminted[firstline=158, lastline=250]
    {c++}{../implementaciones/bps/cifrador_BPS.cpp}
  \caption{Función de descifrado de BPS}
  \label{codigo:bps_descifrado}
\end{listing}

Otro parte crucial de BPS es el cifrador interno BC, por lo cual es importante
mostrar su implementación. El código \ref{codigo:bps_cifrador_bc} esta basado
en el pseudocódigo para el BC descrito en \ref{sec:bps_cifrador_interno_bc}

\begin{listing}
  \inputminted[firstline=111, lastline=218]
    {c++}{../implementaciones/bps/cifrador_BC.cpp}
  \caption{Cifrador interno BC de  BPS}
  \label{codigo:bps_cifrador_bc}
\end{listing}

Como se observa, el cifrador interno usa a su ves un cifrador de ronda, el cual
es un cifrado por bloques (en este caso \gls{gl:aes} o \gls{gl:des}), que se
realizo usando la librería de Crypto++.
