%
% Sección de ideas equivocadas sobre la tokenización,
% Introducción
% Reporte Técnico.
%
% Proyecto Lovelace.
%

\section{La tokenización en otros contextos}

Dado que el sustantivo \gls{gl:token} y el término \textit{tokenización} serán
utilizados una y otra vez a lo largo del presente documento, es imperativo
definirlos claramente. Actualmente, el vocablo \gls{gl:token} es ampliamente
utilizado y su significado cambia dependiendo del contexto en el que se
encuentra; a continuación se enlistan una serie de definiciones que
ejemplifican lo cambiante que puede ser su definición:
\begin{description}
  \item [En el procesamiento de lenguaje natural,] la tokenización es el proceso
    de segmentar un texto en unidades lingüísticas, tales como palabras, signos
    de puntuación, números y símbolos; regularmente, un \gls{gl:token} es una
    palabra lingüísticamente significativa~\cite{miscon_nlp}.
  \item [En compiladores,] la tokenización (o análisis léxico) se refiere al
    proceso de convertir una secuencia de caracteres a una secuencia de
    \glspl{gl:token} (cadenas con significados)~\cite{miscon_comp}.
  \item [En el contexto social,] se le llama tokenismo a la práctica de hacer
    solo un intento simbólico de ser inclusivo con un colectivo discriminado
    para aparentar diversidad (racial, sexual, etc.); quienes pertenecen a la
    minoría son conocidos como \glspl{gl:token}~\cite{miscon_soc}.
  \item [En el contexto bancario,] se le denomina \gls{gl:token} a un pequeño
    aparato encargado de generar llaves para poder establecer una conexión con
    sus servidores.
  \item [En el contexto económico,] se le denomina \gls{gl:token} a una pieza
    metálica o plástica que puede ser utilizada en vez del dinero; por ejemplo,
    las fichas que se utilizan en los casinos.
\end{description}

Probablemente una de las fuentes de confusión más grandes viene del contexto
de las \glspl{gl:criptomoneda}, pues el auge y la presencia que han tenido en
los medios los últimos años las han relacionado intrínsecamente con los
\glspl{gl:token}. Aunado a esto, dentro de este contexto, un \gls{gl:token}
adquiere varios significados: puede utilizarse como sinónimo de
\gls{gl:criptomoneda} (\textit{Ana tiene 42 tokens de Bitcoin}) o puede hacer
referencia a la cadena de números y letras que identifican una transacción.

Después de identificar qué no significa un \gls{gl:token}, a continuación se
provee una definición de lo que es un \gls{gl:token} en el contexto de la
criptografía, particularmente, en el campo de los algoritmos tokenizadores: un
\gls{gl:token} es un valor representativo de un dato sensible (p. ej. un
número de seguridad social, un número de tarjeta) que carece de una relación
directa con el dato sensible en cuestión; dicho de otra manera, partiendo del
\gls{gl:token}, es imposible llegar al dato sensible sin los medios adecuados.
El proceso mediante el cual se genera el \gls{gl:token}, es conocido como
\textit{tokenización} y esta hace uso de algoritmos tokenizadores; en el caso
de este trabajo, los \glspl{gl:token} resultantes de este proceso de
tokenización son cadenas que \textit{parecen} números de tarjetas bancarias
pero que no lo son.
