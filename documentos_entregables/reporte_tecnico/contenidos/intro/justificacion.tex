%
% Sección de justificación, capítulo de introducción,
% reporte técnico.
%
% Proyecto Lovelace.
%

\section{Justificación}

Dado que es un tema relativamente nuevo, la desinformación y la desconfianza
respecto a la generación de \glspl{gl:token} es latente aún, pues, por ejemplo,
aunque el \gls{gl:pci}~\gls{gl:dss} describe los requerimientos que debe cumplir
un sistema tokenizador, no indica qué hacer para satisfacerlos. Muchas empresas
se aprovechan de esta desinformación para promocionarse como sistema
tokenizador sin ser claras sobre sus métodos. A continuación se muestran las
principales deficiencias encontradas como resultado de una investigación
comparativa de las principales soluciones que existen en el mercado.

\begin{description}

  \item[Shift4]
    Esta compañía reclama el crédito de haber inventado el paradigma de
    la tokenización. Denuncia que otras compañías y el mismo
    \gls{gl:pci}~\gls{gl:ssc} desvirtuaron el término; por esta razón, la
    compañía se refiere a su propio método cómo \textit{TrueTokenization}.
    Aunque mencionan en todas partes las supuestas ventajas de sus métodos,
    lo único que dicen con claridad sobre la generación de \glspl{gl:token} es
    que se trata de valores aleatorios, únicos y alfanuméricos. Esta descripción
    no representa en modo alguno una garantía, pues podrían estar usando un
    método cuya seguridad no esté formalmente
    probada~\cite{shif4_uno, shif4_dos}.

  \item[Bluepay TokenShield]
    En este servicio, a pesar de que se ofrecen dos formas distintas de
    tokenizar, nunca se aclara cómo es que funciona alguno de estos procesos.
    Además, en \cite{bluepay_tokenshield} hay una breve descripción del flujo
    de datos del servicio, donde se dice que con el \gls{gl:token} se recupera
    y descifra la información bancaria sensible, lo que genera confusión, pues
    no se sabe si la tokenización se hace por medio de de cifrados que
    preservan el formato\footnotemark, o por medio de bases de datos con tablas
    que contienen pares \gls{gl:pan}-\gls{gl:token}.

    \footnotetext{
      Tipo de cifrado que permite que los mensajes ya cifrados se vean parecidos
      a los originales. Este tema se abarca en la sección \ref{sec:fpe}.
    }

  \item[Braintree]
    Compañía que ofrece distintos \gls{gl:sdk} (para dispositivos móviles y
    para desarrollo web) que permiten interactuar con sus propios servidores
    para procesar pagos con tarjetas de crédito. En~\cite{braintree_uno}
    dan una definición de \gls{gl:token} que no es compatible con el
    \gls{gl:pci}~\gls{gl:ssc}: la única forma de generar \glspl{gl:token},
    según su definición, es por métodos aleatorios. Aún aceptando
    esa definición, no se ofrecen mayores explicaciones: los clientes no
    saben de qué forma se están generando los valores aleatorios.

  \item[Merchant Link]
    La solución de tokenización de Merchant Link es llamada TransactionVault. Su
    página da tres puntos importantes respecto a la tokenización:
    \begin{itemize}
      \item Utilizan tecnología de la \textit{siguiente generación}.
      \item Los \glspl{gl:token} son asociados con una tarjeta y una
        transacción.
      \item Todos los \glspl{gl:token} generados tienen una longitud de 16
        dígitos y pasan los controles de validez de las tarjetas bancarias
        (por ejemplo, el algoritmo de Luhn\footnotemark).
        \footnotetext{
          Este algoritmo es descrito en ~\ref{luhn:1}.
        }
    \end{itemize}

      Aunque en el tercer punto informan al cliente que los \glspl{gl:token}
      creados preservan el formato, no indican cuáles algoritmos son utilizados
      para generarlos; aunado a esto, el segundo punto tampoco aporta mucha
      información: sí, utilizan algoritmos reversibles\footnotemark, pues los
      \glspl{gl:token} quedan ligados a una tarjeta o a una transacción,
      pero siguen sin especificar sus métodos \cite{merchant_link}.

      \footnotetext{
        Algoritmos que, dado un \gls{gl:token}, pueden regresar al \gls{gl:pan}
        correspondiente. Esta clasificación es explicada en~\ref{sec:pci_dss}.
      }

    \item[Jack Henry Card Processing Solution]
      La solución de tokenización ofrecida por Jack Henry Banks indica que se
      encarga de sustituir el número de tarjeta con un \gls{gl:token} numérico
      que pasa los controles de validez  de las tarjetas bancarias; sin
      embargo, no dice qué métodos utiliza para generar los \glspl{gl:token}.
      Indica también que parte de su plataforma está integrada con los servicios
      de tokenización provistos por \gls{gl:mdes} y \gls{gl:vts}; sin embargo,
      estos servicios tampoco indican cómo generan los tokens. Finalmente, no
      queda claro quién se encarga de generar los \glspl{gl:token}, ¿lo hace
      \gls{gl:mdes}, \gls{gl:vts} o Jack
      Henry?~\cite{jack_henry, mdes_1, mdes_2}

    \item[Securosis]
      La publicación en \cite{securosis} tiene por objeto esclarecer el papel
      de la tokenización. Explica de forma bastante clara las ventajas del
      paradigma y el flujo de datos entre tienda, sistema tokenizador y banco;
      sin embargo, al igual que las empresas anteriores, carece de una
      explicación sobre los propios \glspl{gl:token}, y algunas de las
      referencias hacia estos solo consigue confundir. Por ejemplo, presentan
      a la tokenización como una alternativa a la criptografía, hablando
      de los \glspl{gl:token} como valores aleatorios que nada tienen que
      ver con criptografía.

    % Thales Tokenization.
    % Al parecer este si dice cómo es que funciona, usa FPE, mas especifico FF3
    % para tokenizar información basada en texto, y tablas de búsqueda para
    % información numérica de 9 a 19 dígitos.
    % Aquí está la mayoría de la información:
    % (no importa la calidad de las imágenes)
    %http://go.thalesesecurity.com/rs/480-LWA-970/images/Fortrex_Vormetric_VTS_Token_Evaluation.PDF?aliId=59031373

\end{description}

Muchas de las soluciones del mercado intentan argumentar que un \gls{gl:token}
es un valor aleatorio y, por tanto, nada tiene que ver con criptografía;
al parecer esta estrategia publicitaria tuvo mucho éxito, pues uno de los
mensajes más comunes es que la criptografía es cosa del pasado, la tokenización
permite sustituirla. Mientras esto puede ser cierto para los posibles clientes
de un sistema tokenizador (tiendas que procesan pagos en línea), es una mentira
manifiesta cuando se considera a todos los  participantes.

La mayoría de las soluciones tratan a sus métodos como secretos de compañía: no
explican la manera en que generan los \glspl{gl:token}, por lo que esperan que
el trato entre cliente y proveedor esté basado en la confianza que tiene el
primero por el segundo. El modelo basado en la confianza es común al
funcionamiento de casi todas las transacciones monetarias hechas por internet:
dos partes que quieran realizar una transacción necesitan de un tercero (la
institución financiera) para llevarla a cabo. Un estado ideal eliminaría la
necesidad de la tercera parte (es lo que intentan hacer
\glspl{gl:criptomoneda} como \textit{bitcoin}~\cite{bitcoin}), sin embargo,
mientras este estado no se alcance, el modelo basado en la confianza es un mal
necesario, y como tal, debe ser reducido al máximo: un proveedor del
servicio de tokenización debería ser muy claro sobre lo que hace para
tokenizar, permitiendo que sus clientes basen en esto su decisión.

La desinformación que hay alrededor del tema de la tokenización es la principal
justificación para este trabajo: se busca implementar algoritmos públicos
y presentar los resultado, permitiendo a los clientes de sistemas tokenizadores
entender qué hay dentro de la caja negra. Parte de este combate a la
desinformación consiste en dejar bien clara la relación entre la tokenización y
la criptografía: la tokenización es una aplicación de la criptografía, por lo
que los métodos de tokenización deben ser analizados con la misma formalidad con
la que se estudian todas las demás formas de cifrado.
