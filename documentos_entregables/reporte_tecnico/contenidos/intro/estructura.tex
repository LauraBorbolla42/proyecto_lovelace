%
% Organización del documento, capítulo de introducción,
% reporte técnico.
%
% Proyecto Lovelace.
%

\section{Organización del documento}

El capítulo \ref{sec:marco_teorico} provee un conjunto de resúmenes sobre los
temas necesarios para la comprensión de este trabajo; adentrándose en la
criptografía, sus objetivos y los métodos que emplea para la protección de
información, tales como los cifradores por bloque y las funciones hash.

En el capítulo \ref{sec:generacion_de_tokens} se describen algunos de los
estándares pertinentes a la generación de \glspl{gl:token}, así como las
descripciones de los algoritmos existentes para realizar esta labor.

El capítulo \ref{sec:analisis_y_disenio} muestra el análisis técnico para
la implementación del programa tokenizador; abordando sus requerimientos,
la selección de tecnologías, y el diseño del propio programa.

El capítulo \ref{sec:implementacion} contiene los fragmentos de código más
importantes para entender el funcionamiento tanto de los algoritmos de
tokenización como el programa de generación de \glspl{gl:token}; también
presenta los resultados de las comparaciones de desempeño entre los
distintos algoritmos tokenizadores.

Por último, en el capítulo \ref{sec:conclusiones} se concluyen los resultados
del trabajo, hablando sobre los avances que se hicieron y el trabajo que aún
falta por hacerse.
