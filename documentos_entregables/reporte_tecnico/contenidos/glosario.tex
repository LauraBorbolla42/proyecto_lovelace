%
% Definición de entradas del glosario.
% Proyecto Lovelace.
%
% Más información en https://www.sharelatex.com/learn/Glossaries
%

\newglossary*{siglas_cripto}{Criptográficos}
\newglossary*{siglas_bancarias}{Bancarios}
\newglossary*{siglas_computacionales}{Computacionales}
\newglossary*{siglas_instituciones}{Instituciones y agrupaciones}

\makeglossaries

\setglossarypreamble
{%
  \phantomsection%
  \addcontentsline{toc}{chapter}{Glosario}%
  \capituloSinNumero{Glosario}{sec:glosario}{%
    \epigrafe{%
      The most important thing in a programming language is the name. A
      language will not succeed without a good name. I have recently invented a
      very good name and now I am looking for a suitable language.}{%
      \textsc{Donald Knuth}.}}%
  %
  Glosario de términos de computación, criptográficos y matemáticos. Las
  principales fuentes bibliográficas usadas son~\cite{menezes}
  y ~\cite{stallings}; en caso de tratarse de una fuente distinta, se indica
  en la entrada en particular.
}

\newglossaryentry{gl:modo_de_operacion}
{
  name = modo de operación,
  plural = modos de operación,
  description = {
    Construcción que permite extender la funcionalidad de un cifrado a bloques
    para operar sobre tamaños de información arbitrarios%
  }
}

\newglossaryentry{gl:vector_de_inicializacion}
{
  name = vector de inicialización,
  plural = vectores de inicialización,
  description = {
    Cadena de bits de tamaño fijo que sirve como entrada a muchas primitivas
    criptográficas (p. ej. algunos modos de operación). Generalmente
    se requiere que sea generado de forma aleatoria%
  },
  see={gl:modo_de_operacion,gl:primitiva_criptografica}
}

\newglossaryentry{gl:ronda}
{
  name = ronda,
  description = {
    (\textit{Round}) Bloque compuesto por un conjunto de operaciones que es
    ejecutado múltiples veces. Las \textit{rondas} son definidas por el
    algoritmo de cifrado%
  }
}

\newglossaryentry{gl:integridad_datos}
{
  name = integridad de datos,
  description = {
    Propiedad en la que los datos no han sido alterados sin autorización desde
    que fueron creados, transmitidos o almacenados por una fuente autorizada.
    Operaciones que insertan, eliminan, modifican o reordenan bits invalidan
    la \textit{integridad de los datos}. La \textit{integridad de los datos}
    incluye que los datos estén completos y, cuando los datos son divididos en
    bloques, cada bloque debe cumplir con lo mencionado anteriormente%
  }
}

\newglossaryentry{gl:autenticacion_origen}
{
  name = autenticación de origen,
  description = {
    Tipo de autenticación donde se corrobora que una entidad es la fuente
    original de la creación de un conjunto de datos en un tiempo específico.
    Por definición, la \textit{autenticación de origen} incluye la integridad
    de datos, pues cuando se modifican los datos, se tiene una nueva fuente%
  },
  see={gl:integridad_datos}
}

\newglossaryentry{gl:cifrado_iterativo}
{
  name = cifrado iterativo,
  plural = cifrados iterativos,
  description = {
    (\textit{Iterated block cipher}) Cifrado de bloque que involucra la
    repetición secuencial de una función interna llamada función de ronda. Los
    parámetros incluyen el número de rondas, el tamaño de bloque y
    el tamaño de llave%
  },
  see={gl:ronda}
}

\newglossaryentry{gl:entropia}
{
  name = entropía,
  description = {
    Definida para una función de probabilidad de distribución discreta,
    mide cuánta información en promedio es requerida para identificar
    muestras aleatorias de esa distribución%
  },
  see={gl:funcion,gl:distribucion_probabilidad}
}

\newglossaryentry{gl:preimagen}
{
  name = preimagen,
  plural = preimágenes,
  description = {
    Suponga que se tiene $x \in X$ y $y \in Y$ tal que $f(x) = y$;
    se dice entonces que $x$ es \textit{preimagen} de $y$, o que
    $y$ es la imagen de $x$ bajo $f$%
  },
  see={gl:imagen,gl:funcion,gl:dominio,gl:codominio}
}

\newglossaryentry{gl:imagen}
{
  name = imagen,
  plural = imágenes,
  description = {
    Suponga que se tiene $x \in X$ y $y \in Y$ tal que $f(x) = y$;
    se dice entonces que $y$ es la \textit{imagen} de $x$ bajo $f$,
    o que $x$ es preimagen de $y$%
  },
  see={gl:preimagen,gl:funcion,gl:dominio,gl:codominio}
}

\newglossaryentry{gl:biyeccion}
{
  name = biyección,
  plural = biyecciones,
  description = {
    Dicho de las funciones que son inyectivas y
    suprayectivas al mismo tiempo; en otras palabras, que todos los
    elementos del conjunto de salida tengan una imagen distinta en el conjunto
    de llegada y a cada elemento del conjunto de llegada le corresponde
    un elemento del conjunto de salida%
  },
  see={gl:inyectiva,gl:suprayectiva,gl:funcion,gl:imagen}
}

\newglossaryentry{gl:inyectiva}
{
  name = inyectiva,
  description = {
    Una función $f:D_f \rightarrow C_f$ es \textit{inyectiva} (o uno a
    uno) si a diferentes elementos del dominio le corresponden
    diferentes elementos del codominio; se cumple para dos
    valores cualesquiera $x_1, x_2 \in D_f$ que
    $x_1 \neq x_2 \implies f(x_1) \neq f(x_2)$%
  },
  see={gl:funcion,gl:dominio,gl:codominio,gl:biyeccion,gl:suprayectiva}
}

\newglossaryentry{gl:suprayectiva}
{
  name = suprayectiva,
  description = {
    Una función $f:D_f \rightarrow C_f$ es \textit{suprayectiva} si
    todo elemento de su codominio $C_f$ es imagen de
    por lo menos un elemento de su dominio $D_f$: $\forall b \in C_f$
    $\exists a \in D_f$ tal que $f(a)=b$%
  },
  see={gl:funcion,gl:dominio,gl:codominio,gl:imagen,gl:biyeccion,gl:inyectiva}
}

\newglossaryentry{gl:dominio}
{
  name = dominio,
  description = {
    El \textit{dominio} de una función $f(x)$ es el conjunto de
    valores para los cuales la función está definida%
  },
  see={gl:funcion,gl:codominio}
}

\newglossaryentry{gl:codominio}
{
  name = codominio,
  description = {
    Una función mapea a los elementos de un conjunto $A$ con elementos
    de un conjunto $B$; $A$ es el dominio y $B$ es el
    \textit{codominio}%
  },
  see={gl:funcion,gl:dominio}
}

\newglossaryentry{gl:funcion}
{
  name = función,
  plural = funciones,
  description = {
    Regla entre dos conjuntos $A$ y $B$ de manera que a cada elemento del
    conjunto $A$ le corresponda un único elemento del conjunto $B$%
  },
  see={gl:codominio,gl:dominio}
}

\newglossaryentry{gl:token}
{
  name = token,
  description = {
    Valor representativo que se usa en lugar de información valiosa%
  }
}

\newglossaryentry{gl:criptografia_fuerte}
{
  name = criptografía fuerte,
  description = {
    (\textit{Strong cryptography}) De acuerdo al \gls{gl:pci} \gls{gl:ssc}
    (en~\cite{dss_glosario}), es la criptografía basada en algorítmos probados
    y aceptados en la industria, junto con longitudes de llaves fuertes y
    buenas prácticas de administración de llaves%
  }
}

\newglossaryentry{gl:criptologia}
{
  name = criptología,
  description = {
    Estudio de los sistemas, claves y lenguajes secretos u ocultos%
  }
}

\newglossaryentry{gl:permutacion}
{
  name = permutación,
  plural = permutaciones,
  description = {
    Sea $ S $ un conjunto finito de elementos. Una \textit{permutación} $ p $
    en $ S $ es una biyección de $ S $ a sí misma (i. e.
    $ p: S \rightarrow S $)%
  },
  see={gl:biyeccion}
}

\newglossaryentry{gl:computacionalmente_no_factible}
{
  name = computacionalmente no factible,
  description = {
    (\textit{Computationally infeasible}) Se dice que una tarea es
    \textit{computacionalmente no factible} si su costo (medido en términos de
    espacio o de tiempo) es finito pero ridículamente grande%
  }
}

\newglossaryentry{gl:primitiva_criptografica}
{
  name = primitiva criptográfica,
  plural = primitivas criptográficas,
  description = {
    (\textit{Cryptographic primitive}) Algoritmos criptográficos que son usados
    con frecuencia para la construcción de protocolos de seguridad. En
   ~\cite{menezes} se clasifican en tres categorías principales: de llave
    simétrica, de llave pública y sin llave%
  }
}

\newglossaryentry{gl:fuerza_efectiva}
{
  name = fuerza efectiva,
  plural = fuerzas efectivas,
  description = {
    Para un espacio de llaves $ K $, su \textit{fuerza efectiva} es
    $ \log_2 | K | $ (el logaritmo base dos de su cardinalidad)%
  }
}

\longnewglossaryentry{gl:probabilidad_condicional}
{
  name = probabilidad condicional
}
{%
  Sean $ E_1 $ y $ E_2 $ dos eventos, con $ P(E_2) \ge 0 $. La
  \textit{probabilidad condicional} se denota por $ P(E_1 | E_2) $, y es
  igual a
  $$ P(E_1 | E_2) = \frac{P(E_1 \cap E_2)}{P(E_2)} $$
  Esto mide la probabilidad de que ocurra $ E_1 $ sabiendo que ya ocurrió
  $ E_2 $%
}

\newglossaryentry{gl:estidisticamente_independiente}
{
  name = estadísticamente independiente,
  description = {
    Dicho de la ocurrencia de dos eventos $ E_1 $ y $ E_2 $: si
    $ P(E_1 \cap E_2) = P(E_1) P(E_2) $ entonces $ E_1 $ y $ E_2 $ son
    \textit{estadísticamente independientes} entre sí. Es importante notar
    que si esto ocurre, entonces $ P(E_1 | E_2) = P(E_1) $ y
    $ P(E_2 | E_1) = P(E_2) $, es decir, la ocurrencia de uno no tiene ninguna
    influencia en las probabilidades de ocurrencia del otro%
  },
  see={gl:probabilidad_condicional}
}

\newglossaryentry{gl:oraculo}
{
  name = oráculo,
  description = {
    \textit{Oracle machine} Se refiere a una máquina abstracta utilizada para
    estudiar problemas de decisión. Puede verse como una máquina de Turing con
    una caja negra (llamada \textit{oráculo}) que puede resolver ciertos
    problemas de decisión u obtener el valor de una función en una sola
    operación%
  },
  see={gl:maquina_turing}
}

\newglossaryentry{gl:maquina_turing}
{
  name = máquina de Turing,
  plural = máquinas de Turing,
  description = {
    (\textit{Turing machine}) Se considera como una cinta infinita dividida
    en casillas, cada una de las cuales contiene un símbolo. Sobre dicha cinta
    actúa un dispositivo que puede adoptar distintos estados y que, en cada
    instante, lee un símbolo de la casilla sobre la que está situado;
    dependiendo del símbolo leído y del estado en el que se encuentra, la
    máquina realiza las siguientes tres acciones: primero, pasa a un nuevo
    estado; segundo, imprime un símbolo en el lugar del que acaba de leer;
    y, tercero, se desplaza hacia la derecha, hacia la izquierda o se detiene.
    De forma más concreta, una máquina de Turing puede ser vista como una
    máquina de estados finitos con un método de almacenamiento asociado (la
    cinta infinita)~\cite{minsky, DBLP:books/daglib/0023088,
    DBLP:books/daglib/0032222}%
  },
  see={gl:maquina_de_estados_finitos}
}

\newglossaryentry{gl:distribucion_probabilidad}
{
  name = distribución de probabilidad,
  description = {
    Una distribución de probabilidad $P$ en el conjunto de eventos $S$ es
    una secuencia de números positivos $p_1,\: p_2,\: \dots\, p_n$ que
    sumados dan 1. Donde $p_i$ se interpreta como la probabilidad de
    que el evento $s_i$ ocurra%
  }
}

\newglossaryentry{gl:equiprobable}
{
  name = equiprobable,
  plural = equiprobables,
  description = {
    Se dice que un conjunto de eventos es equiprobable cuando cada uno tiene
    la misma probabilidad de ocurrencia%
  }
}

\newglossaryentry{gl:funcion_booleana}
{
  name = función booleana,
  plural = funciones booleanas,
  description = {
    Son las funciones que mapean $f$ a un valor del conjunto booleano ${0,1}$ o
    \textit{verdadero} y \textit{falso}.
    Formalmente, se define como $f: B^n \rightarrow B$, donde $B={0,1}$
    y $n$ un entero no negativo que corresponde al número de argumentos,
    o variables, que necesita la función%
  },
  see={gl:funcion}
}

\newglossaryentry{gl:cifrado_caminata_ciclica}
{
  name = cifrado de caminata cíclica,
  plural = cifrados de caminata cíclica,
  description = {
    (\textit{Cycle-walking cipher}) Método diseñado para cifrar mensajes de un
    espacio $M$ utilizando un algoritmo de cifrado por bloques que actúa en un
    espacio $M^\prime \supset M$ y obtener textos cifrados que están en $M$ al
    cifrar iterativamente hasta que el mensaje cifrado se encuentra en el
    dominio deseado%
  }
}

\newglossaryentry{gl:autenticacion_multifactor}
{
  name = autenticación multifactor,
  description = {
    (\textit{Multi-factor authentication}) Método de autenticación que requiere
    al menos dos métodos (independientes entre sí) para identificar al usuario%
  }
}

\newglossaryentry{gl:autenticacion_mutua}
{
  name = autenticación mutua,
  description = {
    (\textit{Mutual authentication}) Autenticación en la cual cada una de
    las partes identifica a la otra%
  }
}

% No se me ocurre cómo traducir esto... es como ¿valor de un solo uso?
\newglossaryentry{gl:nonce}
{
  name = nonce,
  description = {
    Valor que varía con el tiempo y es improbable que se repita. Por ejemplo,
    puede ser un valor aleatorio generado para cada uso, una etiqueta de tiempo,
    un número de secuencia o una combinación de los tres%
  }
}

\newglossaryentry{gl:semilla}
{
  name = semilla,
  plural = semillas,
  description = {
    (\textit{Seed}) Cadena de bits que es utilizada como entrada para los
    los mecanismos \gls{gl:drbg}. Determina una parte del estado interno
    del \gls{gl:drbg}%
  }
}

\newglossaryentry{gl:material_de_llaves}
{
  name = material de llaves,
  description = {(\textit{keying material})
    Conjunto de llaves criptográficas sin un formato específico%
  }
}

\newglossaryentry{gl:tiempo_polinomial}
{
  name = tiempo polinomial,
  description = {
    Se dice que una función computable o algoritmo es de \textit{tiempo
    polinomial} cuando su complejidad está, en el peor de los casos, acotada
    por arriba por un polinomio sobre el tamaño de sus variables de entrada.
    Si se toma a $f$ como una función computable, esta es de \textit{tiempo
    polinomial} si $f\: \in\: O(n^k)$ donde $k \ge 1$%
  }
}

\newglossaryentry{gl:distribucion_uniforme}
{
  name = distribución uniforme,
  description = {
    Distribución de probabilidad en la que todos los elementos tienen la misma
    probabilidad ocurrencia%
  },
  see={gl:distribucion_probabilidad}
}

\longnewglossaryentry{gl:computacionalmente_indistinguible}
{
  name = computacionalmente indistinguible,
  plural = computacionalmente indistinguibles,
  see={gl:funcion,gl:funcion_despreciable,gl:conjunto_de_distribucion,gl:circuito_booleano}
}
{%
  (\textit{Computational indistinguishability}) Para un $ A $ tomado de
  algún conjunto de distribución y un circuito booleano $ C $ (con las
  suficientes entradas), $ p_C^A $ es la probabilidad de que la salida
  del circuito booleano $ C $ sea 1 para una entrada de $ A $. Se dice
  que los conjuntos de distribución $ \{ A_k \} $ y $ \{ B_k \} $ son
  \textit{computacionalmente indistinguibles} si para cualquier familia
  de circuitos de tamaño polinomial $ C = \{ C_k \} $, la función
  $ e(k) = | p_{A_k}^{C_k} - p_{B_k}^{C_k} |$ es
  despreciable~\cite{DBLP:conf/stoc/BeaverMR90}.
  \\*
  Otra forma de expresarlo, en un contexto más criptográfico, es como la
  incapacidad de un adversario de distingir si la salida de una primitiva
  criptográfica es una permutación aleatoria o una permutación pseudoaleatoria:
  una permutación $ P $ es segura (en términos de un ataque de texto cifrado
  conocido) cuando es \textit{computacionalmente indistingible} de una
  permutación aleatoria.%
}

\newglossaryentry{gl:funcion_despreciable}
{
  name = función despreciable,
  description = {
    (\textit{Negligible function}) Una función $ e: N \rightarrow R $ es
    \textit{despreciable} si, para todos los enteros positivos $ c $,
    existe un entero $ N_c $ tal que para todo $ x \ge N_c $,
    $ |e(x)| \le \frac{1}{x^c} $. Esto significa que $ e(x) $ se desvanece
    más rápido que el inverso de cualquier
    polinomio~\cite{DBLP:conf/stoc/BeaverMR90}%
  },
  see={gl:funcion}
}

\newglossaryentry{gl:conjunto_de_distribucion}
{
  name = conjunto de distribución,
  description = {
    (\textit{Distribution ensemble}) Un \textit{conjunto de distribución}
    $ \{ A_k \} $ es una familia de medidas de probabilidad en $ \{0, 1\}^* $
    para la cual hay un polinomio $ q $ tal que las únicas cadenas de longitud
    mayor a $ q(k) $ tienen una probabilidad distinta de cero en
    $ \{ A_k \} $~\cite{DBLP:conf/stoc/BeaverMR90}%
  }
}

\newglossaryentry{gl:circuito_booleano}
{
  name = circuito booleano,
  description = {
    (\textit{Boolean circuit}) Modelo matemático definido en términos de
    compuertas lógicas digitales (AND, OR, NOT, etc.)%
  }
}

\newglossaryentry{gl:cifrado_con_prefijo}
{
  name = cifrado con prefijo,
  plural = cifrados con prefijo,
  description = {
    Técnica de ordenamiento pseudoaleatorio que consite en seguir el orden dado
    por el texto cifrado de todos los elementos a ordenar%
  }
}

\newglossaryentry{gl:libreta_de_un_solo_uso}
{
  name = libreta de un solo uso,
  plural = libretas de un solo uso,
  description = {
    (\textit{One-time pad}) Algorimo de cifrado donde el texto en claro se combina
    con una llave secreta que es, al menos, de la longitud del mensaje. Si es
    utillizado e implementado correctamente, este algoritmo es indescifrable%
  }
}

\newglossaryentry{gl:registro_de_desplazamiento}
{
  name = registro de desplazamiento,
  description = {
    Es un arreglo de \textit{flip-flops} que se encarga de desplazar la
    información contenida en su estado actual, teniendo una entrada de un
    bit, para reemplazar el valor del primer \textit{flip-flop} del arreglo%
  }
}

\newglossaryentry{gl:registro_de_desplazamiento_con_retroalimentacion_lineal}
{
  name = registro de desplazamiento con retroalimentación lineal,
  description = {
    (\textit{Linear-feedback shift register (LFSR)}) Es un registro de
    desplazamiento en el que su bit de entrada es una función lineal de su
    estado anterior%
  },
  see={gl:registro_de_desplazamiento}
}

\newglossaryentry{gl:acoplamiento}
{
  name = acoplamiento,
  description = {
    (\textit{Coupling}) Medida de la fuerza de asociación establecida por
    la conexión de dos módulos dentro de un sistema. Un acoplamiento
    fuerte complica a un sistema, dado que es más difícil de entender,
    cambiar o corregir. La complejidad de un programa se puede reducir
    buscando el menor acoplamiento posible entre módulos
   ~\cite{DBLP:books/daglib/0019905}%
  }
}

\newglossaryentry{gl:prueba_de_unidad}
{
  name = prueba de unidad,
  plural = pruebas de unidades,
  description = {
    (\textit{Unit testing}) Proceso para probar componentes más simples de un
    programa de forma individual (funciones u objetos aislados)
   ~\cite{DBLP:books/lib/Sommerville07}%
  }
}

\newglossaryentry{gl:prueba_de_componente}
{
  name = prueba de componente,
  plural = pruebas de componentes,
  description = {
    (\textit{Component testing}) Pruebas de componentes hechos a partir de
    objetos que interactúan entre sí
   ~\cite{DBLP:books/lib/Sommerville07}%
  }
}

\newglossaryentry{gl:criptomoneda}
{
  name = criptomoneda,
  description = {
    (\textit{Cryptocurrency}) Moneda digital que funciona con herramientas
    criptográficas modernas. La primera y más exitosa a la fecha, llamada
    \textit{bitcoin}, fue propuesta en~\cite{bitcoin} por el epónimo
    \textit{Satoshi Nakamoto}. La característica más importante de las
    \textit{criptomonedas} es que no están respaldadas por alguna autoridad
    central~\cite{DBLP:journals/iacr/BhattacherjeeS17}%
  }
}

\newglossaryentry{gl:combinacion_lineal}
{
  name = combinación lineal,
  plural = combinaciones lineales,
  description = {
    En álgebra lineal, una combinación lineal es la expresión resultate
    de la suma de $n$ vectores $v_i$ multiplicados por $n$ escalares $c_i$
    de la siguiente forma: $\sum_{i=1}^{n}c_i v_i$%
  }
}

\newglossaryentry{gl:dependencia_lineal}
{
  name = dependencia lineal,
  plural = dependencias lineales,
  description = {
    Un conjunto de vectores es linealmente independiente sin ninguno de
    ellos puede ser representado por una combinación lineal de los vectores
    restantes.
    Si los vectores no son linealmente independientes, se dice que existen
    dependencias lineales%
  },
  see={gl:combinacion_lineal}
}

\newglossaryentry{gl:expresion_regular}
{
  name = expresión regular,
  plural = expresiones regulares,
  description = {
    Forma algebraica de definir patrones. El conjunto de patrones que puede ser
    expresado mediante las expresiones regulares es exactamente el mismo que el
    conjunto de patrones que puede ser descrito por las transiciones de una
    máquina de estados finitos~\cite{DBLP:books/crc/AU1992}%
  },
  see={gl:maquina_de_estados_finitos}
}

\newglossaryentry{gl:maquina_de_estados_finitos}
{
  name = máquina de estados finitos,
  plural = máquinas de estados finitos,
  description = {
   Modelo idealizado de una máquina que opera en tiempos discretos. La operación
   completa de la máquina se define por un conjunto finito de estados junto con
   las reglas de cambios entre estados~\cite{minsky}%
  }
}

\glsaddall
