%
% Proyecto eSTREAM, capítulo de antecedentes
% Proyecto Lovelace.
%

\subsection{El proyecto eSTREAM}

La información aquí expuesta puede ser consultada a mayor detalle en
\cite{resultados_estream_1, resultados_estream_2} y \cite{estream_portafolio}.

El proyecto eSTREAM fue un esfuerzo de la comunidad \gls{gl:ecrypt} para
promover el diseño de cifrados de flujo eficientes y compactos. Como resultado,
se publicó un portafolio en abril de 2008, el cual ha estado bajo continuas
actualizaciones desde entonces; actualmente cuenta con siete algoritmos
(tabla \ref{portafolio_estream}).

El portafolio se divide en dos perfiles: el primero contiene algoritmos
adecuados para aplicaciones de software con requerimientos de rapidez de
procesamiento muy altos; el segundo se enfoca en aplicaciones de hardware con
pocos recursos disponibles.

\begin{table}
  \centering
  \begin{tabular}{| m{0.85in} | m{0.85in} |}
    \hline
    \textbf{Perfil 1} & \textbf{Perfil 2} \\ [0.5ex]
    \hline
    \makecell{HC-128 \\ Rabbit \\ Salsa20/12 \\ Sosemanuk} &
    \makecell{Grain v1 \\ MICKEY v2 \\ Trivium} \\
    \hline
  \end{tabular}
  \caption{Finalistas del proyecto eSTREAM}
  \label{portafolio_estream}
\end{table}

El proyecto se inició después de que Adi Shamir se preguntara si realmente
había necesidad de los cifrados de flujo, en una conferencia de
\gls{gl:rsa} en 2004. El principal argumento a favor fue que, para la gran
mayoría de los casos, el uso de \gls{gl:aes} (sección \ref{sec:aes}) con
una configuración de flujo es una solución adecuada. Este último punto de vista
iba generalmente acompañado de la creencia de que era imposible diseñar cifrados
de flujo seguros (un proyecto anterior similar, \gls{gl:nessie}, terminó
sin resultados después de todos los criptoanálisis hechos). Por otra parte,
como argumentos en contra, Shamir identificó dos áreas en las que los cifrados
de flujo ofrecen ventajas respecto a los cifrados por bloques:

\begin{enumerate}
  \item Cuando se requieren tiempos de procesamiento excepcionalmente rápidos
    (perfil 1).
  \item Cuando los recursos disponibles son muy pocos (perfil 2).
\end{enumerate}

Las palabras de Shamir fueron ampliamente difundidas, y más tarde en ese mismo
año, ECRYPT lanzó el proyecto eSTREAM, cuyo principal objetivo fue ampliar el
conocimiento sobre el análisis y el diseño de cifrados de flujo. Después de un
periodo de estudio, se lanzó una convocatoria que generó un interés
considerable: antes del 29 de abril de 2005 (la fecha límite) se recibieron
34 propuestas; algunas de las cuales intentaban cumplir con los dos perfiles a
la vez (lamentablemente no sobrevivieron mucho tiempo).
