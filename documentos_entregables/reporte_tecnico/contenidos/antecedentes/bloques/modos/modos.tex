%
% Sección de modos de operación, capítulo de antecedentes.
% Proyecto Lovelace.
%

\subsection{Modos de operación}
\label{sec:modos}

La información que aquí se presenta se puede consultar a mayor detalle en
\cite{modos_de_operacion} y \cite{menezes}.

Por sí solos, los cifrados por bloques solamente permiten el cifrado y
descifrado de bloques de información de tamaño fijo; donde, en la mayoría de
los casos, los bloques son de menos de 256 bits, lo cual es equivalente a
alrededor de 8 caracteres. Es fácil darse cuenta de que esta restricción no es
ningún tema menor: en la gran mayoría de las aplicaciones, la longitud de lo
que se quiere ocultar es arbitraria.

Los \glspl{gl:modo_de_operacion} permiten extender la funcionalidad de los
cifrados por bloques para poder aplicarlos a información de tamaño irrestricto:
reciben el texto original (de tamaño arbitrario) y lo cifran, ocupando en el
proceso un cifrado por bloques.

Un primer enfoque (y quizás el más intuitivo) es partir el mensaje original
en bloques del tamaño requerido y después aplicar el algoritmo a cada bloque
por separado; en caso de que la longitud del mensaje no sea múltiplo del
tamaño de bloque, se puede agregar información extra al último bloque para
completar el tamaño requerido. Este es, de hecho, el primero de los modos que
se presentan a continuación, el \gls{gl:ecb}; su uso no es
recomendado, pues es muy inseguro cuando el mensaje original es simétrico a
nivel de bloque. También se enlistan otros tres modos, los cuales junto con
\gls{gl:ecb}, son los más comunes.

% TODO: indagar un poco más en la inseguridad de ECB (dentro de su propia
% scción).

\subimport{/}{ecb}
\subimport{/}{cbc}
\subimport{/}{cfb}
\subimport{/}{ofb}
\subimport{/}{ctr}
