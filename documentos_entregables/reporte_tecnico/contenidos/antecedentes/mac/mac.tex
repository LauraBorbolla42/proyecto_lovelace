%
% Sección de autenticación de mensajes, capítulo de antecedentes.
% Proyecto Lovelace.
%

\section{Códigos de Autenticación de Mensaje (MAC)}

La información presentada a continuación puede consultarse con más profundidad
en las siguientes referencias
\cite{DBLP:series/isc/DelfsK07, menezes, mac_patel}.

Las funciones hash con llave cuyo propósito específico es la
\gls{gl:autenticacion_origen} y garantizar la \gls{gl:integridad_datos} del
mensaje son llamadas \gls{gl:mac}. Estas funciones tienen como entrada
una llave secreta $k$ y un mensaje de longitud arbitraria y dan como resultado
un mensaje de longitud $n$.

\begin{equation}
  \label{funcion_hash_mac}
  h_k: \{0, 1\}^* \longrightarrow \{0,1\}^n
\end{equation}

Las funciones \gls{gl:mac} son la técnica simétrica
estándar utilizada tanto para la autenticación como para la protección de la
integridad de los mensajes. Dependen de unas llaves secretas que son
compartidas entre las partes que se van a comunicar; cada una de las
partes puede producir el \gls{gl:mac} correspondiente para un mensaje
dado. Como se explica a continuación, los \gls{gl:mac} pueden ser
obtenidos mediante cifradores de bloque, cifradores de flujo o de funciones
hash criptográficas.

El algoritmo \gls{gl:mac} más usado basado en un cifrador de bloque
utiliza el modo de operación \gls{gl:cbc} (véase sección \ref{sec:cbc}).
Cuando \gls{gl:des} es utilizado como el cifrado de bloque $E$, el tamaño
de bloque es de 64 bits y la llave \gls{gl:mac} es de 56 bits.

Otra manera de construir \gls{gl:mac} es mediante un algoritmo
de \gls{gl:mdc} que incluya una llave secreta $k$ como parte de la
entrada. Un ejemplo de esto es el algoritmo MD5-MAC; donde la función de
compresión depende de la llave secreta $k$, que interviene en todas las
iteraciones.

El algoritmo \gls{gl:maa} fue diseñado en 1938 específicamente para
obtener \gls{gl:mac} en máquinas de 32 bits. El tiempo de ejecución
es directamente proporcional a la longitud del mensaje y alrededor de cuatro
veces más largo que el \gls{gl:md4}
