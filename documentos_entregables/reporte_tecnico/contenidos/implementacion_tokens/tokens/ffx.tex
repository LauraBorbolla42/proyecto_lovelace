%
% Explicación de código de FFX, reporte técnico.
%
% Proyecto Lovelace.
%

\subsection{Módulo de \texorpdfstring{\acrshort{gl:ffx}}{FFX}}

La operación \gls{gl:ffx} A10 se describió en la sección~\ref{sec:ffx}. La
parte medular de su operación se encuentra en la función de ronda que ocupa
la red Feistel interna (pseudocódigo~\ref{pc:ffx_a10}); la implementación de
esta función se muestra en el código fuente~\ref{codigo:ffx_ronda}.
\gls{gl:ffx} A10 funciona con una red Feistel alternante (sección
\ref{sec:red_feistel_alternante}), por lo que también es importante la
implementación de este proceso; el código fuente
\ref{codigo:red_feistel_operar} muestra la operación de cifrado y
\ref{codigo:red_feistel_deoperar} muestra el descifrado.

\codigoFuente[codigo:ffx_ronda]{158}{206}{c++}{%
  implementaciones/ffx/cabeceras/ronda_ffx_a10.hh}{%
  Función de ronda de \gls{gl:ffx}}

Como se mostró en el diseño del programa (capítulo
\ref{sec:analisis_y_disenio}), la función de ronda implementa la
interfaz de una función con inverso simétrico (diagrama~\ref{clases_ffx}), por
lo que la función \texttt{operar} correspone a la definida por la interfaz.
Otro aspecto importante del código~\ref{codigo:ffx_ronda} es la primitiva
criptográfica usada: una instancia de \gls{gl:aes} \gls{gl:cbc} \gls{gl:mac} de
Crypto++. La operación de esa función se puede dividir en tres partes
claramente distingibles: el armado de la entrada a la primitiva (hasta la línea
160); la operación de la primitiva (hasta la línea 166); y por último, el
formateo de la salida para la preservación del formato (el resto de la función).

\codigoFuente[codigo:red_feistel_operar]{146}{176}{c++}{%
  implementaciones/redes_feistel/cabeceras/red_feistel_alternante.hh}{%
  Cifrado con red Feistel alternante}

\codigoFuente[codigo:red_feistel_deoperar]{178}{209}{c++}{%
  implementaciones/redes_feistel/cabeceras/red_feistel_alternante.hh}{%
  Descifrado con red Feistel alternante}

Una red Feistel alternante es una red Feistel, la cuál a su vez implementa a
una función con inverso; es por esto que los métodos mostrados en los códigos
fuente~\ref{codigo:red_feistel_operar} y~\ref{codigo:red_feistel_deoperar} se
llaman \texttt{operar} y \texttt{deoperar}, respectivamente. La red feistel
alternante tiene dos funciones de  ronda: una para las pares y otra para las
impares. Estas son clases que  tengan la forma de una función con inverso: la
clase de la función de ronda de \gls{gl:ffx} A10, presentada en
\ref{codigo:ffx_ronda}, implementa a una función con inverso simétrico (la cual
a su vez implementa a una función con inverso). De esta forma se puede
instanciar a \gls{gl:ffx} manteniendo totalmente desacoplado al código de las
redes Feistel (ver \gls{gl:acoplamiento}).

%\FloatBarrier
