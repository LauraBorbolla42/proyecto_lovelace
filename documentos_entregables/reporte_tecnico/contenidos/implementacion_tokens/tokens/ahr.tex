%
% Explicación de código de AHR, reporte técnico.
%
% Proyecto Lovelace.
%

\subsection{Módulo de AHR}

En esta sección se explica, de manera general, la implementación del algoritmo
AHR, descrito en la sección~\ref{sec:ahr}.

Partiendo de lo general hacia lo particular, en la sección de código
\ref{codigo:ahr_tok} se ve el método \textit{principal}; primero, dependiendo de
la longitud del \gls{gl:pan} calcula los límites entre los cuales el
\gls{gl:token} será válido. Posteriormente, mediante el método
\textit{crearBloqueT}, descrito en~\ref{codigo:ahr_tok1}, se concatena a la
salida de la función SHA256 (que recibe como entrada el \gls{gl:iin}), la
representación binaria de la entrada $X$, que es el número de cuenta que se
va a tokenizar. Una vez que se tiene el bloque $T$, se cifra mediante
\gls{gl:aes} con una llave de 256 bits y se obtiene la representación decimal
de los últimos bits del bloque cifrado; el resultado de esta operación es
el \gls{gl:token} candidato, pues aún falta revisar si el \gls{gl:token} se
encuentra dentro de los límites obtenidos al principio; en caso de que no
cumpla, se usa \gls{gl:cifrado_caminata_ciclica} hasta obtener un \gls{gl:token}
válido. Finalmente, se revisa en la base de datos si el \gls{gl:token} creado
existe (este algoritmo permite generar varios \glspl{gl:token} para un mismo
\gls{gl:pan}): si no existe, guarda la relación \gls{gl:pan}-\gls{gl:token} en
la base de datos; si sí, aumenta en uno el \gls{gl:iin} y vuelve a ejectuar
el algoritmo desde el inicio. Respecto a la detokenización, como se observa en
\ref{codigo:ahr_tok2}, se realiza directamente una búsqueda en la base de datos,
en caso de no encontrar el \gls{gl:token} ingresado, lanza un error indicando
que no existe en la base de datos.

\codigoFuente[codigo:ahr_tok]{284}{328}{c++}{%
  implementaciones/ahr/ahr.cpp}{%
  Tokenización mediante AHR.}

\codigoFuente[codigo:ahr_tok1]{193}{223}{c++}{%
  implementaciones/ahr/ahr.cpp}{%
  Primer paso para la tokenización con AHR.}

\codigoFuente[codigo:ahr_tok2]{360}{372}{c++}{%
  implementaciones/ahr/ahr.cpp}{%
  Primer paso para la detokenización con AHR.}
