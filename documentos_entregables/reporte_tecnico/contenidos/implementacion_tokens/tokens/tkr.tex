%
% Explicación de código de TKR, reporte técnico.
%
% Proyecto Lovelace.
%

\subsection{Módulo de TKR}

La descripción de los algoritmos que componen a este método se hizo en la
sección~\ref{sec:tkr}. En la sección~\ref{sec:tkr_disenio} se mostró la vista
estática que tiene este módulo. En esta sección se muestran los aspectos más
importantes de su implementación.

\codigoFuente[codigo:tkr_cifrado]{41}{69}{c++}{%
  implementaciones/tkr/tkr.cpp}{%
  Proceso de cifrado de TKR.}

\codigoFuente[codigo:tkr_descifrado]{71}{86}{c++}{%
  implementaciones/tkr/tkr.cpp}{%
  Proceso de descifrado de TKR.}

Los códigos fuentes~\ref{codigo:tkr_cifrado} y~\ref{codigo:tkr_descifrado}
mustran los procesos de cifrado y descifrado, respectivamente. Estas operaciones
están descritas por los pseudocódigos~\ref{tkr2_cifrado} y
\ref{tkr2_descifrado}. La operación de cifrado depende de una función
psudoaleatorioa; dentro del contexto de la clase TKR, esta es cualquier
otra clase que implemente a una función que recibe enteros y regresa
arreglos de dígitos. La operación de descifrado es simplemente una consulta
en la base de datos; en caso de una búsqueda infructuosa, se lanza una
excepción.

Como se muestra en el diagrama~\ref{clases_tkr}, TKR implementa la interfaz
de un algoritmo tokenizador irreversible (que a su vez implementa a un
algoritmo tokenizador), por lo que los métodos mostrados en los códigos
fuente~\ref{codigo:tkr_cifrado} y~\ref{codigo:tkr_descifrado} son
los definidos por la interfaz. Al tratarse de un método irreversible,
ambos procesos (tokenización y detokenización) reciben el \gls{gl:pan} o el
\gls{gl:token} completos, según sea el caso.

\codigoFuente[codigo:funcion_rn]{53}{75}{c++}{%
  implementaciones/tkr/funcion_rn.cpp}{%
  Función RN.}

La función del código fuente~\ref{codigo:funcion_rn} es la implementación del
pseudocódigo descrito en~\ref{tkr2_rn}. Esta clase implementa la
interfaz de una función que recibe enteros y regresa arreglos de dígitos
(siguiendo el modelo de débil acomplamiento dado por las interfaces de
las funciones). Básicamente, se trata de la transformación de
los bytes aleatorios dados por la función interna en el número de
dígitos pedidos. En el contexto de esta clase, la función interna tiene
el mismo aspecto: una función que recibe enteros y regresa arreglos de bytes.

\codigoFuente[codigo:pseudoaleatorio_aes]{51}{82}{c++}{%
  implementaciones/tkr/pseudoaleatorio_aes.cpp}{%
  Generación de bytes pseudoaleatorios basado en \acrshort{gl:aes}.}

Por último, el código fuente~\ref{codigo:pseudoaleatorio_aes} muestra
cómo se puede ocupar un cifrado de bloques (\gls{gl:aes} de 128 bits, en
este caso) para producir arreglos de bytes con aspecto aleatorio. La
operación de esta función es análoga al modo de operación \gls{gl:ctr}
(sección~\ref{sec:ctr})
