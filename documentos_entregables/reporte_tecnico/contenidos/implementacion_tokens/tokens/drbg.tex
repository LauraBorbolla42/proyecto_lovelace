%
% Explicación de código de DRBG, reporte técnico.
%
% Proyecto Lovelace.
%

\subsection{Módulo de DRBG}
\label{sec:implementacion_drbg}

En la sección~\ref{sec:generadores_pseudoaleatorios} se resume el estándar del
\gls{gl:nist} para la  creación de un \gls{gl:drbg}. En~\ref{sec:drbg_lista} se
muestran dos posibles implementaciones concretas: una basada en una función
hash y la  otra basada en un cifrado por bloques. En~\ref{sec:drbg_disenio} se
explica la distribución de las clases creada para este módulo. En esta sección
se muestran los apsectos más significativos de las implementaciones.

\codigoFuente[codigo:drbg]{66}{90}{c++}{%
  implementaciones/drbg/drbg.cpp}{%
  Función pública de generadores pseudoaleatorios.}

Como se muestra en el diagrama~\ref{clases_drbg}, un generador solamente
expone una función pública: \texttt{operar} para la generación de bytes; el
nombre viene dado por el contrato con la interfaz de una función que
recibe enteros y regresa arreglos de bytes. El código fuente~\ref{codigo:drbg}
muestra esta implementación, la cuál depende de \texttt{generarBytes}, que
dentro del contexto de un \gls{gl:drbg} genérico, es abstracta.

De acuerdo a lo explicado en la sección~\ref{sec:intel}, el comportamiento por
defecto del constructor de un DRBG es determinar, en tiempo de ejecución, si el
procesador actual soporta la instrucción de ensamblador RD-SEED. En caso
positivo, la fuente de entropía es una instancia de la clase de aleatoriedad
por hardware. En caso negativo, la entropía es una instancia de la clase de
aleatoriedad trivial. También existe la posibilidad de que el constructor
reciba una fuente de aleatoriedad creada con anterioridad; en este caso, el
DRBG no es responsable de la gestión de la memoria. Esta segunda
característica permite instanciar varios generadores con una misma fuente de
aleatoriedad.

\codigoFuente[codigo:hash_drbg_uno]{117}{135}{c++}{%
  implementaciones/drbg/hash_drbg.cpp}{%
  Función de generación de bytes de hash \gls{gl:drbg}.}

\codigoFuente[codigo:hash_drbg_dos]{166}{186}{c++}{%
  implementaciones/drbg/hash_drbg.cpp}{%
  Función de generación interna de hash \gls{gl:drbg}.}

En los códigos fuente~\ref{codigo:hash_drbg_uno} y~\ref{codigo:hash_drbg_dos}
se muestran las dos operaciones más importantes del generador basado en
funciones hash: la generación de bytes y una función de generación interna.
Como se puede ver en el primero de los códigos, la función de generación de
bytes depende de la función interna: esta genera los bytes pedidos y después,
la función de generación de bytes, se encarga de actualizar el estado interno.

\codigoFuente[codigo:hash_ctr_uno]{82}{101}{c++}{%
  implementaciones/drbg/ctr_drbg.cpp}{%
  Función de generación de bytes de CTR \gls{gl:drbg}.}

\codigoFuente[codigo:hash_ctr_dos]{103}{123}{c++}{%
  implementaciones/drbg/ctr_drbg.cpp}{%
  Función de actualización de estado de CTR \gls{gl:drbg}.}

En los códigos fuente~\ref{codigo:hash_ctr_uno} y~\ref{codigo:hash_ctr_dos}
se muestran las dos operaciones más importantes del generador basado en un
cifrado por bloques. La primera se trata de la función de generación de bytes;
como se puede observar, su operación es idéntica al modo de operación de
contador (sección~\ref{sec:ctr}). La segunda función actualiza el estado
cada vez que hay un cambio de semilla.
