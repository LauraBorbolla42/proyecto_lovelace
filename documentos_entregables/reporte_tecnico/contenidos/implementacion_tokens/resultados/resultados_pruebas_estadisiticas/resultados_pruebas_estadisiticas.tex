%
% Capítulo de Implementaciones,
% Resultados de las pruebas estadisticas para los DRBG.
% Reporte técnico.
% Proyecto Lovelace.
%

\subsection{Resultados de las pruebas estadísticas a los %
  \texorpdfstring{\acrshort{gl:drbg}}{DRBG}}
\label{sec:resultados_pruebas_estadisticas_drgb}

En esta sección se pretende demostrar la aleatoriedad de los \gls{gl:drbg}
que fueron implementados, esto por medio del conjunto de pruebas del
\gls{gl:nist}, pruebas de las que se habla más a fondo en el apéndice
\ref{sec:pruebas_para_generadores}.

En las siguientes 6 tablas
(\ref{resultados_pruebas_drbg_bloque_112_128},
\ref{resultados_pruebas_drbg_bloque_192_256},
\ref{resultados_pruebas_drbg_sha256_112_128},
\ref{resultados_pruebas_drbg_sha256_192_256},
\ref{resultados_pruebas_drbg_sha512_112_128} y
\ref{resultados_pruebas_drbg_sha512_192_256})
se observan los resultados obtenidos para las 15 pruebas estadísticas
descritas en~\cite{nist_pruebas} para 3 tipos distintos de \gls{gl:drbg}
(uno basado en \gls{gl:aes} y otros dos basados en \gls{gl:sha}) en cada
uno de los niveles de seguridad que se tienen (112, 128, 192 y 256).

Para poder comprender la información más claramente, es necesario establecer
los siguientes puntos:

\begin{itemize}

  \item Las pruebas se realizaron con 10 millones de bits obtenidos por cada
    tipo de \gls{gl:drbg}.

  \item Cada prueba realizada consta de 20 ejecuciones, y cada ejecución usa
    medio millón de bits.

  \item Se puede resumir que el valor de P va de $0$ a $1$, y que mientras
    más grande sea este valor, el \gls{gl:drbg} probado es más aleatorio.

  \item Las pruebas de sumas acumulativas, de coincidencia sin superposición,
    de entropía aproximada, de excursiones aleatorias, y su variante constan
    de varias subpruebas.

  \item En las tablas, en las columnas de \textit{proporción}, se pone el
    número de pruebas pasadas con respecto al número de pruebas (subpruebas
    si existen) que se hicieron en total.

  \item Posterior a las columnas de \textit{proporción} se pone un \ding{51}
    en caso de que el \gls{gl:drbg} aprobara de forma general la prueba o un
    \ding{53} en el caso contrario.

\end{itemize}

% Tablas con los resultados.
% =============================================================================

\begin{longtable}{| m{2.4in} | C{0.65in} | C{0.85in} |
C{0.15in} | C{0.65in} | C{0.85in} | C{0.15in} |}

  \hline
  \multicolumn{7}{|c|}{\textbf{Pruebas estadísticas para el \gls{gl:drbg}
  basado en cifrados por bloque (\gls{gl:aes})}} \\
  \hline
  \textbf{Nivel de seguridad}        &
  \multicolumn{3}{c|}{\textbf{112}} &
  \multicolumn{3}{c|}{\textbf{128}} \\
  \hline
  \textbf{Prueba} &
  Valor P         &
  Proporción    & &
  Valor P         &
  Proporción    & \\
  \hline
  \endfirsthead

  \hline
  \multicolumn{7}{|c|}{\textit{Continuación}}\\
  \hline
  \textbf{Nivel de seguridad}        &
  \multicolumn{3}{c|}{\textbf{112}} &
  \multicolumn{3}{c|}{\textbf{128}} \\
  \hline
  \textbf{Prueba} &
  Valor P         &
  Proporción    & &
  Valor P         &
  Proporción    & \\
  \hline
  \endhead

  \multicolumn{7}{|c|}{\textit{Continúa en siguiente página}}\\
  \hline
  \endfoot

  \endlastfoot

  De frecuencia &
  0.991468 &   19/20   & \ding{51} &
  0.437274 &   20/20   & \ding{51} \\\hline

  De frecuencia en un bloque &
  0.017912 &   20/20   & \ding{51} &
  0.048716 &   20/20   & \ding{51} \\\hline

  De sumas acumulativas &
  0.872861 &   38/40   & \ding{51} &
  0.356491 &   40/40   & \ding{51} \\\hline

  De carreras &
  0.964295 &   20/20   & \ding{51} &
  0.350485 &   20/20   & \ding{51} \\\hline

  De la carrera más larga en un bloque &
  0.637119 &   20/20   & \ding{51} &
  0.213309 &   20/20   & \ding{51} \\\hline

  Del rango de matriz binaria &
  0.964295 &   20/20   & \ding{51} &
  0.911413 &   20/20   & \ding{51} \\\hline

  Espectral &
  0.637119 &   20/20   & \ding{51} &
  0.213309 &   20/20   & \ding{51} \\\hline

  De coincidencia sin superposición &
  0.475566 & 2925/2960 & \ding{51} &
  0.513657 & 2930/2960 & \ding{51} \\\hline

  De coincidencia con superposición &
  0.964295 &   20/20   & \ding{51} &
  0.637119 &   20/20   & \ding{51} \\\hline

  Estadística universal de Maurer &
  0.066882 &   20/20   & \ding{51} &
  0.834308 &   19/20   & \ding{51} \\\hline

  De entropía aproximada &
  0.739918 &   20/20   & \ding{51} &
  0.834308 &   20/20   & \ding{51} \\\hline

  De excursiones aleatorias &
  -------- &   70/72   & \ding{51} &
  0.485432 &   96/96   & \ding{51} \\\hline

  Variante de excursiones aleatorias &
  -------- &  159/162  & \ding{51} &
  0.283322 &  215/216  & \ding{51} \\\hline

  Serial &
  0.588596 &   40/40   & \ding{51} &
  0.577037 &   40/40   & \ding{51} \\\hline

  De complejidad lineal &
  0.122325 &   20/20   & \ding{51} &
  0.090936 &   20/20   & \ding{51} \\\hline

  \caption{Resultado de las pruebas estadísticas del \gls{gl:drbg} basado
  en cifrados por bloque (\gls{gl:aes}) para los niveles de seguridad de
  112 y 128.}
  \label{resultados_pruebas_drbg_bloque_112_128}

\end{longtable}

% =============================================================================

\begin{longtable}{| m{2.4in} | C{0.65in} | C{0.85in} |
C{0.15in} | C{0.65in} | C{0.85in} | C{0.15in} |}

  \hline
  \multicolumn{7}{|c|}{\textbf{Pruebas estadísticas para el \gls{gl:drbg}
  basado en cifrados por bloque (\gls{gl:aes})}} \\
  \hline
  \textbf{Nivel de seguridad}        &
  \multicolumn{3}{c|}{\textbf{192}} &
  \multicolumn{3}{c|}{\textbf{256}} \\
  \hline
  \textbf{Prueba} &
  Valor P         &
  Proporción    & &
  Valor P         &
  Proporción    & \\
  \hline
  \endfirsthead

  \hline
  \multicolumn{7}{|c|}{\textit{Continuación}}\\
  \hline
  \textbf{Nivel de seguridad}        &
  \multicolumn{3}{c|}{\textbf{192}} &
  \multicolumn{3}{c|}{\textbf{256}} \\
  \hline
  \textbf{Prueba} &
  Valor P         &
  Proporción    & &
  Valor P         &
  Proporción    & \\
  \hline
  \endhead

  \multicolumn{7}{|c|}{\textit{Continúa en siguiente página}}\\
  \hline
  \endfoot

  \endlastfoot

  De frecuencia &
  0.964295 &   20/20   & \ding{51} &
  0.162606 &   20/20   & \ding{51} \\\hline

  De frecuencia en un bloque &
  0.739918 &   20/20   & \ding{51} &
  0.162606 &   20/20   & \ding{51} \\\hline

  De sumas acumulativas &
  0.637119 &   40/40   & \ding{51} &
  0.325292 &   40/40   & \ding{51} \\\hline

  De carreras &
  0.834308 &   19/20   & \ding{51} &
  0.534146 &   20/20   & \ding{51} \\\hline

  De la carrera más larga en un bloque &
  0.534146 &   19/20   & \ding{51} &
  0.739918 &   20/20   & \ding{51} \\\hline

  Del rango de matriz binaria &
  0.350485 &   20/20   & \ding{51} &
  0.122325 &   20/20   & \ding{51} \\\hline

  Espectral &
  0.637119 &   20/20   & \ding{51} &
  0.637119 &   19/20   & \ding{51} \\\hline

  De coincidencia sin superposición &
  0.496882 & 2925/2960 & \ding{51} &
  0.495414 & 2923/2960 & \ding{51} \\\hline

  De coincidencia con superposición &
  0.122325 &   20/20   & \ding{51} &
  0.437274 &   19/20   & \ding{51} \\\hline

  Estadística universal de Maurer &
  0.350485 &   20/20   & \ding{51} &
  0.437274 &   20/20   & \ding{51} \\\hline

  De entropía aproximada &
  0.834308 &   20/20   & \ding{51} &
  0.534146 &   20/20   & \ding{51} \\\hline

  De excursiones aleatorias &
  0.413551 &   87/88   & \ding{51} &
  0.393867 &   88/88   & \ding{51} \\\hline

  Variante de excursiones aleatorias &
  0.416464 &  198/198  & \ding{51} &
  0.399044 &  197/198  & \ding{51} \\\hline

  Serial &
  0.404927 &   39/40   & \ding{51} &
  0.506506 &   40/40   & \ding{51} \\\hline

  De complejidad lineal &
  0.350485 &   20/20   & \ding{51} &
  0.534146 &   20/20   & \ding{51} \\\hline

  \caption{Resultado de las pruebas estadísticas del \gls{gl:drbg} basado
  en cifrados por bloque (\gls{gl:aes}) para los niveles de seguridad de
  192 y 256.}
  \label{resultados_pruebas_drbg_bloque_192_256}

\end{longtable}

% =============================================================================

\begin{longtable}{| m{2.4in} | C{0.65in} | C{0.85in} |
C{0.15in} | C{0.65in} | C{0.85in} | C{0.15in} |}

  \hline
  \multicolumn{7}{|c|}{\textbf{Pruebas estadísticas para el \gls{gl:drbg}
  basado en funciones hash (\gls{gl:sha}256)}} \\
  \hline
  \textbf{Nivel de seguridad}        &
  \multicolumn{3}{c|}{\textbf{112}} &
  \multicolumn{3}{c|}{\textbf{128}} \\
  \hline
  \textbf{Prueba} &
  Valor P         &
  Proporción    & &
  Valor P         &
  Proporción    & \\
  \hline
  \endfirsthead

  \hline
  \multicolumn{7}{|c|}{\textit{Continuación}}\\
  \hline
  \textbf{Nivel de seguridad}        &
  \multicolumn{3}{c|}{\textbf{112}} &
  \multicolumn{3}{c|}{\textbf{128}} \\
  \hline
  \textbf{Prueba} &
  Valor P         &
  Proporción    & &
  Valor P         &
  Proporción    & \\
  \hline
  \endhead

  \multicolumn{7}{|c|}{\textit{Continúa en siguiente página}}\\
  \hline
  \endfoot

  \endlastfoot

  De frecuencia &
  0.437274 &   20/20   & \ding{51} &
  0.739918 &   20/20   & \ding{51} \\\hline

  De frecuencia en un bloque &
  0.637119 &   20/20   & \ding{51} &
  0.637119 &   19/20   & \ding{51} \\\hline

  De sumas acumulativas &
  0.328236 &   40/40   & \ding{51} &
  0.264105 &   40/40   & \ding{51} \\\hline

  De carreras &
  0.006196 &   19/20   & \ding{51} &
  0.739918 &   20/20   & \ding{51} \\\hline

  De la carrera más larga en un bloque &
  0.275709 &   20/20   & \ding{51} &
  0.739918 &   20/20   & \ding{51} \\\hline

  Del rango de matriz binaria &
  0.350485 &   20/20   & \ding{51} &
  0.637119 &   19/20   & \ding{51} \\\hline

  Espectral &
  0.350485 &   19/20   & \ding{51} &
  0.008879 &   20/20   & \ding{51} \\\hline

  De coincidencia sin superposición &
  0.521362 & 2921/2960 & \ding{53} &
  0.480227 & 2937/2960 & \ding{51} \\\hline

  De coincidencia con superposición &
  0.637119 &   20/20   & \ding{51} &
  0.911413 &   19/20   & \ding{51} \\\hline

  Estadística universal de Maurer &
  0.637119 &   20/20   & \ding{51} &
  0.048716 &   20/20   & \ding{51} \\\hline

  De entropía aproximada &
  0.275709 &   20/20   & \ding{51} &
  0.437274 &   20/20   & \ding{51} \\\hline

  De excursiones aleatorias &
  0.407729 &  102/104  & \ding{51} &
  -------- &   62/64   & \ding{51} \\\hline

  Variante de excursiones aleatorias &
  0.376856 &  232/234  & \ding{51} &
  -------- &  142/144  & \ding{51} \\\hline

  Serial &
  0.404927 &   40/40   & \ding{51} &
  0.637032 &   40/40   & \ding{51} \\\hline

  De complejidad lineal &
  0.637119 &   20/20   & \ding{51} &
  0.637119 &   20/20   & \ding{51} \\\hline

  \caption{Resultado de las pruebas estadísticas del \gls{gl:drbg} basado en
  funciones hash (\gls{gl:sha}256) para los niveles de seguridad de 112 y 128.}
  \label{resultados_pruebas_drbg_sha256_112_128}

\end{longtable}

En la tabla~\ref{resultados_pruebas_drbg_sha256_112_128} se observa que la
única prueba que no pasó el \gls{gl:drbg} basado en \gls{gl:sha}256 con nivel
de seguridad de 112, fue la de coincidencias sin superposición, situación que
viene de no haberse aprobado más del 90\% de una de las subpruebas.

% =============================================================================

\begin{longtable}{| m{2.4in} | C{0.65in} | C{0.85in} |
C{0.15in} | C{0.65in} | C{0.85in} | C{0.15in} |}

  \hline
  \multicolumn{7}{|c|}{\textbf{Pruebas estadísticas para el \gls{gl:drbg}
  basado en funciones hash (\gls{gl:sha}256)}} \\
  \hline
  \textbf{Nivel de seguridad}        &
  \multicolumn{3}{c|}{\textbf{192}} &
  \multicolumn{3}{c|}{\textbf{256}} \\
  \hline
  \textbf{Prueba} &
  Valor P         &
  Proporción    & &
  Valor P         &
  Proporción    & \\
  \hline
  \endfirsthead

  \hline
  \multicolumn{7}{|c|}{\textit{Continuación}}\\
  \hline
  \textbf{Nivel de seguridad}        &
  \multicolumn{3}{c|}{\textbf{192}} &
  \multicolumn{3}{c|}{\textbf{256}} \\
  \hline
  \textbf{Prueba} &
  Valor P         &
  Proporción    & &
  Valor P         &
  Proporción    & \\
  \hline
  \endhead

  \multicolumn{7}{|c|}{\textit{Continúa en siguiente página}}\\
  \hline
  \endfoot

  \endlastfoot

  De frecuencia &
  0.739918 &   20/20   & \ding{51} &
  0.006196 &   20/20   & \ding{51} \\\hline

  De frecuencia en un bloque &
  0.437274 &   20/20   & \ding{51} &
  0.162606 &   20/20   & \ding{51} \\\hline

  De sumas acumulativas &
  0.899301 &   40/40   & \ding{51} &
  0.688519 &   40/40   & \ding{51} \\\hline

  De carreras &
  0.911413 &   17/20   & \ding{53} &
  0.350485 &   20/20   & \ding{51} \\\hline

  De la carrera más larga en un bloque &
  0.834308 &   20/20   & \ding{51} &
  0.911413 &   20/20   & \ding{51} \\\hline

  Del rango de matriz binaria &
  0.911413 &   20/20   & \ding{51} &
  0.122325 &   20/20   & \ding{51} \\\hline

  Espectral &
  0.035174 &   20/20   & \ding{51} &
  0.437274 &   19/20   & \ding{51} \\\hline

  De coincidencia sin superposición &
  0.504889 & 2937/2960 & \ding{51} &
  0.507982 & 2928/2960 & \ding{51} \\\hline

  De coincidencia con superposición &
  0.739918 &   20/20   & \ding{51} &
  0.834308 &   20/20   & \ding{51} \\\hline

  Estadística universal de Maurer &
  0.090936 &   20/20   & \ding{51} &
  0.637119 &   20/20   & \ding{51} \\\hline

  De entropía aproximada &
  0.739918 &   20/20   & \ding{51} &
  0.637119 &   20/20   & \ding{51} \\\hline

  De excursiones aleatorias &
  0.667135 &   78/80   & \ding{51} &
  -------- &   72/72   & \ding{51} \\\hline

  Variante de excursiones aleatorias &
  0.498556 &  174/180  & \ding{51} &
  -------- &  162/162  & \ding{51} \\\hline

  Serial &
  0.637032 &   39/40   & \ding{51} &
  0.534146 &   38/40   & \ding{51} \\\hline

  De complejidad lineal &
  0.911413 &   20/20   & \ding{51} &
  0.350485 &   19/20   & \ding{51} \\\hline

  \caption{Resultado de las pruebas estadísticas del \gls{gl:drbg} basado en
  funciones hash (\gls{gl:sha}256) para los niveles de seguridad de 192 y 256.}
  \label{resultados_pruebas_drbg_sha256_192_256}

\end{longtable}

En la tabla~\ref{resultados_pruebas_drbg_sha256_192_256} se observa que el
\gls{gl:drbg} basado en \gls{gl:sha}256 con nivel de seguridad de 192, no
aprobó de forma general la prueba de carreras, esto por no llegar a las 18
ejecuciones aprobadas de 20.

% =============================================================================

\begin{longtable}{| m{2.4in} | C{0.65in} | C{0.85in} |
C{0.15in} | C{0.65in} | C{0.85in} | C{0.15in} |}

  \hline
  \multicolumn{7}{|c|}{\textbf{Pruebas estadísticas para el \gls{gl:drbg}
  basado en funciones hash (\gls{gl:sha}512)}} \\
  \hline
  \textbf{Nivel de seguridad}        &
  \multicolumn{3}{c|}{\textbf{112}} &
  \multicolumn{3}{c|}{\textbf{128}} \\
  \hline
  \textbf{Prueba} &
  Valor P         &
  Proporción    & &
  Valor P         &
  Proporción    & \\
  \hline
  \endfirsthead

  \hline
  \multicolumn{7}{|c|}{\textit{Continuación}}\\
  \hline
  \textbf{Nivel de seguridad}        &
  \multicolumn{3}{c|}{\textbf{112}} &
  \multicolumn{3}{c|}{\textbf{128}} \\
  \hline
  \textbf{Prueba} &
  Valor P         &
  Proporción    & &
  Valor P         &
  Proporción    & \\
  \hline
  \endhead

  \multicolumn{7}{|c|}{\textit{Continúa en siguiente página}}\\
  \hline
  \endfoot

  \endlastfoot

  De frecuencia &
  0.213309 &   20/20   & \ding{51} &
  0.637119 &   20/20   & \ding{51} \\\hline

  De frecuencia en un bloque &
  0.834308 &   20/20   & \ding{51} &
  0.437274 &   20/20   & \ding{51} \\\hline

  De sumas acumulativas &
  0.911413 &   40/40   & \ding{51} &
  0.637032 &   40/40   & \ding{51} \\\hline

  De carreras &
  0.534146 &   19/20   & \ding{51} &
  0.739918 &   20/20   & \ding{51} \\\hline

  De la carrera más larga en un bloque &
  0.911413 &   20/20   & \ding{51} &
  0.122325 &   19/20   & \ding{51} \\\hline

  Del rango de matriz binaria &
  0.834308 &   20/20   & \ding{51} &
  0.834308 &   19/20   & \ding{51} \\\hline

  Espectral &
  0.637119 &   20/20   & \ding{51} &
  0.739918 &   19/20   & \ding{51} \\\hline

  De coincidencia sin superposición &
  0.512787 & 2934/2960 & \ding{51} &
  0.475790 & 2937/2960 & \ding{51} \\\hline

  De coincidencia con superposición &
  0.437274 &   20/20   & \ding{51} &
  0.275709 &   20/20   & \ding{51} \\\hline

  Estadística universal de Maurer &
  0.739918 &   20/20   & \ding{51} &
  0.275709 &   20/20   & \ding{51} \\\hline

  De entropía aproximada &
  0.834308 &   20/20   & \ding{51} &
  0.739918 &   20/20   & \ding{51} \\\hline

  De excursiones aleatorias &
  -------- &   72/72   & \ding{51} &
  0.442340 &   77/80   & \ding{51} \\\hline

  Variante de excursiones aleatorias &
  -------- &  161/162  & \ding{51} &
  0.523098 &  180/180  & \ding{51} \\\hline

  Serial &
  0.493802 &   40/40   & \ding{51} &
  0.585633 &   40/40   & \ding{51} \\\hline

  De complejidad lineal &
  0.534146 &   20/20   & \ding{51} &
  0.350485 &   20/20   & \ding{51} \\\hline

  \caption{Resultado de las pruebas estadísticas del \gls{gl:drbg} basado en
  funciones hash (\gls{gl:sha}512) para los niveles de seguridad de 112 y 128.}
  \label{resultados_pruebas_drbg_sha512_112_128}

\end{longtable}

% =============================================================================

\begin{longtable}{| m{2.4in} | C{0.65in} | C{0.85in} |
C{0.15in} | C{0.65in} | C{0.85in} | C{0.15in} |}

  \hline
  \multicolumn{7}{|c|}{\textbf{Pruebas estadísticas para el \gls{gl:drbg}
  basado en funciones hash (\gls{gl:sha}512)}} \\
  \hline
  \textbf{Nivel de seguridad}        &
  \multicolumn{3}{c|}{\textbf{192}} &
  \multicolumn{3}{c|}{\textbf{256}} \\
  \hline
  \textbf{Prueba} &
  Valor P         &
  Proporción    & &
  Valor P         &
  Proporción    & \\
  \hline
  \endfirsthead

  \hline
  \multicolumn{7}{|c|}{\textit{Continuación}}\\
  \hline
  \textbf{Nivel de seguridad}        &
  \multicolumn{3}{c|}{\textbf{192}} &
  \multicolumn{3}{c|}{\textbf{256}} \\
  \hline
  \textbf{Prueba} &
  Valor P         &
  Proporción    & &
  Valor P         &
  Proporción    & \\
  \hline
  \endhead

  \multicolumn{7}{|c|}{\textit{Continúa en siguiente página}}\\
  \hline
  \endfoot

  \endlastfoot

  De frecuencia &
  0.162606 &   20/20   & \ding{51} &
  0.534146 &   20/20   & \ding{51} \\\hline

  De frecuencia en un bloque &
  0.739918 &   20/20   & \ding{51} &
  0.739918 &   19/20   & \ding{51} \\\hline

  De sumas acumulativas &
  0.336147 &   40/40   & \ding{51} &
  0.637032 &   40/40   & \ding{51} \\\hline

  De carreras &
  0.534146 &   20/20   & \ding{51} &
  0.637119 &   20/20   & \ding{51} \\\hline

  De la carrera más larga en un bloque &
  0.739918 &   20/20   & \ding{51} &
  0.834308 &   20/20   & \ding{51} \\\hline

  Del rango de matriz binaria &
  0.534146 &   20/20   & \ding{51} &
  0.739918 &   20/20   & \ding{51} \\\hline

  Espectral &
  0.066882 &   19/20   & \ding{51} &
  0.275709 &   19/20   & \ding{51} \\\hline

  De coincidencia sin superposición &
  0.489630 & 2934/2960 & \ding{51} &
  0.484244 & 2929/2960 & \ding{51} \\\hline

  De coincidencia con superposición &
  0.350485 &   20/20   & \ding{51} &
  0.437274 &   20/20   & \ding{51} \\\hline

  Estadística universal de Maurer &
  0.834308 &   20/20   & \ding{51} &
  0.739918 &   19/20   & \ding{51} \\\hline

  De entropía aproximada &
  0.275709 &   20/20   & \ding{51} &
  0.964295 &   20/20   & \ding{51} \\\hline

  De excursiones aleatorias &
  0.503591 &  104/104  & \ding{51} &
  0.481113 &   78/80   & \ding{51} \\\hline

  Variante de excursiones aleatorias &
  0.235885 &  234/234  & \ding{51} &
  0.572188 &  178/180  & \ding{51} \\\hline

  Serial &
  0.688519 &   39/40   & \ding{51} &
  0.555009 &   37/40   & \ding{51} \\\hline

  De complejidad lineal &
  0.637119 &   19/20   & \ding{51} &
  0.739918 &   20/20   & \ding{51} \\\hline

  \caption{Resultado de las pruebas estadísticas del \gls{gl:drbg} basado en
  funciones hash (\gls{gl:sha}512) para los niveles de seguridad de 192 y 256.}
  \label{resultados_pruebas_drbg_sha512_192_256}

\end{longtable}

% =============================================================================

Con base en los resultados de las pruebas estadísticas, se puede decir que
los \gls{gl:drbg} implementados son aleatorios. Aunque hay que tomar en cuenta
que el número de ejecuciones que se realizaron para las pruebas (20) es aun un
número pequeño, pero hacer un mayor número de pruebas consumiría mas recursos
de los que se tienen.
