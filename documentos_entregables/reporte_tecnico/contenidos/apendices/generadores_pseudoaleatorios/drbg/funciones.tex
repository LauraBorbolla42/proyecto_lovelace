%
% Recomendaciones del NIST para la generación de bits pseudoaleatorios,
% capítulo de análisis y diseño para la generación de tokens,
% Funciones
% Proyecto Lovelace.
%

\section{Funciones}

Un \gls{gl:drbg} necesita cinco funciones para poder funcionar correctamente.

\subsection{Instanciación}

Antes de generar bits pseudoaleatorios, el \gls{gl:drbg}
debe ser instanciado; esta función se encarga de revisar que los parámetros
de entrada sean válidos, determina  el nivel de seguridad para la instancia
de \gls{gl:drbg} que se generará, obtiene la entrada de \gls{gl:entropia}
capaz de soportar el nivel de seguridad y el \gls{gl:nonce} (solo si es
requerido), determina el estado interno inicial y, si se tienen varias
instancias del \gls{gl:drbg} simultáneas, obtiene un manejador de estado.
Las entradas de la función son las siguientes:

\begin{enumerate}
  \item Nivel de seguridad requerido.
  \item Bandera que indica si se necesita o no la resistencia de predicción.
  \item Cadena de personalización.
  \item Entrada de \gls{gl:entropia}.
  \item \gls{gl:nonce}
\end{enumerate}

Las primeras entradas deben ser provistas por la aplicación consumidora. La
salida de la función de instanciación a esta aplicación consiste en:

\begin{enumerate}
  \item El estado del proceso; regresa un \textit{EXITOSO} o un estado
    inválido indicando el error que hubo al instanciar al \gls{gl:drbg}.
  \item El manejador de estado, utilizado para identificar el estado interno
    de esta instancia para generar, cambiar la \gls{gl:semilla} y demás
    funciones.
\end{enumerate}

El algoritmo de la función de instanciación se puede observar en el
pseudocódigo~\ref{drbg:1}.

\begin{pseudocodigo}[caption={DRBG, instanciación.}, label={drbg:1}]
    entrada:  $nivel\_seguridad\_requerido$, $bandera\_prediccion$, $cadena\_personalizacion$
              $entrada\_entropia$, $nonce$
    salida:   $estado$, $manejador\_estado$
    inicio
      si $nivel\_seguridad\_requerido > mayor\_nivel\_seguridad\_soportado$:
        regresar $BANDERA\_ERROR, invalido$
      si $bandera\_prediccion = $verdadero Y $resistencia\_prediccion$ no es soportado:
        regresar $BANDERA\_ERROR, invalido$
      si $longitud(cadena\_personalizacion) > longitud\_maxima$:
        regresar $BANDERA\_ERROR, invalido$
      Asignar a $nivel\_seguridad$ el nivel más bajo de seguridad mayor o igual
        a $nivel\_seguridad\_requerido$ del conjunto $\{112, 128, 192, 256\}$
      ($estado, entrada\_entropia$) = $obtener\_entropia(nivel\_seguridad,$...
        ... $longitud\_min, longitud\_max, bandera\_prediccion)$
      si $(estado \neq EXITOSO)$:
        regresar $estado, invalido$
      Obtener nonce
      $estado\_trabajo\_inicial = INSTANCIAR\_ALGORITMO(entrada\_entropia,$...
        ... $nonce, cadena\_personalizacion, nivel\_seguridad)$
      Obtener $manejador\_estado$ para el estado interno vacío.
      si no se encuentra un estado interno vacío:
        regresar $BANDERA\_ERROR, invalido$
      Configurar el estado interno de la nueva instancia con los valores iniciales
        y la información administrativa.
      regresar $(EXITOSO, manejador\_estado)$
    fin
\end{pseudocodigo}

\subsection{Cambio de \gls{gl:semilla}}

Cambiar la \gls{gl:semilla} de una instancia no es requerido, pero es
recomendado que se realice este proceso cada que sea posible. Cambiar la
\gls{gl:semilla} puede:

\begin{itemize}
  \item Ser solicitado expresamente por la aplicación consumidora.
  \item Realizado cuando la aplicación consumidora requiere resistencia de
    predicción.
  \item Disparada por la función generadora cuando se alcanza un número
    predeterminado de salidas generadas.
  \item Disparada por eventos externos.
\end{itemize}

La función se encarga de revisar la validez de los parámetros de entrada,
obtiene la entrada de \gls{gl:entropia} de una fuente de aleatoriedad y,
mediante el algoritmo de cambio de \gls{gl:semilla}, combina el estado de
trabajo actual con la nueva entrada de \gls{gl:entropia} y valores
adicionales para determinar el nuevo estado de trabajo actual. Las entradas
para la función de cambio de \gls{gl:semilla} son las siguientes:

\begin{enumerate}
  \item El manejador de estado.
  \item Bandera que indica si se requiere o no la resistencia de predicción.
  \item Entradas adicionales (opcionales).
  \item Entrada de \gls{gl:entropia}.
  \item Valores del estado interno e información administrativa.
\end{enumerate}

Las primeras tres entradas deben ser provistas por la aplicación
consumidora. La salida consiste en lo siguiente:

\begin{enumerate}
  \item Estado que regresa la función.
  \item Nuevo estado de trabajo interno.
\end{enumerate}

El proceso para cambiar la \gls{gl:semilla} se observa en el pseudocódigo
~\ref{drbg:2}.

\begin{pseudocodigo}[caption={DRBG, cambio de semilla.}, label={drbg:2}]
    entrada:  $bandera\_prediccion$, $manejador\_estado$, $entrada\_adicional$
              $entrada\_entropia$, $estado\_interno$
    salida:   $estado$, $estado\_trabajo\_interno$
    inicio
      si $manejador\_estado$ indica un estado inválido o sin uso:
        regresar $BANDERA\_ERROR$
      si se requiere resistencia de predicción y $bandera\_prediccion=$falso:
        regresar $BANDERA\_ERROR$
      si $longitud(entrada_adicional) > longitud\_max\_entrada\_adicional$:
        regresar $BANDERA\_ERROR$
      ($estado, entrada\_entropia$) = $obtener\_entropia(nivel\_seguridad,$...
        ... $longitud\_min, longitud\_max, bandera\_prediccion)$
      si $(estado \neq EXITOSO)$:
        regresar $estado$
      $estado\_trabajo\_nuevo= ALGORITMO\_CAMBIAR\_SEMILLA(estado\_trabajo,$...
        ...$entrada\_entropia, entrada\_adicional)$
      Reemplazar el valor de $estado\_trabajo$ con $estado\_trabajo\_nuevo$
      regresar $(EXITOSO)$
    fin
\end{pseudocodigo}

\subsection{Generación}
Esta función es utilizada para generar bits pseudoaleatorios después de haber
utilizado la función de instanciación o de cambio de \gls{gl:semilla}. Se
encarga de validar los parámetros de entrada, llamar a la función de cambio de
\gls{gl:semilla} cuando sea requerida \gls{gl:entropia} extra, generar los bits
pseudoaleatorios, actualizar el estado de trabajo y regresar los bits a la
aplicación consumidora que los pidió. La función tiene las siguientes entradas:

\begin{enumerate}
  \item Manejador de estado.
  \item El número de bits que se requieren.
  \item El nivel de seguridad que se necesita.
  \item Si debe ser resistente a predicción o no.
  \item Entradas adicionales.
  \item El estado de trabajo actual e información administrativa.
\end{enumerate}

Después de haber sido generado, la salida consiste en lo siguiente

\begin{enumerate}
  \item Estado.
  \item Bits pseudoaleatorios.
  \item Nuevo estado de trabajo.
\end{enumerate}

El proceso para generar bits se puede observar en el pseudocódigo
~\ref{drbg:3}.

\begin{pseudocodigo}[caption={DRBG, generación.}, label={drbg:3}]
    entrada:  $bandera\_prediccion$, $manejador\_estado$, $entrada\_adicional$
              $nivel\_seguridad\_requerido$, $no\_bits\_requeridos$, $estado\_interno$
    salida:   $estado$, $estado\_trabajo$, $bits\_pseudoaleatorios$
    inicio
      si $manejador\_estado$ indica un estado inválido o sin uso:
        regresar $(BANDERA\_ERROR, NULL)$
      si $no\_bits > no\_bits\_maximo$
        regresar $(BANDERA\_ERROR, NULL)$
      si $nivel\_seguridad\_requerido > nivel\_seguridad$ indicado en el estado interno:
        regresar $(BANDERA\_ERROR, NULL)$
      si $longitud(entrada_adicional) > longitud\_max\_entrada\_adicional$:
        regresar $(BANDERA\_ERROR, NULL)$
      si se requiere resistencia de predicción y $bandera\_prediccion=$falso:
        regresar $(BANDERA\_ERROR, NULL)$
      Limpiar $bandera\_cambio\_semilla\_necesario$
      si $bandera\_cambio\_semilla\_necesario$ o $bandera\_prediccion$ están puestas:
        $estado = CAMBIO\_SEMILLA(manejador\_estado, prediccion\_resistencia, entrada\_adicional)$
        si $estado \neq EXITOSO$:
          regresar $(estado, NULL)$
        Obtener el nuevo estado interno
        $entrada\_adicional = NULL$
        Poner en cero $bandera\_cambio\_semilla\_necesario$
      ($estado, bits\_pseudoaleatorios, nuevo\_estado\_trabajo$) = $ALGORITMO\_GENERAR(estado\_trabajo,$...
        ...$no\_bits\_requeridos, entrada\_adicional)$
      si $estado$ indica que se requiere cambiar la semilla antes de poder generar bits:
        Activar $bandera\_cambio\_semilla\_necesario$
        si se requiere resistencia de predicción:
          Activar $bandera\_prediccion$
        Regresar al paso 7.
      Reemplazar el $estado\_trabajo$ con $nuevo\_estado\_trabajo$.
      regresar $(EXITOSO, bits\_pseudoaleatorios)$
    fin
\end{pseudocodigo}

Hay que tomar en cuenta cuando la implementación no tiene la capacidad de
hacer el cambio de \gls{gl:semilla} en el algoritmo: se quitan los pasos 6 y
7, y, si el estado indica que se necesita hacer el cambio, se regresa un
error que indique que el \gls{gl:drbg} no puede seguir siendo utilizado y
hay que eliminar la instancia. También se debe tener en mente cuando se
termina el ciclo de vida de la \gls{gl:semilla}: cada que se llama al
algoritmo generador, se revisa si el contador ha alcanzado el valor máximo
que se encuentra en el estado interno; en caso de ser así, la función debe
avisar que se requiere un cambio de \gls{gl:semilla}.

\subsection{Desinstanciación}
Esta función se encarga de liberar el estado interno de una instancia al borrar
el contenido de su estado interno. La función requiere del manejador de estado
de la instancia que se va a eliminar y regresa el estado de la función. El
proceso para eliminar una instancia se puede observar en el
pseudocódigo~\ref{drbg:4}.

\begin{pseudocodigo}[caption={DRBG, desinstanciación.}, label={drbg:4}]
    entrada:  $manejador\_estado$
    salida:   $estado$
    inicio
      si $manejador\_estado$ indica un estado inválido:
        regresar $(BANDERA\_ERROR)$
      Eliminar los contenidos del estado interno indicados por $manejador\_estado$
      regresar $(EXITOSO)$
    fin
\end{pseudocodigo}
