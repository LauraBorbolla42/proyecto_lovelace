%
% Recomendaciones del NIST para la generación de bits pseudoaleatorios,
% capítulo de análisis y diseño para la generación de tokens,
% Garantías
% Proyecto Lovelace.
%
\section{Garantías}

Los usuarios de un \gls{gl:drbg} requieren una garantía de que el generador
en verdad produce bits pseudoaleatorios, que el diseño, implementación y uso
de servicios criptográficos son adecuados para proteger la información del
usuario y, finalmente, necesita una garantía de que el generador sigue
operando correctamente. La implementación debe ser validada por un laboratorio
acreditado por \gls{gl:nvlap} para tener la certeza de que el mecanismo está
bien implementado. Se requiere, además, para garantizar el funcionamiento del
mecanismo, lo siguiente:
\begin{description}
  \item[Documentación mínima] se debe proveer, al menos, el siguiente conjunto
    de documentos; una gran parte podrían pertenecer al manual de usuario:
    \begin{enumerate}
      \item Documentación del método para obtener la entrada de entropía.
      \item Documentar cómo la implementación fue diseñada para permitir la
        validación de la implementación y revisar su estado.
      \item Documentar el tipo de mecanismo \gls{gl:drbg} y las primitivas
        criptográficas utilizadas.
      \item Documentar los niveles de seguridad soportados por la
        implementación.
      \item Documentar las características que soporta la implementación.
      \item Si las funciones del \gls{gl:drbg} están distribuidas, especificar
        los mecanismos que se usan para proteger la confidencialidad e
        integridad de las partes del estado interno que son trasnferidas entre
        las partes distribuidas del \gls{gl:drbg}.
      \item Indicar si se utiliza una función de derivación; si no se utiliza,
        documentar que la implementacion solo puede utilizarse cuando la entrada
        de \gls{gl:entropia} completa está disponible.
      \item Documentar todas las funciones de soporte.
      \item Si se requieren hacer pruebas periódicas para la función generadora,
        documentar los intervalos y justificarlos.
      \item Documentar si las funciones del \gls{gl:drbg} pueden ser puestas a
        prueba sobre demanda.
    \end{enumerate}

  \item[Pruebas de validación de la implementación] El mecanismo \gls{gl:drbg}
    debe ser diseñado para ser probado y poder asegurar que el producto está
    correctamente implementado. Debe proveerse una interfaz para realizarle
    pruebas que permita insertar los datos de entrada y obtener los datos de
    salida. Todas las funciones que se incluyen en la implementación deben ser
    probadas en la funcionalidad de las pruebas de salud.

  \item[Pruebas de salud] La implementación \gls{gl:drbg} debe realizarse
    pruebas de salud a sí mismo para asegurarse de que continua operando
    correctamente. Se deben realizar pruebas del tipo \textit{known answer},
    donde se tiene una entrada para la que ya se sabe la respuesta correcta.

  \item[Manejo de errores] Se indica para cada función del mecanismo qué errores
    son los esperados; es menester indicar el tipo de error que ocurrió.

\end{description}
