%
% Apéndice con contenido de páginas estáticas para sistema tokenizador.
% Reporte técnico.
% Proyecto Lovelace.
%

\capitulo[EST]{Páginas estáticas}{sec:paginas_estaticas}{%
  \epigrafe{%
    Our intellectual powers are rather geared to master static relations and
    that our powers to visualize processes evolving in time are relatively
    poorly developed. For that reason we should do (as wise programmers aware
    of our limitations) our utmost to shorten the conceptual gap between the
    static program and the dynamic process, to make the correspondence between
    the program (spread out in text space) and the process (spread out in time)
    as trivial as possible.}{%
    \textsc{Edsger Dijkstra}.}}%
%
\begin{parrafoConListaCorta}{%
  \noindent
  En este apéndice se muestran los contenidos de las páginas estáticas de
  la aplicación web del servicio tokenizador: la página de inicio y la
  página de documentación. El contenido de estas dos páginas está regido por
  los requerimientos \hipervinculo{rq:mostrar_promocion} y
  \hipervinculo{rq:mostrar_documentacion}, respectivamente. Estas páginas se
  encuentran disponibles en:}
  \item \url{https://ricardo-quezada.159.65.96.59.xip.io/lovelace/}
  \item \url{https://ricardo-quezada.159.65.96.59.xip.io/lovelace/documentacion}
\end{parrafoConListaCorta}

Para evitar el duplicado de información se ocuparon las siguientes
dependencias. La fuente de información original es el archivo \texttt{.tex},
mientras que el que ocupa la aplicación web es un \texttt{.html} dependiente
del primero.

\dependencia[dep:pandoc]{Traductor entre lenguajes de marcado}{Pandoc}{%
  \acrshort{gl:gpl} v2}{https://github.com/jgm/pandoc}{%
    Permite hacer converciones entre archivos de marcado. En este caso,
    se ocupa para pasar de un archivo fuente de \LaTeX~a un \gls{gl:html}.}

\dependencia{Filtro para archivos de~\LaTeX}{Pandoc-latex-levelup}{%
  \acrshort{gl:bsd}}{https://github.com/daamien/pandoc-latex-levelup}{%
    Filtro para \hipervinculo{dep:pandoc}. Permite hacer corrimientos entre
    la profundidad de los archivos de origen y los archivos destino.}

\section{Página de inicio}
\subimport{inicio/}{inicio.tex}

\subsubsection{\texorpdfstring{\acrshort{gl:ffx}}{FFX}}
\subimport{inicio/}{ffx.tex}

\subsubsection{\texorpdfstring{\acrshort{gl:bps}}{BPS}}
\subimport{inicio/}{bps.tex}

\subsubsection{TKR}
\subimport{inicio/}{tkr.tex}

\subsubsection{AHR}
\subimport{inicio/}{ahr.tex}

\subsubsection{\texorpdfstring{\acrshort{gl:drbg}}{DRBG}}
\subimport{inicio/}{drbg.tex}

\subimport{inicio/}{comparacion.tex}

\section{Página de documentación}
\subimport{/}{documentacion.tex}
