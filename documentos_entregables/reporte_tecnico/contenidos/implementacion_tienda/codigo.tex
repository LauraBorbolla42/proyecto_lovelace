%
% Código de módulo de tienda en línea,
% Reporte técnico.
% Proyecto Lovelace.
%

\section{Aspectos relevenates de la implementación}

Al estar desarrollada sobre las mismas tecnologías, la implementación de la
tienda en línea es bastante similar a la del sistema tokenizador. Por esta
razón, el enfoque para mostrar ejemplos de código es distinto al de la sección
\ref{sec:codigo_sistema_tokenizador} (los aspectos relevantes de la
implementación del sistema tokenizador), en donde se mostró un ejemplo de cada
módulo del programa, tanto en el servidor como en el cliente. Aquí se muestran
los códigos fuente relacionados con la implementación de un solo caso de uso:
\hipervinculo{lib_cu:comprar}.

El primer paso consiste en la acción de inicio del usuario: el cliente presiona
el botón \textit{Finalizar compra} en \hipervinculo{lib_iu:finalizar_compra}.
En el código~\ref{codigo:pantalla_carrito} se muestra el fragmento de
\gls{gl:html} correspondiente a ese botón. Como se puede observar en la
línea~122, al presionar el botón se llama a la función \texttt{finalizarCompra}.
Esta función se muestra en el código~\ref{codigo:controlador_carrito}, que es
el controlador asociado a la ventana del carrito.

\codigoFuente[codigo:pantalla_carrito]{107}{127}{html}{%
  tienda/archivos_web/html/carrito.html}{%
  Botón \textit{Finalizar compra}, en pantalla de carrito.}

\codigoFuente[codigo:controlador_carrito]{24}{50}{js}{%
  tienda/archivos_web/js/controladores/carrito.controlador.js}{%
  Función de finalización de compra.}

El siguiente paso consiste en mostrar la interfaz
\hipervinculo{lib_iu:seleccionar_forma_de_pago}. Esto se logra con la llamada
de la función del código~\ref{codigo:controlador_carrito}. Lo primero que hace
esta función es verificar una de las condiciones de inicio del caso
\hipervinculo{lib_cu:iniciar_sesion}: si el usuario presiona
\textit{Finalizar compra} sin haber iniciado sesión antes, se le redirige a la
pantalla \hipervinculo{lib_iu:iniciar_sesion}. Si el usuario ya inició sesión,
se abre la ventana emergente con el proceso de finalización de compra. El
controlador asociado recibe, entre otras cosas, un objeto con los libros del
carrito, un arreglo con las tarjetas asociadas al usuario (paso número dos del
caso de uso) y un arreglo con las direcciones de entrega del usuario.

El controlador de la ventanta del proceso de finalización de compra
(\texttt{controladorFinalizarCompra}) le pide al servidor el \gls{gl:html} de
la ventana a mostrar (\texttt{finalizar\_compra.ventana.html}). En el código
fuente~\ref{codigo:inicio_controlador_finalizar_compra} se muestra la secuencia
de inicio de \textit{javascript} ejecutada al momento de cargar la ventana. La
variable \texttt{\$scope.secuencia} controla las interfaces mostradas
al usuario:

\begin{enumerate}
  \item \hipervinculo{lib_iu:seleccionar_forma_de_pago}.
  \item \hipervinculo{lib_iu:seleccionar_direccion_de_entrega}.
  \item \hipervinculo{lib_iu:resumen_de_compra}
\end{enumerate}

\codigoFuente[codigo:inicio_controlador_finalizar_compra]{77}{110}{js}{%
  tienda/archivos_web/js/controladores/secundarios/%
  finalizar_compra.controlador.js}{%
  Secuencia de inicio de controlador de finalización de compra.}

El objeto \texttt{temporal} guarda la forma de pago y la dirección
seleccionadas por el usuario. El objeto \texttt{libros} guarda la selección de
libros del usuario. En el último fragmento del
código~\ref{codigo:inicio_controlador_finalizar_compra} se puede observar una
de las condiciones de inicio del caso de uso
\hipervinculo{lib_cu:agregar_forma_de_pago}: si el cliente no tiene ninguna
forma de pago asociada, se muestra la interfaz
\hipervinculo{lib_iu:formulario_informacion_bancaria}.

El siguiente paso consiste en la selección de la forma de pago. En el
codigo fuente~\ref{codigo:seleccion_forma_de_pago} se muestra el fragmento
de la ventana en donde se encuentran los botones radiales con las formas de
pago asociadas al cliente en sesión. También se muestra el botón para agregar
una nueva forma de pago. Esto último representa otra condición de inicio del
caso de uso \hipervinculo{lib_cu:agregar_forma_de_pago}: si el cliente presiona
el botón, se muestra la interfaz
\hipervinculo{lib_iu:formulario_informacion_bancaria}.

\codigoFuente[codigo:seleccion_forma_de_pago]{22}{56}{html}{%
  tienda/archivos_web/html/ventanas/finalizar_compra.ventana.html}{%
  Selección de forma de pago.}

Después de seleccionar la forma de pago, el cliente debe de presionar el botón
\textit{Continuar}. Este botón se muestra en el fragemnto del código
fuente~\ref{codigo:finalizar_compra_acciones_ventana}; al ser presionado se
llama a la función \texttt{continuar} (código
fuente~\ref{codigo:finalizar_compra_acciones_controlador}). Se trata del primer
caso del condicional, en donde se muestra la interfaz
\hipervinculo{lib_iu:seleccionar_direccion_de_entrega} y se verifica una de las
condiciones de inicio del caso de uso
\hipervinculo{lib_cu:agregar_direccion_entrega}: en caso de que el cliente no
tenga direcciones asociadas, se muestra la ventana
\hipervinculo{lib_iu:formulario_direccion_de_entrega}.

\codigoFuente[codigo:finalizar_compra_acciones_ventana]{114}{134}{html}{%
  tienda/archivos_web/html/ventanas/finalizar_compra.ventana.html}{%
  Acciones de ventana de finalización de compra.}

\codigoFuente[codigo:finalizar_compra_acciones_controlador]{36}{75}{js}{%
  tienda/archivos_web/js/controladores/secundarios/finalizar_compra.controlador.js}{%
  Funciones de controlador de finalización de compra.}

En los códigos fuente~\ref{codigo:finalizar_compra_acciones_ventana} y
\ref{codigo:finalizar_compra_acciones_controlador} se muestra la trayectoria
alternativa de cancelación. Si el cliente presiona el botón \textit{Cancelar}
se regresa a la interfaz \hipervinculo{lib_iu:finalizar_compra}.

El siguiente paso consiste en la selección de la dirección de entrega. En el
código fuente~\ref{codigo:seleccionar_direccion_de_entrega} se muestra el
fragmento de la ventana en donde se encuentran los botones radiales con las
direcciones asociadas. También se muestra el botón para agregar una nueva
dirección. Esto último es una condición de inicio más del caso de uso
\hipervinculo{lib_cu:agregar_direccion_entrega}: si el cliente presiona el
botón, se muestra la interfaz
\hipervinculo{lib_iu:formulario_direccion_de_entrega}.

\codigoFuente[codigo:seleccionar_direccion_de_entrega]{57}{86}{html}{%
  tienda/archivos_web/html/ventanas/finalizar_compra.ventana.html}{%
  Selección de dirección de entrega.}

En la línea~77 del código~\ref{codigo:seleccionar_direccion_de_entrega} se
puede ver el uso de un componente propio: \texttt{ego-direccion}. Los
componentes de \textit{Angular} permiten reutilizar código a lo largo de una
aplicación web~\cite{angular_componentes}. En este caso, el componente de la
dirección está definiendo el formato con el que se muestran las direcciones: un
cambio en este formato afecta a todos los puntos de la aplicación que muestran
direcciones. En el código fuente~\ref{codigo:componente_registro} se muestra el
registro del componente con la aplicación, y en el
código~\ref{codigo:componente_contenido} se muestra el contenido del componente.

\codigoFuente[codigo:componente_registro]{1}{13}{js}{%
  tienda/archivos_web/js/componentes/direccion.componente.js}{%
  Registro de componente de dirección.}

\codigoFuente[codigo:componente_contenido]{1}{13}{html}{%
  tienda/archivos_web/html/componentes/direccion.componente.html}{%
  Contenido de componente de dirección.}

Continuando con el caso de uso, después de seleccionar la dirección de entrega,
el siguiente paso consiste en presionar el botón de \textit{Continuar} (código
fuente~\ref{codigo:finalizar_compra_acciones_ventana}), que ejecuta el segundo
bloque del condicional del
código~\ref{codigo:finalizar_compra_acciones_controlador}. En este caso,
simplemente se muestra la interfaz \hipervinculo{lib_iu:resumen_de_compra}. En
el código fuente~\ref{codigo:ventana_resumen} se muestra el \gls{gl:html} de
esta interfaz.

\codigoFuente[codigo:ventana_resumen]{88}{114}{html}{%
  tienda/archivos_web/html/ventanas/finalizar_compra.ventana.html}{%
  Resumen de compra.}

El siguiente paso es volver a presionar el botón \textit{Continuar}. En este
caso se pasa al tercer bloque del condicional de la función del código
fuente~\ref{codigo:finalizar_compra_acciones_controlador}. Aquí se llama a la
función \texttt{api.registrarCompra}, que es la encargada de comunicarse con el
programa servidor. En el código fuente~\ref{codigo:api_tienda} se muestra un
fragmento de las funciones que se comunican con el servidor. En el caso de
\texttt{registrarCompra}, se hace una petición \texttt{POST} a la \gls{gl:url}
\texttt{/api/tienda/compra}, con un objeto \gls{gl:json} que representa
la compra.

\codigoFuente[codigo:api_tienda]{54}{76}{js}{%
  tienda/archivos_web/js/servicios/api.servicio.js}{%
  Fragmento de funciones de comunicación con el servidor.}

En el código fuente~\ref{codigo:direcciones_generales} se muestra un fragmento
de la configuración global de direcciones de la aplicación de la tienda. En el
caso de la petición para registrar una compra, se pasa a la configuración de
direcciones de la aplicación llamada \textit{tienda}. En el código
fuente~\ref{codigo:direcciones_tienda} se muestra un fragmento de estas
configuraciones. En el caso de la petición de registro de compra (línea~32), se
llama a la función \texttt{funciones\_web.registrarCompra}.

\codigoFuente[codigo:direcciones_generales]{1}{23}{python}{%
  tienda/direcciones.py}{%
  Configuración global de direcciones.}

\codigoFuente[codigo:direcciones_tienda]{21}{39}{python}{%
  tienda/tienda/direcciones.py}{%
  Configuración de direcciones de aplicación \textit{tienda}.}

La función de registro de compra se muestra en el
código~\ref{codigo:registrar_compra}. La primera parte de la función consiste
en el registro en la base de datos de la compra (y las tablas intermedias
asociadas); cada libro comprado se resta de las existencias del inventario. Por
último, se llama a la función \texttt{negocio.detokenizar}, que es la encargada
de comunicarse con el sistema tokenizador.

\codigoFuente[codigo:registrar_compra]{203}{241}{python}{%
  tienda/tienda/funciones_web.py}{%
  Función de registro de compra.}

El código fuente~\ref{codigo:funcion_detokenizacion} muestra la función de la
capa de negocios para hacer detokenizaciones. En caso de que la respuesta del
sistema tokenizador sea un código de respuesta \texttt{200}, el flujo del caso
de uso continúa por la trayectoria principal: se muestra el mensaje
\hipervinculo{lib_msj:compra_registrada}.

\codigoFuente[codigo:funcion_detokenizacion]{168}{182}{python}{%
  tienda/tienda/negocio.py}{%
  Función de comunicación con sistema tokenizador: operación de detokenización.}
