%
% Redes Feistel desbalanceadas, capítulo de antecedentes.
% Proyecto Lovelace.
%

\subsubsection{Redes Feistel desbalanceadas}

Este tipo de redes (presentadas en~\cite{DBLP:conf/fse/SchneierK96}) permite
que los bloques izquierdos y derechos sean de distintas longitudes ($ m $ y
$ n $, respectivamente). En la figura~\ref{feistel:desbalanceadas} se muestra
un esquema del proceso, el cual es bastante similar al de la figura
\ref{feistel} con la única diferencia de que la función $ f $ debe cambiar la
longitud de su entrada: de $ n $ a $ m $. Si $ m \le n $, la red es pesada del
lado de la fuente, y la función actúa como contracción (la entrada es más
grande que la salida). Por otra parte, si $ m \ge n $, la red es pesada del
lado del objetivo, y la función actúa como una expansión (la entrada es más
pequeña que la salida). El caso especial en el que $ m = n $ es en el que la
red está balanceada y corresponde al presentado originalmente (figura
\ref{feistel}); es por esto que las redes Feistel desbalanceadas son
consideradas una generalización del esquema inicial.

Para los esquemas pesados del lado de la fuente, la seguridad aumenta
proporcionalmente al grado de desbalanceo. Por el lado contrario, en los
esquemas pesados del lado del objetivo, entre más balanceada la red, mejor.
