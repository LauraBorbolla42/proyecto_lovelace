%
% Redes Feistel alternantes, capítulo de antecedentes.
% Proyecto Lovelace.
%

\subsubsection{Redes Feistel alternantes}
\label{sec:red_feistel_alternante}

Un inconveniente de las redes desbalanceadas es el costo extra de hacer las
particiones de los bloques intermedios. Las redes Feistel alternantes
(figura~\ref{feistel:alternantes}, presentadas en
\cite{DBLP:conf/fse/AndersonB96a} y~\cite{DBLP:conf/fse/Lucks96}) eliminan
este inconveniente utilizando dos tipos de funciones, una contractora y la
otra de expansión, en \glspl{gl:ronda} alternas.

Es importante notar que las redes alternantes son también una generalización
del esquema original, en la cuál la partición de los bloques es al centro, y
se utliza una sola función.
