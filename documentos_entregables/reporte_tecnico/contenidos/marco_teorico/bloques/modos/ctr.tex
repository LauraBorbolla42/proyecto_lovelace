%
% Modo de operación CTR, capítulo de antecedentes.
% Proyecto Lovelace.
%

\subsubsection{\textit{Counter Mode} (CTR)}
\label{sec:ctr}

Este toma a la entrada un \gls{gl:vector_de_inicializacion} y en cada iteración
lo incrementa y lo cifra. El resultado se obtiene combinando el $ VI $ cifrado
con el bloque de texto cifrado (mediante una operación \verb|xor|). El proceso
de cifrado y descifrado se detallan en el pseudocódigo~\ref{ctr:1}.

\begin{pseudocodigo}[%
    caption={\Gls{gl:modo_de_operacion} \gls{gl:ctr}%
      (cifrado y descifrado).},
    label={ctr:1}%
  ]
    entrada: llave $ k $; vector de inicialización $ VI $;
             bloques de mensaje (cifrado o descifrado) $ Bm_1, Bm_2 \dots Bm_n $.
    salida:  bloques de mensaje (cifrado o descifrado) $ Bc_1, Bc_2 \dots Bc_n $.
    inicio
      para_todo $Bm$
        $Bc_i$ $\gets$ E_k($ (VI + i) \mod 2^n $) $\oplus$ $Bm_i$
      fin
      regresar $Bc$
    fin
\end{pseudocodigo}

En términos de eficiencia, el \gls{gl:ctr} es mejor que \gls{gl:cbc}, ya que
sus operaciones (ambas) se pueden hacer en paralelo. La implementación es
prácticamente la misma para el cifrado y descifrado (solamente se ocupa el
cifrado del algoritmo por bloques subyacente).
