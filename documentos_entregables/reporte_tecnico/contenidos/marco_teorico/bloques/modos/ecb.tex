%
% Modo de operación ECB, capítulo de antecedentes.
% Proyecto Lovelace.
%

\subsubsection{\texorpdfstring{\acrfull{gl:ecb}}{Electronic Codebook (ECB)}}

La figura~\ref{figura:ecb} muestra un diagrama esquemático de este
\gls{gl:modo_de_operacion}. El algoritmo recibe a la entrada una llave y un
mensaje de longitud arbitraria: la llave se pasa sin ninguna modificación a
cada función del cifrado por bloques; el mensaje se debe de partir en bloques
($ M = Bm_1 || Bm_2 || \dots || Bm_n $).

\begin{figure}
  \centering
  \begin{subfigure}{0.45\textwidth}
    \begin{center}
      \subimport{diagramas/}{modo_ecb.tikz.tex}
      \caption{Cifrado.}
    \end{center}
  \end{subfigure}
  \begin{subfigure}{0.45\textwidth}
    \begin{center}
      \subimport{diagramas/}{modo_ecb_inverso.tikz.tex}
      \caption{Descifrado.}
    \end{center}
  \end{subfigure}
  \caption{\Gls{gl:modo_de_operacion} \acrshort{gl:ecb}.}
  \label{figura:ecb}
\end{figure}

\begin{pseudocodigo}[%
    caption={\Gls{gl:modo_de_operacion} \acrshort{gl:ecb}, cifrado.}%
  ]
    entrada: llave $ k $; bloques de mensaje $ Bm_1, Bm_2 \dots Bm_n $.
    salida:  bloques de mensaje cifrado $ Bc_1, Bc_2 \dots Bc_n $.
    inicio
      para_todo $Bm$
        $Bc_i$ $\gets$ E_k($Bm_i$)
      fin
      regresar $Bc$
    fin
\end{pseudocodigo}

\begin{pseudocodigo}[%
    caption={\Gls{gl:modo_de_operacion} \acrshort{gl:ecb}, descifrado.}%
  ]
    entrada: llave $ k $; bloques de mensaje cifrado $ Bc_1, Bc_2 \dots Bc_n $.
    salida:  bloques de mensaje original $ B_1, B_2 \dots B_n $.
    inicio
      para_todo $Bc$
        $Bm_i$ $\gets$ $D_k$($Bc_i$)
      fin
      regresar $Bm$
    fin
\end{pseudocodigo}
