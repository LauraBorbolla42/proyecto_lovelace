%
% Sección de cifrado por bloques, capítulo de antecedentes.
% Proyecto Lovelace.
%

\section{Cifrados por bloques}
\label{sec:bloques}

La información presentada a continuación puede consultarse con más profundidad
en las siguientes referencias~\cite{menezes, DBLP:series/isc/DelfsK07}.

Los cifrados por bloque son esquemas de cifrado que, como bien
lo explica su nombre, operan mediante bloques de datos. Normalmente los
bloques tienen una longitud de 64 o de 128 bits, mientras que las llaves
pueden ser de 56, 128, 192 o 256 bits.

En muchos sistemas criptográficos, los cifrados por bloque simétricos
son elementos importantes, pues su versatilidad permite construir con
ellos generadores de números pseudoaleatorios, cifrados de flujo MACs y
funciones hash. Sirven también como componentes centrales en técnicas de
autenticación de mensajes, mecanismos de integridad de datos, protocolos
de autenticación de entidad y esquemas de firma electrónica que usan
llaves simétricas.

Los cifrados por bloque están limitados en la práctica por varios
factores, tales como el límite de memoria, la velocidad requerida o
restricciones impuestas por el hardware o el software en el que se
implementan. Normalmente, se debe escoger entre eficiencia y seguridad

Idealmente, al cifrar por bloques, cada bit del bloque cifrado depende
de todos los bits de la llave y del texto en claro; no debería existir
una relación estadística evidente entre el texto en claro y el texto
cifrado; el alterar tan solo un bit en el texto en claro o en la llave
debería alterar cada uno de los bits del texto cifrado con una
probabilidad de $\frac{1}{2}$; y alterar un bit del texto cifrado
debería provocar resultados impredecibles al recuperar el texto en
claro.

\subsection{Definición}

\begin{equation}
  \label{cifrado_bloques_def}
  \begin{aligned}
    E: & \{0,1\}^r \times \{0,1\}^n \longrightarrow \{0,1\}^n \\
       & (k,m) \longmapsto  E(k,m)
  \end{aligned}
\end{equation}

Utilizando una llave secreta $k$ de longitud binaria $r$ el algoritmo de
cifrado $E$ cifra bloques en claro $m$ de una longitud binaria fija $n$ y
da como resultado bloques cifrados $c = E (k,m)$ cuya longitud también es
$n$. $n$ es el tamaño de bloque del cifrado.
El espacio de llave está dado por $K = \{0,1\}^r$, para cada llave existe una
función $D_k(c)$ que permite tomar un bloque cifrado $c$ y regresarlo a su
forma original $m$.

Generalmente, los cifrados por bloque procesan el texto claro en bloques
relativamente grandes ($n \geq 64$), contrastando con los cifrados de
flujo, que toman bit por bit. Cuando la longitud del mensaje en claro excede
el tamaño de bloque, se utilizan los \glspl{gl:modo_de_operacion}.

\begin{parrafoConListaCorta}{%
  Los parámetros más importantes de los cifrados por bloque son los
  siguientes:}
  \item Tamaño de bloque
  \item Tamaño de llave
\end{parrafoConListaCorta}

\subsection{Criterios para evaluar los cifrados por bloque}

A continuación se listan algunos de los criterios que pueden ser tomados
en cuenta para evaluar estos cifrados:
\begin{enumerate}
  \item \textbf{Nivel de seguridad.} La confianza que se le tiene a un
    cifrado va creciendo con el tiempo, pues va siendo analizado y
    sometido a pruebas.
  \item \textbf{Tamaño de llave.} La \gls{gl:entropia} del espacio de la
    llave define un límite superior en la seguridad del cifrado al tomar en
    cuenta la búsqueda exhaustiva. Sin embargo, hay que tener cuidado
    con su tamaño, pues también aumentan los costos de generación,
    transmisión, almacenamiento, etcétera.
  \item \textbf{Tamaño de bloque.} Impacta la seguridad, pues entre más
    grandes, mejor; sin embargo, tiene repercusiones en el costo de la
    implementación, además de que puede afectar el rendimiento del
    cifrado.
  \item \textbf{Expansión de datos.} Es extremadamente deseable que los
    datos cifrados no aumenten su tamaño respecto a los datos en claro.
  \item \textbf{Propagación de error.} Descifrar datos que contienen
    errores de bit puede llevar a recuperar incorrectamente el texto en
    claro, además de propagar errores en los bloques pendientes por
    descifrar. Normalmente, el tamaño de bloque afecta el error de
    propagación.
\end{enumerate}

% A continuación se listan algunos algoritmos de cifrado por bloques.
Entre los algoritmos de cifrado por bloque más usados se encuentran:
\gls{gl:des}, \gls{gl:aes}, \gls{gl:feal}, \gls{gl:idea}, \gls{gl:safer}
y RC5. A continuación se presentan las redes Fesitel, las cuales se utilizan
para la construcción de otros cifrados por bloques, \gls{gl:des} y \gls{gl:aes}.

\subimport{feistel/}{feistel}
\subimport{/}{DES}
\subimport{/}{AES}
\subimport{modos/}{modos}
% \subimport{/}{FEAL}
% \subimport{/}{IDEA}
% \subimport{/}{SAFER}
% \subimport{/}{RC5}
