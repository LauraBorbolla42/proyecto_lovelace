%
% Cifrado de flujo RC4, capítulo de antecedentes.
% Proyecto Lovelace.
%

\subsection{RC4}

RC4 es un cifrado de flujo diseñado por Ron L. Rivest en 1987 para la empresa
\gls{gl:rsa}. Es usado en varios protocolos de seguridad comunes:
\gls{gl:ssl}/\gls{gl:tls}, \gls{gl:wep} y \gls{gl:wpa}; los
dos últimos son parte del estándar \gls{gl:ieee} 802.11 para
comunicaciones \gls{gl:lan} inalámbricas. RC4 se mantenía como secreto de
compañía hasta que, en septiembre de 1994, fue filtrado de forma anónima en
Internet.

En el pseudocódigo~\ref{cod:rc4} se describe el proceso de cifrado del
algoritmo. $ S $ es un vector de estado; $ T $ es un vector temporal.

\begin{pseudocodigo}[%
    caption={Proceso de cifrado de RC4.},
    label={cod:rc4}%
  ]
    entrada: llave k; mensaje original $ m_1, m_2 \dots m_n $.
    salida:  mensaje cifrado $ c_1, c_2 \dots c_n $.
    inicio
      /* Inicialización */
      para_todo i entre 0 y 255:
        S[i] $\gets$ i
        T[i] $\gets$ K[i $\mod$ longitud_de_llave]
      fin

      /* Permutación inicial */
      j $\gets$ 0
      para_todo i entre 0 y 255:
        j = (j + S[i] + T[i]) $\mod$ 256
        intercambiar(S[i], S[j])
      fin

      /* Proceso de cifrado */
      i, j $\gets$ 0
      para_todo m:
        i $\gets$ (i + 1) $\mod$ 256
        j $\gets$ (j + S[i]) $\mod$ 256
        intercambiar(S[i], S[j])
        k $\gets$ S[(S[i] + S[j]) $\mod$ 256]
        c $\gets$ m $\oplus$ k
      fin
      regresar c
    fin
\end{pseudocodigo}

Se han hecho varias publicaciones que analizan métodos para atacar RC4, ninguna
de las cuales presenta algo práctico cuando se utiliza una llave mayor a
128 bits. Sin embargo, en~\cite{ataque_wep} se reporta un problema más serio
sobre la implementación que se hace de RC4 en el protocolo \gls{gl:wep};
este problema en particular no ha demostrado afectar a otras aplicaciones que
usan RC4.
