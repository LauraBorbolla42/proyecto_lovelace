%
% Cifrados de flujo, capítulo de antecedentes.
% Proyecto Lovelace.
%

\section{Cifrados de flujo}
\label{sec:flujo}

La información de esta sección puede ser encontrada con mayor detalle en
\cite{menezes, stallings} y~\cite{alan_konheim}.

A diferencia de los cifrados de bloque, que trabajan sobre grupos enteros de
bits a la vez, los cifrados de flujo trabajan sobre bits individuales,
cifrándolos uno por uno. Una manera de verlos es como cifrados por bloques con
un tamaño de bloque igual a 1.

Un cifrado de flujo aplica transformaciones de acuerdo a un flujo de llave:
una secuencia de símbolos pertenecientes al espacio de llaves. El flujo de
llave puede ser generado tanto de manera aleatoria, como por un algoritmo
pseudoaleatorio que reciba a la entrada, o bien una semilla, o bien una
semilla y algunos bits del texto cifrado.

Entre las ventajas de los cifrados de flujo sobre los cifrados de bloque se
encuentra el hecho de que son más rápidos en hardware y más útiles cuando
el buffer es limitado o se necesita procesar la información al momento de
llegada. La propagación de los errores es limitada o nula, por lo que también
son más apropiados en casos en los que hay probabilidades altas de errores en
la transmisión.

Los cifrados de bloques funcionan sin ninguna clase de memoria (por sí solos);
en contraste, la función de cifrado de un cifrado de flujo puede variar
mientras se procesa el texto en claro, por lo cuál tienen un mecanismo de
memoria asociado. Otra denominación para estos cifrados es \textit{de estado},
por que la salida no depende solamente del texto en claro y de la llave, sino
que también depende del estado actual.

% \subimport{/}{clasificacion}
% \subimport{/}{RC4}
% \subimport{/}{estream}
