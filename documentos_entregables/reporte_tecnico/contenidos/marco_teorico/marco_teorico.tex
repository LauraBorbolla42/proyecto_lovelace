%
% Capítulo de antecedentes.
% Proyecto Lovelace.
%

\chapter{Marco teórico}
\label{sec:marco_teorico}
El marco teórico contiene la mayoría de los conceptos utilizados para la
implementación de los algoritmos tokenizadores. Comienza con la introducción a
la criptografía, sus objetivos, ataques y tipos; posteriormente, se tiene un
resumen de los cifrados por bloque, haciendo énfasis en el funcionamiento de las
redes feistel y el cifrador AES. y los modos de operación ECB, CBC y CTR;
después, se agregan unas nociones básicas sobre cifradores por flujos, funciones
hash, códigos \gls{gl:mac} y \gls{gl:tes}. El capítulo finaliza con los cifrados
que preservan el formato. Cada sección indica al inicio las fuentes
bibliográficas que fueron consultadas en caso de que el lector desee profundizar
en algún tema tratado. 

\subimport{intro/}{intro}
\subimport{bloques/}{bloques}
\subimport{cifrados_de_flujo/}{cifrados_de_flujo}
\subimport{hash/}{hash}
\subimport{mac/}{mac}
\subimport{tes/}{tes}
\subimport{fpe/}{fpe}
\subimport{tarjetas/}{tarjetas}
