%
% Capítulo de antecedentes.
% Proyecto Lovelace.
%

\capitulo{Marco teórico}{sec:marco_teorico}{%
  \epigrafe{%
    Desvarío laborioso y empobrecedor el de componer vastos libros; el de
    explayar en quinientas páginas una idea cuya perfecta exposición oral cabe
    en pocos minutos. Mejor procedimiento es simular que esos libros ya existen
    y ofrecer un resumen, un comentario.}{%
    El jardín de senderos que se bifurcan,
    \textsc{Jorge Luis Borges}.}}%
%
El marco teórico contiene la mayoría de los conceptos utilizados para la
implementación de los algoritmos generadores de \glspl{gl:token}. Comienza con
la introducción a la criptografía, sus objetivos, ataques y tipos.
Posteriormente, se tiene un resumen de los cifrados por bloque, haciendo énfasis
en el funcionamiento de las redes Feistel y el cifrado \gls{gl:aes} y los modos
de operación \gls{gl:ecb}, \gls{gl:cbc} y \gls{gl:ctr}; después, se agregan los
cifrados que preservan el formato y finaliza con una descripción sobre la
composición de un número de tarjeta y los estándares del \gls{gl:nist} y
\gls{gl:pci} \gls{gl:ssc} que se utilizaron en este trabajo. Cada sección
indica al inicio las fuentes bibliográficas que fueron consultadas en caso de
que el lector desee profundizar en algún tema tratado.

\subimport{intro/}{intro}
\subimport{bloques/}{bloques}
%\subimport{cifrados_de_flujo/}{cifrados_de_flujo}
\subimport{hash/}{hash}
% \subimport{mac/}{mac}
% \subimport{tes/}{tes}
\subimport{fpe/}{fpe}
\subimport{tarjetas/}{tarjetas}
\subimport{nist_pci/}{nist_pci}
