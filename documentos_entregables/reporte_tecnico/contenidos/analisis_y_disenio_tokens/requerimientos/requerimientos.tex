%
% Requerimientos de tokens, análisis y diseño de generación de tokens.
% Proyecto Lovelace.
%

\section{Requerimientos}
\label{sec:requerimientos}

El \gls{gl:pci} \gls{gl:dss} define cuatro \textit{dominios} de seguridad para
el proceso de tokenización:

\begin{enumerate}

  \item \label{dm:gen_tokens} \textbf{Generación de \glspl{gl:token}.}
    Para cada clase tokenizadora, este dominio se encarga de definir
    consideraciones para la generación segura de \glspl{gl:token}. Cubre los
    dispositivos, procesos, mecanismos y algoritmos que son utilizados para
    crear los \glspl{gl:token}.

  \item \label{dm:mapeo_tokens} \textbf{Mapeo de \glspl{gl:token}.}
    Este dominio, que se refiere al mapeo de los \glspl{gl:token} con su
    \gls{gl:pan} origen, aplica solamente a los procesos de tokenización
    reversibles. Entre otras cosas, provee guías respecto a control de acceso
    necesarios para las peticiones de tokenización.

  \item \label{dm:card_data} \textbf{Bóveda de datos de tarjeta.}
    Como el dominio pasado, solo aplica a las implementaciones de tokenización
    reversibles. Cubre el cifrado obligado del \gls{gl:pan} y los controles de
    acceso necesarios para entrar a la \gls{gl:cdv}.

  \item \label{dm:man_llaves} \textbf{Manejo criptográfico de llaves.}
    Define las buenas prácticas para el manejo criptográfico de las llaves y
    las operaciones realizadas con ellas por el producto tokenizador.

\end{enumerate}

En esta sección se detallan los requerimientos que deben de satisfacer los
\glspl{gl:token} según las guías de tokenización del \gls{gl:pci}
\gls{gl:dss}~\cite{pci_tokens}. Para comenzar, se enlistan los requerimientos
aplicables a todos los \glspl{gl:token}, sin importar su categoría; después de
esto, se agrupan los requerimientos correspondientes a cada una de las
categorías presentadas en la sección \ref{sec:pci_dss}.

Para su posterior seguimiento, cada requerimiento se clasifica según la
entidad responsable de su cumplimiento; estas categorías son:

\begin{enumerate}

  \item \textbf{\hypertarget{clasificacion:algoritmos}{Algoritmos}.}
    Requerimiento cuyo cumplimiento es responsabilidad del diseñador de un
    algoritmo tokenizador.

  \item \textbf{\hypertarget{clasificacion:implementaciones}{Implementaciones}.}
    Requerimiento cuyo cumplimiento es responsabilidad del implementador de un
    algoritmo tokenizador.

  \item \textbf{\hypertarget{clasificacion:sistema}{Sistema tokenizador}.}
    Requerimiento cuyo cumplimiento es responsabilidad de un proveedor de
    servicios de tokenización.

\end{enumerate}

Los requerimientos de la primera categoría se encuentran fuera de los
propósitos de este trabajo. En las categorías dos y tres se agrupan
los requerimientos que, en efecto, un trabajo como este debe de cumplir.
A los de la segunda categoría se les da segumiento en la sección
\ref{sec:seguimiento_token}; los de la tercera son parte del siguiente módulo
de este proyecto y se les da seguimiento en la sección
\ref{sec:seguimiento_api}.

% GT1
\requerimientoConClasificacion[rq_pci:productos_de_hardware]
{Validación de productos de hardware}
{clasificacion:sistema}
{
  Si se usa un producto de hardware para la tokenización, este debe de ser
  validado por \gls{gl:fips} 140-2 nivel 3 o superior (descrito en
 ~\cite{nist_modulos_criptograficos}).
}

% GT2 se lo chutaron.
% GT3
\requerimientoConClasificacion[rq_pci:productos_de_software]
{Validación de productos de software}
{clasificacion:sistema}
{
  Si se usa un producto de software para la tokenización, este debe de ser
  validado por \gls{gl:fips} 140-2 nivel 2 o superior (descrito en
 ~\cite{nist_modulos_criptograficos}).
}

% GT4
\requerimientoConClasificacion[rq_pci:resistencia_texto_claro_conocido]
{Resistencia a texto claro conocido}
{clasificacion:algoritmos}
{
  Un atacante con acceso a múltiples pares de \glspl{gl:token} y
  \gls{gl:pan} no debe de ser capaz de determinar otros \gls{gl:pan} a partir
  de solamente \glspl{gl:token}. En otras palabras, los \glspl{gl:token}
  deben ser resistentes a ataques con texto en claro conocido (sección
 ~\ref{sec:criptoanalisis}).
}

% GT5
% Me parece que esto es reduntante: si GT4 ya estableció la necesidad de
% resistencia a texto claro conocido, la resistencia a sólo texto cifrado
% ya va incluida. De hecho, en la misma redacción de GT5 incluyen claramente
% a GT4 con una disyunción.
% Creo que ya había leído esto mismo en alguno de los artículos que nos pasó
% Sandra.
\requerimientoConClasificacion[rq_pci:resistencia_solo_texto_cifrado]
{Resistencia a sólo texto cifrado}
{clasificacion:algoritmos}
{
  Recuperar un \gls{gl:pan} a partir de un \gls{gl:token} debe de ser
  \gls{gl:computacionalmente_no_factible} (resistencia a ataques con sólo
  texto cifrado, sección~\ref{sec:criptoanalisis}).
}

% GT6
\requerimientoConClasificacion[rq_pci:deteccion_anomalias]
{Detección de anomalías}
{clasificacion:sistema}
{
  Se deben de implementar disparadores que permitan detectar
  irregularidades en el sistema (anomalías, funcionamientos erróneos,
  comportamientos sospechosos). El producto debe registrar dichos eventos y
  avisar al personal correspondiente.
}

% GT7
\requerimientoConClasificacion[rq_pci:distincion_token_pan]
{Distinción entre \glspl{gl:token} y \gls{gl:pan}}
{clasificacion:implementaciones}
{
  Se debe contar con un mecanismo para distinguir entre \glspl{gl:token}
  y \gls{gl:pan}. Los proveedores del servicio de tokenización deben
  compartir este mecanismo con la entidad (o entidades) que usa los
  \glspl{gl:token}.
}

% GT8
\requerimientoConClasificacion[rq_pci:guia_de_instalacion]
{Guía de instalación}
{clasificacion:sistema}
{
  Se debe de contar con una guía de instalación y uso para el correcto
  funcionamiento del producto de tokenización.
}

% GT9
\requerimientoConClasificacion[rq_pci:integridad_ejecutables]
{Integridad del proceso de tokenización}
{clasificacion:implementaciones}
{
  Deben de implementarse mecanismos que garanticen la integridad del proceso
  de generación de \glspl{gl:token}.
}

% GT10, RC2A, RN2B y RN3B
\requerimientoConClasificacion[rq_pci:acceso_a_producto]
{Acceso al proceso detokenización}
{clasificacion:sistema}
{
  Solo los usuarios autenticados y componentes del sistema tienen permitido
  acceder al proceso de tokenización y detokenización. Los métodos utilizados
  deben ser al menos tan rigurosos como lo indicado en el requerimiento 8 del
  \gls{gl:pci} \gls{gl:dss}~\cite{pci_dss}.
}

% GT10.1, RC2A-1, RC2A-2, RN2B, RN3B
\subrequerimiento[rq_pci:control_de_peticiones]
{Control de peticiones}
{
  Todas las peticiones deben pasar a través de una \gls{gl:api} que controle
  todos los intentos de acceso y aplique de manera uniforme reglas de
  control de acceso.
}

% GT10.2
\subrequerimiento[rq_pci:registros_de_acceso]
{Registros de acceso}
{
  Se deben registrar todos los eventos de acceso, de tokenización y
  de detokenización. Esta funcionalidad debe ser configurable de manera
  segura. Para esto se debe seguir el requerimiento 4 del
  \gls{gl:pa}-\gls{gl:dss}~\cite{dss_pa}.
}

% GT10.3
\subrequerimiento[rq_pci:autenticacion_multifactor]
{\Gls{gl:autenticacion_multifactor}}
{
  El sistema debe soportar \gls{gl:autenticacion_multifactor} para todos
  los tipos de usuario: accesos administrativos, operaciones de tokenización
  y detokenización, mantenimiento, etcétera.
}

% GT10.4
\subrequerimiento[rq_pci:acceso_de_sistema]
{Accesos a nivel de sistema}
{
  Todos los accesos a nivel de sistema deben soportar
  \gls{gl:autenticacion_mutua}, incluyendo a las peticiones de tokenización
  y detokenización.
}

% GT10.5
\subrequerimiento[rq_pci:acceso_administrativo]
{Accesos administrativos}
{
  Se debe utilizar \gls{gl:criptografia_fuerte} para todos los accesos
  administrativos que no se hagan desde consola.
}

% GT11
\requerimientoConClasificacion[rq_pci:token_a_token_prohibido]
{Mapeos de \gls{gl:token} a \gls{gl:token} prohibidos}
{clasificacion:implementaciones}
{
  No se debe poder pasar de un primer \gls{gl:token} válido a un segundo,
  también válido; forzosamente debe existir un estado intermedio: del primer
  \gls{gl:token} se pasa al \gls{gl:pan} correspondiente (operación de
  detokenización) y de este se pasa al segundo \gls{gl:token}.
}

% GT12
\requerimientoConClasificacion[rq_pci:vulnerabilidades_comunes]
{Protección contra vulnerabilidades comunes}
{clasificacion:implementaciones}
{
  Se deben implementar medidas en contra de las vulnerabilidades de
  seguridad más comunes (\cite{dss_pa}, requerimiento 5.2). Algunas de estas
  medidas pueden ser el uso de herramientas de análisis de código estático,
  o el uso de lenguajes de programación especializados.
}

% GT13
\requerimientoConClasificacion[rq_pci:primitivas_usadas]
{Primitivas criptográficas usadas}
{clasificacion:implementaciones}
{
  Las primitivas criptográficas que se usen deben estar basadas en
  estándares nacionales (referentes a Estados Unidos) o internacionales (p. ej.
  \gls{gl:aes}). Ver sección~\ref{sec:primitivas}.
}

% TODO:
% - Hablar sobre repeticiones en documento de PCI
% - ¿Porqué RC1A no le aplica a los irreversibles también?

% Para eliminar repeticiones:

% IT4A, RC4B, RN4B
\requerimientoConClasificacion[rq_pci:manejo_de_llaves]
{Sobre el manejo adecuado de llaves}
{clasificacion:sistema}
{
  En donde se usen llaves para la generación y protección de \gls{gl:token},
  se deben seguir buenas prácticas criptográficas para la administración de
  estas. En particular, se deben cumplir con las recomendaciones del
  \gls{gl:nist} en~\cite{nist_llaves} y~\cite{nist_disenio_llaves}.
}

% IT4A-1, RC4B-1
\subrequerimiento[rq_pci:ciclo_de_vida_llaves]
{Sobre el ciclo de vida}
{
  La llave tokenizadora debe seguir la política de los ciclos de llaves
  descritos en el \acrshort{gl:iso}/\acrshort{gl:iec} 115681 (ver
  sección~\ref{sec:administracion_llaves}).
}

% IT4A-2, RC4B-2
\subrequerimiento[rq_pci:periodo_llaves]
{Descripción del periodo criptográfico activo}
{
  La política sobre el tiempo de vida de la llave debe incluir una
  descripción sobre el periodo criptográfico activo de la llave
  tokenizadora en cuestión.
}

% IT4A-3, RC4B-3
\subrequerimiento[rq_pci:destruccion_de_llaves]
{Sobre la destrucción de las llaves}
{
  El proveedor debe incorporar una función que permita la destrucción
  de sus llaves criptográficas sin tener que alterar o abrir el
  dispositivo.
}

% RC1A-1
\subrequerimiento[rq_pci:exportar_llaves_en_claro]
{Exportar llave en claro prohibido}
{
  Las llaves usadas para generar \glspl{gl:token} no se deben poder
  exportar en claro desde el programa.
}

% RC1A-2
% TODO: ¿Cómo funciona la entropía de una fuente generadora?
\subrequerimiento[rq_pci:entropia_generacion_llaves]
{Entropía de generación de llaves}
{
  La fuente generadora de llaves debe tener, al menos, 128 bits de
  \gls{gl:entropia}.
}

% RC1A-3
\subrequerimiento[rq_pci:llaves_de_uso_unico]
{Llaves de uso único}
{
  Las llaves criptográficas usadas para generar \glspl{gl:token} no
  deben ser usadas para ningún otro fin.
}

\subimport{/}{irreversibles}
\subimport{/}{criptograficos_reversibles}
\subimport{/}{no_criptograficos_reversibles}

%En la tabla~\ref{resumen_general} se clasifican los requerimientos presentados
%hasta ahora. Las categorías posibles son: \textit{aplicables}, \textit{no
%aplicables} y \textit{fuera de alcance}. Los \textit{aplicables} son aquellos
%que  se  encuentran dentro del alcance de este trabajo y que, por lo tanto, son
%adoptados como requerimientos propios. Los \textit{fuera de alcance} se dejan
%fuera por están más allá de las posibilidades de este trabajo.

Los siguientes requerimientos son referentes al cumplimiento
de distintos estándares y recomendaciones (principalmente del \gls{gl:nist});
en la sección~\ref{parte:apendices} se resume el contenido de cada uno de
estos:

\begin{itemize}
  \item \hipervinculo{rq_pci:productos_de_hardware}.
  \item \hipervinculo{rq_pci:productos_de_software}.
  \item \hipervinculo{rq_pci:acceso_a_producto}.
  \item \hipervinculo{rq_pci:manejo_de_llaves}.
  \item \hipervinculo{rq_pci:cr_aleatoriedad_digitos}.
\end{itemize}

La tabla~\ref{resumen_general} es una relación entre la lista de
requerimientos aquí presentada y la notación del \gls{gl:pci} \gls{gl:ssc} en
\cite{pci_tokens}. Sobre todo en cuanto a los requerimientos
\hipervinculo{rq_pci:acceso_a_producto} y
\hipervinculo{rq_pci:manejo_de_llaves} el documento del \gls{gl:pci} es
bastante repetitivo: se colocan 3 versiones (una para cada categoría) con
prácticamente el mismo contenido.

\newcommand{\columnaUno}{4.0in}
\newcommand{\columnaDos}{2.0in}

\clearpage
\begin{longtable}{| m{\columnaUno} | m{\columnaDos} |}

  \hline
  \textbf{Requerimiento} &
  \textbf{Equivalente \gls{gl:pci}} \\
  \hline
  \endfirsthead

  \hline
  \multicolumn{2}{|c|}{\textit{Continuación}}\\
  \hline
  \textbf{Requerimiento} &
  \textbf{Equivalente \gls{gl:pci}} \\
  \hline
  \endhead

  \multicolumn{2}{|c|}{\textit{Continúa en siguiente página}}\\
  \hline
  \endfoot

  \endlastfoot

  % Generales:

  \celdaConParrafo{\columnaUno}{%
    \hipervinculo{rq_pci:productos_de_hardware}} &
  \celdaConParrafo{\columnaDos}{GT1} \\\hline

  \celdaConParrafo{\columnaUno}{%
    \hipervinculo{rq_pci:productos_de_software}} &
  \celdaConParrafo{\columnaDos}{GT3} \\\hline

  \celdaConParrafo{\columnaUno}{%
    \hipervinculo{rq_pci:resistencia_texto_claro_conocido}} &
  \celdaConParrafo{\columnaDos}{GT4} \\\hline

  \celdaConParrafo{\columnaUno}{%
    \hipervinculo{rq_pci:resistencia_solo_texto_cifrado}} &
  \celdaConParrafo{\columnaDos}{GT5} \\\hline

  \celdaConParrafo{\columnaUno}{%
    \hipervinculo{rq_pci:deteccion_anomalias}} &
  \celdaConParrafo{\columnaDos}{GT6} \\\hline

  \celdaConParrafo{\columnaUno}{%
    \hipervinculo{rq_pci:distincion_token_pan}} &
  \celdaConParrafo{\columnaDos}{GT7} \\\hline

  \celdaConParrafo{\columnaUno}{
    \hipervinculo{rq_pci:guia_de_instalacion}} &
  \celdaConParrafo{\columnaDos}{GT8} \\\hline

  \celdaConParrafo{\columnaUno}{%
    \hipervinculo{rq_pci:integridad_ejecutables}} &
  \celdaConParrafo{\columnaDos}{GT9} \\\hline

  \celdaConParrafo{\columnaUno}{%
    \hipervinculo{rq_pci:control_de_peticiones}} &
  \celdaConParrafo{\columnaDos}{GT10.1, RC2A-1, RC2A-2 RN2B y RN3B} \\\hline

  \celdaConParrafo{\columnaUno}{%
    \hipervinculo{rq_pci:registros_de_acceso}} &
  \celdaConParrafo{\columnaDos}{GT10.2} \\\hline

  \celdaConParrafo{\columnaUno}{%
    \hipervinculo{rq_pci:autenticacion_multifactor}} &
  \celdaConParrafo{\columnaDos}{GT10.3} \\\hline

  \celdaConParrafo{\columnaUno}{%
    \hipervinculo{rq_pci:acceso_de_sistema}} &
  \celdaConParrafo{\columnaDos}{GT10.4} \\\hline

  \celdaConParrafo{\columnaUno}{%
    \hipervinculo{rq_pci:acceso_administrativo}} &
  \celdaConParrafo{\columnaDos}{GT10.5} \\\hline

  \celdaConParrafo{\columnaUno}{%
    \hipervinculo{rq_pci:token_a_token_prohibido}} &
  \celdaConParrafo{\columnaDos}{GT11} \\\hline

  \celdaConParrafo{\columnaUno}{%
    \hipervinculo{rq_pci:vulnerabilidades_comunes}} &
  \celdaConParrafo{\columnaDos}{GT12} \\\hline

  \celdaConParrafo{\columnaUno}{%
    \hipervinculo{rq_pci:primitivas_usadas}} &
  \celdaConParrafo{\columnaDos}{GT13} \\\hline

  \celdaConParrafo{\columnaUno}{%
    \hipervinculo{rq_pci:ciclo_de_vida_llaves}} &
  \celdaConParrafo{\columnaDos}{IT4A-1 y RC4B-1} \\\hline

  \celdaConParrafo{\columnaUno}{%
    \hipervinculo{rq_pci:periodo_llaves}} &
  \celdaConParrafo{\columnaDos}{IT4A-2 y RC4B-2} \\\hline

  \celdaConParrafo{\columnaUno}{%
    \hipervinculo{rq_pci:destruccion_de_llaves}} &
  \celdaConParrafo{\columnaDos}{IT4A-3 y RC4B-3} \\\hline

  \celdaConParrafo{\columnaUno}{%
    \hipervinculo{rq_pci:exportar_llaves_en_claro} }&
  \celdaConParrafo{\columnaDos}{RC1A-1} \\\hline

  \celdaConParrafo{\columnaUno}{%
    \hipervinculo{rq_pci:entropia_generacion_llaves}} &
  \celdaConParrafo{\columnaDos}{RC1A-2} \\\hline

  \celdaConParrafo{\columnaUno}{%
    \hipervinculo{rq_pci:llaves_de_uso_unico}} &
  \celdaConParrafo{\columnaDos}{RC1A-3} \\\hline

  % Irreversibles:

  \celdaConParrafo{\columnaUno}{%
    \hipervinculo{rq_pci:ir_mecanismo_generador}} &
  \celdaConParrafo{\columnaDos}{IT1A-1} \\\hline

  \celdaConParrafo{\columnaUno}{%
    \hipervinculo{rq_pci:ir_contenido_en_claro}} &
  \celdaConParrafo{\columnaDos}{IT1A-2} \\\hline

  \celdaConParrafo{\columnaUno}{%
    \hipervinculo{rq_pci:ir_diccionario_imposible}} &
  \celdaConParrafo{\columnaDos}{IT1A-3} \\\hline

  \celdaConParrafo{\columnaUno}{%
    \hipervinculo{rq_pci:ir_busqueda_exhaustiva}} &
  \celdaConParrafo{\columnaDos}{IT1A-4} \\\hline

  % Criptográficos:

  \celdaConParrafo{\columnaUno}{%
    \hipervinculo{rq_pci:cr_distribucion_uniforme}} &
  \celdaConParrafo{\columnaDos}{RC1B-1} \\\hline

  \celdaConParrafo{\columnaUno}{%
    \hipervinculo{rq_pci:cr_permutacion_aleatoria}} &
  \celdaConParrafo{\columnaDos}{RC1B-2} \\\hline

  \celdaConParrafo{\columnaUno}{%
    \hipervinculo{rq_pci:cr_cambio_de_llave}} &
  \celdaConParrafo{\columnaDos}{RC1B-3} \\\hline

  \celdaConParrafo{\columnaUno}{%
    \hipervinculo{rq_pci:cr_cambio_de_pan}} &
  \celdaConParrafo{\columnaDos}{RC1B-4} \\\hline

  \celdaConParrafo{\columnaUno}{%
    \hipervinculo{rq_pci:cr_cambio_de_pan}} &
  \celdaConParrafo{\columnaDos}{RC1B-4} \\\hline

  \celdaConParrafo{\columnaUno}{%
    \hipervinculo{rq_pci:cr_aleatoriedad_digitos}} &
  \celdaConParrafo{\columnaDos}{RC1B-5} \\\hline

  \celdaConParrafo{\columnaUno}{%
    \hipervinculo{rq_pci:cr_almacenamiento_tokens}} &
  \celdaConParrafo{\columnaDos}{RC1C} \\\hline

  \celdaConParrafo{\columnaUno}{%
    \hipervinculo{rq_pci:cr_administracion_llaves}} &
  \celdaConParrafo{\columnaDos}{RC4A} \\\hline

  \celdaConParrafo{\columnaUno}{%
    \hipervinculo{rq_pci:cr_longitud_llaves}} &
  \celdaConParrafo{\columnaDos}{RC4C} \\\hline

  \celdaConParrafo{\columnaUno}{%
    \hipervinculo{rq_pci:cr_independencia_estadistica}} &
  \celdaConParrafo{\columnaDos}{RC4D} \\\hline

  % No criptográficos:

  \celdaConParrafo{\columnaUno}{%
    \hipervinculo{rq_pci:nc_generacion_y_almacenamiento}} &
  \celdaConParrafo{\columnaDos}{RN1A} \\\hline

  \celdaConParrafo{\columnaUno}{%
    \hipervinculo{rq_pci:nc_distribucion_equiprobable}} &
  \celdaConParrafo{\columnaDos}{RN1B-1} \\\hline

  \celdaConParrafo{\columnaUno}{%
    \hipervinculo{rq_pci:nc_permutaciones_aleatorias}} &
  \celdaConParrafo{\columnaDos}{RN1B-2} \\\hline

  \celdaConParrafo{\columnaUno}{%
    \hipervinculo{rq_pci:nc_parametros_de_tokenizacion}} &
  \celdaConParrafo{\columnaDos}{RN1B-3} \\\hline

  \celdaConParrafo{\columnaUno}{%
    \hipervinculo{rq_pci:nr_aleatoriedad_digitos}} &
  \celdaConParrafo{\columnaDos}{RN1B-5} \\\hline

  \celdaConParrafo{\columnaUno}{%
    \hipervinculo{rq_pci:nc_distribucion_imparcial}} &
  \celdaConParrafo{\columnaDos}{RN1C} \\\hline

  \celdaConParrafo{\columnaUno}{%
    \hipervinculo{rq_pci:nc_independencia_estadistica}} &
  \celdaConParrafo{\columnaDos}{RN1D} \\\hline

  \celdaConParrafo{\columnaUno}{%
    \hipervinculo{rq_pci:nc_proceso_detokenizacion}} &
  \celdaConParrafo{\columnaDos}{RN2A} \\\hline

  \celdaConParrafo{\columnaUno}{%
    \hipervinculo{rq_pci:nc_cifrado_de_base}} &
  \celdaConParrafo{\columnaDos}{RN3A} \\\hline

  \celdaConParrafo{\columnaUno}{%
    \hipervinculo{rq_pci:nc_administracion_llaves}} &
  \celdaConParrafo{\columnaDos}{RN4A} \\\hline

  \caption{Resumen de requerimientos del \gls{gl:pci} \gls{gl:ssc} para
    los sistemas tokenizadores.}
  \label{resumen_general}

\end{longtable}
