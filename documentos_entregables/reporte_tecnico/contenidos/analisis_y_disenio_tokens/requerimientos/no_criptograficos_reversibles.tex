%
% Requerimientos para tokens no criptográficos reversibles,
% análisis y diseño de generación de tokens.
% Proyecto Lovelace.
%

\subsection{No criptográficos reversibles}

% RN 1A
\requerimiento[rq_pci:nc_generacion_y_almacenamiento]
{Generación y almacenamiento de tokens (no criptográficos)}
{
  La generación de un \gls{gl:token} debe realizarse independientemente de
  su \gls{gl:pan}, y la relación entre un \gls{gl:pan} y su \gls{gl:token}
  sólo tiene que estar almacenada en la base de datos (\gls{gl:cdv})
  establecida. \\
  \nopagebreak[4]
  \textbf{Clasificación:} \hipervinculo{clasificacion:algoritmos}.
}

% RN 1B
\requerimiento[rq_pci:nc_seguridad]
{Probabilidad de encontrar un PAN (no criptográficos)}
{
  La probabilidad de encontrar un \gls{gl:pan} a partir de su respectivo
  \gls{gl:token} debe de ser menor que $1$ en $10^6$. \\
  \nopagebreak[4]
  \textbf{Clasificación:} \hipervinculo{clasificacion:algoritmos}.

  % RN 1B-1
  \subrequerimiento[rq_pci:nc_distribucion_equiprobable]
  {Distribución equiprobable (no criptográficos)}
  {
    Para un \gls{gl:pan} dado, todos sus \glspl{gl:token} respectivos
    deben ser \glspl{gl:equiprobable}, esto es que el sistema
    \textit{tokenizador} no debe exhibir patrones probabilísticos que
    lo vulneren a un ataque estadístico.
  }

  % RN 1B-2
  \subrequerimiento[rq_pci:nc_permutaciones_aleatorias]
  {Permutaciones aleatorias (no criptográficos)}
  {
    El método de tokenización debe actuar como una familia de
    \glspl{gl:permutacion} aleatoria en el espacio efectivo de los
    \glspl{gl:pan} al espacio de \glspl{gl:token}.
  }

  % RN 1B-3
  \subrequerimiento[rq_pci:nc_parametros_de_tokenizacion]
  {Parámetros de tokenización (no criptográficos)}
  {
    El método de tokenización debe incluir parámetros tales que, un
    cambio en estos parámetros resulte en un \gls{gl:token} diferente;
    por ejemplo, un cambio en la instancia del proceso debe derivar en
    una secuencia de \glspl{gl:token} distintos, incluso cuando es usada
    la misma secuencia de \glspl{gl:token}.
  }

  % RN 1B-4
  % Contradoctoria con RN 1A, si el token es independiete, no debe
  % reflejar cambios.
  %\subrequerimiento{Reflejo de cambios}
  %{
  %  Al cambiar parte de un \gls{gl:pan}, debe cambiar su \gls{gl:token}
  %  resultante.
  %}

  % RN 1B-5
  \subrequerimiento[rq_pci:nr_aleatoriedad_digitos]
  {Verificación de la aleatoriedad (no criptográficos)}
  {
    Se debe tener un medio para verificar de forma práctica la aleatorización
    de dígitos, de acuerdo a lo establecido en \gls{gl:nist} 800-90A
    \cite{nist_aleatorios}.
  }
}

En \cite{pci_tokens} se establece un subrequerimiento más de
\hipervinculo{rq_pci:nc_seguridad}: \textit{Al cambiar parte de un
\gls{gl:pan}, debe cambiar su \gls{gl:token} resultante}. Es un requerimiento
análogo a \hipervinculo{rq_pci:cr_cambio_de_pan}, sin embargo en el contexto
de los  no criptográficos reversibles, tal restricción no tiene sentido, dado
que  la generación del \gls{gl:token} es independiente del \gls{gl:pan}
(requerimiento \hipervinculo{rq_pci:nc_generacion_y_almacenamiento}).

% RN 1C
% TODO: ¿Diferencias con RB1B-1?
\requerimiento[rq_pci:nc_distribucion_imparcial]
{Distribución imparcial (no criptográficos)}
{
  El proceso de generación de \glspl{gl:token} debe garantizar una
  distribución de \glspl{gl:token} imparcial, esto significa que la
  probabilidad de cualquier par \gls{gl:pan}/\gls{gl:token} debe ser igual. \\
  \nopagebreak[4]
  \textbf{Clasificación:} \hipervinculo{clasificacion:algoritmos}.
}

% RN 1D
\requerimiento[rq_pci:nc_independencia_estadistica]
{Instancias estadísticamente independientes (no criptográficos)}
{
  Si varias o diferentes instancias de la bases de datos (\gls{gl:cdv}) son
  usadas, cada una de estas debe ser
  \glspl{gl:estidisticamente_independiente}. \\
  \nopagebreak[4]
  \textbf{Clasificación:} \hipervinculo{clasificacion:sistema}.
}

% RN 2A
\requerimiento[rq_pci:nc_proceso_detokenizacion]
{Proceso de detokenización (no criptográficos)}
{
  El proceso de detokenización debe realizarse por medio de una búsqueda de
  datos o un índice dentro de la base de datos (\gls{gl:cdv}), y no por medio
  de métodos criptográficos. \\
  \nopagebreak[4]
  \textbf{Clasificación:} \hipervinculo{clasificacion:algoritmos}.

  % RN 2A-1
  %\subrequerimiento{Asignación probabilísticamente independiente}
  %{
  %  El \gls{gl:pan} y el \gls{gl:token} debe ser probabilísticamente
  %  independientes. Cualquier método lógico o matemático no debe ser
  %  usado para \textit{tokenizar} el \gls{gl:pan} o \textit{detokenizar}
  %  el \gls{gl:token}.
  %}
}

Un subrequerimiento de \hipervinculo{rq_pci:nc_proceso_detokenizacion} que
aquí se omite establece: «\textit{El \gls{gl:pan} y el \gls{gl:token} debe ser
probabilísticamente independientes. Cualquier método lógico o matemático no
debe ser usado para \textit{tokenizar} el \gls{gl:pan} o \textit{detokenizar}
el \gls{gl:token}}». La independencia entre \gls{gl:pan} y \gls{gl:token} ya
se establece en \hipervinculo{rq_pci:nc_generacion_y_almacenamiento}. No es
clara la ascepción de \textit{método lógico matemático}; una solución
común para generar \glspl{gl:token} no criptográficos es usar \gls{gl:prng},
los cuales son métodos matemáticos.

% RN3A
\requerimiento[rq_pci:nc_cifrado_de_base]
{Cifrado de la base de datos (no criptográficos)}
{
  Dentro de la base de datos (\gls{gl:cdv}), los \gls{gl:pan} deben ser
  cifrados con una llave de mínimo 128 bits de \gls{gl:fuerza_efectiva}. \\
  \nopagebreak[4]
  \textbf{Clasificación:} \hipervinculo{clasificacion:implementaciones}.
}

% RN4A
\requerimiento[rq_pci:nc_administracion_llaves]
{Seguridad de la administración de llaves (no criptográficos)}
{
  Todas la operaciones sobre la administración de las llaves criptográficas
  deben realizarse en un dispositivo criptográfico seguro y aprobado: el
  \gls{gl:pci} \gls{gl:ssc} se encarga de hacer validaciones; también puede ser
  cualquier dispositivo validado por \gls{gl:fips} 140-2 nivel 3 o superior
  \cite{nist_modulos_criptograficos} o por la \gls{gl:iso} 13491-1. \\
  \nopagebreak[4]
  \textbf{Clasificación:} \hipervinculo{clasificacion:sistema}.
}
