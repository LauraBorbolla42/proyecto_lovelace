%
% Seguimiento de requerimientos de tokens,
% Capítulo de análisis y diseño de tokens
% Proyecto Lovelace.
%

\subsection{Seguimiento de requerimientos de 
  \texorpdfstring{\glspl{gl:token}}{tokens}}
\label{sec:seguimiento_token}

A continuación se encuentran los requerimientos que caen en la clasificación
del \hipervinculo{clasificacion:implementaciones} mencionados en la sección
\ref{sec:requerimientos}. Se indica, para cada requerimiento, si el
requerimiento fue satisfecho o no y cómo o por qué no.

\begin{itemize}
  \item \textbf{\hipervinculo{rq_pci:distincion_token_pan}.}
    El programa tokenizador es capaz de distinguir entre un \acrshort{gl:pan}
    y un \gls{gl:token} porque el segundo tiene el dígito de verificación
    calculado con el algoritmo de Luhn desfasado en uno (le suma uno al
    resultado). Por lo tanto, este requerimiento es satisfecho.

  \item \textbf{\hipervinculo{rq_pci:integridad_ejecutables}.}
    No hay mecanismos que garanticen la integridad del proceso de generación
    de \glspl{gl:token}; por lo tanto, este requerimiento no es satisfecho.

  \item \textbf{\hipervinculo{rq_pci:token_a_token_prohibido}.}
    El programa verifica que, cuando se va a realizar una tokenización, la
    entrada sea un \acrshort{gl:pan}, por lo que no es posible pasar de un
    \gls{gl:token} a otro \gls{gl:token} sin haber detokenizado primero. Por
    lo tanto, este requerimiento es satisfecho.

  \item \textbf{\hipervinculo{rq_pci:vulnerabilidades_comunes}.}
    \textit{Probablemente este requerimiento sea satisfecho por las banderitas
      del compilador}, RQF7.

  \item \textbf{\hipervinculo{rq_pci:primitivas_usadas}.}
    Los algoritmos que utiliza el programa tokenizador implementan estándares,
    por ejemplo, los cinco algoritmos utilizan AES y varios utilizan SHA-256.
    Por lo tanto, este algoritmo es satisfecho.

  \item \textbf{\hipervinculo{rq_pci:cr_longitud_llaves}.}
    Las llaves utilizadas en los algoritmos son de, mínimo, 128 bits
    (en \ref{sec:ahr}, se utiliza una de 256 bits); por lo tanto, este
    requerimiento es satisfecho.

  \item \textbf{\hipervinculo{rq_pci:nc_cifrado_de_base}.}
    Dado que este es un estudio propio de los algoritmos y su desempeño, y no
    de un sistema tokenizador, se decidió no cifrar los \glspl{gl:token} en la
    base, pues ralentizaban mucho el desempeño. Por lo tanto, este
    requerimiento no es satisfecho.

\end{itemize}
