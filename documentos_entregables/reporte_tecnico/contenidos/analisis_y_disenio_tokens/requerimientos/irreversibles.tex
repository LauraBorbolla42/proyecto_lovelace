%
% Requerimientos para tokens irreversibles,
% análisis y diseño de generación de tokens.
% Proyecto Lovelace.
%

\subsection{Irreversibles}

% IT1A
\requerimientoConClasificacion
{Sobre la generación de 
  \texorpdfstring{\glspl{gl:token}}{tokens} (irreversibles)}
{clasificacion:algoritmos}
{
  El mecanismo utilizado para la generación de los \glspl{gl:token}
  no es reversible (o improbable).
}

% IT1A-1
\subrequerimiento[rq_pci:ir_mecanismo_generador]
{Sobre el mecanismo generador (irreversibles)}
{
  El proceso para crear \glspl{gl:token} clasificado como irreversible
  debe asegurar que el mecanismo, proceso o algoritmo utilizado para
  crear el \gls{gl:token} no sea reversible. Si una función hash (véase
  sección~\ref{sec:hash}) es utilizada, esta debe ser una
  \gls{gl:primitiva_criptografica} y utilizar una llave secreta $k$ tal que
  el mero conocimiento de la función hash no permita la creación de un
  \gls{gl:oraculo}.
}

% IT1A-2
\subrequerimiento[rq_pci:ir_contenido_en_claro]
{Contenido en claro (irreversibles)}
{
  Los \glspl{gl:token} irreversibles no deben contener dígitos en claro del
  \gls{gl:pan} original, excepto que estos dígitos sean una coincidencia.
}

% IT1A-3
\subrequerimiento[rq_pci:ir_diccionario_imposible]
{Creación de un diccionario (irreversibles)}
{
  La creación de una tabla o \textit{diccionario} de \glspl{gl:token}
  estáticos debería ser imposible, o, al menos, al punto de satisfacer que
  la probabilidad de predecir correctamente el \gls{gl:pan} sea menor que
  $\frac{1}{10^6}$.
}

% IT1A-4
\subrequerimiento[rq_pci:ir_busqueda_exhaustiva]
{Sobre el proceso de autenticación (irreversibles)}
{
  En el caso de los \glspl{gl:token} autenticables, el proceso de
  autenticación no debe revelar información suficiente para realizar
  búsquedas, excepto una exhaustiva (\gls{gl:pan} por \gls{gl:pan}) y se
  deben implementar controles para detectar estas últimas.
}

%En la tabla~\ref{resumen_irreversibles} se clasifican los requerimientos
%que en el documento del \gls{gl:pci} \gls{gl:ssc}~\cite{pci_tokens} están
%catalogados como irreversibles. En la lista de este documento se procura no
%hacer tal división, para evitar las repeticiones en exceso que~\cite{pci_tokens}
%manifiesta en algunas ocasiones. Por ejemplo, el requerimiento
%\cite{rq_pci:manejo_de_llaves} es aplicable tanto a los irreversibles como a
%los reversibles criptográficos (y de hecho también a los no criptográficos, dado
%que ahí también se ocupan llaves en la mayoría de los casos), sin embargo,
%en las recomendaciones de \gls{gl:pci} \gls{gl:ssc} se manejan dos versiones
%de lo mismo.
