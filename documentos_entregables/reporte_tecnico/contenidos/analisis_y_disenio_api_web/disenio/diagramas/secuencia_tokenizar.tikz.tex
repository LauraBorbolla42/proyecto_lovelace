%
% Diagrama de secuencia de operación de tokenización.
% Capítulo de análisis y diseño de aplicación web.
% Reporte técnico.
% Proyecto Lovelace.
%

\begin{tikzpicture}[]
  \begin {umlseqdiag}

  % Líneas de vida
  \umlactor[no ddots]{Tienda}
  \umlcontrol[class = Servicio]{servicioTokenizador}
  \umlentity[class = Negocio]{negocioTokenizador}
  \umldatabase[no ddots]{Modelo}
  \umlentity[no ddots]{programaTokenizador}

  % Petición de tokenización
  \begin{umlcall}[%
    op = {GET /tokenizar/\{pan\}},%
    return = 200 OK token]{Tienda}{servicioTokenizador}
    \begin{umlcall}[%
      op = {validarEstado(cliente)},%
      return = true]{servicioTokenizador}{negocioTokenizador}
    \end{umlcall}
    \begin {umlfragment}[%
      type = alt,
      label = irreversible,%
      inner xsep = 5,%
      inner ysep = 2]
      \begin{umlcall}[%
        op = {validarTarjeta(tarjeta)},%
        return = true]{servicioTokenizador}{negocioTokenizador}
        \begin{umlcall}[%
          op = {buscarTarjeta(tarjeta)},%
          return = false]{negocioTokenizador}{Modelo}
        \end{umlcall}
      \end{umlcall}
      \begin{umlcall}[%
        op = {generarToken(método)},%
        return = token]{servicioTokenizador}{programaTokenizador}
        \begin{umlcall}[%
          op = {token.guardar()},%
          return = true]{programaTokenizador}{Modelo}
        \end{umlcall}
      \end{umlcall}
      \umlfpart[reversible]
      \begin{umlcall}[%
        op = {generarToken(método)},%
        return = token]{servicioTokenizador}{programaTokenizador}
      \end{umlcall}
    \end{umlfragment}
  \end{umlcall}

  \end{umlseqdiag}
\end{tikzpicture}
