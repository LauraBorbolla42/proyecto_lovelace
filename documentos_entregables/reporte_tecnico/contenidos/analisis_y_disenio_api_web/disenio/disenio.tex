%
% Sección de diseño, capítulo de análisis y diseño de api web.
% Proyecto Lovelace.
%

\section{Diseño}

En esta sección se detallan los aspectos más importantes del diseño de la
aplicación web que interactúa con el programa generador de tokens.

\subsection{Vista dinámica}

En las figuras \ref{fig:secuencia_registrar_cliente} y
\ref{fig:secuencia_tokenizar} se muestran, a modo de ejemplo, diagramas de
secuencia de los casos de uso para registrar un cliente
(\hipervinculo{cu:registrar_cliente}) y para tokenizar un número de tarjeta
(\hipervinculo{cu:tokenizar}); todos los demás casos de uso mantienen una
estructura análoga. En la figura \ref{fig:secuencia_generia} se muestra el
comportamiento dinámico del sistema para las operaciones \gls{gl:crud}.

\begin{figure}
  \begin{center}
    \subimport{diagramas/}{secuencia_generica.tikz.tex}
    \caption{Diagrama de secuencia para operaciones genéricas.}
    \label{fig:secuencia_generia}
  \end{center}
\end{figure}

\begin{figure}
  \begin{center}
    \subimport{diagramas/}{registrar_cliente.tikz.tex}
    \caption{Diagrama de secuencia para registrar a un cliente.}
    \label{fig:secuencia_registrar_cliente}
  \end{center}
\end{figure}
