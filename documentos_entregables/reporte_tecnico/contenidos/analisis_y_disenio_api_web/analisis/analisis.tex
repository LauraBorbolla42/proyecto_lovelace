%
% Sección de análisis, capítulo de análisis y diseño de api web.
% Proyecto Lovelace.
%

\section{Análisis}

A lo largo de esta sección se utilizan de forma equivalente los términos
\textit{sistema}, \textit{aplicación}, \textit{aplicación web} y
\textit{programa}. Todos hacen referencia al segundo prototipo del
proyecto.

El desarrollo (y por consiguiente, la lectura) de las secciones de reglas de
negocio, requerimientos, casos de uso, interfaces de usuario y mensajes no fue
hecho de forma secuencial, dado el grado de paralelismo que hay entre estas
secciones. Quizá la forma más sencilla de leerlo sea pasar directamente a los
casos de uso, y de ahí ir buscando las referencias a las demás secciones.

\subimport{/}{comparacion_de_sistemas}
\subimport{/}{reglas_de_negocio}
\subimport{/}{requerimientos}
\subimport{casos_de_uso/}{casos_de_uso}
\subimport{/}{interfaces_de_usuario}
\subimport{/}{mensajes}
