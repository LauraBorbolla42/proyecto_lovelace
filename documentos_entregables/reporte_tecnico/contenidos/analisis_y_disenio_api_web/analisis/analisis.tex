%
% Sección de análisis, capítulo de análisis y diseño de api web.
% Proyecto Lovelace.
%
% Sobre las referencias entre secciones:
% Aunque en un párrafo de aquí abajo doy una explicación sobre el supuesto
% paralelismo entre algunas secciones, esto no aplica 100% a la dependencia que
% hay entre estas. Las reglas de negocio son la entidad de más alto nivel,
% existen por sí solas; luego van los requerimientos y hasta el último los
% casos de uso. En otras palabras, en las reglas de negocio no debe haber
% referencias ni a requerimientos ni a casos de uso; en los requerimientos
% solo puede haber referencias a reglas de negocio, pero no a casos de uso;
% en los casos de uso puede haber referencia a requerimientos y reglas de
% negocio. Esto hablando de los casos más generales: tal vez haya por ahí
% algunas excepciones.
%

\section{Análisis}

A lo largo de esta sección se utilizan de forma equivalente los términos
\textit{sistema}, \textit{aplicación}, \textit{aplicación web} y
\textit{programa}. Todos hacen referencia al segundo prototipo del
proyecto. Los términos \textit{pantalla}, \textit{interfaz},
\textit{interfaz gráfica}, \textit{página}, \textit{ventana} y
\textit{formulario} (los dos últimos solamente en donde aplique la
denominación) son utilizados por igual para referirse al estado gráfico
del navegador en un momento predeterminado.

En esta sección se utilizarán \textbf{negritas} para referirse a estados
(\textbf{verificado, aceptado}) o tipos de un elemento
(\textbf{cliente, administrador}) e \textit{itálicas} para referise a elementos
de las interfaces.

El desarrollo (y por consiguiente, la lectura) de las secciones de reglas de
negocio, requerimientos, casos de uso, interfaces de usuario y mensajes no fue
hecho de forma secuencial, dado el grado de paralelismo que hay entre estas
secciones. Quizá la forma más sencilla de leerlo sea pasar directamente a los
casos de uso, y de ahí ir buscando las referencias a las demás secciones.

%%\subimport{/}{comparacion_de_sistemas}
\subimport{/}{reglas_de_negocio}
\subimport{requerimientos/}{requerimientos}
\subimport{casos_de_uso/}{casos_de_uso}
\subimport{/}{interfaces_de_usuario}
\subimport{mensajes/}{mensajes}
\subimport{/}{modelo_de_datos}
\subimport{/}{seguimiento_de_requerimientos}
