%
% Caso de uso para iniciar sesión,
% capítulo de análisis y diseño de api web,
% Proyecto Lovelace.
%

\casoDeUso[cu:iniciar_sesion]
{Iniciar sesión}
{
  Caso que permite que un visitante se identifique ante el sistema. El
  visitante introduce su correo electrónico y contraseña y, dependiendo de
  si es un usuario o un administrador, es redirigido a la página de
  inicio correspondiente.

  \begin{trayectoriaPrincipal}

    \item[origen] El visitante presiona el botón \textit{iniciar sesión}, en
      cualquiera de las dos pantallas principales del sitio público
      (\hipervinculo{iu:inicio} o \hipervinculo{iu:documentacion}).

    \item[interfaz] El sistema muestra la pantalla
      \hipervinculo{iu:iniciar_sesion} con el botón \textit{iniciar sesión}
      deshabilitado.

    \item[datos] El visitante introduce su correo electrónico y contraseña
      [\hipervinculoLocal{ta:cancelar}].

    \item El sistema habilita el botón \textit{iniciar sesión}.

    \item El visitante presiona el botón \textit{iniciar sesión}
      [\hipervinculoLocal{ta:cancelar}].

    \item El sistema valida las credenciales del usuario[
      \hipervinculoLocal{ta:credenciales_incorrectas},
      \hipervinculoLocal{ta:correo_no_verificado},
      \hipervinculoLocal{ta:cliente_no_aprobado},
      \hipervinculoLocal{ta:cliente_rechazado},
      \hipervinculoLocal{ta:cliente_en_lista_negra}].

    \item El sistema registra al usuario en las variables de sesión.

    \item El sistema identifica el tipo de actor que está iniciando
      sesión [\hipervinculoLocal{ta:administrador}].

    \item El sistema obtiene el estado del cliente.

    \item El sistema muestra la pantalla
      \hipervinculo{iu:administracion_de_tokens} tomando en cuenta el estado
      del cliente.

  \end{trayectoriaPrincipal}

  \begin{trayectoriaAlternativa}[ta:cancelar]
    {El visitante cancela la operación}

    \item El visitante presiona el botón \textit{cancelar}.

    \item El sistema muestra la interfaz de origen del paso
      \referenciaLocal{origen} de la trayectoria principal.

  \end{trayectoriaAlternativa}

  \begin{trayectoriaAlternativa}[ta:credenciales_incorrectas]
    {Credenciales incorrectas}

    \item El correo electrónico no se encuentra registrado en el sistema o la
      contraseña dada no corresponde al correo electrónico dado.

    \item El sistema muestra el mensaje
      \hipervinculo{msj:credenciales_incorrectas}.

    \item Se regresa al paso \referenciaLocal{interfaz} de la trayectoria
      principal.

  \end{trayectoriaAlternativa}

  \begin{trayectoriaAlternativa}[ta:correo_no_verificado]
    {Correo no verificado}

    \item El sistema muestra el mensaje
      \hipervinculo{msj:correo_no_verificado}.

    \item Se regresa al paso \referenciaLocal{datos} de la trayectoria
      principal.

  \end{trayectoriaAlternativa}

  \begin{trayectoriaAlternativa}[ta:cliente_no_aprobado]
    {Cliente no aprobado}

    \item El sistema muestra el mensaje
      \hipervinculo{msj:cliente_no_aprobado}.

    \item Se regresa al paso \referenciaLocal{datos} de la trayectoria
      principal.

  \end{trayectoriaAlternativa}

  \begin{trayectoriaAlternativa}[ta:cliente_rechazado]
    {Cliente rechazado}

    \item El sistema muestra el mensaje
      \hipervinculo{msj:cliente_rechazado}.

    \item Se regresa al paso \referenciaLocal{datos} de la trayectoria
      principal.

  \end{trayectoriaAlternativa}

  \begin{trayectoriaAlternativa}[ta:cliente_en_lista_negra]
    {Cliente en lista negra}

    \item El sistema muestra el mensaje
      \hipervinculo{msj:cliente_en_lista_negra}.

    \item Se regresa al paso \referenciaLocal{datos} de la trayectoria
      principal.

  \end{trayectoriaAlternativa}

  \begin{trayectoriaAlternativa}[ta:administrador]
    {El tipo de actor es un administrador}

    \item El sistema recupera la información de los diez primeros clientes
      de los siguientes estados:
      \begin{itemize}
        \item \textbf{aprobados} o \textbf{en cambio de llaves}.
        \item \textbf{verificados}.
        \item \textbf{en lista negra}.
      \end{itemize}

    \item El sistema muestra la pantalla
      \hipervinculo{iu:administracion} con los datos obtenidos en el paso
      anterior.

  \end{trayectoriaAlternativa}
}
