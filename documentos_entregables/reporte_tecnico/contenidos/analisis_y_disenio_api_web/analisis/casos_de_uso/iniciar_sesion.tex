%
% Caso de uso para iniciar sesión,
% Capítulo de análisis y diseño de api web,
% Proyecto Lovelace.
%

\casoDeUso[cu:iniciar_sesion]{API}
{Iniciar sesión}
{
  Caso que permite que un visitante se identifique ante el sistema. El
  visitante introduce su correo electrónico y contraseña y, dependiendo de
  si es un usuario o un administrador, es redirigido a la página de
  inicio correspondiente. Esto en conformidad con lo dispuesto en
  \hipervinculo{rq_api:identificacion_de_actores} y
  \hipervinculo{rq_api:cambios_de_actores}.

  \begin{trayectoriaPrincipal}

    \item[origen] El visitante presiona el botón \textit{iniciar sesión}, en
      cualquiera de las dos pantallas principales del sitio público
      (\hipervinculo{iu:inicio_sitio_publico} o
      \hipervinculo{iu:documentacion}).

    \item[interfaz] El sistema muestra la pantalla
      \hipervinculo{iu:iniciar_sesion} con el botón \textit{iniciar sesión}
      deshabilitado.

    \item El visitante introduce su correo electrónico y contraseña
      [\hipervinculoLocal{ta:cancelar}].

    \item El sistema habilita el botón \textit{iniciar sesión}.

    \item El visitante presiona el botón \textit{iniciar sesión}
      [\hipervinculoLocal{ta:cancelar}].

    \item El sistema valida las credenciales del usuario
      [\hipervinculoLocal{ta:credenciales_incorrectas}].

    \item El sistema registra al usuario en las variables de sesión.

    \item El sistema identifica el tipo de actor que está iniciando
      sesión [\hipervinculoLocal{ta:administrador}].

    \item El sistema muestra la pantalla \hipervinculo{iu:inicio_usuario}.

  \end{trayectoriaPrincipal}

  \begin{trayectoriaAlternativa}[ta:cancelar]
    {El visitante cancela la operación}

    \item El visitante presiona el botón \textit{cancelar}.

    \item El sistema muestra la interfaz de origen del paso
      \referenciaLocal{origen} de la trayectoria principal.

  \end{trayectoriaAlternativa}

  \begin{trayectoriaAlternativa}[ta:credenciales_incorrectas]
    {Credenciales incorrectas}

    \item El correo electrónico no se encuentra registrado en el sistema o la
      contraseña dada no corresponde al correo electrónico dado.

    \item El sistema muestra el mensaje
      \hipervinculo{msj:credenciales_incorrectas}.

    \item Se regresa al paso \referenciaLocal{interfaz} de la trayectoria
      principal.

  \end{trayectoriaAlternativa}

  \begin{trayectoriaAlternativa}[ta:administrador]
    {El tipo de actor es un administrador}

    \item El sistema muestra la pantalla
      \hipervinculo{iu:inicio_administrador}.

  \end{trayectoriaAlternativa}
}
