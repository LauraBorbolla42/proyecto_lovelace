%
% Caso de uso para registrar un usuario,
% capítulo de análisis y diseño de api web,
% Proyecto Lovelace.
%

\casoDeUso[cu:registrar_cliente]
{Registrar cliente}
{
  Caso que permite que un \textbf{visitante} se registre y pase a ser
  un \textbf{cliente} del sistema tokenizador (en conformidad con la regla de
  negocio \hipervinculo{rn:tipo_de_usuario}).

  \begin{trayectoriaPrincipal}

    \item[origen] El visitante presiona el botón \textit{registrarse}, en
      cualquiera de las dos pantallas principales del sitio público
      (\hipervinculo{iu:inicio} o \hipervinculo{iu:documentacion}).

    \item El sistema muestra la pantalla
      \hipervinculo{iu:registrar_usuario} con el botón \textit{aceptar}
      deshabilitado.

    \item[datos] El visitante introduce un correo electrónico y una contraseña
      (junto con su confirmación); [\hipervinculoLocal{ta:cancelar}].

    \item El sistema habilita el botón \textit{aceptar}.

    \item El visitante presiona el botón \textit{aceptar}.

    \item El sistema valida el correo electrónico de acuerdo a la regla
      \hipervinculo{rn:correo_electronico};
      [\hipervinculoLocal{ta:correo_incorrecto}].

    \item El sistema valida el formato de la contraseña de acuerdo a la regla
      \hipervinculo{rn:formato_de_contrasenias};
      [\hipervinculoLocal{ta:contrasenia_incorrecta}].

    \item El sistema valida que la confirmación de la contraseña coincida con
      la contraseña introducida;
      [\hipervinculoLocal{ta:confirmacion_incorrecta}].

    \item El sistema valida que el correo introducido no se encuentre ya asignado
      a un usuario tomando en cuenta la regla de negocios
      \hipervinculo{rn:unicidad_correo};
      [\hipervinculoLocal{ta:usuario_existente}].

    \item El sistema crea un nuevo cliente en estado \textbf{en espera},
      según la regla de negocios \hipervinculo{rn:estados_cliente}, un nuevo
      correo en estado \textbf{no verificado}, según la regla
      \hipervinculo{rn:estados_correo}, y un nuevo vínvulo con la fecha
      actual, según \hipervinculo{rn:hipervinculo_verificacion}.

    \item El sistema manda un correo al cliente con el hipervínculo de
      verificación.

    \item El sistema muestra la ventana
      \hipervinculo{iu:aviso_de_correo_de_confirmacion}.

    \item El cliente accede a la \gls{gl:url} enviada a su correo.

    \item El sistema verifica, mediante la fecha de creación del vínculo, que no
      haya expirado, tomando en cuenta la regla de negocios
      \hipervinculo{rn:verificacion_de_cuentas_nuevas};
      [\hipervinculoLocal{ta:expiracion_de_vinculo}].

    \item El sistema cambia el estado del correo a \textbf{verificado},
      según la regla de negocios \hipervinculo{rn:estados_correo}.

    \item El sistema muestra la ventana
      \hipervinculo{iu:aviso_de_espera_para_aprobacion}.

  \end{trayectoriaPrincipal}

  %%%%%%%%%%%%%%%%%%%%%%%%%%%%%%%%%%%%%%%%%%%%%%%%%%%%%%%%%%%%%%%%%%%%%%%%%%%%%%

  \begin{trayectoriaAlternativa}[ta:cancelar]
    {El visitante cancela la operación}

    \item El visitante presiona el botón \textit{cancelar}.

    \item El sistema muestra la interfaz de origen del paso
      \referenciaLocal{origen} de la trayectoria principal.

  \end{trayectoriaAlternativa}

  %%%%%%%%%%%%%%%%%%%%%%%%%%%%%%%%%%%%%%%%%%%%%%%%%%%%%%%%%%%%%%%%%%%%%%%%%%%%%%

  \begin{trayectoriaAlternativa}[ta:correo_incorrecto]
    {Correo electrónico incorrecto}

    \item El sistema muestra el mensaje \hipervinculo{msj:correo_incorrecto}.

    \item Se regresa al paso \referenciaLocal{datos} de la trayectoria
      principal.

  \end{trayectoriaAlternativa}

  %%%%%%%%%%%%%%%%%%%%%%%%%%%%%%%%%%%%%%%%%%%%%%%%%%%%%%%%%%%%%%%%%%%%%%%%%%%%%%

  \begin{trayectoriaAlternativa}[ta:contrasenia_incorrecta]
    {Contraseña incorrecta}

    \item El sistema muestra el mensaje
      \hipervinculo{msj:contrasenia_incorrecta}.

    \item Se regresa al paso \referenciaLocal{datos} de la trayectoria
      principal.

  \end{trayectoriaAlternativa}

  %%%%%%%%%%%%%%%%%%%%%%%%%%%%%%%%%%%%%%%%%%%%%%%%%%%%%%%%%%%%%%%%%%%%%%%%%%%%%%

  \begin{trayectoriaAlternativa}[ta:confirmacion_incorrecta]
    {Confirmación de contraseña incorrecta}

    \item El sistema muestra el mensaje
      \hipervinculo{msj:confirmacion_incorrecta}.

    \item Se regresa al paso \referenciaLocal{datos} de la trayectoria
      principal.

  \end{trayectoriaAlternativa}

  %%%%%%%%%%%%%%%%%%%%%%%%%%%%%%%%%%%%%%%%%%%%%%%%%%%%%%%%%%%%%%%%%%%%%%%%%%%%%%

  \begin{trayectoriaAlternativa}[ta:usuario_existente]
    {Correo registrado previamente}

    \item El sistema muestra el mensaje
      \hipervinculo{msj:usuario_existente}.

    \item Se regresa al paso \referenciaLocal{datos} de la trayectoria
      principal.

  \end{trayectoriaAlternativa}

  %%%%%%%%%%%%%%%%%%%%%%%%%%%%%%%%%%%%%%%%%%%%%%%%%%%%%%%%%%%%%%%%%%%%%%%%%%%%%%

  \begin{trayectoriaAlternativa}[ta:expiracion_de_vinculo]
    {Vínculo expirado}

    \item El sistema muestra la ventana
      \hipervinculo{iu:aviso_de_expiracion_de_vinculo}.

    \item El sistema elimina el registro del usuario.

  \end{trayectoriaAlternativa}
}
