%
% Caso de uso para tokenizar un número de tarjeta.
% Capítulo de análisis y diseño de api web,
% Proyecto Lovelace.
%

\casoDeUso[cu:tokenizar_tarjeta]
{Tokenizar un número de tarjeta}
{
  Permite a un usuario tipo \textbf{cliente}, dada una conexión establecida, obtener el
  token de una tarjeta.

  \begin{trayectoriaPrincipal}

    \item El cliente envía una petición con los siguientes datos:
      \begin{itemize}
        \item Número de tarjeta a tokenizar.
        \item Algoritmo tokenizador.
      \end{itemize}

    \item El sistema obtiene el estado del cliente.

    \item El sistema verifica que el estado del cliente sea \textbf{aceptado} o
      \textbf{en cambio de llaves} (tomando en cuenta la regla de negocios
      \hipervinculo{rn:tipo_de_usuario});
      [\hipervinculoLocal{ta:cliente_sin_permisos}].

    \item El sistema verifica que el tipo de algoritmo utilizado sea
      \textbf{irreversible} (tomando en cuenta la regla de negocios
      \hipervinculo{rn:tipo_de_algoritmo});
      [\hipervinculoLocal{ta:tokenizacion_reversible}].

    \item El sistema verifica que la tarjeta no tenga asociado un token con ese
      cliente [\hipervinculoLocal{ta:tarjeta_registrada}].

    \item El sistema obtiene el token utilizando el algoritmo especificado y la
      llave asociada al algoritmo con estado \textbf{actual}, tomando en cuenta
      la regla de negocios \hipervinculo{rn:tokenizacion}
      [\hipervinculoLocal{ta:llave_actual_inexistente}].

    \item El sistema registra el token obtenido en el paso anterior en la base
      de datos con estado \textbf{actual} tomando en cuenta la regla de negocios
      \hipervinculo{rn:estados_token}.

    \item [regreso_token] El sistema regresa al cliente el token generado.

  \end{trayectoriaPrincipal}

  \begin{trayectoriaAlternativa}[ta:cliente_sin_permisos]
    {El cliente tiene un estado distinto a \textbf{aceptado} o
      \textbf{en cambio de llaves}.}

    \item El sistema regresa un valor nulo al cliente.
      % Hay que regresar un error aquí.

  \end{trayectoriaAlternativa}

  \begin{trayectoriaAlternativa}[ta:tokenizacion_reversible]
    {El algoritmo especificado por el cliente es de tipo \textbf{reversible}.}

    \item El sistema obtiene el token con el algoritmo especificado utilizando
      la llave con estado \textbf{actual}, tomando en cuenta la regla de
      negocios \hipervinculo{rn:tokenizacion}
      [\hipervinculoLocal{ta:llave_actual_inexistente}].

    \item Regresar al paso \referenciaLocal{regreso_token} de la trayectoria
      principal.

  \end{trayectoriaAlternativa}

  \begin{trayectoriaAlternativa}[ta:tarjeta_registrada]
    {El cliente ya tiene un token relacionado con esa tarjeta.}

    \item El sistema regresa un valor nulo al cliente.
      % Hay que regresar un error aquí.

  \end{trayectoriaAlternativa}

  \begin{trayectoriaAlternativa}[ta:llave_actual_inexistente]
    {El cliente no tiene una llave asociada con estado \textbf{actual}.}

    \item El sistema regresa un valor nulo al cliente.
      % Hay que regresar un error aquí.

  \end{trayectoriaAlternativa}

  \begin{trayectoriaAlternativa}[ta:error_tokenizacion]
    {Ocurrió un error durante el proceso de tokenización.}

    \item El sistema regresa un valor nulo al cliente.
      % Hay que regresar un error aquí.

  \end{trayectoriaAlternativa}

}
