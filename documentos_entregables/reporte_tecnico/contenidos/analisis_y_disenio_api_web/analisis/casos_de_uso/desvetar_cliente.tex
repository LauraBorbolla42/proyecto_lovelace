%
% Caso de uso para quitar a un cliente en la lista negra,
% Capítulo de análisis y diseño de api web,
% Proyecto Lovelace.
%

\casoDeUso[cu:desvetar_cliente]
{Quitar a un cliente en la lista negra}
{
  Permite a un usuario tipo administrador quitar a un cliente en la lista negra,
  cambiando su estado a \textbf{aprobado} (véase
  \hipervinculo{rn:estados_cliente}).

  \begin{trayectoriaPrincipal}

    \item El administrador presiona el botón \textit{Administrar clientes}
      en el menú disponible en todas las pantallas.

    \item El sistema obtiene los datos de todos los clientes en la base de datos
      cuyo estado es \textbf{aprobado}, \textbf{en cambio de llaves},
      \textbf{en lista negra} o \textbf{rechazado}; véase
      [\hipervinculoLocal{ta:sin_clientes}].

    \item El sistema muestra la interfaz \textit{Inserte pantalla aquí} con la
      información obtenida.

    \item El administrador presiona el botón \textit{Quitar de la lista negra}
      correspondiente al cliente que desea vetar.

    \item El sistema cambia el estado de dicho cliente a \textbf{aprobado}
      tomando en cuenta la regla de negocios \hipervinculo{rn:estados_cliente};
      véase [\hipervinculoLocal{ta:error_actualizar_cliente}].

    \item [aprobados_refrescados] El sistema obtiene los datos de todos los
      clientes en la base de datos cuyo estado es \textbf{aprobado},
      \textbf{en cambio de llaves}, \textbf{en lista negra} o
      \textbf{rechazado}.

    \item El sistema refresca la interfaz \textit{Inserte pantalla aquí}.

  \end{trayectoriaPrincipal}

  \begin{trayectoriaAlternativa}[ta:sin_clientes]
    {No hay clientes listos en el sistema.}

    \item La consulta de clientes con estado \textbf{aprobado},
    \textbf{en cambio de llaves}, \textbf{en lista negra} o \textbf{rechazado}
    es vacía.

    \item El sistema muestra el mensaje
      \hipervinculo{msj:sin_clientes_aprobados}.

  \end{trayectoriaAlternativa}

  \begin{trayectoriaAlternativa}[ta:error_actualizar_cliente]
    {Ocurrió un error al actualizar el estado del cliente.}

    \item El sistema muestra el mensaje
      \hipervinculo{msj:error_actualizar_estado_cliente}.

    \item Regresar al paso \referenciaLocal{aprobados_refrescados} de la
      trayectoria principal.

  \end{trayectoriaAlternativa}
}
