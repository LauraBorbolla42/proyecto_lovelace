%
% Caso de uso para retokenizar un token.
% Capítulo de análisis y diseño de api web,
% Proyecto Lovelace.
%

\casoDeUso[cu:retokenizar_token]
{Retokenizar un \texorpdfstring{\gls{gl:token}}{token}}
{
  Permite a un usuario tipo \textbf{aplicación cliente}, dada una conexión
  establecida, obtener el token nuevo a partir de otro token.

  \begin{trayectoriaPrincipal}

    \item La aplicación cliente envía una petición con los siguientes datos:
      \begin{itemize}
        \item Token a retokenizar.
        \item Algoritmo tokenizador.
      \end{itemize}

    \item El sistema obtiene el estado del cliente.

    \item El sistema verifica que el estado del cliente sea
      \textbf{en cambio de llaves} (tomando en cuenta la regla de negocios
      \hipervinculo{rn:estados_cliente});
      [\hipervinculoLocal{ta:cliente_en_estado_incorrecto}].

    \item El sistema verifica que el el token dado sea válido (regla de
      negocios \hipervinculo{rn:formato_token});
      [\hipervinculoLocal{ta:token_invalido}].

    \item El sistema verifica que el tipo de algoritmo utilizado sea
      \textbf{irreversible} (tomando en cuenta la regla de negocios
      \hipervinculo{rn:tipo_de_algoritmo});
      [\hipervinculoLocal{ta:retokenizacion_reversible}].

    \item El sistema verifica que el estado del token sea \textbf{anterior}
      o \textbf{retokenizado}, tomando en cuenta la regla de negocios
      \hipervinculo{rn:estados_token};
      [\hipervinculoLocal{ta:token_inexistente}].

    \item El sistema obtiene el número de tarjeta relacionado con este token
      realizando una consulta a la base de datos.

    \item El sistema obtiene el token correspondiente al número de tarjeta
      obtenido en el paso anterior utilizando el algoritmo especificado y la
      llave asociada al algoritmo con estado \textbf{actual}, tomando en cuenta
      la regla de negocios \hipervinculo{rn:tokenizacion}.

    \item El sistema registra el token obtenido en el paso anterior en la base
      de datos con estado \textbf{actual} tomando en cuenta la regla de negocios
      \hipervinculo{rn:estados_token}.

    \item El sistema cambia el estado del token recibido en la petición
      a \textbf{retokenizado}.

    \item [regreso_token] El sistema regresa a la aplicación cliente el token
      generado.

  \end{trayectoriaPrincipal}

  %%%%%%%%%%%%%%%%%%%%%%%%%%%%%%%%%%%%%%%%%%%%%%%%%%%%%%%%%%%%%%%%%%%%%%%%%%%%%%

  \begin{trayectoriaAlternativa}[ta:cliente_en_estado_incorrecto]
    {El cliente tiene un estado distinto a \textbf{en cambio de llaves}}

    \item El sistema regresa una respuesta \gls{gl:http} con código 403.

  \end{trayectoriaAlternativa}

  %%%%%%%%%%%%%%%%%%%%%%%%%%%%%%%%%%%%%%%%%%%%%%%%%%%%%%%%%%%%%%%%%%%%%%%%%%%%%%

  \begin{trayectoriaAlternativa}[ta:retokenizacion_reversible]
    {El algoritmo especificado por el cliente es de tipo \textbf{reversible}}

    \item El sistema obtiene el número de tarjeta con el token y la llave
      asociada al algoritmo con estado \textbf{anterior}.

    \item El sistema obtiene el token correspondiente al número de tarjeta
      mediante el algoritmo especificado y la llave asociada con estado
      \textbf{actual}.

    \item Regresar al paso \referenciaLocal{regreso_token} de la trayectoria
      principal.

  \end{trayectoriaAlternativa}

  %%%%%%%%%%%%%%%%%%%%%%%%%%%%%%%%%%%%%%%%%%%%%%%%%%%%%%%%%%%%%%%%%%%%%%%%%%%%%%

  \begin{trayectoriaAlternativa}[ta:token_inexistente]
    {EL token especificado no se encuentra registrado en la base de datos}

    \item El sistema incrementa el contador de malas acciones del usuario
      en tres unidades (regla de negocios
      \hipervinculo{rn:malas_acciones}).
    \item El sistema verifica que el contador de malas acciones del usuario no
      sea igual o mayor al umbral permitido
      (\hipervinculo{rn:top_malas_acciones});
      [\hipervinculoLocal{ta:muchas_malas_acciones}].

    \item El sistema regresa una respuesta \gls{gl:http} con código 400.

  \end{trayectoriaAlternativa}

  %%%%%%%%%%%%%%%%%%%%%%%%%%%%%%%%%%%%%%%%%%%%%%%%%%%%%%%%%%%%%%%%%%%%%%%%%%%%%%

  \begin{trayectoriaAlternativa}[ta:token_invalido]
    {Token con formato inválido}

    \item El sistema incrementa el contador de malas acciones del usuario
      en una unidad (regla de negocios \hipervinculo{rn:malas_acciones}).
    \item El sistema verifica que el contador de malas acciones del usuario no
      sea igual o mayor al umbral permitido
      (\hipervinculo{rn:top_malas_acciones});
      [\hipervinculoLocal{ta:muchas_malas_acciones}].

    \item El sistema regresa una respuesta \gls{gl:http} con código 400.

  \end{trayectoriaAlternativa}

  %%%%%%%%%%%%%%%%%%%%%%%%%%%%%%%%%%%%%%%%%%%%%%%%%%%%%%%%%%%%%%%%%%%%%%%%%%%%%%

  \begin{trayectoriaAlternativa}[ta:muchas_malas_acciones]
    {Límite de malas acciones alcanzado}

    \item El sistema cambia el estado del cliente a \textbf{en lista negra}.

    \item El sistema regresa una respuesta \gls{gl:http} con código 400.

  \end{trayectoriaAlternativa}

}
