%
% Caso de uso para poner a un cliente en la lista negra,
% Capítulo de análisis y diseño de api web,
% Proyecto Lovelace.
%

\casoDeUso[cu:vetar_cliente]
{Poner a un cliente en la lista negra}
{
  Permite a un usuario tipo administrador poner a un cliente en la lista negra,
  cambiando su estado a \textbf{en lista negra} (véase
  \hipervinculo{rn:estados_cliente}).

  \begin{trayectoriaPrincipal}

    \item El administrador presiona el botón \textit{Vetar}
      correspondiente al cliente que desea vetar en la interfaz
      \hipervinculo{iu:administracion}.

    \item El sistema muestra el mensaje \hipervinculo{msj:vetar_cliente}.

    \item El administrador presiona \textit{Aceptar}
      [\hipervinculoLocal{ta:cancelar}].

    \item El sistema cambia el estado de dicho cliente a \textbf{en lista negra}
      tomando en cuenta la regla de negocios \hipervinculo{rn:estados_cliente}.

    \item El sistema manda un correo para avisar al cliente de su
      ingreso a la lista negra.

    \item El sistema refresca la interfaz \hipervinculo{iu:administracion}.

  \end{trayectoriaPrincipal}

  \begin{trayectoriaAlternativa}[ta:cancelar]
    {Cancelar operación}

    \item El administrador presiona \textit{Cancelar}.

  \end{trayectoriaAlternativa}
}
