%
% Caso de uso para actualizar la información de un cliente.
% Capítulo de análisis y diseño de api web,
% Proyecto Lovelace.
%

\casoDeUso[cu:actualizar_cliente]
{Actualizar información de un cliente}
{
  Permite a un usuario tipo \textbf{cliente} actualizar sus datos.

  \begin{trayectoriaPrincipal}

    \item El cliente presiona \textit{Actualizar información} en la
      interfaz \hipervinculo{iu:control}.

    \item El sistema muestra la interfaz
      \hipervinculo{actualizar_cliente} con el botón \textit{aceptar}
      deshabilitado.

    \item[datos] El cliente introduce un correo electrónico y una contraseña
      (junto con su confirmación); [\hipervinculoLocal{ta:cancelar}].

    \item El sistema habilita el botón \textit{aceptar}.

    \item El cliente presiona el botón \textit{aceptar}.

    \item El sistema valida el correo electrónico de acuerdo a la regla
      \hipervinculo{rn:correo_electronico};
      [\hipervinculoLocal{ta:correo_incorrecto}].

    \item El sistema valida el formato de la contraseña de acuerdo a la regla
      \hipervinculo{rn:formato_de_contrasenias};
      [\hipervinculoLocal{ta:contrasenia_incorrecta}].

    \item El sistema valida que la confirmación de la contraseña coincida con
      la contraseña introducida;
      [\hipervinculoLocal{ta:confirmacion_incorrecta}].

    \item El sistema valida que el correo introducido no se encuentre ya
      asignado a un usuario, tomando en cuenta la regla de negocios
      \hipervinculo{rn:unicidad_correo};
      [\hipervinculoLocal{ta:usuario_existente}].

    \item El sistema registra la nueva dirección del cliente en estado
      \textbf{no verificado}, según la regla de negocios
      \hipervinculo{rn:estados_correo} y un nuevo vínculo tomando en cuenta
      la regla de negocios
      \hipervinculo{rn:hipervinculo_verificacion}.

    \item El sistema elimina el correo con estado con estado \textbf{verificado}
      (regla de negocion \hipervinculo{rn:estados_correo}).

    \item El sistema manda un correo al cliente con un hipervínculo de
      verificación.

    \item El sistema muestra la ventana
      \hipervinculo{iu:aviso_de_correo_de_confirmacion}.

    \item El cliente accede a la \gls{gl:url} enviada a su correo.

    \item El sistema verifica el correo, tomando en cuenta la regla de negocios
      \hipervinculo{rn:verificacion_de_actualizacion_de_cuentas}.

    \item El sistema cambia el estado del correo del cliente a
      \textbf{verificado}, según la regla de negocios
      \hipervinculo{rn:estados_correo}.

    \item El sistema muestra la ventana
      \hipervinculo{iu:aviso_de_verificacion_exitosa}.

  \end{trayectoriaPrincipal}

  %%%%%%%%%%%%%%%%%%%%%%%%%%%%%%%%%%%%%%%%%%%%%%%%%%%%%%%%%%%%%%%%%%%%%%%%%%%%%%

  \begin{trayectoriaAlternativa}[ta:cancelar]
    {El visitante cancela la operación}

    \item El visitante presiona el botón \textit{cancelar}.

    \item El sistema muestra la interfaz de origen del paso
      \referenciaLocal{origen} de la trayectoria principal.

  \end{trayectoriaAlternativa}

  %%%%%%%%%%%%%%%%%%%%%%%%%%%%%%%%%%%%%%%%%%%%%%%%%%%%%%%%%%%%%%%%%%%%%%%%%%%%%%

  \begin{trayectoriaAlternativa}[ta:correo_incorrecto]
    {Correo electrónico incorrecto}

    \item El sistema muestra el mensaje \hipervinculo{msj:correo_incorrecto}.

    \item Se regresa al paso \referenciaLocal{datos} de la trayectoria
      principal.

  \end{trayectoriaAlternativa}

  %%%%%%%%%%%%%%%%%%%%%%%%%%%%%%%%%%%%%%%%%%%%%%%%%%%%%%%%%%%%%%%%%%%%%%%%%%%%%%

  \begin{trayectoriaAlternativa}[ta:contrasenia_incorrecta]
    {Contraseña incorrecta}

    \item El sistema muestra el mensaje
      \hipervinculo{msj:contrasenia_incorrecta}.

    \item Se regresa al paso \referenciaLocal{datos} de la trayectoria
      principal.

  \end{trayectoriaAlternativa}

  %%%%%%%%%%%%%%%%%%%%%%%%%%%%%%%%%%%%%%%%%%%%%%%%%%%%%%%%%%%%%%%%%%%%%%%%%%%%%%

  \begin{trayectoriaAlternativa}[ta:confirmacion_incorrecta]
    {Confirmación de contraseña incorrecta}

    \item El sistema muestra el mensaje
      \hipervinculo{msj:confirmacion_incorrecta}.

    \item Se regresa al paso \referenciaLocal{datos} de la trayectoria
      principal.

  \end{trayectoriaAlternativa}

  %%%%%%%%%%%%%%%%%%%%%%%%%%%%%%%%%%%%%%%%%%%%%%%%%%%%%%%%%%%%%%%%%%%%%%%%%%%%%%

  \begin{trayectoriaAlternativa}[ta:usuario_existente]
    {Correo registrado previamente}

    \item El sistema muestra el mensaje
      \hipervinculo{msj:usuario_existente}.

    \item Se regresa al paso \referenciaLocal{datos} de la trayectoria
      principal.

  \end{trayectoriaAlternativa}

}
