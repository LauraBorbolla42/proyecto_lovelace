%
% Caso de uso para iniciar el refresco de llaves.
% Capítulo de análisis y diseño de api web,
% Proyecto Lovelace.
%

\casoDeUso[cu:iniciar_refresco]
{Iniciar el refresco de llaves}
{
  Permite a un usuario tipo \textbf{cliente} iniciar el refresco de llaves.

  \begin{trayectoriaPrincipal}

    \item El cliente presiona \textit{Iniciar refresco de llaves} en la interfaz
      \textit{inserte pantalla aquí}.

    \item El sistema obtiene el estado del cliente.

    \item El sistema verifica que el estado del cliente sea
      \textbf{aceptado} (tomando en cuenta la regla de negocios
      \hipervinculo{rn:estados_cliente});
      [\hipervinculoLocal{ta:cliente_en_cambio}].

    \item El sistema cambia el estado de todos los tokens relacionados a él, con
      estado \textbf{actual} a \textbf{anterior}
      [\hipervinculoLocal{ta:error_actualizar_token}].

    \item El sistema cambia el estado de todas las llaves asociadas con el
      usuario en estado \textbf{actual} a \textbf{anterior}
      [\hipervinculoLocal{ta:error_actualizar_llave}].

    \item El sistema crea un juego nuevo de llaves y las asocia al cliente
      con estado \textbf{actual} [\hipervinculoLocal{ta:error_crear_llave}].

    \item El sistema cambia el estado del usuario a
      \textbf{en cambio de llaves}
      [\hipervinculoLocal{ta:error_actualizar_estado_cliente}].

  \end{trayectoriaPrincipal}

  \begin{trayectoriaAlternativa}[ta:cliente_en_cambio]
    {El cliente se encuentra en estado \textbf{en cambio de llaves}.}

    \item El sistema muestra el mensaje
      \hipervinculo{msj:error_cliente_ya_esta_en_cambio}.

  \end{trayectoriaAlternativa}

  \begin{trayectoriaAlternativa}[ta:error_actualizar_token]
    {Ocurrió un error al actualizar el estado de los tokens.}

    \item El sistema muestra el mensaje
      \hipervinculo{msj:error_iniciar_refresco}.

    %% Se asume que estamos usando transacciones en la base de datos y, si no
    %% se completan exitosamente, se hace un rollback; si no, va a quedar
    %% toda inconsistente nuestra base. También podemos asumir que todos los
    %% tokens tenian estado actual, así que un paso aquí sería poner todos en
    %% estado actual de nuevo.

  \end{trayectoriaAlternativa}

  \begin{trayectoriaAlternativa}[ta:error_actualizar_llave]
    {Ocurrió un error al actualizar el estado de las llaves.}

    \item El sistema cambia el estado de todos los tokens con estado
      \textbf{anterior} a \textbf{actual}.

    \item El sistema muestra el mensaje
      \hipervinculo{msj:error_iniciar_refresco}.

    %% Aqui también se asume que estamos usando transacciones en la base de
    %% datos y, si no se completan exitosamente, se hace un rollback; si no, va
    %% a quedar toda inconsistente nuestra base.

  \end{trayectoriaAlternativa}

  \begin{trayectoriaAlternativa}[ta:error_crear_llave]
    {Ocurrió un error al crear una llave.}

    \item El sistema cambia el estado de todos los tokens con estado
      \textbf{anterior} a \textbf{actual}.

    \item El sistema elimina todas las llaves asociadas con estado
      \textbf{actual}.

    \item El sistema cambia el estado de todas las llaves con estado
      \textbf{anterior} a \textbf{actual}.

    \item El sistema muestra el mensaje
      \hipervinculo{msj:error_iniciar_refresco}.

    %% Aqui también se asume que estamos usando transacciones en la base de
    %% datos y, si no se completan exitosamente, se hace un rollback; si no, va
    %% a quedar toda inconsistente nuestra base.

  \end{trayectoriaAlternativa}

  \begin{trayectoriaAlternativa}[ta:error_actualizar_estado_cliente]
    {Ocurrió un error al actualizar el estado del cliente.}

    \item El sistema cambia el estado de todos los tokens con estado
      \textbf{anterior} a \textbf{actual}.

    \item El sistema elimina todas las llaves asociadas con estado
      \textbf{actual}.

    \item El sistema cambia el estado de todas las llaves con estado
      \textbf{anterior} a \textbf{actual}.

    \item El sistema muestra el mensaje
      \hipervinculo{msj:error_iniciar_refresco}.

    %% Aqui también se asume que estamos usando transacciones en la base de
    %% datos y, si no se completan exitosamente, se hace un rollback; si no, va
    %% a quedar toda inconsistente nuestra base.

  \end{trayectoriaAlternativa}
}
