%
% Caso de uso para terminar el refresco de llaves.
% Capítulo de análisis y diseño de api web,
% Proyecto Lovelace.
%

\casoDeUso[cu:terminar_refresco]
{Terminar el refresco de llaves}
{
  Permite a un usuario tipo \textbf{cliente} terminar el refresco de llaves.

  \begin{trayectoriaPrincipal}

    \item El cliente presiona \textit{Terminar refresco de llaves} en la
      interfaz \textit{inserte pantalla aquí}.

    \item El sistema obtiene el estado del cliente.

    \item El sistema verifica que el estado del cliente sea
      \textbf{en cambio de llaves} (tomando en cuenta la regla de negocios
      \hipervinculo{rn:estados_cliente});
      [\hipervinculoLocal{ta:cliente_no_en_cambio}].

    \item El sistema verifica que no haya tokens relacionados con el cliente
      que se encuentren en estado \textbf{anterior}.
      [\hipervinculoLocal{ta:retokenizacion_incompleta}].

    \item[llaves] El sistema elimina todas las llaves asociadas al cliente con
      estado \textbf{anterior} [\hipervinculoLocal{ta:error_eliminar_llave}].

    \item El sistema cambia el estado de todos los tokens relacionados con el
      cliente que se encuentran en estado \textbf{retokenizado} a
      \textbf{actual} [\hipervinculoLocal{ta:error_actualizar_token}].

    \item El sistema cambia el estado del usuario a
      \textbf{aceptado} [\hipervinculoLocal{ta:error_actualizar_estado_cliente}].

  \end{trayectoriaPrincipal}

  \begin{trayectoriaAlternativa}[ta:cliente_no_en_cambio]
    {El cliente se encuentra en un estado distinto a
      \textbf{en cambio de llaves}.}

    \item El sistema muestra el mensaje
      \hipervinculo{msj:error_cliente_no_esta_en_cambio}.

  \end{trayectoriaAlternativa}

  \begin{trayectoriaAlternativa}[ta:retokenizacion_incompleta]
    {El cliente aún tiene tokens relacionados con estado \textbf{anterior}.}

    \item El sistema muestra el mensaje
      \hipervinculo{msj:adv_retokenizacion_incompleta}.

    \item El usuario presiona \textit{Aceptar}
      [\hipervinculoLocal{ta:cancelar_terminar_refresco}].

    \item El sistema elimina todos los tokens relacionados con el cliente que se
      encuentran en estado \textbf{anterior}.

    \item El sistema regresa al paso \referenciaLocal{llaves} de la trayectoria
      principal.

  \end{trayectoriaAlternativa}

  \begin{trayectoriaAlternativa}[ta:error_eliminar_llave]
    {Ocurrió un error al eliminar las llaves.}

    \item El sistema muestra el mensaje
      \hipervinculo{msj:error_terminar_refresco}.

  \end{trayectoriaAlternativa}

  \begin{trayectoriaAlternativa}[ta:error_actualizar_token]
    {Ocurrió un error al actualizar el estado de los tokens.}

    \item El sistema muestra el mensaje
      \hipervinculo{msj:error_terminar_refresco}.

  \end{trayectoriaAlternativa}

  \begin{trayectoriaAlternativa}[ta:error_actualizar_estado_cliente]
    {Ocurrió un error al actualizar el estado del cliente.}

    \item El sistema muestra el mensaje
      \hipervinculo{msj:error_terminar_refresco}.

  \end{trayectoriaAlternativa}

  \begin{trayectoriaAlternativa}[ta:cancelar_terminar_refresco]
    {El cliente cancela el terminar con el refresco de llaves.}

    \item El usuario presiona \textit{Cancelar}.

    \item El sistema muestra la interfaz de inicio
      \textit{inserte pantalla aquí}.

  \end{trayectoriaAlternativa}
}
