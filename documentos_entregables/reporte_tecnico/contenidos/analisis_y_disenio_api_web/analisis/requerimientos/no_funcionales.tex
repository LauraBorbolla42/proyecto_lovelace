%
% Lista de requerimientos funcionales,
% Capítulo de análisis y diseño de api web.
% Proyecto Lovelace
%

\subsubsection{No funcionales}

\requerimientoNoFuncional[rqnf_api:tamanios_de_pantalla]
{Sobre los tamaños de pantallas}
{
  Las interfaces gráficas de la aplicación deben estar diseñadas para
  responder a distintos tamaños de pantalla. En particular, se debe
  mostrar cómo se ve la interfaz para los siguientes tamaños (las unidades están
  en pixeles): muy grande (1920x1080), grande (1280x800), mediano (900x960),
  pequeño (600x960) y muy pequeño (480x800).
}

\requerimientoNoFuncional[rqnf_api:alamcenamiento_de_contrasenias]
{Sobre el almacenamiento de contraseñas}
{
  Las contraseñas nunca se almacenan en la base de datos. Lo que se almacena es
  un hash (sección \ref{sec:hash}); al momento de identificar a un usuario
  (\hipervinculo{rq:iniciar_sesion}) se calcula el hash de la
  contraseña introducida y se compara contra el que está almacenado en la base
  de datos. El algoritmo utilizado para estos hash debe de ser \gls{gl:sha}-256.
}

\requerimientoNoFuncional[rqnf_api:encabezados_de_autenticacion]
{Sobre los encabezados de autenticación}
{
  Todas las peticiones de tokenización, detokenización y retokenización deben
  hacer uso de los encabezados de autenticación provistos por el protocolo
  HTTPS.
}
