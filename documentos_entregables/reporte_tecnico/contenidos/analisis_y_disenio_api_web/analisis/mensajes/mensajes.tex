%
% Catálogo de mensajes para sistema tokenizador
% Capítulo de análisis y diseño de api web
% Prpyecto Lovelace.
%
% El contenido del mensaje es un archivo separado para poder manejarlo
% como recurso estático en la aplicación web. #DRY
%

\subsection{Catálogo de mensajes}

% Iniciar sesión %%%%%%%%%%%%%%%%%%%%%%%%%%%%%%%%%%%%%%%%%%%%%%%%%%%%%%%%%%%%%%

% La disyunción no aplica de forma directa: evidentemente no se puede dar el
% caso de que el nombre de usuario sea incorrecto pero la contraseña sí sea
% correcta.
\mensaje[msj:credenciales_incorrectas]
{Credenciales incorrectas}{\subimport{/}{credenciales_incorrectas}}

\mensaje[msj:cliente_no_verificado]
{Cliente no verificado}{\subimport{/}{cliente_no_verificado}}

\mensaje[msj:cliente_no_aprobado]
{Cliente no aprobado}{\subimport{/}{cliente_no_aprobado}}

\mensaje[msj:cliente_rechazado]
{Cliente rechazado}{\subimport{/}{cliente_rechazado}}

\mensaje[msj:cliente_en_lista_negra]
{Cliente en lista negra}{\subimport{/}{cliente_en_lista_negra}}

% Registrar usuario %%%%%%%%%%%%%%%%%%%%%%%%%%%%%%%%%%%%%%%%%%%%%%%%%%%%%%%%%%%

<<<<<<< HEAD
=======
% TODO:
% * Terminar de definir expresión regular del correo electrónico.
% * ¿Mostrarle una expresión regular al usuario?
>>>>>>> laborb/analisis
\mensaje[msj:correo_incorrecto]
{Correo electrónico incorrecto}{\subimport{/}{correo_incorrecto}}

\mensaje[msj:contrasenia_incorrecta]
{Contraseña incorrecta}{\subimport{/}{contrasenia_incorrecta}}

\mensaje[msj:confirmacion_incorrecta]
{Confirmación de contraseña incorrecta}{\subimport{/}{confirmacion_incorrecta}}

\mensaje[msj:usuario_existente]
{Correo registrado previamente}{\subimport{/}{usuario_existente}}
