%
% Reglas de negocio de sistema tokenizador,
% Capítulo de análisis y diseño de api web.
% Proyecto Lovelace
%

\subsection{Reglas de negocio}

El formato de las \glspl{gl:expresion_regular}, por cuestiones de optimización
en tiempos de desarrollo, se presenta según el formato de \textit{java script},
definido en \cite{mozilla_er}.

% \reglaDeNegocio[rn:servicio_tokenizacion]
% {Servicio de tokenizacion}
% {
%   El servicio de tokenización abarca dos operaciones básicas:
%   \textit{tokenización}, esto es, dado un número de tarjeta, regresar el token
%   correspondiente; y \textit{detokenización}, que es el proceso inverso, dado un
%   token, regresar el número de tarjeta asociado.
%
%   Esta regla está asociada a \hipervinculo{rn:formato_numero_de_tarjeta} y
%   \hipervinculo{rn:formato_token}.
% }

\reglaDeNegocio[rn:formato_numero_de_tarjeta]
{Composición de números de tarjeta}
{
  Un número de tarjeta queda definido por la expresión regular
  \texttt{[0-9]\{12, 19\}}. El número se divide en tres partes: el
  identificador del emisor (primeros 6 dígitos), el número de cuenta y el dígito
  verificador (último dígito). Para más detalles sobre este tema consultar la
  sección \ref{sec:composicion_tarjeta}. El dígito verificador se obtiene de tal
  forma que el algoritmo de Luhn (sección \ref{sec:algoritmo_luhn}) dé como
  resultado cero.
}

\reglaDeNegocio[rn:formato_token]
{Composición de tokens}
{
  El formato de un token es el mismo que para un número de tarjeta
  (\hipervinculo{rn:formato_numero_de_tarjeta}) con la única excepción de que el
  último dígito, el dígito verificador, se calcule de forma que el algoritmo de
  Luhn (sección \ref{sec:algoritmo_luhn}) dé como resultado un uno.
}

% TODO: encontrar una manera para mostrar esto.
\reglaDeNegocio[rn:correo_electronico]
{Formato de correos electrónicos}
{
  Un nombre de usuario queda definido por la expresión regular
  %% Está medio loca esta validación, ¿no?
  %\begin{verbatim}
  %^(([^<>()\[\]\\.,;:\s@"]+(\.[^<>()\[\]\\.,;:\s@"]+)*)|(".+"))@((\[[0-9]{1,3}\.[0-9]{1,3}\.[0-9]{1,3}\.[0-9]{1,3}\])|(([a-zA-Z\-0-9]+\.)+[a-zA-Z]{2,}))$
  %\end{verbatim}
}

% TODO: ¿Agregar símbolos a expresión regular de contraseñas?

\reglaDeNegocio[rn:formato_de_contrasenias]
{Formato de contraseñas}
{
  El formato de una contraseña está regido por la expresión regular
  \texttt{0-9a-zA-Z}\{8, 24\}.
}

\reglaDeNegocio[rn:tipo_de_usuario]
{Tipos de usuario}
{
  Existen tres tipos de usuarios:
  \begin{itemize}
    \item Visitante
    \item Cliente
    \item Administrador
  \end{itemize}

  \begin{hangparas}{2em}{0}
    Al abrir la aplicación, todos los usuarios son visitantes; pues los clientes
    y los administradores deben iniciar sesión para ser identificados por el
    sistema; un vistante puede convertirse en cliente al registrarse y ser
    aprobado. Las transiciones entre los tipos de usuarios pueden observarse en
    el diagrama \ref{estados_actores}.
  \end{hangparas}
}

\begin{figure}
  \begin{center}
    \subimport{diagramas/}{estados_actores.tikz.tex}
    \caption{Diagrama de estados de actores.}
    \label{estados_actores}
  \end{center}
\end{figure}

\reglaDeNegocio[rn:estados_cliente]
{Estados de un cliente}
{
  Un cliente puede tener los siguientes estados:
  \begin{itemize}
    \item No verificado
    \item Verificado
    \item Aprobado
    \item Rechazado
    \item En cambio de llave
    \item En lista negra
  \end{itemize}

  \begin{hangparas}{2em}{0}
    Cuando un cliente se registra, se encuentra en estado \textbf{no verificado},
    cuando verifica su dirección de correo electrónico, pasa a
    \textbf{verificado}; una vez que el administrador lo aprueba, pasa a
    \textbf{aprobado}. Cuando inicia un refresco de llaves, pasa a
    \textbf{en cambio de llaves}. Finalmente, el administrador puede cambiar su
    estado a \textbf{en lista negra} (y regresarlo a \textbf{aprobado} o
    \textbf{en cambio de llaves}).
  \end{hangparas}
}

\reglaDeNegocio[rn:estados_token]
{Estados de un token}
{
  Un token puede tener los siguientes estados:
  \begin{itemize}
    \item Actual
    \item Anterior
    \item Retokenizado
  \end{itemize}

  \begin{hangparas}{2em}{0}
    Cuando un token es creado, su estado es \textbf{actual}.
    Cuando un cliente inicia el refresco de llaves, todos sus tokens cambian de
    estado a \textbf{anterior}.
    Cuando un cliente solicita la retokenización de un token, se cambia su estado
    a \textbf{retokenizado}.
  \end{hangparas}
}

\reglaDeNegocio[rn:estados_llave]
{Estados de una llave}
{
  Una llave puede tener los siguientes estados:
  \begin{itemize}
    \item Actual
    \item Anterior
  \end{itemize}

  \begin{hangparas}{2em}{0}
    Cuando una llave es creada, su estado es \textbf{actual}. Si el cliente
    tenía asociada otra llave, el estado de la llave existente cambia a
    \textbf{anterior}.
  \end{hangparas}
}

\reglaDeNegocio[rn:tokenizacion_en_refresco]
{Petición de tokenización en proceso de refresco}
{
  Cuando un cliente solicite una operación de tokenización y se encuentre en
  estado \textbf{en cambio de llaves}, se debe utilizar la llave nueva para
  obtener el token.
}

\reglaDeNegocio[rn:detokenizacion_en_refresco]
{Petición de detokenización en proceso de refresco}
{
  Cuando un cliente solicite una operación de detokenización y se encuentre en
  estado \textbf{en cambio de llaves}, se debe especificar en la petición si
  se desea utilizar la llave nueva o anterior. En caso de no ser especificada,
  se realizará la operación utilizando la llave anterior.
}

\reglaDeNegocio[rn:habilitacion_de_retokenizacion]
{Habilitación de retokenización}
{
  La operación de retokenización solo está habilitada para los clientes cuyo
  estado es \textbf{en cambio de llaves}.

}

\reglaDeNegocio[rn:criptoperiodo_llave]
{Criptoperiodo de una llave}
{
  El criptoperiodo de todas las llaves usadas para la tokenización (y
  detokenización) es de seis meses.
}

\reglaDeNegocio[rn:unicidad_tarjeta_cliente]
{Unicidad de número de tarjeta por cliente}
{
  Un cliente solo puede tener un token por tarjeta; esto es, no puede realizar,
  para un mismo número de tarjeta, la operación de tokenización más de una vez.
}

\reglaDeNegocio[rn:aprobacion_cliente]
{Aprobación de clientes}
{
  Un administrador puede aprobar o rechazar usuarios tipo \textbf{cliente}
cuando estos tengan estado \textbf{verificado}.
}

\reglaDeNegocio[rn:acceso_no_autorizado]
{Acceso no autorizados}
{
  Un acceso no autorizado se da cuando un usuario hace una petición al servidor
  para la cual no tiene los privilegios necesarios.
}

\reglaDeNegocio[rn:acceso_fidedigno]
{Acceso fidedigno}
{
  Un acceso fidedigno es un acceso no autorizado
  (\hipervinculo{rn:acceso_no_autorizado}) que se puede producir de manera
  legítima en el uso de la aplicación. Algunas posibles causas son la expiración
  de una sesión o su cierre accidental.
}

% TODO: agregar causas o ejemplos.

\reglaDeNegocio[rn:acceso_malintencionado]
{Acceso malintencionado}
{
  Un acceso malintencionado es un acceso no autorizado
  (\hipervinculo{rn:acceso_no_autorizado}) que no se puede producir de manera
  legítima en el uso de la aplicación.
}

\reglaDeNegocio[rn:acceso_erroneo]
{Acceso erróneo}
{
  Un acceso erróneo es un acceso no autorizado
  (\hipervinculo{rn:acceso_no_autorizado}) en el que no hay elementos
  suficientes para clasificarlo como fidedigno
  (\hipervinculo{rn:acceso_fidedigno}) o como malintencionado
  (\hipervinculo{rn:acceso_malintencionado}).
}
