%
% Reglas de negocio de sistema tokenizador,
% Capítulo de análisis y diseño de api web.
% Proyecto Lovelace
%

\subsection{Reglas de negocio}

El formato de las \glspl{gl:expresion_regular}, por cuestiones de optimización
en tiempos de desarrollo, se presenta según el formato de \textit{JavaScript},
definido en~\cite{mozilla_er}.

\reglaDeNegocio[rn:formato_numero_de_tarjeta]
{Composición de números de tarjeta}
{
  Un número de tarjeta queda definido por la expresión regular:

  \expresionRegular{expresiones_regulares/numero_de_tarjeta.txt}

  El número se divide en tres partes: el identificador del emisor (primeros 6
  dígitos), el número de cuenta y el dígito verificador (último dígito). Para
  más detalles sobre este tema consultar la sección
 ~\ref{sec:composicion_tarjeta}. El dígito verificador se obtiene de tal forma
  que el algoritmo de Luhn (sección~\ref{sec:algoritmo_luhn}) dé como resultado
  cero.
}

\reglaDeNegocio[rn:formato_token]
{Composición de tokens}
{
  El formato de un token es el mismo que para un número de tarjeta
  (\hipervinculo{rn:formato_numero_de_tarjeta}) con la única excepción de que el
  último dígito, el dígito verificador, se calcule de forma que el algoritmo de
  Luhn (sección~\ref{sec:algoritmo_luhn}) dé como resultado un uno.
}

\reglaDeNegocio[rn:unicidad_correo]
{Unicidad del correo electrónico}
{
  El correo electrónico con el que se registra a los usuarios \textbf{cliente}
  y \textbf{administrador} debe ser único.
}

\reglaDeNegocio[rn:correo_electronico]
{Formato de correos electrónicos}
{
  Un correo electrónico queda definido, de acuerdo al \gls{gl:rfc} 5322
 ~\cite{DBLP:journals/rfc/rfc5322}, por la expresión regular:

  \expresionRegular{expresiones_regulares/correo.txt}
}

\reglaDeNegocio[rn:formato_de_contrasenias]
{Formato de contraseñas}
{
  El formato de una contraseña está regido por la expresión regular:

  \expresionRegular{expresiones_regulares/contrasenia.txt}

  Esto es, entre 8 y 24 caracteres \gls{gl:ascii} imprimibles.
}

\reglaDeNegocio[rn:tipo_de_algoritmo]
{Tipos de algoritmos tokenizadores}
{
  Existen dos tipos de algoritmos tokenizadores:
  \begin{itemize}
    \item Reversibles
    \item Irreversibles
  \end{itemize}
}

\reglaDeNegocio[rn:tipo_de_usuario]
{Tipos de usuario}
{
  Existen cuatro tipos de usuarios:
  \begin{itemize}
    \item Visitante
    \item Cliente
    \item Administrador
    \item Aplicación cliente
  \end{itemize}

  \begin{hangparas}{2em}{0}
    Al abrir la aplicación, todos los usuarios son visitantes; pues los clientes
    y los administradores deben iniciar sesión para ser identificados por el
    sistema; un vistante puede convertirse en cliente al registrarse y ser
    aprobado. Las transiciones entre los tipos de usuarios pueden observarse en
    el diagrama~\ref{estados_actores}. Una aplicación cliente es aquella que
    utiliza el servicio de tokenización (por ejemplo, una tienda en línea); cada
    cliente representa a una aplicación cliente.
  \end{hangparas}
}

\begin{figure}
  \begin{center}
    \subimport{diagramas/}{estados_actores.tikz.tex}
    \caption{Diagrama de estados de actores.}
    \label{estados_actores}
  \end{center}
\end{figure}

\reglaDeNegocio[rn:estados_cliente]
{Estados de un cliente}
{
  Un cliente puede tener los siguientes estados:
  \begin{itemize}
    \item En espera
    \item Aprobado
    \item Rechazado
    \item En cambio de llave
    \item En lista negra
  \end{itemize}

  \begin{hangparas}{2em}{0}
    Cuando un cliente se registra, se encuentra en estado \textbf{en espera},
    con correo \textbf{no verificado}; cuando verifica su dirección de correo
    electrónico, pasa a \textbf{en espera} con correo electrónico
    \textbf{verificado}; una vez que el administrador lo aprueba, pasa a
    \textbf{aprobado}; estando en estado \textbf{en espera} con correo
    \textbf{verificado}, un administrador puede pasar a un cliente al estado
    \textbf{rechazado} . Cuando inicia un refresco de llaves, pasa a
    \textbf{en cambio de llaves}. Finalmente, el administrador puede cambiar su
    estado a \textbf{en lista negra} (y regresarlo a \textbf{aprobado} o
    \textbf{en cambio de llaves}). Estas transiciones se pueden observar en la
    Figura~\ref{estados_cliente}.
  \end{hangparas}
}

%
% TODO:
% El sidewaysfigure implica, evidentemente, una mezcla entre forma y contenido.
% Se tendría que poder determinar, en tiempo de compilación, si la imagen dada
% tiene la proporción y tamaño adecuados para ocupar el entorno.
%

\begin{sidewaysfigure}
  \begin{center}
    \subimport{diagramas/}{estados_cliente.tikz.tex}
    \caption{Diagrama de estados de un cliente.}
    \label{estados_cliente}
  \end{center}
\end{sidewaysfigure}

\reglaDeNegocio[rn:estados_token]
{Estados de un token}
{
  Un token puede tener los siguientes estados:
  \begin{itemize}
    \item Actual
    \item Anterior
    \item Retokenizado
  \end{itemize}

  \begin{hangparas}{2em}{0}
    Cuando un token es creado, su estado es \textbf{actual}. Cuando un cliente
    inicia el refresco de llaves, todos sus tokens cambian de estado a
    \textbf{anterior}. Cuando un cliente solicita la retokenización de un token,
    se cambia su estado a \textbf{retokenizado}. Al terminar el refresco de
    llaves o al eliminar la cuenta de un cliente, el token se elimina. Estas
    transiciones se pueden observar en la Figura~\ref{estados_token}.
  \end{hangparas}
}

\begin{figure}
  \begin{center}
    \subimport{diagramas/}{estados_tokens.tikz.tex}
    \caption{Diagrama de estados de tokens.}
    \label{estados_token}
  \end{center}
\end{figure}

\reglaDeNegocio[rn:estados_llave]
{Estados de una llave}
{
  Una llave puede tener los siguientes estados:
  \begin{itemize}
    \item Actual
    \item Anterior
  \end{itemize}

  \begin{hangparas}{2em}{0}
    Cuando una llave es creada, su estado es \textbf{actual}; cuando el cliente
    inicia el refresco de llaves, la llave existente cambia a estado
    \textbf{anterior}. La llave se elimina cuando termina el refresco de llaves
    o cuando se elimina la cuenta de un cliente. Estas transiciones se pueden
    observar mejor en la Figura~\ref{estados_llave}.
  \end{hangparas}
}

\begin{figure}
  \begin{center}
    \subimport{diagramas/}{estados_llaves.tikz.tex}
    \caption{Diagrama de estados de llaves.}
    \label{estados_llave}
  \end{center}
\end{figure}

\reglaDeNegocio[rn:estados_correo]
{Estados de un correo electrónico}
{
  Un correo electrónico puede tener los siguientes estados:
  \begin{itemize}
    \item Verificado
    \item No verificado
  \end{itemize}

  \begin{hangparas}{2em}{0}
    Cuando un correo es registrado o actualizado , su estado es \textbf{no
    verificado}; cambia su estado a \textbf{verificado} cuando el cliente
    presiona un vínculo de validación válido (véase
    \hipervinculo{rn:verificacion_de_cuentas_nuevas}); se elimina cuando se
    borra la cuenta del cliente o cuando este actualiza su información. Estas
    transiciones se pueden observar mejor en la Figura~\ref{estados_correo}.
  \end{hangparas}
}

\begin{figure}
  \begin{center}
    \subimport{diagramas/}{estados_correos.tikz.tex}
    \caption{Diagrama de estados de un correo electrónico.}
    \label{estados_correo}
  \end{center}
\end{figure}

\reglaDeNegocio[rn:malas_acciones]
{Contador de malas acciones}
{
  Todo cliente tiene asociado un cliente de malas acciones que aumenta de la
  siguiente manera:
  \begin{itemize}
    \item Aumentar en tres cuando un cliente intenta detokenizar tokens válidos
      que no son suyos.
    \item Aumentar en uno cuando un cliente intenta detokenizar tokens no
      válidos.
    \item Aumentar en uno cuando un cliente intente acceder a páginas que no
      tiene acceso.
  \end{itemize}

  \begin{hangparas}{2em}{0}
    Cuando un cliente es creado, el contador está en cero.
  \end{hangparas}
}

\reglaDeNegocio[rn:top_malas_acciones]
{Sancionar cliente}
{
  Cuando el contador de malas acciones de un cliente sea mayor o igual a 10,
  el estado del cliente deberá ser actualizado a  \textbf{en lista negra}.
}

\reglaDeNegocio[rn:tokenizacion]
{Petición de tokenización}
{
  Cuando un cliente solicite una operación de tokenización, se debe utilizar la
  llave con estado \textbf{actual} para obtener el token.
}

\reglaDeNegocio[rn:detokenizacion]
{Petición de detokenización}
{
  Cuando un cliente en estado \textbf{aceptado} solicite una operación de
  detokenización, se debe utilizar la llave con estado \textbf{actual}
  para obtener el número de tarjeta.
}

\reglaDeNegocio[rn:detokenizacion_en_refresco]
{Petición de detokenización en proceso de refresco}
{
  Cuando un cliente solicite una operación de detokenización y se encuentre en
  estado \textbf{en cambio de llaves}, se debe especificar en la petición si
  se desea utilizar la llave nueva o anterior. En caso de no ser especificada,
  se realizará la operación utilizando la llave con estado \textbf{actual}.
}

\reglaDeNegocio[rn:habilitacion_de_retokenizacion]
{Habilitación de retokenización}
{
  La operación de retokenización solo está habilitada para los clientes cuyo
  estado es \textbf{en cambio de llaves}.

}

\reglaDeNegocio[rn:criptoperiodo_llave]
{Criptoperiodo de una llave}
{
  El criptoperiodo de todas las llaves usadas para la tokenización (y
  detokenización) es de seis meses (ver el Apéndice
 ~\ref{sec:administracion_llaves}).
}

\reglaDeNegocio[rn:unicidad_tarjeta_cliente]
{Unicidad de número de tarjeta por cliente}
{
  Un cliente solo puede tener un token por tarjeta; esto es, no puede realizar,
  para un mismo número de tarjeta, la operación de tokenización más de una vez.
}

\reglaDeNegocio[rn:hipervinculo_verificacion]
{Hipervínculo de verificación}
{
  Un vínculo de confirmación está conformado por dos partes; la dirección base
  del sitio y un valor hash obtenido mediante el correo y la contraseña del
  usuario que se va a verificar.
}

\reglaDeNegocio[rn:verificacion_de_cuentas_nuevas]
{Expiración de vínculos de verificación de registro}
{
  El vínculo para verificar una cuenta es efectivo solamente durante
  24 horas; después de este tiempo, el vínculo deja de ser válido y se elimina
  el registro del usuario.
}

\reglaDeNegocio[rn:verificacion_de_actualizacion_de_cuentas]
{Expiración de vínculos de verificación de actualización}
{
  El vínculo para verificar la actualizacion de una cuenta es efectivo
  indefinidamente.
}

\reglaDeNegocio[rn:aprobacion_cliente]
{Aprobación de clientes}
{
  Un administrador puede aprobar o rechazar usuarios tipo \textbf{cliente}
  cuando estos tengan estado \textbf{en espera} con correo \textbf{verificado}
  en un periodo de 24 HRS desde que el cliente verificó su correo.
}

% TODO:
% Este newpage no debería estar aquí
% \newpage
% Por alguna extraña razón, el diagrama de casos de uso (a bastantes páginas)
% de aquí, genera un espacio en blanco (de media página) entre la penúltima y
% la última página.
