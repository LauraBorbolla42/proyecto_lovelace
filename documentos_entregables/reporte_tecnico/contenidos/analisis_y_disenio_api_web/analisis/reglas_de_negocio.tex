%
% Reglas de negocio de sistema tokenizador,
% Capítulo de análisis y diseño de api web.
% Proyecto Lovelace
%

\subsection{Reglas de negocio}

El formato de las \glspl{gl:expresion_regular}, por cuestiones de optimización
en tiempos de desarrollo, se presenta según el formato de \textit{java script},
definido en \cite{mozilla_er}.

\reglaDeNegocio[rn:servicio_tokenizacion]{API}
{Servicio de tokenizacion}
{
  El servicio de tokenización abarca dos operaciones básicas:
  \textit{tokenización}, esto es, dado un número de tarjeta, regresar el token
  correspondiente; y \textit{detokenización}, que es el proceso inverso, dado un
  token, regresar el número de tarjeta asociado.

  Esta regla está asociada a \hipervinculo{rn:formato_numero_de_tarjeta} y
  \hipervinculo{rn:formato_token}.
}

% TODO:
% El caracter de inicio de cadena, en la expresión regular, no se está mostrado
% como debería.

\reglaDeNegocio[rn:formato_numero_de_tarjeta]{API}
{Composición de números de tarjeta}
{
  Un número de tarjeta queda definido por la expresión regular
  \texttt{\^[0-9]\{12, 19\}\$}. El número se divide en tres partes: el
  identificador del emisor (primeros 6 dígitos), el número de cuenta y el dígito
  verificador (último dígito). Para más detalles sobre este tema consultar la
  sección \ref{sec:composicion_tarjeta}. El dígito verificador se obtiene de tal
  forma que el algoritmo de Luhn (sección \ref{sec:algoritmo_luhn}) dé como
  resultado cero.
}

\reglaDeNegocio[rn:formato_token]{API}
{Composición de tokens}
{
  El formato de un token es el mismo que para un número de tarjeta
  (\hipervinculo{rn:formato_numero_de_tarjeta}) con la única excepción de que el
  último dígito, el dígito verificador, se calcule de forma que el algoritmo de
  Luhn (sección \ref{sec:algoritmo_luhn}) dé como resultado un uno.
}


