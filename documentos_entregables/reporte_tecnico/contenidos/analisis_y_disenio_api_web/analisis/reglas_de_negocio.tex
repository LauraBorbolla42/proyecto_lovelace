%
% Reglas de negocio de sistema tokenizador,
% Capítulo de análisis y diseño de api web.
% Proyecto Lovelace
%

\subsection{Reglas de negocio}

El formato de las \glspl{gl:expresion_regular}, por cuestiones de optimización
en tiempos de desarrollo, se presenta según el formato de \textit{java script},
definido en \cite{mozilla_er}.

\reglaDeNegocio[rn:servicio_tokenizacion]
{Servicio de tokenizacion}
{
  El servicio de tokenización abarca dos operaciones básicas:
  \textit{tokenización}, esto es, dado un número de tarjeta, regresar el token
  correspondiente; y \textit{detokenización}, que es el proceso inverso, dado un
  token, regresar el número de tarjeta asociado.

  Esta regla está asociada a \hipervinculo{rn:formato_numero_de_tarjeta} y
  \hipervinculo{rn:formato_token}.
}

\reglaDeNegocio[rn:formato_numero_de_tarjeta]
{Composición de números de tarjeta}
{
  Un número de tarjeta queda definido por la expresión regular
  \texttt{[0-9]\{12, 19\}}. El número se divide en tres partes: el
  identificador del emisor (primeros 6 dígitos), el número de cuenta y el dígito
  verificador (último dígito). Para más detalles sobre este tema consultar la
  sección \ref{sec:composicion_tarjeta}. El dígito verificador se obtiene de tal
  forma que el algoritmo de Luhn (sección \ref{sec:algoritmo_luhn}) dé como
  resultado cero.
}

\reglaDeNegocio[rn:formato_token]
{Composición de tokens}
{
  El formato de un token es el mismo que para un número de tarjeta
  (\hipervinculo{rn:formato_numero_de_tarjeta}) con la única excepción de que el
  último dígito, el dígito verificador, se calcule de forma que el algoritmo de
  Luhn (sección \ref{sec:algoritmo_luhn}) dé como resultado un uno.
}

% TODO: encontrar una manera para mostrar esto.
\reglaDeNegocio[rn:correo_electronico]
{Formato de correos electrónicos}
{
  Un nombre de usuario queda definido por la expresión regular
  %\begin{verbatim}
  %^(([^<>()\[\]\\.,;:\s@"]+(\.[^<>()\[\]\\.,;:\s@"]+)*)|(".+"))@((\[[0-9]{1,3}\.[0-9]{1,3}\.[0-9]{1,3}\.[0-9]{1,3}\])|(([a-zA-Z\-0-9]+\.)+[a-zA-Z]{2,}))$
  %\end{verbatim}
}

% TODO: ¿Agregar símbolos a expresión regular de contraseñas?

\reglaDeNegocio[rn:formato_de_contrasenias]
{Formato de contraseñas}
{
  El formato de una contraseña está regido por la expresión regular
  \texttt{0-9a-zA-Z}\{8, 24\}.
}

\reglaDeNegocio[rn:acceso_no_autorizado]
{Acceso no autorizados}
{
  Un acceso no autorizado se da cuando un usuario hace una petición al servidor
  para la cual no tiene los privilegios necesarios.
}

\reglaDeNegocio[rn:acceso_fidedigno]
{Acceso fidedigno}
{
  Un acceso fidedigno es un acceso no autorizado
  (\hipervinculo{rn:acceso_no_autorizado}) que se puede producir de manera
  legítima en el uso de la aplicación. Algunas posibles causas son la expiración
  de una sesión o su cierre accidental.
}

% TODO: agregar causas o ejemplos.

\reglaDeNegocio[rn:acceso_malintencionado]
{Acceso malintencionado}
{
  Un acceso malintencionado es un acceso no autorizado
  (\hipervinculo{rn:acceso_no_autorizado}) que no se puede producir de manera
  legítima en el uso de la aplicación.
}

\reglaDeNegocio[rn:acceso_erroneo]
{Acceso erroneo}
{
  Un acceso erroneo es un acceso no autorizado
  (\hipervinculo{rn:acceso_no_autorizado}) en el que no hay elementos
  suficientes para clasificarlo como fidedigno
  (\hipervinculo{rn:acceso_fidedigno}) o como malintencionado
  (\hipervinculo{rn:acceso_malintencionado}).
}
