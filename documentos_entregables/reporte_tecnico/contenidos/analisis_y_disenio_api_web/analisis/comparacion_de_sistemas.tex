%
% Comparación de sistemas tokenizadores en el mercado.
% Capítulo de análisis y diseño de api web.
% Proyecto Lovelace
%

\subsection{Comparación de sistemas tokenizadores}

El primer problema en la planeación de la interfaz web al programa tokenizador
del primer módulo consiste en la enumeración de requerimientos del sistema. Un
primer enfoque podría haber sido plantear los requerimientos de forma intuitiva:
¿qué se espera que un sistema tokenizador haga? Sin embargo, este primer
enfoque cae muy rápido en una lista de requerimientos demasiado simples. Esto
es un indicio de que las soluciones existentes en el mercado agregan algunas
funcionalidades secundarias, por llamarlas de alguna forma, al proceso simple de
tokenización y detokenización en red. Para explorar esta posibilidad y poder
plantear un sistema tokenizador con perspectivas de uso real se hizo una
investigación sobre las características de los sistemas tokenizadres que existen
actualmente en el mercado. En esta sección se presentan los resultados.

\subsubsection{Shift4}

\url{https://www.shift4.com/dotn/4tify/trueTokenization.cfm}

Características principales:

\begin{itemize}

  \item El servicio de tokenización se integra a un sistema de procesamiento de
    pagos en línea llamado \textit{dollars on the net}.

    \begin{itemize}
      \item
    \end{itemize}

\end{itemize}

% Esto no se va a quedar, evidentemente.

Algunas de las barrabasadas:

\begin{itemize}

  \item \textit{In 2005, Shift4 hosted a security conference in Las Vegas
    attended by software developers and representatives from all of the credit
    card associations that would change the fundamentals of data protection and
    PCI requirements for merchants. It was there that Shift4 unveiled
    tokenization, a new technology integrated with Shift4's Software as a
    Service (SaaS) DOLLARS ON THE NET® payment gateway. Since the introduction
    of tokenization, Shift4 has processed more than five billion tokenized
    transactions through hundreds of tokenized interfaces} \cite{shift4_uno}.

    Todo un aire épico histórico para su historia para que al final solo digan
    que la tokenización es una «nueva tecnología integrada con dollars on the
    net».

  \item \textit{While various companies have jumped on the bandwagon with
    solutions that they call tokenization, most use the word to mean virtually
    any technology that enables the card number to be replaced or modified. Some
    of these so-called tokenization solutions are card number encryption,
    hashing, truncation, or merely obfuscation. Consultants, Qualified Security
    Assessors (QSAs), and even the PCI council itself have clouded the issue of
    what tokenization really is} \cite{shift4_uno}.

    ¿Si la tokenización no es ninguna de las cosas que mencionan, entonces qué
    es? ¿En dónde está explicado lo que para shift4 es tokenización?

  \item \textit{At first glance, tokenization may seem like encryption, but it's
    not – TrueTokens are not encrypted data. Although encryption does enable
    companies to comply with PCI guidelines, Shift4 believes storing sensitive
    data using encryption alone is not sufficient enough to protect CHD and
    provide REAL security.}

    ¿La criptografía por sí sola no es suficiente? ¿Qué es la criptografía para
    estos tipos? La generación de cosas aleatorias, por métodos determinísticos,
    también es campo de estudio de la criptografía. Aunque fuera un generador
    realmente aleatorio, al final de cuentas es una permutación en el mismo
    dominio.

\end{itemize}



