%
% Diagrama de modelo de datos,
% Aplicación web de sistema tokenizador.
% Proyecto Lovelace.
%

\begin{tikzpicture}[
  clase_simple/.style = {
    rectangle,
    anchor = north,
    draw = black,
    thin,
    inner sep = 2mm,
    align = center,
    rectangle split,
    rectangle split parts = 2,
    rectangle split part align = {center, left}}]

  \node[clase_simple]
    (usuario)
    {
      \textbf{Usuario}
      \nodepart{second}
      \begin{tabular}{l}
        Contraseña \\
        Correo electrónico \\
        Estado \\
        Tipo
      \end{tabular}
    };

    \node[clase_simple]
      (estado_de_usuario)
      [right = 0.6in of usuario]
      {
        \textbf{Estado de usuario}
        \nodepart{second}
        \begin{tabular}{l}
          Nombre
        \end{tabular}
      };

    \draw[Diamond-]
      (usuario.east)
      node[
        above,
        anchor = south west]
        {Estado}
      --
      (estado_de_usuario.west);

    \node[clase_simple]
      (tipo_de_usuario)
      [left = 0.6in of usuario]
      {
        \textbf{Tipo de usuario}
        \nodepart{second}
        \begin{tabular}{l}
          Nombre
        \end{tabular}
      };

    \draw[Diamond-]
      (usuario.west)
      node[
        above,
        anchor = south east]
        {Tipo}
      --
      (tipo_de_usuario.east);

    \node[clase_simple]
      (token)
      at ($(usuario.south)!0.5!(tipo_de_usuario.south) - (0, 0.6in)$)
      {
        \textbf{Token}
        \nodepart{second}
        \begin{tabular}{l}
          Token \\
          Número de tarjeta \\
          Usuario \\
          Estado
        \end{tabular}
      };

    \draw[Diamond-]
      (token.north)
      node[
        left,
        anchor = south east]
        {Usuario}
      --
      ($(token.north) + (0, 0.25in)$)
      -|
      ($(usuario.south) - (0.2in, 0)$);

    \node[clase_simple]
      (llave)
      at ($(usuario.south)!0.5!(estado_de_usuario.south) - (0, 0.6in)$)
      {
        \textbf{Llave}
        \nodepart{second}
        \begin{tabular}{l}
          Llave \\
          Criptoperiodo \\
          Fecha de creación \\
          Usuario \\
          Estado \\
          Algoritmo
        \end{tabular}
      };

    \draw[Diamond-]
      (llave.north)
      node[
        right,
        anchor = south west]
        {Usuario}
      --
      ($(llave.north) + (0, 0.25in)$)
      -|
      ($(usuario.south) + (0.2in, 0)$);

    \node[clase_simple]
      (estado_de_token)
      [left = 0.6in of token]
      {
        \textbf{Estado de token}
        \nodepart{second}
        \begin{tabular}{l}
          Nombre
        \end{tabular}
      };

    \draw[Diamond-]
      (token.west)
      node[
        above,
        anchor = south east]
        {Estado}
      --
      (estado_de_token.east);

    \node[clase_simple]
      (estado_de_llave)
      [right = 0.6in of llave]
      {
        \textbf{Estado de llave}
        \nodepart{second}
        \begin{tabular}{l}
          Nombre
        \end{tabular}
      };

    \draw[Diamond-]
      (llave.east)
      node[
        above,
        anchor = south west]
        {Estado}
      --
      (estado_de_llave.west);

    \node[clase_simple]
      (algoritmo)
      [below = 0.5in of llave]
      {
        \textbf{Algoritmo}
        \nodepart{second}
        \begin{tabular}{l}
          Nombre \\
          Longitud de llave \\
          Tipo
        \end{tabular}
      };

    \draw[Diamond-]
      (llave.south)
      node[
        right,
        anchor = north west]
        {Algoritmo}
      --
      (algoritmo.north);

    \node[clase_simple]
      (tipo_de_algoritmo)
      [left = 0.6in of algoritmo]
      {
        \textbf{Tipo de algoritmo}
        \nodepart{second}
        \begin{tabular}{l}
          Nombre
        \end{tabular}
      };

    \draw[Diamond-]
      (algoritmo.west)
      node[
        above,
        anchor = south east]
        {Tipo}
      --
      (tipo_de_algoritmo.east);


\end{tikzpicture}
