%
% Diagrama de modelo de datos,
% Aplicación web de sistema tokenizador.
% Proyecto Lovelace.
%

\tikzset{
    font=\scriptsize}

\begin{tikzpicture}[
  clase_simple/.style = {
    rectangle,
    anchor = north,
    draw = black,
    thin,
    inner sep = 1mm,
    align = center,
    rectangle split,
    rectangle split parts = 2,
    rectangle split part align = {center, left}},
  comentario/.style = {
    rectangle,
    anchor = north,
    draw = black,
    thin,
    inner sep = 1mm,
    text width = 1in}]

  \node[clase_simple]
    (usuario)
    {
      \textbf{Usuario}
      \nodepart{second}
      \begin{tabular}{l}
        Correo \\
        Estado \\
        Tipo \\
        Cnt. malas acciones
      \end{tabular}
    };

    \node[clase_simple]
      (estado_de_usuario)
      [right = 0.6in of usuario]
      {
        \textbf{Estado de usuario}
        \nodepart{second}
        \begin{tabular}{l}
          Nombre
        \end{tabular}
      };

    \draw[Diamond-]
      (usuario.east)
      node[
        above,
        anchor = south west]
        {Estado}
      --
      (estado_de_usuario.west);

    \node[comentario, text width = 1.2in]
      (contenido_estado_de_usuario)
      [above = 0.5in of estado_de_usuario]
      {
        \begin{tabular}{l}
          En espera \\
          Aprobado \\
          Rechazado \\
          En cambio de llave \\
          En lista negra
        \end{tabular}
      };

    \draw
      ($(contenido_estado_de_usuario.north east) - (7mm, 0.3mm)$) --
      ($(contenido_estado_de_usuario.north east) - (0.2mm, 4mm)$) --
      ($(contenido_estado_de_usuario.north east) - (2mm, 5mm)$) --
      cycle;

    \draw[{Circle[open]}-, dashed]
      (estado_de_usuario.north)
      --
      node[
        right,
        text width = 1.2in,
        align = left]
        {\hipervinculo{rn:estados_cliente}}
      (contenido_estado_de_usuario.south);

    \node[clase_simple]
      (tipo_de_usuario)
      [left = 0.6in of usuario]
      {
        \textbf{Tipo de usuario}
        \nodepart{second}
        \begin{tabular}{l}
          Nombre
        \end{tabular}
      };

    \draw[Diamond-]
      (usuario.west)
      node[
        above,
        anchor = south east]
        {Tipo}
      --
      (tipo_de_usuario.east);

    \node[comentario]
      (contenido_tipo_de_usuario)
      [above = 0.5in of tipo_de_usuario]
      {
        \begin{tabular}{l}
          Visitante \\
          Cliente \\
          Administrador
        \end{tabular}
      };

    \draw
      ($(contenido_tipo_de_usuario.north east) - (7mm, 0.3mm)$) --
      ($(contenido_tipo_de_usuario.north east) - (0.2mm, 4mm)$) --
      ($(contenido_tipo_de_usuario.north east) - (2mm, 5mm)$) --
      cycle;

    \draw[{Circle[open]}-, dashed]
      (tipo_de_usuario.north)
      --
      node[
        left,
        text width = 1.0in,
        align = right]
        {\hipervinculo{rn:tipo_de_usuario}}
      (contenido_tipo_de_usuario.south);

    \node[clase_simple]
      (token)
      at ($(usuario.south)!0.5!(tipo_de_usuario.south) - (0, 0.6in)$)
      {
        \textbf{Token}
        \nodepart{second}
        \begin{tabular}{l}
          Token \\
          Número de tarjeta \\
          Usuario \\
          Estado
        \end{tabular}
      };

    \draw[Diamond-]
      (token.north)
      node[
        left,
        anchor = south east]
        {Usuario}
      --
      ($(token.north) + (0, 0.25in)$)
      -|
      ($(usuario.south) - (0.2in, 0)$);

    \node[clase_simple]
      (llave)
      at ($(usuario.south)!0.5!(estado_de_usuario.south) - (0, 0.6in)$)
      {
        \textbf{Llave}
        \nodepart{second}
        \begin{tabular}{l}
          Llave \\
          Criptoperiodo \\
          Fecha de creación \\
          Usuario \\
          Estado \\
          Algoritmo
        \end{tabular}
      };

    \draw[Diamond-]
      (llave.north)
      node[
        right,
        anchor = south west]
        {Usuario}
      --
      ($(llave.north) + (0, 0.25in)$)
      -|
      ($(usuario.south) + (0.2in, 0)$);

    \node[clase_simple]
      (estado_de_token)
      [left = 0.6in of token]
      {
        \textbf{Estado de token}
        \nodepart{second}
        \begin{tabular}{l}
          Nombre
        \end{tabular}
      };

    \draw[Diamond-]
      (token.west)
      node[
        above,
        anchor = south east]
        {Estado}
      --
      (estado_de_token.east);

    \node[comentario]
      (contenido_estado_de_token)
      [below = 0.7in of estado_de_token]
      {
        \begin{tabular}{l}
          Actual \\
          Anterior \\
          Retokenizador
        \end{tabular}
      };

    \draw
      ($(contenido_estado_de_token.north east) - (7mm, 0.3mm)$) --
      ($(contenido_estado_de_token.north east) - (0.2mm, 4mm)$) --
      ($(contenido_estado_de_token.north east) - (2mm, 5mm)$) --
      cycle;

    \draw[{Circle[open]}-, dashed]
      (estado_de_token.south)
      --
      node[
        right,
        text width = 0.8in,
        align = left]
        {\hipervinculo{rn:estados_token}}
      (contenido_estado_de_token.north);

    \node[clase_simple]
      (estado_de_llave)
      [right = 0.6in of llave]
      {
        \textbf{Estado de llave}
        \nodepart{second}
        \begin{tabular}{l}
          Nombre
        \end{tabular}
      };

    \draw[Diamond-]
      (llave.east)
      node[
        above,
        anchor = south west]
        {Estado}
      --
      (estado_de_llave.west);

    \node[comentario]
      (contenido_estado_de_llave)
      [below = 0.7in of estado_de_llave]
      {
        \begin{tabular}{l}
          Actual \\
          Anterior
        \end{tabular}
      };

    \draw
      ($(contenido_estado_de_llave.north east) - (7mm, 0.3mm)$) --
      ($(contenido_estado_de_llave.north east) - (0.2mm, 4mm)$) --
      ($(contenido_estado_de_llave.north east) - (2mm, 5mm)$) --
      cycle;

    \draw[{Circle[open]}-, dashed]
      (estado_de_llave.south)
      --
      node[
        left,
        text width = 0.8in,
        align = right]
        {\hipervinculo{rn:estados_llave}}
      (contenido_estado_de_llave.north);

    \node[clase_simple]
      (algoritmo)
      [below = 0.5in of llave]
      {
        \textbf{Algoritmo}
        \nodepart{second}
        \begin{tabular}{l}
          Nombre \\
          Longitud de llave \\
          Tipo
        \end{tabular}
      };

    \draw[Diamond-]
      (llave.south)
      node[
        right,
        anchor = north west]
        {Algoritmo}
      --
      (algoritmo.north);

    \node[comentario]
      (contenido_algoritmo)
      [below = 0.5in of algoritmo]
      {
        \begin{tabular}{l}
          FFX \\
          BPS \\
          TKR \\
          AHR \\
          DRBG \\
        \end{tabular}
      };

    \draw
      ($(contenido_algoritmo.north east) - (7mm, 0.3mm)$) --
      ($(contenido_algoritmo.north east) - (0.2mm, 4mm)$) --
      ($(contenido_algoritmo.north east) - (2mm, 5mm)$) --
      cycle;

    \draw[{Circle[open]}-, dashed]
      (algoritmo.south)
      --
      node[
        right,
        text width = 1.5in]
        {\hipervinculo{rq:tokenizar_tarjeta}}
      (contenido_algoritmo.north);

    \node[clase_simple]
      (tipo_de_algoritmo)
      [left = 0.6in of algoritmo]
      {
        \textbf{Tipo de algoritmo}
        \nodepart{second}
        \begin{tabular}{l}
          Nombre
        \end{tabular}
      };

    \draw[Diamond-]
      (algoritmo.west)
      node[
        above,
        anchor = south east]
        {Tipo}
      --
      (tipo_de_algoritmo.east);

    \node[comentario]
      (contenido_tipo_de_algoritmo)
      [below = 0.5in of tipo_de_algoritmo]
      {
        \begin{tabular}{l}
          Reversible \\
          Irreversible
        \end{tabular}
      };

    \draw
      ($(contenido_tipo_de_algoritmo.north east) - (7mm, 0.3mm)$) --
      ($(contenido_tipo_de_algoritmo.north east) - (0.2mm, 4mm)$) --
      ($(contenido_tipo_de_algoritmo.north east) - (2mm, 5mm)$) --
      cycle;

    \draw[{Circle[open]}-, dashed]
      (tipo_de_algoritmo.south)
      --
      node[
        left,
        text width = 1.7in,
        align = right]
        {\hipervinculo{rn:tipo_de_algoritmo}}
      (contenido_tipo_de_algoritmo.north);

    \node[clase_simple]
      (correo)
      [above = 1.4in of usuario]
      {
        \textbf{Correo}
        \nodepart{second}
        \begin{tabular}{l}
          Correo \\
          Contraseña \\
          Vínculo \\
          Estado
        \end{tabular}
      };

    \draw[Diamond-]
      (usuario.north)
      node[
        above,
        anchor = south east]
        {Correo}
      --
      (correo.south);

    \node[clase_simple]
      (estado_de_correo)
      [right = 0.6in of correo]
      {
        \textbf{Estado de correo}
        \nodepart{second}
        \begin{tabular}{l}
          Nombre
        \end{tabular}
      };

    \draw[Diamond-]
      (correo.east)
      node[
        above,
        anchor = south west]
        {Estado}
      --
      (estado_de_correo.west);

    \node[comentario]
      (contenido_estado_de_correo)
      [above = 0.5in of estado_de_correo]
      {
        \begin{tabular}{l}
          Verificado \\
          No verificado
        \end{tabular}
      };

    \draw
      ($(contenido_estado_de_correo.north east) - (7mm, 0.3mm)$) --
      ($(contenido_estado_de_correo.north east) - (0.2mm, 4mm)$) --
      ($(contenido_estado_de_correo.north east) - (2mm, 5mm)$) --
      cycle;

    \draw[{Circle[open]}-, dashed]
      (estado_de_correo.north)
      --
      node[
        right,
        text width = 1.4in,
        align = left]
        {\textsc{RNAPI-11 Estados de un correo electrónico}}
      (contenido_estado_de_correo.south);

    \node[clase_simple]
      (vinculo)
      [left = 0.6in of correo]
      {
        \textbf{Vínculo}
        \nodepart{second}
        \begin{tabular}{l}
          Vínculo \\
          Fecha
        \end{tabular}
      };

    \draw[Diamond-]
      (correo.west)
      node[
        above,
        anchor = south east]
        {Vínculo}
      --
      (vinculo.east);


\end{tikzpicture}

\tikzset{
    font=\footnotesize}
