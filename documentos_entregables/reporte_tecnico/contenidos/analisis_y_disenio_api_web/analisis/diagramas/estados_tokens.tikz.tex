%
% Diagrama de estados de tokens,
% Aplicación web de sistema tokenizador.
% Proyecto Lovelace.
%

\begin{tikzpicture}[
  estado/.style = {
    rectangle,
    draw = black,
    thin,
    inner sep = 4mm,
    text width = 25mm,
    align = center,
    minimum height = 1.5cm},
  estado_inicial/.style = {
    circle,
    minimum size = 15pt,
    fill = black,
  },
  etiqueta_transicion/.style = {
    align = center,
    text width = 15mm
  }]

  % Estados
  \node[estado_inicial]
    (estado_inicial) {};
  \node[estado]
    (actual)
    [right = 1.8in of estado_inicial]
    {Actual};
  \node[estado]
    (anterior)
    [below right = 0.6in of actual]
    {Anterior};
  \node[estado]
    (retokenizado)
    [below left = 0.6in of anterior]
    {Retokenizado};

  % Transiciones
  \draw[-Stealth]
    (estado_inicial.east)
    --
    node[
      etiqueta_transicion,
      text width = 1.6in,
      above]
      {\hipervinculo{cu:tokenizar_tarjeta}}
    (actual.west);
  \draw[-Stealth]
    (actual.east)
    -|
    node[
      etiqueta_transicion,
      text width = 1.6in,
      below right]
      {\hipervinculo{cu:iniciar_refresco}}
    (anterior.north);
  \draw[-Stealth]
    (anterior.south)
    |-
    node[
      etiqueta_transicion,
      text width = 1.6in,
      above right]
      {\hipervinculo{cu:retokenizar_token}}
    (retokenizado.east);
  \draw[-Stealth]
    (retokenizado.north)
    --
    node[
      etiqueta_transicion,
      text width = 1.6in,
      left]
      {\hipervinculo{cu:terminar_refresco}}
    (actual.south);

\end{tikzpicture}
