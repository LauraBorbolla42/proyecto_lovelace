%
% Diagrama de estados de tokens,
% Aplicación web de sistema tokenizador.
% Proyecto Lovelace.
%
% TODO:
% * Agregar transición de estado retokenizado a estado actual: caso de uso de
%   terminar refresco de llaves.
% * Redistribuir nodos para tener más espacio en el ancho de las etiquetas de
%   transiciones y para que no se salga de los márgenes del documento..
%

\begin{tikzpicture}[
    estado/.style = {
        rectangle,
        draw = black,
        thin,
        inner sep = 4mm,
        text width = 25mm,
        align = center,
        minimum height = 1.5cm},
      estado_inicial/.style = {
        circle,
        minimum size = 15pt,
        fill = black,
      },
      etiqueta_transicion/.style = {
        align = center,
        text width = 15mm
      }]

  % Estados
  \node[estado_inicial]
    (estado_inicial) {};
  \node[estado]
    (actual)
    [right = 1in of estado_inicial]
    {Actual};
  \node[estado]
    (anterior)
    [right = 1in of actual]
    {Anterior};
  \node[estado]
    (retokenizado)
    [right = 1in of anterior]
    {Retokenizado};

  % Transiciones
  \draw[-Stealth]
    (estado_inicial.east)
    --
    node[
      etiqueta_transicion,
      text width = 0.8in,
      above]
      {\hipervinculo{cu:tokenizar_tarjeta}}
    (actual.west);
  \draw[-Stealth]
    (actual.east)
    --
    node[
      etiqueta_transicion,
      text width = 0.8in,
      above]
      {\hipervinculo{cu:iniciar_refresco}}
    (anterior.west);
  \draw[-Stealth]
    (anterior.east)
    --
    node[
      etiqueta_transicion,
      text width = 0.8in,
      above]
      {\hipervinculo{cu:retokenizar_token}}
    (retokenizado.west);

\end{tikzpicture}
