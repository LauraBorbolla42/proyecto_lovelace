%
% Capítulo de análisis y diseño de módulo de caso de uso <<tienda en línea>>
% Proyecto Lovelace
%

\capitulo[LIB]{Análisis y diseño de tienda en línea}
   {sec:analisis_y_disenio_tienda}
{
  \epigrafe
  {%
    Lolita, luz de mi vida, fuego de mis entrañas. Pecado mío, alma mía.
    Lo-li-ta: la punta de la lengua emprende un viaje de tres pasos paladar
    abajo hasta apoyarse, en el tercero, en el borde de los dientes.
    Lo. Li. Ta.%
  }
  {%
    Lolita,
    \textsc{Vladimir Nabokov}.%
  }
}

\noindent
En este capítulo se abordan los temas referentes al análisis y al diseño del
tercer módulo del proyecto: el caso de uso de una tienda en línea.

Es importante resaltar que este tercer módulo es un \textbf{caso de prueba},
por lo que solo se implementarán las acciones necesarias para que se comunique
con los módulos anteriores: guardar la información bancaria tokenizada de los
clientes y realizar compras donde el servicio de tokenización fingirá realizar
las transacciones bancarias correspondientes.

Finalmente, se ha decidido que, para aterrizar el caso de prueba, la tienda en
línea venderá libros; en otras palabras, será el portal de una librería.

\subimport{analisis/}{analisis}
%\subimport{disenio/}{disenio}
