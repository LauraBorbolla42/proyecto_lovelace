%
% Caso para agregar un libro al carrito,
% análisis y diseño de tienda en línea.
% Proyecto Lovelace.
%

\casoDeUso[lib_cu:agregar_libro]
{Agregar libro}
{
  Caso que permite que un \textbf{cliente} o un \textbf{visitante} agreguen
  libros al carrito de compra.

  \begin{trayectoriaPrincipal}

    \item El usuario presiona el botón \textit{Agregar al carrito} en
      cualquiera de estas interfaces:
      \begin{itemize}
        \item En \hipervinculo{lib_iu:inicio}, en el botón correspondiente
          al libro deseado.
        \item En \hipervinculo{lib_iu:detalles_de_libro}.
      \end{itemize}

      \item El sistema actualiza la interfaz para reflejar que el libro
        seleccionado se encuentra en el carrito.

      \item El sistema guarda el libro en los datos de sesión.

      \item El sistema muestra una entrada numérica para introducir la cantidad
        de ejemplares a comprar con un valor por defecto de uno;
        [\hipervinculoLocal{lib_ta:modificar_cantidad}].

  \end{trayectoriaPrincipal}

  %%%%%%%%%%%%%%%%%%%%%%%%%%%%%%%%%%%%%%%%%%%%%%%%%%%%%%%%%%%%%%%%%%%%%%%%%%%%%

  \begin{trayectoriaAlternativa}[lib_ta:modificar_cantidad]
    {El usuario modifica la cantidad por defecto}

    \item El usuario introduce la cantidad de ejemplares del libro deseado.

    \item El sistema guarda el dato introducido en los datos de sesión.

  \end{trayectoriaAlternativa}
}
