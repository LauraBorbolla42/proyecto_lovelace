%
% Caso de uso para editar la información de un cliente,
% Capítulo de análisis y diseño de tienda en línea,
% Proyecto Lovelace.
%

\casoDeUso[lib_cu:actualizar_informacion_personal]
{Actualizar información de cliente}
{
  Caso que permite que un \textbf{cliente} actualice su información en el
  sistema. Puede actualizar los siguientes datos:
  \begin{itemize}
    \item Nombre
    \item Correo electrónico
    \item Contraseña
  \end{itemize}

  \begin{trayectoriaPrincipal}

    \item[origen] El \textbf{cliente} presiona el botón de
      \textit{Actualizar información} en la interfaz
      \hipervinculo{lib_iu:informacion_personal}.

    \item El sistema muestra la interfaz
      \hipervinculo{lib_iu:formulario_informacion_personal} con la información
      del cliente, una lista con las direcciones de entrega y el botón
      \textit{Guardar cambios} inhabilitado.

    \item[datos] El \textbf{cliente} actualiza los siguientes campos
      [\hipervinculoLocal{lib_ta:cancelar}]:
      \begin{itemize}
        \item Nombre
        \item Correo electrónico
        \item Contraseña (y su confirmación)
      \end{itemize}

    \item El sistema habilita el botón de \textit{aceptar}.

    \item El \textbf{cliente} presiona el botón de \textit{aceptar}.

    \item EL sistema valida que el correo electrónico sea válido
      al cumplir con el \gls{gl:rfc}
      5322~\cite{DBLP:journals/rfc/rfc5322}, mediante la expresión
      regular

      \expresionRegular{expresiones_regulares/correo.txt}

      [\hipervinculoLocal{lib_ta:correo_incorrecto}]

    \item El sistema valida que el correo electrónico no esté ya
      registrado en el sistema; pues dos \textbf{usuarios} no pueden
      compartir el mismo correo;
      [\hipervinculoLocal{lib_ta:usuario_existente}].

    \item El sistema valida el formato de la contraseña de acuerdo con
      la expresión regular

      \expresionRegular{expresiones_regulares/contrasenia.txt}

      [\hipervinculoLocal{lib_ta:contrasenia_incorrecta}].

    \item El sistema verifica que la confirmación de contraseña
      ingresada coincida con la contraseña.
      [\hipervinculoLocal{lib_ta:confirmacion_incorrecta}].

    \item El sistema registra la nueva dirección del cliente, la nueva
      contraseña y un nuevo vínculo de verificación (asociado al cliente) con
      la fecha actual, cambiando el estado a \textbf{no verificado}.

    \item El sistema envía un correo al cliente con el hipervínculo de
      verificación.

    \item El sistema muestra la ventana
      \hipervinculo{lib_iu:aviso_de_correo_de_confirmacion}.

    \item El \textbf{cliente} accede a la \gls{gl:url} enviada a su
      correo.

    \item El sistema verifica que no haya más de 72 HRS de diferencia
      entre la fecha de creación del vínculo y la hora actual;
      [\hipervinculoLocal{lib_ta:expiracion_de_vinculo}].

    \item El sistema cambia el estado del \textbf{cliente} a
      \textbf{verificado}.

    \item El sistema muestra la pantalla \hipervinculo{lib_iu:inicio}.

  \end{trayectoriaPrincipal}

  %%%%%%%%%%%%%%%%%%%%%%%%%%%%%%%%%%%%%%%%%%%%%%%%%%%%%%%%%%%%%%%%%%%%%%%%%%%%%%

  \begin{trayectoriaAlternativa}[lib_ta:cancelar]
    {El visitante cancela la operación}

    \item El visitante presiona el botón \textit{cancelar}.

    \item El sistema muestra la interfaz de origen del paso
      \referenciaLocal{origen} de la trayectoria principal.

  \end{trayectoriaAlternativa}

  %%%%%%%%%%%%%%%%%%%%%%%%%%%%%%%%%%%%%%%%%%%%%%%%%%%%%%%%%%%%%%%%%%%%%%%%%%%%%%

  \begin{trayectoriaAlternativa}[lib_ta:correo_incorrecto]
    {Correo electrónico incorrecto}

    \item El sistema muestra el mensaje
      \hipervinculo{lib_msj:correo_invalido}.

    \item Se regresa al paso \referenciaLocal{datos} de la trayectoria
      principal.

  \end{trayectoriaAlternativa}

  %%%%%%%%%%%%%%%%%%%%%%%%%%%%%%%%%%%%%%%%%%%%%%%%%%%%%%%%%%%%%%%%%%%%%%%%%%%%%%

  \begin{trayectoriaAlternativa}[lib_ta:contrasenia_incorrecta]
    {Contraseña incorrecta}

    \item El sistema muestra el mensaje
      \hipervinculo{lib_msj:contrasenia_incorrecta}.

    \item Se regresa al paso \referenciaLocal{datos} de la trayectoria
      principal.

  \end{trayectoriaAlternativa}

  %%%%%%%%%%%%%%%%%%%%%%%%%%%%%%%%%%%%%%%%%%%%%%%%%%%%%%%%%%%%%%%%%%%%%%%%%%%%%%

  \begin{trayectoriaAlternativa}[lib_ta:confirmacion_incorrecta]
    {Confirmación de contraseña incorrecta}

    \item El sistema muestra el mensaje
      \hipervinculo{lib_msj:confirmacion_incorrecta}.

    \item Se regresa al paso \referenciaLocal{datos} de la trayectoria
      principal.

  \end{trayectoriaAlternativa}

  %%%%%%%%%%%%%%%%%%%%%%%%%%%%%%%%%%%%%%%%%%%%%%%%%%%%%%%%%%%%%%%%%%%%%%%%%%%%%%

  \begin{trayectoriaAlternativa}[lib_ta:usuario_existente]
    {Correo registrado previamente}

    \item El sistema muestra el mensaje
      \hipervinculo{lib_msj:usuario_existente}.

    \item Se regresa al paso \referenciaLocal{datos} de la trayectoria
      principal.

  \end{trayectoriaAlternativa}

  %%%%%%%%%%%%%%%%%%%%%%%%%%%%%%%%%%%%%%%%%%%%%%%%%%%%%%%%%%%%%%%%%%%%%%%%%%%%%%

  \begin{trayectoriaAlternativa}[lib_ta:expiracion_de_vinculo]
    {Vínculo expirado}

    \item El sistema muestra la ventana
      \hipervinculo{lib_iu:aviso_de_correo_de_confirmacion}.

    \item El sistema elimina el registro del usuario.

  \end{trayectoriaAlternativa}

}
