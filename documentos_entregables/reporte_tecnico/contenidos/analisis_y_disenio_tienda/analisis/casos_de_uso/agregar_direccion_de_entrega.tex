%
% Caso de uso para editar la información de un cliente,
% Capítulo de análisis y diseño de tienda en línea,
% Proyecto Lovelace.
%

\casoDeUso[lib_cu:agregar_direccion_entrega]
{Agregar dirección de entrega}
{
  Caso que permite que un \textbf{cliente} agregue una dirección de entrega.

  \begin{trayectoriaPrincipal}

    \item[origen] El \textbf{cliente} presiona el botón de
      \textit{Agregar dirección de entrega} en la interfaz
      \hipervinculo{lib_iu:cuenta}, \textit{Nueva dirección} en
      la interfaz \hipervinculo{lib_iu:seleccionar_direccion_de_entrega} o
      presiona \textit{Continuar} en
      \hipervinculo{lib_iu:seleccionar_forma_de_pago} sin tener direcciones
      de entrega guardadas.

    \item El sistema muestra la interfaz
      \hipervinculo{lib_iu:formulario_direccion_de_entrega} con el botón
      \textit{Guardar dirección} inhabilitado.

    \item[datos] El \textbf{cliente} ingresa la siguiente información:
      \begin{itemize}
        \item Calle
        \item Número exterior e interior
        \item Colonia o localidad
        \item Delegación o municipio
        \item Estado
        \item Código postal
      \end{itemize}

    \item El sistema habilita el botón \textit{Guardar dirección}.

    \item El \textbf{cliente} presiona el botón \textit{Guardar dirección};
      [\hipervinculoLocal{lib_ta:cancelar}].

    \item El sistema valida que esa dirección no esté asociada ya al cliente;
      [\hipervinculoLocal{lib_ta:direccion_existente}].

    \item El sistema crea un nuevo registro con la dirección ingresada,
      estado \textbf{activo} y la asocia al cliente.

    \item[mensaje] El sistema muestra el mensaje
      \hipervinculo{lib_msj:direccion_de_entrega_guardada}

    \item El sistema muestra la interfaz del paso \referenciaLocal{origen}
      de la trayectoria principal.

  \end{trayectoriaPrincipal}

  %%%%%%%%%%%%%%%%%%%%%%%%%%%%%%%%%%%%%%%%%%%%%%%%%%%%%%%%%%%%%%%%%%%%%%%%%%%%%%

  \begin{trayectoriaAlternativa}[lib_ta:cancelar]
    {El visitante cancela la operación}

    \item El visitante presiona el botón \textit{cancelar}.

    \item El sistema muestra la interfaz de origen del paso
      \referenciaLocal{origen} de la trayectoria principal.

  \end{trayectoriaAlternativa}

  %%%%%%%%%%%%%%%%%%%%%%%%%%%%%%%%%%%%%%%%%%%%%%%%%%%%%%%%%%%%%%%%%%%%%%%%%%%%%%

  \begin{trayectoriaAlternativa}[lib_ta:direccion_existente]
    {Dirección ya existente}

    \item El sistema verifica que la dirección tenga estado \textbf{activo};
      [\hipervinculoLocal{lib_ta:direccion_inactiva_existente}].

    \item El sistema muestra el mensaje
      \hipervinculo{lib_msj:direccion_de_entrega_existente}.

    \item Se regresa al paso \referenciaLocal{datos} de la trayectoria
      principal.

  \end{trayectoriaAlternativa}

  %%%%%%%%%%%%%%%%%%%%%%%%%%%%%%%%%%%%%%%%%%%%%%%%%%%%%%%%%%%%%%%%%%%%%%%%%%%%%%

  \begin{trayectoriaAlternativa}[lib_ta:direccion_inactiva_existente]
    {Dirección existente pero inactiva}

    \item El sistema cambia el estado de la dirección a \textbf{activo}.

    \item Se regresa al paso \referenciaLocal{mensaje} de la trayectoria
      principal.

  \end{trayectoriaAlternativa}
}
