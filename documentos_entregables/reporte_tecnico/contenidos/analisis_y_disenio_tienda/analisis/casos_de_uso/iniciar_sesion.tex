%
% Caso de uso para iniciar sesión,
% Capítulo de análisis y diseño de tienda en línea,
% Proyecto Lovelace.
%

\casoDeUso[lib_cu:iniciar_sesion]
{Iniciar sesión}
{
  Caso que permite que un \textbf{visitante} se identifique ante el sistema. El
  \textbf{visitante} introduce su correo electrónico y contraseña y,
  dependiendo de dónde se encontraba (en inicio o finalizando una compra) es
  redirigido a la página correspondiente.

  \begin{trayectoriaPrincipal}

    \item[origen] El visitante presiona el botón \textit{Iniciar sesión}, en
      la pantalla principal \hipervinculo{lib_iu:inicio} o
      \textit{Finalizar compra} en \hipervinculo{lib_iu:finalizar_compra} sin
      haber iniciado sesión.

    \item[interfaz] El sistema muestra la pantalla
      \hipervinculo{lib_iu:iniciar_sesion} con el botón \textit{iniciar sesión}
      deshabilitado.

    \item[datos] El visitante introduce su correo electrónico y contraseña;
      [\hipervinculoLocal{lib_ta:cancelar}].

    \item El sistema habilita el botón \textit{iniciar sesión}.

    \item El visitante presiona el botón \textit{iniciar sesión};
      [\hipervinculoLocal{lib_ta:cancelar}].

    \item El sistema valida las credenciales del usuario; [%
      \hipervinculoLocal{lib_ta:credenciales_incorrectas},
      \hipervinculoLocal{lib_ta:correo_no_verificado}].

    \item El sistema registra al usuario en las variables de sesión.

    \item El sistema verifica que el usuario no esté realizando una compra
      [\hipervinculoLocal{lib_ta:mitad_de_compra}].

    \item El sistema muestra la pantalla \hipervinculo{lib_iu:inicio}.

  \end{trayectoriaPrincipal}

  %%%%%%%%%%%%%%%%%%%%%%%%%%%%%%%%%%%%%%%%%%%%%%%%%%%%%%%%%%%%%%%%%%%%%%%%%%%%%%

  \begin{trayectoriaAlternativa}[lib_ta:cancelar]
    {El visitante cancela la operación}

    \item El visitante presiona el botón \textit{cancelar}.

    \item El sistema muestra la interfaz de origen del paso
      \referenciaLocal{origen} de la trayectoria principal.

  \end{trayectoriaAlternativa}

  %%%%%%%%%%%%%%%%%%%%%%%%%%%%%%%%%%%%%%%%%%%%%%%%%%%%%%%%%%%%%%%%%%%%%%%%%%%%%%

  \begin{trayectoriaAlternativa}[lib_ta:credenciales_incorrectas]
    {Credenciales incorrectas}

    \item El correo electrónico no se encuentra registrado en el sistema o la
      contraseña dada no corresponde al correo electrónico dado.

    \item El sistema muestra el mensaje
      \hipervinculo{lib_msj:credenciales_incorrectas}.

    \item Se regresa al paso \referenciaLocal{interfaz} de la trayectoria
      principal.

  \end{trayectoriaAlternativa}

  %%%%%%%%%%%%%%%%%%%%%%%%%%%%%%%%%%%%%%%%%%%%%%%%%%%%%%%%%%%%%%%%%%%%%%%%%%%%%%

  \begin{trayectoriaAlternativa}[lib_ta:correo_no_verificado]
    {Cliente no verificado}

    \item El sistema muestra el mensaje
      \hipervinculo{lib_msj:correo_no_verificado}.

    \item Se regresa al paso \referenciaLocal{datos} de la trayectoria
      principal.

  \end{trayectoriaAlternativa}

  %%%%%%%%%%%%%%%%%%%%%%%%%%%%%%%%%%%%%%%%%%%%%%%%%%%%%%%%%%%%%%%%%%%%%%%%%%%%%%

  \begin{trayectoriaAlternativa}[lib_ta:mitad_de_compra]
    {El cliente estaba finalizando una compra}

    \item El sistema obtiene los datos de todos los libros que se encuentran en
      el carrito del usuario.

    \item El sistema arma y muestra la interfaz
      \hipervinculo{lib_iu:finalizar_compra} con los datos obtenidos en el paso
      anterior.

  \end{trayectoriaAlternativa}
}
