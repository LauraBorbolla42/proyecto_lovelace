%
% Caso de uso para realizar una compra,
% análisis y diseño de tienda en línea.
% Proyecto Lovelace.
%

\casoDeUso[lib_cu:comprar]
{Comprar}
{
  Caso que permite que un \textbf{cliente} realice la compra de los libros que
  hay en el carrito.

  Como caso de prueba del sistema tokenizador, al momento de registrar la
  compra, se hace una operación de detokenización para simular la posible
  interacción con el entorno bancario. Una implementación real se podría
  hacer de dos maneras:

  \begin{itemize}
    \item El sistema tokenizador funciona como intermediario entre la tienda
      y el entorno bancario. En este caso, en lugar de la operación de
      detokenización, se haría una petición de transacción.
    \item Es la tienda la que se ocupa de interactuar con el banco. En este
      caso, después de la operación de detokenización, la tienda tendría que
      implementar la transacción bancaria.
  \end{itemize}

  \begin{trayectoriaPrincipal}

    \item El cliente presiona el botón \textit{Finalizar compra} en la
      interfaz \hipervinculo{lib_iu:finalizar_compra}.

    \item El sistema obtiene las formas de pago del cliente en sesión.

    \item El sistema muestra la interfaz
      \hipervinculo{lib_iu:seleccionar_forma_de_pago}.

    \item El cliente selecciona la forma de pago deseada;
      [\hipervinculoLocal{lib_ta:cancelar}].

    \item El cliente presiona el botón \textit{Continuar};
      [\hipervinculoLocal{lib_ta:cancelar}].

    \item El sistema muestra la interfaz
      \hipervinculo{lib_iu:seleccionar_direccion_de_entrega}.

    \item El cliente selecciona la dirección de entrega deseada;
      [\hipervinculoLocal{lib_ta:cancelar}].

    \item El cliente presiona el botón \textit{Continuar};
      [\hipervinculoLocal{lib_ta:cancelar}].

    \item El sistema muestra la interfaz
      \hipervinculo{lib_iu:resumen_de_compra}.

    \item El cliente presiona el botón \textit{Aceptar};
      [\hipervinculoLocal{lib_ta:cancelar}].

    % Supongo que en estos últimos pasos se refleja lo incompleto de la tienda:
    % una tienda más en serio tendría que preocuparse por el envío de los
    % libros.

    \item El sistema envía la petición de detokenización del token del
      cliente en sesión\footnotemark.

    \item El sistema registra la compra y resta de las existencias los libros
      comprados.

    \item El sistema muestra el mensaje
      \hipervinculo{lib_msj:compra_registrada}.

  \end{trayectoriaPrincipal}


  \footnotetext{
    Después de este paso es en donde una tienda real tendría que hacer la
    transacción bancaria.}

  %%%%%%%%%%%%%%%%%%%%%%%%%%%%%%%%%%%%%%%%%%%%%%%%%%%%%%%%%%%%%%%%%%%%%%%%%%%%%%

  \begin{trayectoriaAlternativa}[lib_ta:cancelar]
    {El cliente cancela la operación}

    \item El cliente presiona el botón \textit{cancelar}.

    \item El sistema muestra la interfaz
      \hipervinculo{lib_iu:finalizar_compra}.

  \end{trayectoriaAlternativa}

}
