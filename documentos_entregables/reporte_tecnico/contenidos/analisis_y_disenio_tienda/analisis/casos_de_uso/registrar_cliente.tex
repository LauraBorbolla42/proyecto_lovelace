%
% Caso de uso para registrar a un cliente,
% Capítulo de análisis y diseño de tienda en línea,
% Proyecto Lovelace.
%

\casoDeUso[lib_cu:registrar_cliente]
{Registrar cliente}
{
  Caso que permite que un \textbf{visitante} se registre en el sistema,
  permiténdole identificarse como un \textbf{cliente} una vez terminado el
  proceso e iniciado sesión. Debe proveer su nombre, correo electrónico y
  una contraseña.

  \begin{trayectoriaPrincipal}

    \item[origen] El \textbf{visitante} presiona el botón de
      \textit{Registrarse} en cualquier interfaz en la que se
      encuentre.

    \item El sistema muestra la interfaz \hipervinculo{lib_iu:registro}
      con el botón \textit{Aceptar} inhabilitado.

    \item[datos] El \textbf{visitante} introduce los siguientes datos
      [\hipervinculoLocal{lib_ta:cancelar}]:
      \begin{itemize}
        \item Nombre
        \item Correo electrónico
        \item Contraseña (y su confirmación)
      \end{itemize}

    \item El sistema habilita el botón de \textit{aceptar}.

    \item El \textbf{cliente} presiona el botón de \textit{aceptar}.

    \item El sistema valida que el correo electrónico sea válido
      al cumplir con el \gls{gl:rfc}
      5322~\cite{DBLP:journals/rfc/rfc5322}, mediante la expresión
      regular

      \expresionRegular{expresiones_regulares/correo.txt}

      [\hipervinculoLocal{lib_ta:correo_incorrecto}].

    \item El sistema valida que el correo electrónico no esté ya
      registrado en el sistema; pues dos \textbf{usuarios} no pueden
      compartir el mismo correo;
      [\hipervinculoLocal{lib_ta:usuario_existente}].


    \item El sistema valida el formato de la contraseña de acuerdo con
      la expresión regular

      \expresionRegular{expresiones_regulares/contrasenia.txt}

      [\hipervinculoLocal{lib_ta:contrasenia_incorrecta}].

    \item El sistema verifica que la confirmación de contraseña
      ingresada coincida con la contraseña.
      [\hipervinculoLocal{lib_ta:confirmacion_incorrecta}].

    \item El sistema crea un nuevo cliente con estado
      \textbf{no verificado} con la información ingresada y un nuevo
      vínculo de verificación (asociado al cliente) con la fecha actual.

    \item El sistema envía un correo al cliente con el hipervínculo de
      verificación.

    \item El sistema muestra la ventana
      \hipervinculo{lib_iu:aviso_de_correo_de_confirmacion}.

    \item El \textbf{cliente} accede a la \gls{gl:url} enviada a su
      correo.

    \item El sistema verifica que no haya más de 72 HRS de diferencia
      entre la fecha de creación del vínculo y la hora actual;
      [\hipervinculoLocal{lib_ta:expiracion_de_vinculo}].

    \item El sistema cambia el estado del \textbf{cliente} a
      \textbf{verificado}.

    \item El sistema muestra la pantalla \hipervinculo{lib_iu:inicio}.

  \end{trayectoriaPrincipal}

  %%%%%%%%%%%%%%%%%%%%%%%%%%%%%%%%%%%%%%%%%%%%%%%%%%%%%%%%%%%%%%%%%%%%%%%%%%%%%%

  \begin{trayectoriaAlternativa}[lib_ta:cancelar]
    {El visitante cancela la operación}

    \item El visitante presiona el botón \textit{cancelar}.

    \item El sistema muestra la interfaz de origen del paso
      \referenciaLocal{origen} de la trayectoria principal.

  \end{trayectoriaAlternativa}

  %%%%%%%%%%%%%%%%%%%%%%%%%%%%%%%%%%%%%%%%%%%%%%%%%%%%%%%%%%%%%%%%%%%%%%%%%%%%%%

  \begin{trayectoriaAlternativa}[lib_ta:correo_incorrecto]
    {Correo electrónico incorrecto}

    \item El sistema muestra el mensaje
      \hipervinculo{lib_msj:correo_invalido}.

    \item Se regresa al paso \referenciaLocal{datos} de la trayectoria
      principal.

  \end{trayectoriaAlternativa}

  %%%%%%%%%%%%%%%%%%%%%%%%%%%%%%%%%%%%%%%%%%%%%%%%%%%%%%%%%%%%%%%%%%%%%%%%%%%%%%

  \begin{trayectoriaAlternativa}[lib_ta:contrasenia_incorrecta]
    {Contraseña incorrecta}

    \item El sistema muestra el mensaje
      \hipervinculo{lib_msj:contrasenia_incorrecta}.

    \item Se regresa al paso \referenciaLocal{datos} de la trayectoria
      principal.

  \end{trayectoriaAlternativa}

  %%%%%%%%%%%%%%%%%%%%%%%%%%%%%%%%%%%%%%%%%%%%%%%%%%%%%%%%%%%%%%%%%%%%%%%%%%%%%%

  \begin{trayectoriaAlternativa}[lib_ta:confirmacion_incorrecta]
    {Confirmación de contraseña incorrecta}

    \item El sistema muestra el mensaje
      \hipervinculo{lib_msj:confirmacion_incorrecta}.

    \item Se regresa al paso \referenciaLocal{datos} de la trayectoria
      principal.

  \end{trayectoriaAlternativa}

  %%%%%%%%%%%%%%%%%%%%%%%%%%%%%%%%%%%%%%%%%%%%%%%%%%%%%%%%%%%%%%%%%%%%%%%%%%%%%%

  \begin{trayectoriaAlternativa}[lib_ta:usuario_existente]
    {Correo registrado previamente}

    \item El sistema muestra el mensaje
      \hipervinculo{lib_msj:usuario_existente}.

    \item Se regresa al paso \referenciaLocal{datos} de la trayectoria
      principal.

  \end{trayectoriaAlternativa}

  %%%%%%%%%%%%%%%%%%%%%%%%%%%%%%%%%%%%%%%%%%%%%%%%%%%%%%%%%%%%%%%%%%%%%%%%%%%%%%

  \begin{trayectoriaAlternativa}[lib_ta:expiracion_de_vinculo]
    {Vínculo expirado}

    \item El sistema muestra el mensaje
      \hipervinculo{lib_msj:vinculo_expirado}.

    \item El sistema elimina el registro del usuario.

  \end{trayectoriaAlternativa}

}
