%
% Caso de uso para eliminar una forma de pago,
% Capítulo de análisis y diseño de tienda en línea,
% Proyecto Lovelace.
%

\casoDeUso[lib_cu:eliminar_forma_de_pago]
{Eliminar forma de pago}
{
  Caso que permite que un \textbf{cliente} elimine una de las formas de pago
  que tiene guardadas. Para mantener consistencia en la base, el registro de
  la tarjeta no será eliminado; sino que cambiará su estado a
  \textbf{inactivo}.

  \begin{trayectoriaPrincipal}

    \item[origen] El \textbf{cliente} presiona el botón de \textit{Eliminar}
      en la interfaz \hipervinculo{lib_iu:informacion_bancaria}.

    \item El sistema muestra el mensaje emergente
      \hipervinculo{lib_msj:adv_eliminar_forma_de_pago}.

    \item El \textbf{cliente} presiona \textit{Aceptar} en el mensaje
      emergente; [\hipervinculoLocal{lib_ta:cancelar}].

    \item El sistema cambia el estado de la tarjeta a \textbf{inactivo}.

    \item El sistema verifica que la dirección de facturación asociada a esta
      forma de pago no tenga otra tarjeta y cambia su estado
      a \textbf{inactivo}.

    \item El sistema muestra la interfaz
      \hipervinculo{lib_iu:informacion_bancaria}.

  \end{trayectoriaPrincipal}

  %%%%%%%%%%%%%%%%%%%%%%%%%%%%%%%%%%%%%%%%%%%%%%%%%%%%%%%%%%%%%%%%%%%%%%%%%%%%%%

  \begin{trayectoriaAlternativa}[lib_ta:cancelar]
    {El \textbf{cliente} cancela la operación}

    \item El \textbf{cliente} presiona el botón \textit{cancelar}.

    \item El sistema muestra la interfaz
      \hipervinculo{lib_iu:informacion_bancaria}.

  \end{trayectoriaAlternativa}
}
