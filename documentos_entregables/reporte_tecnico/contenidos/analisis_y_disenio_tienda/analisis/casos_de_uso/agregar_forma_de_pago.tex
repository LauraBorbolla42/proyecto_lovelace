%
% Caso de uso para agregar una forma de pago,
% Capítulo de análisis y diseño de tienda en línea,
% Proyecto Lovelace.
%

\casoDeUso[lib_cu:agregar_forma_de_pago]
{Agregar forma de pago}
{
  Caso que permite que un \textbf{cliente} agregue una forma de pago; debe
  proporcionar la siguiente información relacionada a la tarjeta a agregar:
  \begin{itemize}
    \item Nombre del titular de la tarjeta
    \item Dirección de facturación de la tarjeta
    \item Tipo de la tarjeta (crédito/débito)
    \item Emisor de la tarjeta (American Express, VISA, MasterCard, ...)
    \item Número de la tarjeta
    \item Fecha de vencimiento
  \end{itemize}

  \begin{hangparas}{2em}{0}
    Ya que este es un caso de prueba para el sistema tokenizador, el
    \textbf{cliente} puede escoger el método con el que se creará el token.
    En un ambiente de producción, es más probable que todos sus tokenizaciones
    sean hechas con un mismo algoritmo.

    El sistema no guardará el número de tarjeta, sino que realizará una
    operación de tokenización con el número de tarjeta en el servicio tokenizador
    y guardará el token que regrese como resultado y el algoritmo que fue
    utilizado.

    El sistema también debe mostrar un aviso que aclare qué servicio de
    tokenización se está utilizando y aclarar que la tienda no guardará el
    número de tarjeta (sino el sistema tokenizador).
  \end{hangparas}

  \begin{trayectoriaPrincipal}

    \item[origen] El \textbf{cliente} presiona el botón de
      \textit{Agregar forma de pago} en la interfaz
      \hipervinculo{lib_iu:informacion_bancaria}, en
      \hipervinculo{lib_iu:seleccionar_forma_de_pago} o presiona
      \textit{Finalizar compra} en \hipervinculo{lib_iu:finalizar_compra}
      sin tener formas de pago asociadas.

    \item El sistema muestra la interfaz
      \hipervinculo{lib_iu:formulario_informacion_bancaria} con el botón
      \textit{Agregar tarjeta} inhabilitado y una lista con las direcciones
      que se tienen registradas (si se tienen direcciones registradas).

    \item[datos] El \textbf{cliente} ingresa la siguiente información;
      [\hipervinculoLocal{lib_ta:usar_direccion_existente},
      \hipervinculoLocal{lib_ta:cancelar}]:
      \begin{itemize}
        \item Nombre del titular de la tarjeta
        \item Tipo de la tarjeta (crédito/débito)
        \item Emisor de la tarjeta (American Express, VISA, MasterCard, ...)
        \item Número de la tarjeta
        \item Algoritmo tokenizador
        \item Fecha de vencimiento
        \item Dirección de facturación de la tarjeta
      \end{itemize}

    \item El sistema habilita el botón \textit{Agregar tarjeta}.

    \item El \textbf{cliente} presiona el botón \textit{Agregar tarjeta};
      [\hipervinculoLocal{lib_ta:cancelar}].

    \item El sistema valida que el número de la tarjeta sea válido tomando en
      cuenta la siguiente expresión regular y asegurándose de que el dígito
      verificador sea correcto mediante el algoritmo de Luhn:

      \expresionRegular{expresiones_regulares/numero_de_tarjeta.txt}

      [\hipervinculoLocal{lib_ta:tarjeta_invalida}].

    \item El sistema valida que la fecha de vencimiento no sea menor o igual a
      la fecha actual; [\hipervinculoLocal{lib_ta:tarjeta_expirada}].

    \item El sistema verifica que no exista otra tarjeta relacionada al
      \textbf{cliente} que tenga el mismo tipo, emisor, titular y últimos
      cuatro dígitos; [\hipervinculoLocal{lib_ta:tarjeta_existente}].

    \item El sistema envía la petición de tokenización del número de tarjeta
      ingresado y el algoritmo seleccionado al servicio de tokenización;
      [\hipervinculoLocal{lib_ta:error_al_tokenizar}].

    \item El sistema crea un nuevo registro de forma de pago con el token, la
      información proporcionada en el formulario y estado \textbf{activo}.

    \item El sistema muestra el mensaje
      \hipervinculo{lib_msj:metodo_de_pago_agregado}.

    \item El sistema muestra la interfaz de origen del paso
      \referenciaLocal{origen} de la trayectoria principal.

  \end{trayectoriaPrincipal}

  %%%%%%%%%%%%%%%%%%%%%%%%%%%%%%%%%%%%%%%%%%%%%%%%%%%%%%%%%%%%%%%%%%%%%%%%%%%%%

  \begin{trayectoriaAlternativa}[lib_ta:usar_direccion_existente]
    {El \textbf{cliente} selecciona una de las direcciones ya guardadas}

    \item El \textbf{cliente} selecciona una de las direcciones ya
      guardadas.

    \item El sistema completa con la información de la dirección seleccionada
      los campos de la dirección.

    \item Se regresa al paso \referenciaLocal{datos} de la trayectoria
      principal.

  \end{trayectoriaAlternativa}

  %%%%%%%%%%%%%%%%%%%%%%%%%%%%%%%%%%%%%%%%%%%%%%%%%%%%%%%%%%%%%%%%%%%%%%%%%%%%%

  \begin{trayectoriaAlternativa}[lib_ta:cancelar]
    {El \textbf{cliente} cancela la operación}

    \item El \textbf{cliente} presiona el botón \textit{cancelar}.

    \item El sistema muestra la interfaz de origen del paso
      \referenciaLocal{origen} de la trayectoria principal.

  \end{trayectoriaAlternativa}

  %%%%%%%%%%%%%%%%%%%%%%%%%%%%%%%%%%%%%%%%%%%%%%%%%%%%%%%%%%%%%%%%%%%%%%%%%%%%%

  \begin{trayectoriaAlternativa}[lib_ta:tarjeta_invalida]
    {La tarjeta ingresada es inválida}

    \item El sistema muestra el mensaje
      \hipervinculo{lib_msj:tarjeta_invalida}.

    \item Se regresa al paso \referenciaLocal{datos} de la trayectoria
      principal.

  \end{trayectoriaAlternativa}

  %%%%%%%%%%%%%%%%%%%%%%%%%%%%%%%%%%%%%%%%%%%%%%%%%%%%%%%%%%%%%%%%%%%%%%%%%%%%%

  \begin{trayectoriaAlternativa}[lib_ta:tarjeta_expirada]
    {La tarjeta ingresada está expirada}

    \item El sistema muestra el mensaje
      \hipervinculo{lib_msj:tarjeta_expirada}.

    \item Se regresa al paso \referenciaLocal{datos} de la trayectoria
      principal.

  \end{trayectoriaAlternativa}

  %%%%%%%%%%%%%%%%%%%%%%%%%%%%%%%%%%%%%%%%%%%%%%%%%%%%%%%%%%%%%%%%%%%%%%%%%%%%%

  \begin{trayectoriaAlternativa}[lib_ta:tarjeta_existente]
    {La tarjeta ingresada ya ha sido almacenada}

    \item El sistema verifica que el registro de la tarjeta existente sea
      \textbf{activo}; [\hipervinculoLocal{lib_ta:tarjeta_existente_inactiva}].

    \item El sistema verifica que la fecha de vencimiento ingresada coincida
      con la que se tiene almacenada;
      [\hipervinculoLocal{lib_ta:tarjeta_existente_con_distinta_expiracion}].

    \item El sistema muestra el mensaje
      \hipervinculo{lib_msj:tarjeta_existente}.

    \item El sistema muestra la interfaz de origen del paso
      \referenciaLocal{origen} de la trayectoria principal.

  \end{trayectoriaAlternativa}

  %%%%%%%%%%%%%%%%%%%%%%%%%%%%%%%%%%%%%%%%%%%%%%%%%%%%%%%%%%%%%%%%%%%%%%%%%%%%%

  \begin{trayectoriaAlternativa}[lib_ta:tarjeta_existente_inactiva]
    {La tarjeta ingresada ya ha sido almacenada y se encuentra inactiva}

    \item El sistema actualiza el registro de la tarjeta con la fecha de
      vencimiento y dirección ingresada.

    \item El sistema actualiza el registro del estado de la tarjeta a
      \textbf{activo}.

    \item El sistema muestra el mensaje
      \hipervinculo{lib_msj:metodo_de_pago_agregado}.

    \item El sistema muestra la interfaz de origen del paso
      \referenciaLocal{origen} de la trayectoria principal.

  \end{trayectoriaAlternativa}

  %%%%%%%%%%%%%%%%%%%%%%%%%%%%%%%%%%%%%%%%%%%%%%%%%%%%%%%%%%%%%%%%%%%%%%%%%%%%%

  \begin{trayectoriaAlternativa}[lib_ta:error_al_tokenizar]
    {El código HTTP de respuesta no es un 2XX}

    \item El sistema muestra el mensaje
      \hipervinculo{lib_msj:error_al_tokenizar}.

    \item El sistema muestra la interfaz de origen del paso
      \referenciaLocal{origen} de la trayectoria principal.

  \end{trayectoriaAlternativa}

  %%%%%%%%%%%%%%%%%%%%%%%%%%%%%%%%%%%%%%%%%%%%%%%%%%%%%%%%%%%%%%%%%%%%%%%%%%%%%

  \begin{trayectoriaAlternativa}
    [lib_ta:tarjeta_existente_con_distinta_expiracion]
    {Hay una tarjeta existente y activa con los mismos datos, excepto la fecha
    de vencimiento}

    \item El sistema muestra el mensaje
      \hipervinculo{lib_msj:tarjeta_existente_con_distinta_expiracion}.

    \item El sistema muestra la interfaz de origen del paso
      \referenciaLocal{origen} de la trayectoria principal.

  \end{trayectoriaAlternativa}
}
