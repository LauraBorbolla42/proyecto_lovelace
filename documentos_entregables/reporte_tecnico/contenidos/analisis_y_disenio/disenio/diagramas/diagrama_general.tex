%
% Diagrama de clases general.
% Análisis y diseño de programa tokenizador, reporte técnico.
%
% Proyecto Lovelace.
%

\documentclass{standalone}
\usepackage[T1]{fontenc}
\usepackage{tikz-uml}

\tikzumlset
{
  fill class=white,
  fill package=white
}

%
% El tipo de retorno de la división entre arreglos no tendría que tener
% espacios: el problema es que si aparecen juntos, LaTeX pone una comilla de
% cierre española.
%

\begin{document}
  \begin{tikzpicture}
    \begin{umlpackage}{Utilidades}

      \umlclass[x=12, template=tipo]{Arreglo}
      {
        \# mArregloInterno : tipo* = nullptr \\
        \# mNumeroDeElementos : unsigned int = 0
      }
      {
        + Arreglo(numeroDeElementos : unsigned int) \\
        + operator[](indice : unsigned int) : IntermediarioDeArreglo<tipo> \\
        + operator/(marcasDivisorias : Arreglo<unsigned int>\&)
          : Arreglo< Arreglo<tipo> > \\
        + obtenerNumeroDeElementos() : unsigned int \\
        + obtenerCopiaDeArreglo() : tipo* \\
        + obtenerApuntador() : const tipo* \\
        + colocarErrores(constante : tipo) : void
      }

      \umlclass[x=12, y=-6]{ArregloDeDigitos}
      {
        -- mCadena : string \\
        -- mNumero : entero \\
        -- mBase : unsigned int = 10
      }
      {
        + ArregloDeDigitos(cadena : string) \\
        + ArregloDeDigitos(numero : entero) \\
        + operator[](indice : unsigned int) : IntermediarioDeArregloDeDigitos \\
        + obtenerCadena() : string \\
        + obtenerCadenaEfectiva() : string \\
        + obtenerNumero() : entero
      }

      \umlclass[template=tipo]{IntermediarioDeArreglo}
      {
        \# mArreglo : Arreglo\& \\
        \# mIndice : unisgned int
      }
      {
        + operator=(elemento : tipo) : tipo
      }

      \umlclass[y=-6]{IntermediarioDeArregloDeDigitos}
      {}
      {
        + operator=(elemento : tipo) : tipo
      }

      \umlinherit{ArregloDeDigitos}{Arreglo}
      \umlinherit{IntermediarioDeArregloDeDigitos}{IntermediarioDeArreglo}

    \end{umlpackage}
  \end{tikzpicture}
\end{document}
