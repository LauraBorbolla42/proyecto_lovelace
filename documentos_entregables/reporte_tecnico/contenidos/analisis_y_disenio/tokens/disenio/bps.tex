%
% Clases de módulo de BPS.
% Análisis y diseño de programa tokenizador, reporte técnico.
%
% Proyecto Lovelace.
%

\paragraph{Clases de BPS}

Primero, se tiene que mencionar que a pesar de que BPS puede verse como una
versión mas específica de \gls{gl:ffx}, esta clase no se realizó de forma
directa a partir de la clase de la red Feistel.

A BPS se le puede considerar como un modo de operación que usa un cifrador
interno BC, que a su vez usa un cifrador de ronda, situación por la que BPS
es una clase que implementa el modo de operación descrito en
\ref{sec:bps_modo_operacion}, que utiliza la clase BC, cuyo funcionamiento se
describe en \ref{sec:bps_cifrador_interno_bc}, el cual se puede resumir como
una red Feistel que opera sobre un cadenas construidas a partir de cualquier
alfabeto, en vez de sobre cadenas binarias.

En cuanto al cifrador de ronda, esta clase se construyó usando la librería de
Cripto++, dando la posibilidad de usar \gls{gl:aes} o \gls{gl:des}.
