%
% Listado de algoritmos implementados, análisis y diseño de generación de tokens.
% Proyecto Lovelace.
%

\subsection{Algoritmos implementados}
\label{sec:algoritmos}

En esta sección se documentan los algoritmos para generar \glspl{gl:token}
que serán implementados; cada apartado incluye una descripción del
algoritmo y su pseudocódigo. Antes de la documentación, se presentan
dos clasificaciones de los algoritmos tokenizadores: la primera es la propuesta
por el \gls{gl:pci} \gls{gl:ssc} y mencionada en la
sección~\ref{sec:requerimientos}; la segunda clasificación es propuesta por los
autores de este documento, tomando en cuenta un enfoque más criptográfico.

\paragraph{Clasificación del \gls{gl:pci} \gls{gl:ssc}}
\begin{itemize}
  \item \Glspl{gl:token} reversibles:
    \begin{itemize}
      \item \Glspl{gl:token} criptográficos:
        \begin{enumerate}
          \item BPS (véase sección~\ref{sec:bps}).
          \item FFX (véase sección~\ref{sec:ffx}).
        \end{enumerate}
      \item \Glspl{gl:token} no criptográficos:
        \begin{enumerate}
          \item AHR (véase sección~\ref{sec:ahr}).
          \item TKR (véase sección~\ref{sec:tkr}).
          \item \gls{gl:drbg}
        \end{enumerate}
    \end{itemize}
\end{itemize}

Cabe resaltar que, de acuerdo con esta clasificación, los \glspl{gl:token}
generados mediante \gls{gl:drbg} pueden quedar también en la clasificación
de \gls{gl:token} no reversible, no autenticable.

\paragraph{Clasificación propuesta}
\begin{itemize}
  \item Criptografía convencional:
    \begin{enumerate}
      \item BPS (véase sección~\ref{sec:bps}).
      \item FFX (véase sección~\ref{sec:ffx}).
    \end{enumerate}
  \item Criptografía no convencional:
    \begin{enumerate}
      \item AHR (véase sección~\ref{sec:ahr}).
      \item TKR (véase sección~\ref{sec:tkr}).
    \end{enumerate}
  \item No criptográficos:
    \begin{enumerate}
      \item \gls{gl:drbg}
    \end{enumerate}
\end{itemize}

La segunda clasificación tiene un enfoque distinto, pues toma en cuenta que todos
utilizan, en mayor o menor medida, funciones criptográficas. Consiste en tres
categorías: la primera (criptografía convencional), incluye algoritmos bien
conocidos o \textit{tradicionales}, publicados con relativa antigüedad;
la segunda categoría (criptografía no convencional) incluye algoritmos cuya
publicación es relativamente nueva; finalmente, están los
\textit{no criptograficos}, denominados así no porque no utilicen criptografía,
sino porque dependen, principalmente, de la pseudoaleatoriedad.

\subimport{ffx/}{ffx}
\subimport{bps/}{bps}
\subimport{/}{tkr}
\subimport{/}{rht}
