%
% Algoritmo de tipo híbrido reversible
% «Several Proofs of Security for a Tokenization Algorithm»
% Proyecto Lovelace.
%

\subsubsection{Algoritmo Hibrido Reversible}
\label{sec:ahr}
Longo, Aragona y Sala~\cite{aragona}, propusieron en 2017, un algoritmo de
tipo híbrido reversible. Este está basado en un cifrado de bloques con una
llave secreta y una entrada adicional.

Se sabe que el número de una tarjeta (\gls{gl:pan}) está conformado por tres
partes concatenadas: el número que identifica al emisor de la tarjeta, el que
identifica la cuenta y un número de verificación. En este algoritmo se reemplaza
la primera parte con un \gls{gl:token} \gls{gl:bin} y se cifra solo la parte
que identifica la cuenta. Al final, se calcula un nuevo dígito de verificación.

Las entradas del algoritmo son la parte del \gls{gl:pan} a cifrar y una entrada
adicional. Esta última actúa como un \textit{tweak} (véase
sección~\ref{sec:tes}), pues permite que se generen varios \glspl{gl:token}
para el mismo \gls{gl:pan}.

El algoritmo necesita una función $f$ pública que, dada una cadena
de longitud $m$ regrese una de longitud $n$ (véase sección de funciones
hash~\ref{sec:hash}). Se toman solo cifrados cuyo tamaño de bloque sea de
mínimo 128 bits. La función $f$ se encarga de poner el relleno en la entrada
para completar el bloque del cifrado y permitir la creación de varios
\glspl{gl:token} para el mismo \gls{gl:pan} utilizando la misma llave en el
proceso de cifrado. Finalmente, el algoritmo necesita una base de datos segura
que se encargará de contener los pares \gls{gl:pan}-\gls{gl:token}. Al momento
de crear los \glspl{gl:token}, se necesita acceder a la base de datos
mediante una \gls{gl:funcion_booleana} \textit{comprobar} que revisa si el
\gls{gl:token} generado ya está almacenado en la base.

Como se desea obtener un \gls{gl:token} que tenga el mismo número de dígitos
que el \gls{gl:pan} (longitud $l$) ingresado, se deben tomar en cuenta solo una
fracción de las posibles salidas del cifrado $E$; para resolver este problema,
se utiliza un método conocido como el \gls{gl:cifrado_caminata_ciclica}.

\paragraph{Notación}
A continuación se definen una serie de notaciones que se utilizarán en el
algoritmo:
\begin{itemize}
  \item $ M $ Tamaño de bloque del cifrador por bloques que se usará.
  \item $ l $ Longitud de la entrada. En este caso, $13 \geq l \geq 19$.
  \item $ n $ Número de bits necesarios para representar a la entrada:
    $n = log_2(10^l)$.
  \item $ [y]^s_b $ Indica que $y$ es menor que $b^s$: $y < b^s$.
  \item $\bar{x}$ Representación de $x$ en una cadena binaria cuando $x$ es
    representado en su forma decimal y viceversa.
\end{itemize}


El algoritmo de tokenización es el siguiente:
\begin{pseudocodigo}[%
    caption={Híbrido reversible, método de tokenización}
  ]
  entrada: PAN p; entrada_adicional u; llave k
  salida:  token
  inicio
    $t = f(u, p) || [\bar{p}]^s_b$ (paso 1)
    $c = E(k, t)$ (paso 2)
    si $(\bar{c} \mod 2^n) \geq 10^l$
      $t = c$
      Regresar al paso 2.
    fin
    $token = {[\bar{c} \mod 2^n]}^l_{10}$
    si $comprobar(token) =$ verdadero
      $u = u + 1$
      Regresar al paso 1.
    fin
    regresar token
  fin
\end{pseudocodigo}
