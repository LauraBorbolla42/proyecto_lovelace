%
% Construcción de tkr2 (y tkr2a) de
% «A Cryptographic Study of Tokenization Systems»
% Proyecto Lovelace.
%

\subsubsection{Algoritmo TKR2}
\label{sec:tkr}

En \cite{doc_sandra} se analiza formalmente el problema de la generación de
\glspl{gl:token} y se propone un algoritmo que no está basado en \gls{gl:fpe}.
Hasta antes de la publicación de este documento, los únicos métodos para
generar \glspl{gl:token} cuya seguridad estaba formalmente demostrada eran los
basados en \gls{gl:fpe}.

El algoritmo propuesto usa \glspl{gl:primitiva_criptografica} para generar
\glspl{gl:token} aleatorios y almacena en una base de datos (\gls{gl:cdv}) la
relación original de estos con los \gls{gl:pan}. Según la clasificación del
\gls{gl:pci} se trata de una algoritmo tokenizador reversible \textit{no
criptográfico}; sin embargo tratarlo de esta forma (como \textit{no
criptográfico}) resulta un tanto confuso, dado que toda la contrucción del
algoritmo y las pruebas de su seguridad se basan en fundamentos de la
criptografía.

En el pseudocódigo \ref{tkr2_cifrado} se muestra el proceso de tokenización,
mientras que en \ref{tkr2_descifrado} está la detokenización.

\begin{pseudocodigo}[%
    caption={\textit{TKR2}, método de tokenización},
    label={tkr2_cifrado}%
  ]
  entrada: PAN p; información asociada d; llave k
  salida:  token
  inicio
    $ S_1 $ $ \gets $ buscarPAN(p)
    $ S_2 $ $ \gets $ buscarInfoAsociada(d)
    si $ S_1 $ y $ S_2 $ = 0:
      t $ \gets $ RN(k)
      insertar(t, p, d)
    sino:
      t $ \gets $ $ S_1 $
    fin
    regresar t
  fin
\end{pseudocodigo}

\begin{pseudocodigo}[%
    caption={\textit{TKR2}, método de detokenización},
    label={tkr2_descifrado}%
  ]
  entrada: token t; información asociada d; llave k
  salida:  PAN
  inicio
    $ S_1 $ $ \gets $ buscarToken(t)
    $ S_2 $ $ \gets $ buscarInfoAsociada(d)
    si $ S_1 $ y $ S_2 $ = 0:
      regresar error
    sino:
      p $ \gets $ $ S_1 $
    fin
    regresar p
  fin
\end{pseudocodigo}

La mayor parte del proceso de tokenización y toda la detokenización son
bastante fáciles de comprender; lo único que queda por esclarecer es la
función generadora de \glspl{gl:token} aleatorios $ RN_k $. Idealmente,
esta función debe regresar un elemento uniformemente aleatorio del espacio
de \glspl{gl:token}. La propuesta que se hace en \cite{doc_sandra} para
instanciar esta función se presenta en el pseudocódigo \ref{tkr2_rn}.
Aquí, la variable \textit{contador} mantiene un estado del algoritmo (mantiene
su valor a lo largo de las distintas llamadas); el espacio de \glspl{gl:token}
contiene cadenas de longitud fija $ \mu $ de un alfabeto $ AL $ cuya
cardinalidad es $ l $; el número de bits necesarios para enumerar a todo el
alfabeto se guardan en $ \lambda = \lceil \log_2 l \rceil$.

\begin{pseudocodigo}[%
    caption={\textit{ACSTS}, generación de \glspl{gl:token} aleatorios},
    label={tkr2_rn}%
  ]
  entrada: llave k
  salida:  token
  inicio
    $X$ $\gets$ f(k, contador)
    $X_1, X_2, \dots, X_m$ $\gets$ cortar($X$, $\lambda$)
    t $\gets$ ""
    i $\gets$ 0
    mientras $|t| \neq \mu$:
      si entero(X_i) $\le$ $l$:
        t $\gets$ t + entero(X_i)
      fin
      i $\gets$ i  + 1
    fin
    contador $\gets$ contador + 1
    regresar t
  fin
\end{pseudocodigo}

En el pseudocódigo \ref{tkr2_rn} lo primero que se hace es utilizar una
\gls{gl:primitiva_criptografica} $ f $ para generar una cadena binaria
(hablaremos sobre este punto más adelante); esta cadena (nombrada $ X $) se
parte en subcadenas de $ \lambda $ bits; después se itera de manera
consecutiva sobre estas subcadenas binarias, si la representación entera de la
i-ésima está en el rango del alfabeto de los \glspl{gl:token}, entonces se
concatena al token resultado, sino, se pasa a la siguiente subcadena.
La longitud de la cadena regresada por $ f $ debe ser, aproximadamente,
$ 3 \mu \lambda $: dado que se espera que el comportamiento de $ f $ sea
\gls{gl:equiprobable}, entonces el ciclo correrá un promedio de $ 2 \mu $
veces.

Existen varios candidatos viables para $ f $: un cifrado de flujo (sección
\ref{sec:flujo}), pues el flujo de llave de estos produce cadenas de aspecto
aleatorio; un cifrado por bloques (sección \ref{sec:bloques}), utilizando un
modo de operación de contador (sección \ref{sec:ofb}); un \gls{gl:trng} para
obtener secuencias de bits verdaderamente aleatorias.

Por último hay que aclarar que el algoritmo presentado en el pseudocódigo
\ref{tkr2_cifrado} debe recibir un par de modificaciones más: al momento de
generar un \gls{gl:token} debe existir una validación que verifique que este
sea único (para evitar que dos \gls{gl:pan} tengan un mismo \gls{gl:token});
la base de datos debe estar cifrada, por lo que, antes de hacer inserciones y
después de hacer consultas, deben existir las operaciones correspondientes.
