%
% Requerimientos para tokens criptográficos reversibles,
% análisis y diseño de generación de tokens.
% Proyecto Lovelace.
%

\subsubsection{Criptográficos reversibles}

% RC1B
% TODO: el documento explica por qué esta cantidad; aún no lo entiendo.
\requerimiento{Probabilidad de adivinar relaciones (criptográficos)}
{
  La probabilidad de adivinar la relación entre un \gls{gl:token} y un
  \gls{gl:pan} debe de ser menor que $ 1 $ en $ 10^6 $.

  % RC1B-1
  \subrequerimiento[rq_pci:cr_distribucion_uniforme]
  {Distribución uniforme (criptográficos)}
  {
    Para un \gls{gl:pan} dado, todos los \glspl{gl:token} deben ser
    equiprobables; esto es, el mecanismo tokenizador no debe exhibir
    tendencias probabilísticas que lo expongan a ataques estadísticos.
  }

  % RC1B-2 permutación aleatoria
  % TODO: ¿Cuál es la diferencia entre una «permutación aleatoria» y una
  % «permutación aleatoria fuerte»?
  % Si el mapeo es entre dos espacios distintos... ya no se llama
  % permutación, ¿o sí?
  \subrequerimiento[rq_pci:cr_permutacion_aleatoria]
  {Permutación aleatoria (criptográficos)}
  {
    El método de tokenización debe actuar como una familia de
    \glspl{gl:permutacion} aleatoria desde el espacio de \gls{gl:pan} al
    espacio de \glspl{gl:token}.
  }

  % RC1B-3
  \subrequerimiento[rq_pci:cr_cambio_de_llave]
  {Cambio de llave (criptográficos)}
  {
    Un cambio en la llave se debe ver reflejado en un cambio en el
    \gls{gl:token} resultado.
  }

  % RC1B-4
  \subrequerimiento[rq_pci:cr_cambio_de_pan]
  {Cambio de \gls{gl:pan} (criptográficos)}
  {
    Un cambio en el \gls{gl:pan} se debe ver reflejado en un cambio en el
    \gls{gl:token} resultado.
  }

  % RC1B-5
  \subrequerimiento[rq_pci:cr_aleatoriedad_digitos]
  {Verificación de la aleatoriedad (criptográficos)}
  {
    Se debe tener un medio para verificar de forma práctica la aleatorización
    de dígitos, de acuerdo a lo establecido en \gls{gl:nist} 800-90A
    \cite{nist_aleatorios}.
  }

}

% RC1C almacenamiento duplicado
% Este no lo entiendo: primero, ¿qué no se suponía que en los criptográficos
% nunca se almacenaba el token?; segundo, aún suponiendo que este
% requerimiento fuera en realidad para los no criptográficos, ¿Cuál es la
% diferencia entre guardar el PAN completo y guardar el PAN truncado?
\requerimiento[rq_pci:cr_almacenamiento_tokens]
{Almacenamiento de tokens (criptográficos)}
{
  Los \glspl{gl:token} generados no se deben almacenar en ningún punto del
  sistema.
}

El requerimiento anterior (\hipervinculo{rq_pci:cr_almacenamiento_tokens},
RC1C en \cite{pci_tokens}) es un tanto difícil de interpretar; la versión
original establece: \textit{los \glspl{gl:token} basados en el \gls{gl:pan}
completo no se deben almacenar si el producto tokenizador también almacena
su \gls{gl:pan} truncado correspondiente}. Es un requerimiento de los
criptográficos reversibles, por lo que el \gls{gl:token} no se debería
almacenar bajo ninguna circunstancia (según la propia clasificación del
\gls{gl:pci} \gls{gl:ssc}); la redacción del requerimiento se cambió para
reflejar este hecho.

% RC4A
\requerimiento[rq_pci:cr_administracion_llaves]
{Seguridad de la administración de llaves (criptográficos)}
{
  Todas la operaciones sobre la administración de las llaves criptográficas
  deben realizarse en un dispositivo criptográfico seguro y aprobado: el
  \gls{gl:pci} \gls{gl:ssc} se encarga de hacer validaciones; también puede ser
  cualquier dispositivo validado por \gls{gl:fips} 140-2 nivel 3 o superior
  \cite{nist_modulos_criptograficos} o por la \gls{gl:iso} 13491-1.
}

% RC4C-1 y 2
\requerimiento[rq_pci:cr_longitud_llaves]
{Sobre la longitud de las llaves (criptográficos)}
{
  Las llaves para \textit{tokenizar} deben tener una
  \gls{gl:fuerza_efectiva} de, al menos, 128 bits. Cualquier llave utilizada
  para proteger o para derivar la llave del \gls{gl:token} debe de ser de igual
  o mayor \gls{gl:fuerza_efectiva}.
}

% RC4D
\requerimiento[rq_pci:cr_independencia_estadistica]
{Independencia estadística (criptográficos)}
{
  Si el espacio de llaves es usado para producir \glspl{gl:token} es dos
  contextos distintos (e. g. para distintos comerciantes), estas deben ser
  \glspl{gl:estidisticamente_independiente}.
}
