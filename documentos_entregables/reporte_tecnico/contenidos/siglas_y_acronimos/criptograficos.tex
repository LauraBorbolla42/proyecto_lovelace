%
% Siglas y acrónimos relacionados con la criptografía, reporte técnico.
% Proyecto Lovelace.
%

\setglossarypreamble[siglas_cripto]
{%
  \addcontentsline{toc}{section}{Criptográficos}
}

\newacronym[type=siglas_cripto]{gl:abl}{ABL}{Arbitrary Block Length Mode}

\newacronym[type=siglas_cripto]{gl:aes}{AES}{Advanced Encryption Standard}

\newacronym[type=siglas_cripto]{gl:cavp}{CAVP}
  {Cryptographic Algorithm Validation Program}

\newacronym[type=siglas_cripto]{gl:cbc}{CBC}{Cipher-block Chaining}

\newacronym[type=siglas_cripto]{gl:cfb}{CFB}{Cipher Feedback}

\newacronym[type=siglas_cripto]{gl:cmac}{CMAC}
  {Cipher-based \gls{gl:mac}}

\newacronym[type=siglas_cripto]{gl:cmc}{CMC}{\gls{gl:cbc}-Mask-\gls{gl:cbc}}

\newacronym[type=siglas_cripto]{gl:crhf}{CRHF}
  {Collision-Resistant Hash Function}

\newacronym[type=siglas_cripto]{gl:csp}{CSP}{Critical Security Parameter}

\newacronym[type=siglas_cripto]{gl:ctr}{CTR}{Counter Mode}

\newacronym[type=siglas_cripto]{gl:des}{DES}{Data Encryption Standard}

\newacronym[type=siglas_cripto]{gl:dh}{DH}{Diffie-Hellman}

\newacronym[type=siglas_cripto]{gl:dhe}{DHE}{\gls{gl:dh} Ephemeral}

\newacronym[type=siglas_cripto]{gl:drbg}{DRBG}
  {Deterministic Random Bit Generator}

\newacronym[type=siglas_cripto]{gl:dsa}{DSA}{Digital Signature Algorithm}

\newacronym[type=siglas_cripto]{gl:ecb}{ECB}{Electronic Codebook}

\newacronym[type=siglas_cripto]{gl:ecc}{ECC}{Elliptic Curve Cryptosystem}

\newacronym[type=siglas_cripto]{gl:ecdh}{ECDH}{Elliptic Curve \gls{gl:dh}}

\newacronym[type=siglas_cripto]{gl:ecdsa}{ECDSA}{Elliptic Curve \gls{gl:dsa}}

\newacronym[type=siglas_cripto]{gl:ecies}{ECIES}{
  Elliptic Curve Integrated Encryption Scheme}

\newacronym[type=siglas_cripto]{gl:ecmqv}{ECMQV}
  {Elliptic Curve Menezes-Qu-Vanstone}

\newacronym[type=siglas_cripto]{gl:eme}{EME}{\gls{gl:ecb}-Mask-\gls{gl:ecb}}

\newacronym[type=siglas_cripto]{gl:feal}{FEAL}
  {Fast Data Encipherment Algorithm}

% ¿FFn? No lo encuentro (búsqueda superficial en duck duck go)
% Hasta ahora caigo en la cuenta: «FFn» es como el NIST y el PCI se refieren
% a una instanciación de los parámetros de FFX.
% \newacronym{gl:ffn}{FFn}{Counther Mode}

\newacronym[type=siglas_cripto]{gl:ffx}{FFX}
  {Format-preserving Feistel-based Encryption}

\newacronym[type=siglas_cripto]{gl:fpe}{FPE}{Format Preserving Encryption}

\newacronym[type=siglas_cripto]{gl:hctr}{HCTR}{Hash \gls{gl:ctr}}

\newacronym[type=siglas_cripto]{gl:hmac}{HMAC}
  {Keyed-Hashed Message Authentication Code}

\newacronym[type=siglas_cripto]{gl:idea}{IDEA}
  {International Data Encryption Algorithm}

\newacronym[type=siglas_cripto]{gl:kdf}{KDF}{Key Derivation Function}

\newacronym[type=siglas_cripto]{gl:maa}{MAA}{Message Authenticator Algorithm}

\newacronym[type=siglas_cripto]{gl:mac}{MAC}{Message Authentication Code}

\newacronym[type=siglas_cripto]{gl:mdc}{MDC}{Message Digest Cipher}

\newacronym[type=siglas_cripto]{gl:mdc2}{MDC-2}{Modification Detection Code-2}

\newacronym[type=siglas_cripto]{gl:md4}{MD4}{Message Digest-4}

\newacronym[type=siglas_cripto]{gl:md5}{MD5}{Message Digest-5}

\newacronym[type=siglas_cripto]{gl:nrbg}{NRBG}
  {Non-deterministic Random Bit Generator}

\newacronym[type=siglas_cripto]{gl:nmac}{NMAC}{Nested \gls{gl:mac}}

\newacronym[type=siglas_cripto]{gl:oaep}{OAEP}
  {Optimal Asymmetric Encryption Padding}

\newacronym[type=siglas_cripto]{gl:ocb}{OCB}{Offset Codebook}

\newacronym[type=siglas_cripto]{gl:ofb}{OFB}{Output Feedback}

\newacronym[type=siglas_cripto]{gl:omac}{OMAC}{One-key \gls{gl:mac}}

\newacronym[type=siglas_cripto]{gl:owhf}{OWHF}{One-Way Hash Function}

\newacronym[type=siglas_cripto]{gl:pmac}{PMAC}{Parallel \gls{gl:mac}}

\newacronym[type=siglas_cripto]{gl:prf}{PRF}{Pseudorandom Function}

\newacronym[type=siglas_cripto]{gl:prng}{PRNG}{Pseudorandom Number Generator}

\newacronym[type=siglas_cripto]{gl:rbg}{RBG}{Random Bit Generator}

\newacronym[type=siglas_cripto]{gl:rng}{RNG}{Random Number Generator}

\newacronym[type=siglas_cripto]{gl:rsa}{RSA}
  {Ron Rivest, Adi Shamir, Leonard Adleman}

% No estoy seguro de que la ES realmente sea eso; no lo dicen claramente
% en ningún lado:
% http://www.inf.pucrs.br/~calazans/graduate/TPVLSI_I/RSA-oaep_spec.pdf
% Creo que sí, yo juraría que era una especie de juego de palabras con AES,
% pero creo que ño.
\newacronym[type=siglas_cripto]{gl:rsaes}{RSAES}
  {\gls{gl:rsa} Encryption Scheme}

\newacronym[type=siglas_cripto]{gl:safer}{SAFER}
  {Secure And Fast Encryption Routine}

\newacronym[type=siglas_cripto]{gl:sha}{SHA}{Secure Hash Algorithm}

\newacronym[type=siglas_cripto]{gl:tdes}{TDES}{Triple \gls{gl:des}}

\newacronym[type=siglas_cripto]{gl:tbc}{TBC}{Tweakable Block Cipher}

\newacronym[type=siglas_cripto]{gl:tbc_modo}{TBC}{Tweakable Block Chaining}

\newacronym[type=siglas_cripto]{gl:tes}{TES}{Tweakable Encyphering Scheme}

\newacronym[type=siglas_cripto]{gl:trng}{TRNG}{True Random Number Generator}

\newacronym[type=siglas_cripto]{gl:xcb}{XCB}{Extended Codebook}
