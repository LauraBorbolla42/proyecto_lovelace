%
% Sección introducción a la criptografía, capítulo de antecedentes.
% Proyecto Lovelace.
%

\section{Introducción a la criptografía}

La información presentada a continuación puede consultarse con más profundidad
en las siguientes referencias
\cite{menezes, DBLP:series/isc/DelfsK07}.

La palabra criptografía proviene de las etimologías griegas \textit{Kriptos}
(ocultar) y \textit{Graphos} (escritura), y es definida por la \gls{gl:rae}
como «el arte de escribir con clave secreta o de un modo enigmático».
De manera más formal se puede definir a la criptografía como la ciencia
encargada de estudiar y diseñar por medio de técnicas matemáticas, métodos y
modelos capaces de resolver problemas en la seguridad de la información, como
la confidencialidad de esta, su integridad y la autenticación de su origen.

\subimport{/}{definicion_objetivos}
\subimport{/}{criptoanalisis}
\subimport{/}{clasificacion}
