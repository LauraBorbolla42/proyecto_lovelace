%
% Explicación de FEAL, capítulo de antecedentes.
% Proyecto Lovelace.
%

\subsection{Fast Data Encipherment Algorithm (FEAL)}

Es una familia de algoritmos que ha tenido una participación crítica
en el desarrollo y refinamiento de varias técnicas del criptoanálisis,
tales como el criptoanálisis lineal y diferencial. \gls{gl:feal}-N mapea
bloques de texto en claro de 64 bits a bloques de 64 bits de texto
cifrado mediante una llave secreta de 64 bits. Es un cifrado Feistel de
$n-$\glspl{gl:ronda} parecido a \gls{gl:des}, pero con una función $f$ más
simple.

\gls{gl:feal} fue diseñado para ser veloz y simple, especialmente para
microprocesadores de 8 bits: usa operaciones orientadas a bytes, evita
el uso de permutaciones de bit y tablas de consulta. La versión inicial
de cuatro \glspl{gl:ronda} (FEAL-4), propuesto como una alternativa rápida a
\gls{gl:des}, fue encontrado mucho más inseguro de lo planeado; por lo
que se propuso realizar más \glspl{gl:ronda} (FEAL-16 y FEAL-32) para compensar
y ofrecer un nivel de seguridad parecido a \gls{gl:des}; sin embargo, el
rendimiento se ve fuertemente afectado mientras el número de \glspl{gl:ronda}
aumenta; y, mientras \gls{gl:des} puede mejorar su velocidad con tablas de
consulta, resulta más complicado para \gls{gl:feal}.

\begin{pseudocodigo}[caption={FEAL-8, cifrado.}, label={feal8:1}]
  entrada:    64 bits de texto en claro $M = m_1 \dots m_{64}$;
              llave de 64 bits $K = k_1 \dots k_{64}$.
  salida:     bloque de texto cifrado de 64 bits $C = c_1 \dots c_{64}$.
  inicio
    Calcular 16 subllaves de 16 bits para $K$.
    Definir $M_L = m_1 \dots m_{32}; M_R = m_{33} \dots m_{64}$.
    $(L_0, R_0) \leftarrow (M_L, M_R) \oplus ((K_8, K_9), (K_{10}, K_{11}))$
    $R_0 \leftarrow R_0 \oplus L_0$.
    para_todo $i$ desde 1 hasta 8:
      $L_i \leftarrow R_{i-1}$
      $R_i \leftarrow L_{i-1} \oplus f(R_{i-1}, K_{i-1})$
    fin
    $L_8 \leftarrow L_8 \oplus R_8$
    $(R_8, L_8) \leftarrow (R_8, L_8) \oplus ((K_{12}, K_{13}),(K_{14},K_{15}))$
    $C \leftarrow (R_8, L_8)$.
  fin
\end{pseudocodigo}

% ¿Qué demonios? Las tuplas de llaves se salen del margen.

Para descifrar se utiliza el mismo algoritmo, con la misma llave $K$ y el
texto cifrado $C = (R_8, L_8)$ se utiliza como la entrada $M$; sin
embargo, la generación de llaves se hace al revés: las subllaves
$ ((K_{12}, K_{13}), \allowbreak (K_{14}, K_{15})) $ se utilizan para la
$\oplus$ inicial, las $ ((K_8, K_9), (K_{10}, K_{11})) $ para la $\oplus$
final y en las \glspl{gl:ronda} se utiliza de la subllave $K_7$ a la $K_0$.

\gls{gl:feal} con una llave de 64 bits puede ser generalizado a un número
$n$ de rondas \glspl{gl:ronda} con $n$ par, aunque se recomienda $n = 2^x$.
