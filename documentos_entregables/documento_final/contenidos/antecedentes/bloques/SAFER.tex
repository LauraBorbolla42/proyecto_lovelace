%
% Explicación de SAFER, capítulo de antecedentes.
% Proyecto Lovelace.
%

\subsection{Secure And Fast Encryption Routine (SAFER)}

El cifrado \gls{gl:safer} K-64 es un cifrado por bloques de 64 bits
iterativo. Consiste en $r$ \glspl{gl:ronda} idénticas seguidas por una
transformación. Originalmente se recomendaban $6$ \glspl{gl:ronda} seguidas,
sin embargo, ahora se utiliza una generación de claves ligeramente modificada y
el uso de $8$ \glspl{gl:ronda} (máximo 10). Ambas generaciones de llaves
expanden la llave de 64 bits en $2r+1$ subllaves, cada una de 64 bits (dos por
cada ronda y una más para la transformación de salida).

Este cifrado consiste completamente en operaciones de bytes, por lo que
es adecuado para procesadores con tamaños de palabra pequeños, como los
chips de tarjetas.

%   Lo siento, pero si corto la línea de la entrada, la entrada queda a
%  la mitad y se ve muy raro.

\begin{pseudocodigo}[caption={SAFER K-64, cifrado.}, label={safer:1}]
  entrada: $r, 6\leq r \leq10$; $64-$bits de datos en claro $M = m_1 \dots m_{64}$; $ K = k_1 \dots k_{64}$.
  salida: bloque cifrado de $64-$bits $Y = (Y_1, \dots, Y_8)$.
  inicio
    Calcular las subllaves $K_1, \dots, K_{2r+1}$
    $(X_1, X_2, \dots X_8) \leftarrow (m_1 \dots m_8, m_9 \dots m_{16}, \dots, m_{57} \dots m_{64})$
    para_todo $i$ desde $1$ hasta $r$:
      a) Para $j = 1, 4, 5, 8: X_j \leftarrow X_j \oplus K_{2i-1}[j]$
        Para $j = 2, 3, 6, 7: X_j \leftarrow X_j + K_{2i-1}[j]$$mod$ $2^8$
      b) Para $j = 1, 4, 5, 8: X_j \leftarrow S$[$X_j$]
        Para $j = 2, 3, 6, 7: X_j \leftarrow S_{inversa}X_j$
      c) Para $j = 1, 4, 5, 8: X_j \leftarrow X_j + K_{2i}[j]$$mod$ $2^8$
        Para $j = 2, 3, 6, 7: X_j \leftarrow X_j \oplus K_{2i}[j]$
      d) Para $j = 1, 3, 5, 7: (X_j, X_{j+1}) \leftarrow f(X_j, X_{j+1})$.
      e) $(Y_1, Y_2 ) \leftarrow f(X_1, X_3), (Y_3, Y_4 ) \leftarrow f(X_5, X_7)$,
        $(Y_5, Y_6 ) \leftarrow f(X_2, X_4), (Y_7, Y_8 ) \leftarrow f(X_6, X_8 )$.
        Para $j$ desde 1 hasta 8: $X_j \leftarrow Y_j$
      f) $(Y_1, Y_2) \leftarrow f(X_1, X_3), (Y_3, Y_4) \leftarrow f(X_5, X_7)$,
        $(Y_5, Y_6 ) \leftarrow f(X_2, X_4), (Y_7, Y_8) \leftarrow f(X_6, X_8)$.
        Para $j$ desde 1 hasta 8: $X_j \leftarrow Y_j$.
    fin
    Para $j = 1, 4, 5, 8: Y_j \leftarrow X_j \oplus K_{2r+1}[j]$.
    Para $j = 2, 3, 6, 7: Y_j \leftarrow X_j + K_{2r+1} [j] \mod 2^8$.
  fin
\end{pseudocodigo}

Para descifrar, se utiliza la misma llave $K$ y las subllaves $K_i$
que fueron utilizadas al cifrar. Cada paso del cifrado se hace en orden
inverso, del último al primero; comenzando con una transformación de entrada
utilizando la llave $K_{2r+1}$ para deshacer la transformación de salida, se
sigue con las \glspl{gl:ronda} de descifrado utilizando las llaves de $K_{2r}$
a $K_1$, invirtiendo los pasos cada ronda.
