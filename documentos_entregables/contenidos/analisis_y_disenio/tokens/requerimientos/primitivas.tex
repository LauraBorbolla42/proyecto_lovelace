%
% Requerimientos para las primitivas criptográficas usadas,
% análisis y diseño de generación de tokens.
% Proyecto Lovelace.
% Anexo C de las guías para tokens del PCI.
%

\subsubsection{Primitivas criptográficas}
\label{sec:primitivas}

En esta sección se resumen los requerimientos mínimos que debe de tener
cualquier \gls{gl:primitiva_criptografica} que se use dentro del sistema
\textit{tokenizador}. Esta información se presenta en el anexo C de 
\cite{pci_tokens}, el cual está clasificado como \textit{informativo} solamente
(no \textit{normativo}). Con respecto a las \glspl{gl:primitiva_criptografica},
la parte \textit{normativa} está controlada por el programa \gls{gl:cavp} del
\gls{gl:nist}; el cual evalúa las implementaciones usadas de una serie de 
primitivas comunes\footnote{Esta lista se puede encontrar en 
\url{https://csrc.nist.gov/Projects/Cryptographic-Algorithm-Validation-Program}}.

En la tabla \ref{minimo_llaves} se colocan los tamaños mínimos de llaves y
\glspl{gl:modo_de_operacion} (sección \ref{sec:modos}) asociados
para las \glspl{gl:primitiva_criptografica} de los «algoritmos
criptográficos»\footnote{El \gls{gl:pci} \gls{gl:ssc} parece dividir a las
\glspl{gl:primitiva_criptografica} en \textit{algoritmos criptográficos},
\textit{funciones hash} y \textit{generadores de números pseudoaleatorios};
esto resulta confuso dado que las tres categorías pertenecen al campo de
estudio de la criptografía.} permitidos. Estos son: \gls{gl:aes} (sección
\ref{sec:aes}), \gls{gl:rsa} (sección \ref{sec:clasificacion}), \gls{gl:ecc} y
\gls{gl:dsa}/\gls{gl:dh}. Se hace especial énfasis en que \gls{gl:tdes} no
está permitido.

% La última lista en el párrafo es para evitar que las siglas de los
% protocolos no enlistados antes aparezcan expandidos en la tabla y para poder
% poner referencias cruzadas (en las celdas se ven raras). Para cuando no haya
% nada que hacer, estaría bien agregar a los antecedentes a las curvas
% elípticas, y para el intercambio de llaves Diffie-Hellman.
% Lástima que son muchos modos de operación para hacer lo mismo; creo que
% aquí sí ocuparé \acrshort (ahora sí que crecerá nuestra sección de siglas
%y acrónimos).

% Para AES había un FFn que no encuentro en ningún otro lado.
% Luego el PCI les cambia el nombre, tengo que buscar más a fondo.
% Misma historia para EHD.

\begin{table}[H]
  \centering
  \begin{tabular}{| c | c | c |}
    \hline
    \textbf{Algoritmo} &
    \textbf{Tamaño de llave} &
    \textbf{Modo de operación} \\ [0.5ex]
    \hline
    \textbf{\gls{gl:aes}} &
    128 &
    \acrshort{gl:ctr},
    \acrshort{gl:ocb},
    \acrshort{gl:cbc},
    \acrshort{gl:ofb},
    \acrshort{gl:cfb} \\
    \hline
    \textbf{\gls{gl:rsa}} &
    3072 &
    \acrshort{gl:rsaes}-\acrshort{gl:oaep} \\
    \hline
    \textbf{\gls{gl:ecc}} &
    256 &
    \acrshort{gl:ecdh},
    \acrshort{gl:ecmqv},
    \acrshort{gl:ecdsa},
    \acrshort{gl:ecies} \\
    \hline
    \textbf{\gls{gl:dsa}/\gls{gl:dh}} &
    3072/256 &
    \acrshort{gl:dhe}\\
    \hline
  \end{tabular}
  \caption{Longitudes de llave mínimas y \glspl{gl:modo_de_operacion}
      permitidos para algoritmos criptográficos}
  \label{minimo_llaves}
\end{table}

% ¿Qué son los bits de seguridad? (glosario)
La tabla \ref{hash_permitidos} enlista los algoritmos hash (sección
\ref{sec:hash}) permitidos. Para evitar introducir fallas de seguridad a
través de las funciones hash, estas deben proveer al menos tantos bits de
seguridad como el algoritmo criptográfico usado, y en cualquier caso, no
menos de 128 bits (lo que deja fuera a \gls{gl:md4} y \gls{gl:md5}).

\begin{table}[H]
  \centering
  \begin{tabular}{| c | c |}
    \hline
    \textbf{Bits de seguridad} &
    \textbf{Algoritmo hash} \\ [0.5ex]
    \hline
    128 & \gls{gl:sha}-256 \\
    \hline
    128 & \gls{gl:sha}3-256 \\
    \hline
    192 & \gls{gl:sha}3-384 \\
    \hline
    256 & \gls{gl:sha}-512 \\
    \hline
    256 & \gls{gl:sha}3-512 \\
    \hline
  \end{tabular}
  \caption{Algoritmos hash permitidos}
  \label{hash_permitidos}
\end{table}

% Hmmm... está sección cada vez se hace más necesaria en nuestros antecedentes:
% generadores de números aleatorios.

El número de bits de entropía utilizados para los generadores de números
aleatorios debe de ser mayor o igual al número de bits de seguridad utilizados
para las primitivas anteriores. Cuando se utilicen generadores determinísticos,
estos deben seguir las recomendaciones del \gls{gl:nist} en
\cite{nist_aleatorios}.
