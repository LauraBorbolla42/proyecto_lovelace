%
% Requerimientos para tokens no criptográficos reversibles,
% análisis y diseño de generación de tokens.
% Proyecto Lovelace.
%

\subsubsection{No criptográficos reversibles}

  % RN 1A
  \requerimiento{Generación y almacenamiento de tokens}{}{
    La generación de un \gls{gl:token} debe realizarse independientemente de 
    su \gls{gl:pan}, y la relación entre un \gls{gl:pan} y su \gls{gl:token} 
    sólo tiene que estar almacenada en la base de datos (\gls{gl:cdv}) 
    establecida.
  }

  % RN 1B
  \requerimiento{Probabilidad de encontrar un PAN}{
    La probabilidad de encontrar un \gls{gl:pan} a partir de su respectivo 
    \gls{gl:token} debe de ser menor que $1$ en $10^6$.

      % RN 1B-1
      \subrequerimiento{Distribución equiprobable}{
        Para un \gls{gl:pan} dado, todos sus \glspl{gl:token} respectivos 
        deben ser \glspl{gl:equiprobable}, esto es que el sistema 
        \textit{tokenizador} no debe exhibir patrones probabilísticos que 
        lo vulneren a un ataque estadístico.
      }

      % RN 1B-2
      \subrequerimiento{Permutaciones aleatorias}{
        El método de tokenización debe actuar como una familia de 
        \glspl{gl:permutacion} aleatoria en el espacio efectivo de los 
        \glspl{gl:pan} al espacio de \glspl{gl:token}.
      }

      % RN 1B-3
      \subrequerimiento{Parámetros de tokenización}{
        El método de tokenización debe incluir parámetros tales que, un 
        cambio en estos parámetros resulte en un \gls{gl:token} diferente; 
        por ejemplo, un cambio en la instancia del proceso debe derivar en 
        una secuencia de \glspl{gl:token} distintos, incluso cuando es usada 
        la misma secuencia de \glspl{gl:token}.
      }

      % RN 1B-4
      % Contradoctoria con RN 1A, si el token es independiete, no debe 
      % reflejar cambios.
      \subrequerimiento{Reflejo de cambios}{
        Al cambiar parte de un \gls{gl:pan}, debe cambiar su \gls{gl:token} 
        resultante.
      }
  }

  % RN 1C
  \requerimiento{Distribución imparcial}{
    El proceso de generación de \glspl{gl:token} debe garantizar una 
    distribución de \glspl{gl:token} imparcial, esto significa que la 
    probabilidad de cualquier par \gls{gl:pan}/\gls{gl:token} debe ser igual.
  }

  % RN 1D
  \requerimiento{Instancias estadísticamente independientes}{
    Si varias o diferentes instancias de la bases de datos (\gls{gl:cdv}) son 
    usadas, cada una de estas debe ser \glspl{gl:estidisticamente_independiente}.
  }

  % RN 2A
  \requerimiento{Asignación de tokens}{
    La asignación de un \gls{gl:token} a un \gls{gl:pan} debe realizarse 
    por medio de una búsqueda de datos o un índice dentro de la base de 
    datos (\gls{gl:cdv}), y no por medio de métodos criptográficos.
      
      % RN 2A-1
      % No se entiende como no usar métodos logicos en la asignación.
      \subrequerimiento{Asignación probabilísticamente independiente}{
        El \gls{gl:pan} y el \gls{gl:token} debe ser probabilísticamente 
        independientes. Cualquier método lógico o matemático no debe ser 
        usado para \textit{tokenizar} el \gls{gl:pan} o \textit{detokenizar} 
        el \gls{gl:token}.
      }
  }

  % RN 2B y RN 3B
  \requerimiento{Controles de acceso basados en roles}{
    Para obtener el \gls{gl:pan} de su \gls{gl:token} asociado, o acceder 
    a la base de datos (\gls{gl:cdv}), se deben de requerir controles de 
    acceso basados en roles o \gls{gl:rbacs} por sus siglas en inglés.
  }

  % RN 3A
  \requerimiento{Cifrado de la base de datos}{
    Dentro de la base de datos (\gls{gl:cdv}), los \gls{gl:pan} deben ser 
    cifrados con una llave de mínimo 128 bits de \gls{gl:fuerza_efectiva}.
  }

  % RN 4A
  % Fuera de alcance.
  \requerimiento{Seguridad de la administración de llaves}{
    Todas la operaciones sobre la administración de las llaves criptográficas 
    deben realizarse en un dispositivo criptográfico seguro y aprobado.
  }

  % RN 4B
  \requerimiento{Administración de llaves de acuerdo al estándar}{
    La administración de las llaves criptográficas debe realizarse de 
    acuerdo con el estándar \gls{gl:nist}/\acrshort{gl:iso}; por ejemplo 
    el \gls{gl:nist} SP 800-57, el \acrshort{gl:iso}/\acrshort{gl:iec} 
    11770 o el \gls{gl:nist} SP 800-130.
  }

