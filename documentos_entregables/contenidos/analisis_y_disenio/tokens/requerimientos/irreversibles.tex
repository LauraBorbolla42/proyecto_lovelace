%
% Requerimientos para tokens irreversibles,
% análisis y diseño de generación de tokens.
% Proyecto Lovelace.
%

\subsubsection{Irreversibles}

% IT1A
\requerimiento{Sobre la generación de \glspl{gl:token}}
{
  El mecanismo utilizado para la generación de los \glspl{gl:token}
  no es reversible (o improbable).

  % IT1A-1
  \subrequerimiento{Sobre el mecanismo generador}
  {
    El proceso para crear \glspl{gl:token} clasificado como irreversible
    debe asegurar que el mecanismo, proceso o algoritmo utilizado para 
    crear el \gls{gl:token} no sea reversible. Si una función hash (véase
    sección \ref{sec:hash}) es utilizada, esta debe ser una 
    \gls{gl:primitiva_criptografica} y utilizar una llave secreta $k$ tal que
    el mero conocimiento de la función hash no permita la creación de un
    \gls{gl:oraculo}.
  }

  % IT1A-2
  \subrequerimiento{Contenido en claro}
  {
    Los \glspl{gl:token} no deben contener dígitos en claro del \gls{gl:pan}
    original, excepto que estos dígitos sean una coincidencia.
  }

  % IT1A-3
  \subrequerimiento{Creación de un diccionario}
  {
    La creación de una tabla o \textit{diccionario} de \glspl{gl:token}
    estáticos debería ser imposible, o, al menos, al punto de satisfacer
    que la probabilidad de predecir correctamente el \gls{gl:pan} debería ser 
    menor que $\frac{1}{10^6}$.
  }

  % IT1A-4
  \subrequerimiento{Sobre el proceso de autenticación}
  {
    Si un \gls{gl:token} autenticable no reversible es utilizado, el
    proceso de autenticación no debe revelar información suficiente para
    realizar búsquedas, excepto una exhaustiva (\gls{gl:pan} por 
    \gls{gl:pan}) y se deben implementar controles para detectar estas
    últimas.
  }
}

% IT4A
\requerimiento{Sobre el manejo adecuado de llaves}
{
  Se deben seguir las buenas prácticas criptográficas respecto a la
  administración de llaves.
  
  % IT4A-1
  \subrequerimiento{Sobre el ciclo de vida}
  {
    La llave tokenizadora debe seguir la política de los ciclos de llaves
    descritos en el \acrshort{gl:iso}/\acrshort{gl:iec} 115681.
  }

  % IT4A-2
  \subrequerimiento{Descripción del periodo criptográfico activo}
  {
    La política sobre el tiempo de vida de la llave debe incluir una 
    descripción sobre el periodo criptográfico activo de la llave 
    tokenizadora en cuestión.
  }

  % IT4A-3
  \subrequerimiento{Sobre la destrucción de las llaves}
  {
   	El proveeddor debe incorporar una función que permita la destrucción
   	de sus llaves criptográficas sin tener que alterar o abrir el
   	dispositivo.  
  }
}

\begin{table}[H]
  \centering
  \begin{tabular}{| p{5.5cm} | p{2cm} | p{4cm} | p{4cm} |}

    \hline
      \textbf{Requerimiento}    &
      \textbf{PCI}              &
      \textbf{Clasificación}    &
      \textbf{Observación}     \\ [0.8ex]
    \hline

    SUBREQUTO-13/1 Sobre el mecanismo generador.  
    &  IT 1A-1  &  \hyperref[dm:gen_tokens]{Generación de tokens}              &  Aplicable        \\ \hline
    SUBREQUTO-13/2 Contenido en claro.  
    &  IT 1A-2  &  \hyperref[dm:gen_tokens]{Generación de tokens}              &  Aplicable        \\ \hline
    SUBREQUTO-13/3 Creación de un diccionario.  
    &  IT 1A-3  &  \hyperref[dm:gen_tokens]{Generación de tokens}              &  Aplicable        \\ \hline
    SUBREQUTO-13/4 Sobre el proceso de autenticación.  
    &  IT 1A-4  &  \hyperref[dm:gen_tokens]{Generación de tokens}              &  Aplicable        \\ \hline
    SUBREQUTO-14/1 Sobre el ciclo de vida.  
    &  IT 4A-1  &  \hyperref[dm:man_llaves]{Manejo criptográfico de llaves}    &  Fuera de alcance \\ \hline
    SUBREQUTO-14/2 Descripción del periodo criptográfico activo.  
    &  IT 4A-2  &  \hyperref[dm:man_llaves]{Manejo criptográfico de llaves}    &  Fuera de alcance \\ \hline
    SUBREQUTO-14/3 Sobre la destrucción de las llaves.  
    &  IT 4A-3  &  \hyperref[dm:man_llaves]{Manejo criptográfico de llaves}    &  -                \\ \hline

  \end{tabular}
  \caption{Requerimientos para tokens irreversibles.}
\end{table}

