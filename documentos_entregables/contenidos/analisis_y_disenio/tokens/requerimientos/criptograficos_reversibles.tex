%
% Requerimientos para tokens criptográficos reversibles,
% análisis y diseño de generación de tokens.
% Proyecto Lovelace.
%

\subsubsection{Criptográficos reversibles}

\begin{itemize}

  % RC1A seguridad de la administración de llaves.
  \item La administración de las llaves criptográficas debe ser segura.

  \begin{itemize}

    % RC1A-1 exportar llave en claro prohibido
    \item Las llaves usadas para generar \glspl{gl:token} no se deben poder
      exportar en claro desde el programa.

    % RC1A-2 entropía de generación de llaves.
    % TODO: ¿Cómo funciona la entropía de una fuente generadora?
    \item La fuente generadora de llaves debe tener, al menos, 128 bits de
      \gls{gl:entropia}.

    % RC1A-3 llaves de uso único.
    \item Las llaves criptográficas usadas para generar \glspl{gl:token} no
      deben ser usadas para ningún otro fin.

  \end{itemize}

  % RC1B probabilidad de adivinar relaciones
  % TODO: el documento explica porque esta cantidad; aún no lo entiendo.
  \item La probabilidad de adivinar la relación entre un \glspl{gl:token} y un
    \gls{gl:pan} debe de ser menor que $ 1 $ en $ 10^6 $.

  \begin{itemize}

    % RC1B-1 distribución uniforme (relacionado con RC1A-2, creo)
    \item Para un \gls{gl:pan} dado, todos los \glspl{gl:token} deben ser
      equiprobables; esto es, el mecanismo tokenizador no debe exhibir
      tendencias probabilísticas que lo expongan a ataques estadísticos.

    % RC1B-2 permutación aleatoria
    % TODO: ¿Cuál es la diferencia entre una «permutación aleatoria» y una
    % «permutación aleatoria fuerte»?
    % Si el mapeo es entre dos espacios distintos... ya no se llama
    % permutación, ¿o sí?
    \item El método de tokenización debe actuar como una familia de
      \glspl{gl:permutacion} aleatoria desde el espacio de \gls{gl:pan} al
      espacio de \glspl{gl:token}.

    % RC1B-3 Cambio de llave
    \item Un cambio en la llave se debe ver reflejado en un cambio en el
      \gls{gl:token} resultado.

    % RC1B-4 Cambio de PAN
    \item Un cambio en el \gls{gl:pan} se debe ver reflejado en un cambio en el
      \gls{gl:token} resultado.

    % RC1B-5 Aleatoriedad de dígitos; esto aún no lo entiendo.

  \end{itemize}

  % RC1C almacenamiento duplicado
  % ¿Que no se suponía que en los criptográficos NUNCA se almacenaban los
  % tokens? ¿Qué hace este requerimiento aquí?
  \item Los \glspl{gl:token} que se basan en el \gls{gl:pan} completo no se
    deben almacenar si ya hay guardado un \gls{gl:token} equivalente del
    \gls{gl:pan} truncado.

  % Los del dominio 2 son requerimeintos para la api web.

\end{itemize}
