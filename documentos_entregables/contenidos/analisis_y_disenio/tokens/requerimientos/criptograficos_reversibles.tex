%
% Requerimientos para tokens criptográficos reversibles,
% análisis y diseño de generación de tokens.
% Proyecto Lovelace.
%

\subsubsection{Criptográficos reversibles}

% RC1A
\requerimiento{Seguridad de la administración de llaves}
{
  la administración de las llaves criptográficas debe ser segura.

  % RC1A-1
  \subrequerimiento{Exportar llave en claro prohibido}
  {
    Las llaves usadas para generar \glspl{gl:token} no se deben poder
    exportar en claro desde el programa.
  }

  % RC1A-2
  % TODO: ¿Cómo funciona la entropía de una fuente generadora?
  \subrequerimiento{Entropía de generación de llaves}
  {
    La fuente generadora de llaves debe tener, al menos, 128 bits de
    \gls{gl:entropia}.
  }

  % RC1A-3
  \subrequerimiento{Llaves de uso único}
  {
    Las llaves criptográficas usadas para generar \glspl{gl:token} no
    deben ser usadas para ningún otro fin.
  }
}

% RC1B
% TODO: el documento explica porque esta cantidad; aún no lo entiendo.
\requerimiento{Probabilidad de adivinar relaciones}
{
  La probabilidad de adivinar la relación entre un \gls{gl:token} y un
  \gls{gl:pan} debe de ser menor que $ 1 $ en $ 10^6 $.

  % RC1B-1 (relacionado con RC1A-2, creo)
  \subrequerimiento{Distribución uniforme}
  {
    Para un \gls{gl:pan} dado, todos los \glspl{gl:token} deben ser
    equiprobables; esto es, el mecanismo tokenizador no debe exhibir
    tendencias probabilísticas que lo expongan a ataques estadísticos.
  }

  % RC1B-2 permutación aleatoria
  % TODO: ¿Cuál es la diferencia entre una «permutación aleatoria» y una
  % «permutación aleatoria fuerte»?
  % Si el mapeo es entre dos espacios distintos... ya no se llama
  % permutación, ¿o sí?
  \subrequerimiento{Permutación aleatoria}
  {
    El método de tokenización debe actuar como una familia de
    \glspl{gl:permutacion} aleatoria desde el espacio de \gls{gl:pan} al
    espacio de \glspl{gl:token}.
  }

  % RC1B-3
  \subrequerimiento{Cambio de llave}
  {
    Un cambio en la llave se debe ver reflejado en un cambio en el
    \gls{gl:token} resultado.
  }

  % RC1B-4
  \subrequerimiento{Cambio de \gls{gl:pan}}
  {
    Un cambio en el \gls{gl:pan} se debe ver reflejado en un cambio en el
    \gls{gl:token} resultado.
  }

  % RC1B-5 Aleatoriedad de dígitos, se debe validar que se sigue la
  % recomendación del NIST para los generadores de números pseudoaleatorios;
  % fuera de alcance.
  \subrequerimiento{Verificación de la aleatoriedad}
  {
    Se debe tener un medio para verificar de forma práctica la aleatorización
    de dígitos.
  }

}

% RC1C almacenamiento duplicado
% Este no lo entiendo: primero, ¿qué no se suponía que en los criptográficos
% nunca se almacenaba el token?; segundo, aún suponiendo que este
% requerimiento fuera en realidad para los no criptográficos, ¿Cuál es la
% diferencia entre guardar el PAN completo y guardar el PAN truncado?
\requerimiento{Almacenamiento de tokens}
{
  Los \glspl{gl:token} generados a partir de un \gls{gl:pan} completo no
  deben se ser almacenados en el producto de tokenización ni sistemas
  dependientes.
}

%\item Los \glspl{gl:token} que se basan en el \gls{gl:pan} completo no se
%  deben almacenar si ya hay guardado un \gls{gl:token} equivalente del
%  \gls{gl:pan} truncado.

% Los del dominio 2 son requerimeintos para la api web.

% RC4A Protección de llaves
% Se debe ocupar un SCD (secure cryptographyc device) para administrar
% llaves; puro capitalismo: puedes comprar los del PCI o hacer que el NIST
% valide los tuyos; fuera de alcance, evidentemente.
% De cualquier forma, sería requerimiento de la api web.
\requerimiento{Seguridad de la administración de llaves}
{
  Todas la operaciones sobre la administración de las llaves criptográficas
  deben realizarse en un dispositivo criptográfico seguro y aprobado.
}


% RC4B
% Claros y concisos: el primer documento son unas recomendaciones 160 hojas
% y la ISO ni siquiera la puedo ver (más capitalismo); puedo «comprar» el
% estándar, ¿qué clase de estándar se vende?
\requerimiento{Prácticas para la administración de llaves}
{
  La administración de llaves criptográficas debe llevarse a cabo de
  acuerdo al estándar del \gls{gl:nist} \cite{nist_llaves} y a
  \acrshort{gl:iso}/\acrshort{gl:iec} 11770.

  % RC4B1 Ciclo de vida de las llaves.
  % Un ISO más... empiezo a extrañar los RFC de la IETF.
  % TODO: incluir sección con resumen del contenido del anexo D.
  \subrequerimiento[rq:cripto:ciclo_de_vida]{Ciclo de vida de las llaves}
  {
    Las llaves para \textit{tokenizar} deben seguir una política de
    ciclo de vida como la descrita en  \acrshort{gl:iso}/\acrshort{gl:iec}
    11568-1.
  }

  % RC4B2
  \subrequerimiento{Descripción de periodos para cada llave}
  {
    La política de la vida útil de cada llave debe incluir una
    descripción del periodo activo.
  }

  % RC4B3
  \subrequerimiento{Permitir la destrucción de todas las llaves}
  {
    El sistema debe permitir la destrucción de las llaves sin necesidad
    de una manipulación externa.
  }
}

% RC4C-1 y -2 Longitud de llaves
% (esto se repite con la sección de primitivas).
\requerimiento{Sobre la longitud de las llaves}
{
  Las llaves para \textit{tokenizar} deben tener una
  \gls{gl:fuerza_efectiva} de, al menos, 128 bits. Cualquier llave utilizada
  para proteger o para derivar la llave del token debe de ser de igual o
  mayor \gls{gl:fuerza_efectiva}.
}

% RC4D
\requerimiento{Independencia estadística}
{
  Si el espacio de llaves es usado para producir \glspl{gl:token} es dos
  contextos distintos (e. g. para distintos comerciantes), estas deben ser
  \glspl{gl:estidisticamente_independiente}.
}
