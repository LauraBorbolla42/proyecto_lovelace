%
% Recomendaciones del NIST para la generación de bits pseudoaleatorios,
% capítulo de análisis y diseño para la generación de tokens,
% Funciones
% Proyecto Lovelace.
%

Un \gls{gl:drbg} necesita cinco funciones para poder funcionar correctamente.
\begin{description}
  \item [Instanciación] Antes de generar bits pseudoaleatorios, el \gls{gl:drbg}
    debe ser instanciado; esta función se encarga de revisar que los parámetros
    de entrada sean válidos, determina  el nivel de seguridad para la instancia
    de \gls{gl:drbg} que se generará, obtiene la entrada de \gls{gl:entropia}
    capaz de soportar el nivel de seguridad y el \gls{gl:nonce} (solo si es
    requerido), determina el estado interno inicial y, si se tienen varias
    instancias del \gls{gl:drbg} simultáneas, obtiene un manejador de estado.
    Las entradas de la función son las siguientes:
    \begin{enumerate}
      \item Nivel de seguridad requerido.
      \item Bandera que indica si se necesita o no la resistencia de predicción.
      \item Cadena de personalización.
      \item Entrada de \gls{gl:entropia}. 
      \item \gls{gl:nonce}
    \end{enumerate}
    Las primeras entradas deben ser provistas por la aplicación consumidora. La
    salida de la función de instanciación a esta aplicación consiste en:
    \begin{enumerate}
      \item El estado del proceso; regresa un \textit{EXITOSO} o un estado
        inválido indicando el error que hubo al instanciar al \gls{gl:drbg}.
      \item El manejador de estado, utilizado para identificar el estado interno
        de esta instancia para generar, cambiar la \gls{gl:semilla} y demás
        funciones.
    \end{enumerate}
    \begin{pseudocodigo}[caption={DRBG, instanciación.}, label={drbg:1}]
      entrada:  $nivel\_seguridad\_requerido$, $bandera\_prediccion$, $cadena\_personalizacion$
                $entrada\_entropia$, $nonce$ 
      salida:   $estado$, $manejador\_estado$
      inicio
        si $nivel\_seguridad\_requerido$ > $mayor\_nivel\_seguridad\_soportado$:
          regresar $BANDERA\_ERROR, invalido$
        si $bandera\_prediccion = $verdadero Y $resistencia\_prediccion$ no es soportado:
          regresar $BANDERA\_ERROR, invalido$
        si $longitud(cadena\_personalizacion) > longitud\_maxima$:
          regresar $BANDERA\_ERROR, invalido$
        Asignar a $nivel\_seguridad$ el nivel más bajo de seguridad mayor o igual
          a $nivel\_seguridad\_requerido$ del conjunto $\{112, 128, 192, 256\}$
        ($estado, entrada\_entropia$) = $obtener\_entropia(nivel\_seguridad,$...
          ... $longitud\_min, longitud\_max, bandera\_prediccion)$
        si $(estado \neq EXITOSO)$:
          regresar $estado, invalido$ 
        Obtener nonce
        $estado\_trabajo\_inicial = INSTANCIAR\_ALGORITMO(entrada\_entropia,$...
          ... $nonce, cadena\_personalizacion, nivel\_seguridad)$
        Obtener $manejador\_estado$ para el estado interno vacío. 
        si no se encuentra un estado interno vacío:
          regresar $BANDERA\_ERROR, invalido$
        Configurar el estado interno de la nueva instancia con los valores iniciales
          y la información administrativa.
        regresar $(EXITOSO, manejador\_estado)$
      fin
    \end{pseudocodigo}
\end{description}