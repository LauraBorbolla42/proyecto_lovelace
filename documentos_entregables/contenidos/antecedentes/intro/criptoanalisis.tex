
\subsection{Criptoanálisis y ataques}

  La criptografía forma parte de una ciencia más general llamada criptología, 
  la cual tiene otras ramas de estudio, como es el criptoanálisis que es la 
  ciencia encargada de estudiar los posibles ataques a sistemas criptográficos, 
  que son capaces de contrariar sus servicios ofrecidos. Entre los principales 
  objetivos del criptoanálisis están interceptar la información que circula 
  en un canal de comunicación, alterar dicha información y suplantar la 
  identidad, rompiendo con los servicios de confidencialidad, integridad y 
  autenticación respectivamente.

  Los ataques que se realizan a sistemas criptográficos dependen de la cantidad 
  de recursos o conocimientos con los que cuenta el adversario que realiza dicho 
  ataque, dando como resultado la siguiente clasificación.

  \begin{enumerate}
 
    \item \textbf{Ataque con sólo texto cifrado}

      En este ataque el adversario solamente es capaz de obtener la información
      cifrada, y tratará de conocer su contenido en claro a partir de ella.
      Esta forma de atacar es la más básica, y todos los métodos
      criptográficos deben poder soportarla.

    \item \textbf{Ataque con texto en claro conocido}

      Esta clase de ataques ocurren cuando el adversario puede obtener pares
      de información cifrada y su correspondiente información en claro, y
      por medio de su estudio, trata de descifrar otra información cifrada
      para la cual no conoce su contenido.

    \item \textbf{Ataque con texto en claro elegido}

      Este ataque es muy parecido al anterior, con la diferencia de que en
      este el adversario es capaz de obtener los pares de información
      cifrada y en claro con el contenido que desee.

    \item \textbf{Ataque con texto en claro conocido adaptativo}

      En este ataque el adversario es capaz de obtener los pares de
      información cifrada y en claro con el contenido que desee, además
      tiene amplio acceso o puede usar de forma repetitiva el mecanismo de
      cifrado.

    \item \textbf{Ataque con texto en claro elegido adaptativo}

      En este caso el adversario puede elegir información cifrada y conocer
      su contenido, dado que tiene acceso a los mecanismos de descifrado.

  \end{enumerate}
