
La palabra criptografía proviene de las etimologías griegas \textit{Kriptos}
(ocultar) y \textit{Graphos} (escritura), y es definida por la Real Academia
Española como el arte de escribir con clave secreta o de un modo enigmático.
De manera más formal se puede definir a la criptografía como la ciencia
encargada de estudiar y diseñar por medio de técnicas matemáticas, métodos y
modelos capaces de resolver problemas en la seguridad de la información, como
la confidencialidad de esta, su integridad y la autenticación de su origen.

  \subsection{Objetivos de la criptografía}

    La criptografía tiene como finalidad cumplir los siguientes cuatro
    servicios.

    \begin{enumerate}

      \item \textbf{Confidencialidad}

        Es el servicio encargado de mantener legible la información solo a
        aquellos que estén autorizados a visualizarla.

      \item \textbf{Integridad}

        Este servicio se encarga de evitar la alteración de la información de
        forma no autorizada, esto incluye la inserción, sustitución y
        eliminación de los datos.

      \item \textbf{Autenticación}

        Este servicio se refiere a la identificación tanto de las personas que
        establecen una comunicación, garantizando que cada una es quien dice
        ser; como del origen de la información que se maneja, garantizando la
        veracidad de la hora y fecha de origen, el contenido, tiempos de
        envío, entre otros.

      \item \textbf{No repudio}

        Es el servicio que evita que el autor de la información o de alguna
        acción determinada, pueda negar su validez, ayudando así a prevenir
        situaciones de disputa.

    \end{enumerate}
