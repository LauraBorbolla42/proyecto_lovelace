%
% Sección introducción a la criptografía, capítulo de antecedentes.
% Proyecto Lovelace.
%

\section{Introducción a la criptografía}

La palabra criptografía proviene de las etimologías griegas \textit{Kriptos} 
(ocultar) y \textit{Graphos} (escritura), y es definida por la Real Academia 
Española como el arte de escribir con clave secreta o de un modo enigmático. 
De manera más formal se puede definir a la criptografía como la ciencia 
encargada de estudiar y diseñar por medio de técnicas matemáticas métodos y 
modelos capaces de resolver problemas en la seguridad de la información, como 
la confidencialidad de esta, su integridad y la autenticación de su origen.

  \subsection{Objetivos de la Criptografía}
    
    La criptografía tiene como finalidad cumplir los siguientes cuatro 
    servicios.
    
    \begin{enumerate}
      
      \item \textbf{Confidencialidad}
        
        Es el servicio encargado de mantener legible la información solo a 
        aquellos que estén autorizados a visualizarla.
    
      \item \textbf{Integridad}
        
        Este servicio se encarga de evitar la alteración de la información de 
        forma no autorizada, esto incluye la inserción, sustitución y 
        eliminación de los datos.
    
      \item \textbf{Autenticación}
        
        Este servicio se refiere a la identificación tanto de las personas que 
        establece una comunicación, garantizando que cada una es quien dice 
        ser; como del origen de la información que se maneja, garantizando la 
        veracidad de la hora y fecha de origen, el contenido, tiempos de 
        envió, entre otros.

      \item \textbf{No repudio}
        
        Es el servicio que evita que el autor de la información o de alguna 
        acción determinada, pueda negar su validez, ayudando así a prevenir 
        situaciones de disputa.

    \end{enumerate}

  \subsection{Criptoanálisis y Ataques}
  
    La criptografía forma parte de una ciencia más general llamada 
    criptología, la cual tiene otras ramas de estudio, como es el 
    criptoanálisis, que es la ciencia encargada de estudiar los posibles 
    ataques a sistemas criptográficos, que son capaces de contrariar sus 
    servicios ofrecidos. 
    Los ataques que realizan a sistemas criptográficos dependen de la cantidad 
    de recursos o conocimientos con los que cuenta el adversario que realiza 
    dicho ataque, dando así a la siguiente clasificación.

    \begin{enumerate}
      
      \item \textbf{Ciphertext-only attack}
      
        En este ataque el adversario solo es capaz de obtener la información 
        cifrado, y tratara de conocer su contenido en claro a partir de ella.
        Esta forma de atacar es la más básica, y todos los métodos 
        criptográficos deben poder soportarla.
      
      \item \textbf{Known-plaintext attack}
      
        Esta clase de ataques ocurren cuando el adversario puede obtener pares 
        de información cifrada y su correspondiente información en claro, y 
        por medio de su estudio, trata de descifrar otra información cifrada 
        para la cual no conoce su contenido.

      \item \textbf{Chosen-plaintext attack}
      
        Este ataque es muy parecido al anterior, con la diferencia de que en 
        este el adversario es capaz de obtener los pares de información 
        cifrada y en claro con el contenido que desee.

      \item \textbf{Adaptively-chosen-plaintext attack}
      
        En este ataque el adversario es capaz de obtener los pares de 
        información cifrada y en claro con el contenido que desee y además 
        tiene amplio acceso o puede usar de forma repetitiva el mecanismo de 
        encriptación.
      
      \item \textbf{Chosen and adaptively-chosen-ciphertext attack}
        En este caso el adversario puede elegir información cifrada y conocer 
        su contenido, dado que tiene acceso a los mecanismos de descifrado.
    
    \end{enumerate}
  
  \subsection{Clasificación de la criptografía}
  
    La criptografía puede clasificarse de forma histórica en dos categorías, 
    la criptografía clásica y la criptografía moderna. La criptografía clásica 
    es aquella que se utilizó desde la antigüedad, teniéndose registro de su 
    usado desde hace más 4000 años por los egipcios, hasta la mitad del siglo 
    XX. En esta los métodos utilizados para cifrar eran variados, pero en su 
    mayoría usaban la transposición y la sustitución, además de que la mayoría 
    se mantenían en secretos. Mientras que la criptografía moderna es la que 
    se inició después la publicación de la \textit{Teoría de la información} 
    por Claude Elwood Shannon, dado que esta sentó las bases matemáticas para 
    la criptología en general.
 
    Una manera de clasificar es de acuerdo a las técnicas y métodos empleados 
    para cifrar la información, esta clasificación se puede observar en la 
    siguiente figura.
  
    Entrando dentro de la clasificación de la criptografía clásica, se tienen 
    los cifrados por transposición, los cuales se basan en técnicas de 
    permutación de forma que los caracteres de la información en claro se 
    reordenen mediante algoritmos específicos, y los cifrados por sustitución, 
    que utilizan técnicas de modificación de los caracteres por otros 
    correspondiente a un alfabeto específico para el cifrado. 
  
    En cuanto a la criptografía moderna, esta tiene dos vertientes, la 
    criptografía simétrica o de llave secreta y la asimétrica o de llave 
    publica. Hablando de la primer vertiente, se puede decir que es aquella 
    que utiliza un modelo matemático para cifrar y descifrar un mensaje 
    utilizando únicamente una llave que permanece secreta.
  
    En la figura anterior se puede observar el proceso para establecer una 
    comunicación segura por medio de la criptografía simétrica. Primero, tanto 
    Alice como Bob deben de establecer una llave única y compartida $k$,para 
    que después, Alice, actuando como el emisor, cifra un mensaje $m$ usando 
    la llave $k$ por medio del algoritmo de cifrado $E(k,m)$ para obtener el 
    mensaje cifrado $c$ y enviárselo a Bob. Posteriormente Bob, como receptor, 
    se encarga de descifrar $c$ con ayuda de la llave $k$ por medio del 
    algoritmo de descifrado $D(k,c)$ para obtener el mensaje original $m$.
  
    Entre los beneficios de este tipo de criptografía está su utilidad para 
    cifrar archivos personales, su relativa facilidad de uso y para garantizar 
    la confidencialidad e integridad debido a el uso de una llave, y su 
    rapidez, pero en contraparte, su uso genera problemas para organizar y 
    compartir las llaves secretas de una forma segura y eficiente.
  
    Ahora, adentrándose en la criptografía asimétrica, se tiene que su idea 
    principal es el uso de 2 llaves distintas para cada persona, una llave 
    pública para cifrar que este disponible para cualquier otra persona, y una 
    llave privada para descifrar, que se mantiene disponible solo para su 
    propietario.
  
    El proceso para establecer una comunicación segura por medio de este tipo 
    de criptografía es el siguiente, primero, Alice nuevamente como el emisor, 
    cifra un mensaje $m$ con la llave publica de de Bob $pk$ usa el algoritmo 
    de cifrado $E(pk,m)$ para obtener $c$ y enviarlo. Después Bob como 
    receptor, se encarga de descifrar $c$ por medio del algoritmo de 
    descifrado $D(sk,c)$ haciendo uso de su llave privada $sk$. Este proceso 
    se refleja gráficamente el la siguiente figura.

    Entre los uso que se le da a esta criptografía está el mantener la 
    distribución de llaves privada segura, y establecer métodos que garantizan 
    la autenticación y el no repudio, como por ejemplo en las firmas y 
    certificados digitales.
