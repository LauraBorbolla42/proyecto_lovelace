%
% Sección introducción a la criptografía, capítulo de antecedentes.
% Proyecto Lovelace.
%

\section{Introducción a la criptografía}

La palabra criptografía proviene de las etimologías griegas \textit{Kriptos} 
(ocultar) y \textit{Graphos} (escritura), y es definida por la Real Academia 
Española como el arte de escribir con clave secreta o de un modo enigmático. 
De manera más formal se puede definir a la criptografía como la ciencia 
encargada de estudiar y diseñar por medio de técnicas matemáticas métodos y 
modelos capaces de resolver problemas en la seguridad de la información, como 
la confidencialidad de esta, su integridad y la autenticación de su origen.

  \subsection{Objetivos de la Criptografía}
    
    La criptografía tiene como finalidad cumplir los siguientes cuatro 
    servicios.
    
    \begin{enumerate}
      
      \item \textbf{Confidencialidad}
        
        Es el servicio encargado de mantener legible la información solo a 
        aquellos que estén autorizados a visualizarla.
    
      \item \textbf{Integridad}
        
        Este servicio se encarga de evitar la alteración de la información de 
        forma no autorizada, esto incluye la inserción, sustitución y 
        eliminación de los datos.
    
      \item \textbf{Autenticación}
        
        Este servicio se refiere a la identificación tanto de las personas que 
        establece una comunicación, garantizando que cada una es quien dice 
        ser; como del origen de la información que se maneja, garantizando la 
        veracidad de la hora y fecha de origen, el contenido, tiempos de 
        envió, entre otros.

      \item \textbf{No repudio}
        
        Es el servicio que evita que el autor de la información o de alguna 
        acción determinada, pueda negar su validez, ayudando así a prevenir 
        situaciones de disputa.

    \end{enumerate}

  \subsection{Criptoanálisis y Ataques}
  
    La criptografía forma parte de una ciencia más general llamada 
    criptología, la cual tiene otras ramas de estudio, como es el 
    criptoanálisis, que es la ciencia encargada de estudiar los posibles 
    ataques a sistemas criptográficos, que son capaces de contrariar sus 
    servicios ofrecidos. 
    Los ataques que realizan a sistemas criptográficos dependen de la cantidad 
    de recursos o conocimientos con los que cuenta el adversario que realiza 
    dicho ataque, dando así a la siguiente clasificación.

    \begin{enumerate}
      
      \item \textbf{Ciphertext-only attack}
      
        En este ataque el adversario solo es capaz de obtener la información 
        cifrado, y tratara de conocer su contenido en claro a partir de ella.
        Esta forma de atacar es la más básica, y todos los métodos 
        criptográficos deben poder soportarla.
      
      \item \textbf{Known-plaintext attack}
      
        Esta clase de ataques ocurren cuando el adversario puede obtener pares 
        de información cifrada y su correspondiente información en claro, y 
        por medio de su estudio, trata de descifrar otra información cifrada 
        para la cual no conoce su contenido.

      \item \textbf{Chosen-plaintext attack}
      
        Este ataque es muy parecido al anterior, con la diferencia de que en 
        este el adversario es capaz de obtener los pares de información 
        cifrada y en claro con el contenido que desee.

      \item \textbf{Adaptively-chosen-plaintext attack}
      
        En este ataque el adversario es capaz de obtener los pares de 
        información cifrada y en claro con el contenido que desee y además 
        tiene amplio acceso o puede usar de forma repetitiva el mecanismo de 
        encriptación.
      
      \item \textbf{Chosen and adaptively-chosen-ciphertext attack}
        En este caso el adversario puede elegir información cifrada y conocer 
        su contenido, dado que tiene acceso a los mecanismos de descifrado.
    
    \end{enumerate}
  
