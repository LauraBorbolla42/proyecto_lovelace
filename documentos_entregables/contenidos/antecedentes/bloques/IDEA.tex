\subsection{International Data Encryption Algorithm (IDEA)}

Cifra bloques de 64 bits utilizando una llave de 128 bits. Este cifrado
está basado en una generalización de la estructura Feistel y consiste en
8 rondas idénticas seguidas por una transformación. Cada ronda $r$ 
utiliza 6 subllaves $K^{(r)}_i$ ($1 \leq i \leq 6$) de 16 bits que se 
encargan de transformar una entrada $X$ de 64 bits en una salida de 
cuatro bloques de 16-bits, que son utilizados como entrada en la 
siguiente ronda. La salida de la ronda 8 tiene como entrada la 
transformación de salida que, al emplear cuatro llaves adicionales 
$K^{(9)}_i$ ($1 \leq i \leq 4$), produce los datos cifrados 
$Y = (Y_1, Y_2, Y_3, Y_4)$.

%   Lo siento, pero si corto la línea de la entrada, la entrada queda a 
%  la mitad y se ve muy raro.

\begin{pseudocodigo}[caption={IDEA, cifrado.}, label={idea:1}]
  entrada: $64-$bits de datos en claro $M = m_1 \dots m_64$; llave de $128-$bits $ K = k_1 \dots k_{128}$.
  salida: bloque cifrado de $64-$bits $Y = (Y_1, Y_2, Y_3, Y_4)$.
  inicio
    Calcular las subllaves $K^{(r)}_1, \dots, K^{(r)}_{6}$ para las rondas $1 \leq r \leq 8$ y $K^{(9)}_1, \dots, K^{(9)}_{4}$
    para la transformación de salida.
    $(X_1, X_2, X_3, X_4) \leftarrow (m_1 \dots m_{16}, m_{17} \dots m_{32}, m_{33} \dots m_{48}, m_{49} \dots m_{64})$ donde $X_i$ almacena 16 bits.
    Para la ronda $r$ desde 1 hasta 8:
      a) $X_1 \leftarrow X_1 \times K_1^{(r)} mod 2^{16} + 1$
         $X_4 \leftarrow X_4 \times K_4^{(r)} mod 2^{16} + 1$
         $X_2 \leftarrow X_2 + K_2^{(r)} mod 2^{16}$
         $X_3 \leftarrow X_3 + K_3^{(r)} mod 2^{16}$
      b) $t_0 \leftarrow K_5^{(r)} \times (X_1 \oplus X_3) mod 2^{16} + 1$
         $t_1 \leftarrow K_6^{(r)} \times (t_0 + (X_2 \oplus X_4)) mod 2^{16} + 1$
         $t_2 \leftarrow t_0 + t_1 mod 2^{16}$
      c) $X_1 \leftarrow X_1 \oplus t_1$
         $X_4 \leftarrow X_4 \oplus t_2$
         $a \leftarrow X_2 \oplus t_2$
         $X_2 \leftarrow X_3 \oplus t_1$
         $X_3 \leftarrow a$
    fin
    Realizar la transformación de salida:
      $Y_1 \leftarrow X_1 \times K_1^{(9)} mod 2^{16} + 1$
      $Y_4 \leftarrow X_4 \times K_4^{(9)} mod 2^{16} + 1$
      $Y_2 \leftarrow X_3 + K_2^{(9)} mod 2^{16}$
      $Y_3 \leftarrow X_2 + K_3^{(9)} mod 2^{16}$
  fin
\end{pseudocodigo}

El descifrado se realiza con el mismo algoritmo de cifrado, pero 
utilizando como entrada los datos cifrados $Y$ como entrada $M$. Se usa la
misma llave $K$; aunque las subllaves sufren una modificación al ser
generadas, pues se utiliza una tabla y se realizan las operaciones 
contrarias (inverso de la adición y el inverso del producto).

Descartando los ataques a las llaves débiles, no hay un mejor ataque 
publicado para el IDEA de 8 rondas que el de la búsqueda exhaustiva
en el espacio de llave. Por lo que la seguridad está ligada a la 
creciente debilidad de su tamaño de bloque relativamente pequeño.