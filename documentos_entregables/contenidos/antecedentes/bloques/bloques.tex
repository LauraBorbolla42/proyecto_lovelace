%
% Sección de cifrado por bloques, capítulo de antecedentes.
% Proyecto Lovelace.
%

\newpage
\section{Cifrados por bloques}

Los cifrados por bloque son esquemas de cifrado simétricos que, como bien 
lo explica su nombre, operan mediante bloques de datos. Normalmente los 
bloques tienen una longitud de 64 o de 128, mientras que las llaves 
pueden ser de 56, 128, 192 o 256.

En muchos sistemas criptográficos, estos cifrados son elementos 
importantes, pues su versatilidad permite construir con ellos 
generadores de números pseudoaleatorios, cifrados de flujo MACs y 
funciones hash. Sirven también como componentes centrales en técnicas de 
autenticación de mensajes, mecanismos de integridad de datos, protocolos 
de autenticación de entidad y esquemas de firma electrónica que usan 
llaves simétricas. 

Los cifrados por bloque están limitados en la práctica por varios 
factores, tales como el límite de memoria, la velocidad requerida o 
restricciones impuestas por el hardware o el software en el que se 
implementan. Normalmente, se debe escoger entre eficiencia y seguridad

Idealmente, al cifrar por bloques, cada bit del bloque cifrado depende
de todos los bits de la llave y del texto en claro; no debería existir
una relación estadística evidente entre el texto en claro y el texto
cifrado; el alterar tan solo un bit en el texto en claro o en la llave 
debería alterar cada uno de los bits del texto cifrado con una 
probabilidad de $\frac{1}{2}$; y alterar un bit del texto cifrado
debería provocar resultados impredecibles al recuperar el texto en
claro.

\subsection{Definición}

\begin{equation}
  \label{cifrado_bloques_def}
 	E: \{0,1\}^r \times \{0,1\}^n \longrightarrow \{0,1\}^n, (k,m) 
	\longmapsto  E(k,m)
\end{equation}


Utilizando una llave secreta $k$ de longitud binaria $r$ el algoritmo de 
cifrado $E$ cifra bloques en claro $m$ de una longitud binaria fija $n$ y 
da como resultado bloques cifrados $c = E (k,m)$ cuya longitud también es 
$n$. $n$ es el tamaño de bloque del cifrado. 
El espacio de llave $K = \{0,1\}^r$. Para cada llave, existe una función 
$D_k(c)$ que permite tomar un bloque cifrado $c$ y regresarlo a su forma 
original $m$.

Generalmente, los cifrados por bloque procesan el texto claro en bloques 
relativamente grandes ($n \geq 64$), contrastando con los cifradores de 
flujo. Cuando la longitud del mensaje en claro excede el tamaño de 
bloque, se utilizan los modos de operación.

Los parámetros más importantes de los cifrados por bloque son los 
siguientes:
\begin{itemize}
	\item Tamaño de bloque
	\item Tamaño de llave
\end{itemize}

\subsection{Criterios para evaluar los cifrados por bloque}

A continuación se listan algunos de los criterios que pueden ser tomados en cuenta para evaluar estos cifrados:
\begin{enumerate}
	\item \textbf{Nivel de seguridad.} La confianza que se le tiene a un 
		cifrado va creciendo con el tiempo, pues va siendo analizado y 
		sometido a pruebas.
	\item \textbf{Tamaño de llave.} La entropía del espacio de la llave 
		define un límite superior en la seguridad del cifrado al tomar en 
		cuenta la búsqueda exhaustiva. Sin embargo, hay que tener cuidado 
		con su tamaño, pues también aumentan los costos de generación, 
		transmisión, almacenamiento, etcétera.
	\item \textbf{Tamaño de bloque.} Impacta la seguridad, pues entre más 
		grandes, mejor; sin embargo, tiene repercusiones en el costo de la 
		implementación, además de que puede afectar el rendimiento del 
		cifrado.
	\item \textbf{Expansión de datos.} Es extremadamente deseable que los 
		datos cifrados no aumenten su tamaño respecto a los datos en claro.
	\item \textbf{Propagación de error.} Descifrar datos que contienen 
		errores de bit puede llevar a recuperar incorrectamente el texto en
		claro, además de propagar errores en los bloques pendientes por 
		descifrar. Normalmente, el tamaño de bloque afecta el error de 
		propagación.
\end{enumerate}


\subsection{Data Encryption Standard (DES)}

Este es, probablemente, el cifrado simétrico por bloques más conocido;
ya que en la década de los 70 estableció un precedente al ser el primer
algoritmo a nivel comercial que publicó abiertamente sus
especificaciones y detalles de implementación. Se encuentra definido
en el estándar americano \acrshort{gl:fips} 46-2.

\acrshort{gl:des} es un cifrado Feistel que procesa bloques de $n=64$ bits y
produce bloques cifrados de la misma longitud. Aunque la llave es de 64 bits,
8 son de paridad, por lo que el tamaño \textit{efectivo} de la llave es de
56 bits. Las $2^{56}$ llaves implementan, máximo, $2^{56}$ de las
$2^{64}!$ posibles \glspl{gl:biyeccion} en bloques de 64 bits.

Con la llave $K$ se generan 16 subllaves $K_i$ de 48 bits; una para cada
\gls{gl:ronda}. En cada \gls{gl:ronda} se utilizan 8 \textit{cajas-s}
(mapeos de sustitución de 6 a 4 bits). La entrada de 64 bits es dividida por la
mitad en $L_0$ y $R_0$. Cada \gls{gl:ronda} $i$ va tomando las entradas
$L_{i-1}$ y $R_{i-1}$ de la \gls{gl:ronda} anterior y produce salidas de 32
bits $L_i$ y $R_i$ mientras $1 \leq i \leq 16$ de la siguiente manera:

\begin{equation}
  \label{cifrado_des}
  \begin{aligned}
    L_i = {} & R_{i-1} \\
    R_i = {} & L_{i-1} \oplus f(R_{i-1}, K_i) \\
    & donde \quad f(R_{i-1}, K_i) = P(S(E(R_{i-1})\oplus K_i))
  \end{aligned}
\end{equation}

$E$ se encarga de expandir $R_{i-1}$ de 32 bits a 48, $P$ es una
permutación de 32 bits y $S$ son las cajas-s. 

\begin{pseudocodigo}[caption={DES, cifrado.}, label={des:1}]
  entrada:  64 bits de texto en claro $M = m_1 \dots m_{64}$;
            llave de 64 bits $K = k_1 \dots k_{64}$.
  salida:   bloque de texto cifrado de 64 bits $C = c_1 \dots c_{64}$.
  inicio
    Calcular 16 subllaves $K_i$ de 48 bits partiendo de $K$.
    Obtener $(L_0, R_0)$ de la tabla de permutaciones iniciales $IP(m_1m_2\dots m_{64})$
    para_todo $i$ desde 1 hasta 16: 
      L_i = R_{i-1}
      Obtener $f(R_{i-1}, K_i)$:
        a) Expandir $R_{i-1} = r_1r_2\dots r_{32}$ de 32 a 48 bits
          usando $E$: $T \leftarrow E(R_{i-1})$.
        b) $T^\prime \leftarrow T \oplus K_i$. Donde $T^\prime$ es representado
          como ocho cadenas de 6 bits cada una $(B_1, \dots, B_8)$.
        c) $T'' \leftarrow (S_1(B_1), S_2(B_2), \dots S_8(B_8))$
        d) $T''' \leftarrow P(T'')$
      R_i = L_{i-1} \oplus f(R_{i-1}, K_i)
    fin
    $b_1b_2 \dots b_{64} \leftarrow (R_{16}, L{16})$.
    $C \leftarrow IP^{-1}(b_1b_2 \dots b_{64})$
  fin
\end{pseudocodigo}

El descifrado \acrshort{gl:des} consiste en el mismo algoritmo de cifrado,
con la misma llave $K$, pero utilizando las subllaves en orden inverso:
$K_{16}, K_{15}, \dots, K_1$.

\subsubsection{Llaves débiles}
Tomando en cuenta las siguientes definiciones

\begin{itemize}
  \item Llave débil: una llave $K$ tal que $E_K(E_K(M)) = M$ para toda
    $x$; en otras palabras, una llave débil permite que, al cifrar dos
    veces con la misma llave, se obtenga de nuevo el mensaje en claro.
  \item Llaves semidébiles: se tiene un par de llaves $K_1, K_2$ tal que
    $E_{K_1}(E_{K_2}(x)) = x$.
\end{itemize}

\acrshort{gl:des} tiene cuatro llaves débiles y seis pares de llaves
semidébiles. Las cuatro llaves débiles generan subllaves $K_i$ iguales y,
debido a que \acrshort{gl:des} es un cifrado Feistel, el cifrado es
autorreversible. O sea que al final se obtiene de nuevo el texto en claro,
pues cifrar dos veces con la misma llave regresa la entrada original.
Respecto a los pares semidébiles, el cifrado con una de las llaves del
par es equivalente al descifrado  con la otra (o viceversa).

\subsection{Fast Data Encipherment Algorithm (FEAL)}

Es una familia de algoritmos que ha tenido una participación crítica
en el desarrollo y refinamiento de varias técnicas del criptoanálisis,
tales como el criptoanálisis lineal y diferencial. \acrshort{gl:feal}-N mapea
bloques de texto en claro de 64 bits a bloques de 64 bits de texto
cifrado mediante una llave secreta de 64 bits. Es un cifrado Feistel de
$n-$\glspl{gl:ronda} parecido a \acrshort{gl:des}, pero con una función $f$ más
simple.

\acrshort{gl:feal} fue diseñado para ser veloz y simple, especialmente para
microprocesadores de 8 bits: usa operaciones orientadas a bytes, evita
el uso de permutaciones de bit y tablas de consulta. La versión inicial
de cuatro \glspl{gl:ronda} (FEAL-4), propuesto como una alternativa rápida a
\acrshort{gl:des}, fue encontrado mucho más inseguro de lo planeado; por lo
que se propuso realizar más \glspl{gl:ronda} (FEAL-16 y FEAL-32) para compensar
y ofrecer un nivel de seguridad parecido a \acrshort{gl:des}; sin embargo, el
rendimiento se ve fuertemente afectado mientras el número de \glspl{gl:ronda}
aumenta; y, mientras \acrshort{gl:des} puede mejorar su velocidad con tablas de
consulta, resulta más complicado para \acrshort{gl:feal}.

\begin{pseudocodigo}[caption={FEAL-8, cifrado.}, label={feal8:1}]
  entrada:    64 bits de texto en claro $M = m_1 \dots m_{64}$;
              llave de 64 bits $K = k_1 \dots k_{64}$.
  salida:     bloque de texto cifrado de 64 bits $C = c_1 \dots c_{64}$.
  inicio
    Calcular 16 subllaves de 16 bits para $K$.
    Definir $M_L = m_1 \dots m_{32}; M_R = m_{33} \dots m_{64}$.
    $(L_0, R_0) \leftarrow (M_L, M_R) \oplus ((K_8, K_9), (K_{10}, K_{11}))$
    $R_0 \leftarrow R_0 \oplus L_0$.
    para_todo $i$ desde 1 hasta 8:
      $L_i \leftarrow R_{i-1}$
      $R_i \leftarrow L_{i-1} \oplus f(R_{i-1}, K_{i-1})$
    fin
    $L_8 \leftarrow L_8 \oplus R_8$
    $(R_8, L_8) \leftarrow (R_8, L_8) \oplus ((K_{12}, K_{13}),(K_{14},K_{15}))$
    $C \leftarrow (R_8, L_8)$.
  fin
\end{pseudocodigo}

% ¿Qué demonios? Las tuplas de llaves se salen del margen.

Para descifrar se utiliza el mismo algoritmo, con la misma llave $K$ y el
texto cifrado $C = (R_8, L_8)$ se utiliza como la entrada $M$; sin
embargo, la generación de llaves se hace al revés: las subllaves
$((K_{12}, K_{13}), (K_{14}, K_{15}))$ se utilizan para la $\oplus$ inicial,
las $((K_8, K_9), (K_{10}, K_{11}))$ para la $\oplus$ final y en las
\glspl{gl:ronda} se utiliza de la subllave $K_7$ a la $K_0$.

\acrshort{gl:feal} con una llave de 64 bits puede ser generalizado a $N-$
\glspl{gl:ronda} con $N$ par, aunque se recomienda $N = 2^x$.

%
% Explicación de IDEA, capítulo de antecedentes.
% Proyecto Lovelace.
%

\subsection{International Data Encryption Algorithm (IDEA)}

Cifra bloques de 64 bits utilizando una llave de 128 bits. Este cifrado
está basado en una generalización de la estructura Feistel y consiste en
8 \glspl{gl:ronda} idénticas seguidas por una transformación. Cada ronda $r$
utiliza 6 subllaves $K^{(r)}_i$ ($1 \leq i \leq 6$) de 16 bits que se
encargan de transformar una entrada $X$ de 64 bits en una salida de
cuatro bloques de 16-bits, que son utilizados como entrada en la
siguiente ronda. La salida de la ronda 8 tiene como entrada la
transformación de salida que, al emplear cuatro llaves adicionales
$K^{(9)}_i$ ($1 \leq i \leq 4$), produce los datos cifrados
$Y = (Y_1, Y_2, Y_3, Y_4)$.

%   Lo siento, pero si corto la línea de la entrada, la entrada queda a
%  la mitad y se ve muy raro.

\begin{pseudocodigo}[caption={IDEA, cifrado.}, label={idea:1}]
    entrada:   $64-$bits de datos en claro $M = m_1 \dots m_{64}$;
               llave de $128-$bits $ K = k_1 \dots k_{128}$.
    salida:    bloque cifrado de $64-$bits $Y = (Y_1, Y_2, Y_3, Y_4)$.
    inicio
      Calcular las subllaves $K^{(r)}_1, \dots, K^{(r)}_{6}$ para las rondas $1 \leq r \leq 8$ y $K^{(9)}_1, \dots, K^{(9)}_{4}$
      para la transformación de salida.
      $(X_1, X_2, X_3, X_4) \leftarrow (m_1 \dots m_{16}, m_{17} \dots m_{32}, m_{33} \dots m_{48}, m_{49} \dots m_{64})$
          donde $X_i$ almacena 16 bits.
      para_todo $r$ desde 1 hasta 8:
        a) $X_1 \leftarrow X_1 \times K_1^{(r)} \mod2^{16} + 1$
           $X_4 \leftarrow X_4 \times K_4^{(r)} \mod2^{16} + 1$
           $X_2 \leftarrow X_2 + K_2^{(r)} \mod2^{16}$
           $X_3 \leftarrow X_3 + K_3^{(r)} \mod2^{16}$
        b) $t_0 \leftarrow K_5^{(r)} \times (X_1 \oplus X_3) \mod2^{16} + 1$
           $t_1 \leftarrow K_6^{(r)} \times (t_0 + (X_2 \oplus X_4)) \mod2^{16} + 1$
           $t_2 \leftarrow t_0 + t_1 \mod2^{16}$
        c) $X_1 \leftarrow X_1 \oplus t_1$
           $X_4 \leftarrow X_4 \oplus t_2$
           $a \leftarrow X_2 \oplus t_2$
           $X_2 \leftarrow X_3 \oplus t_1$
           $X_3 \leftarrow a$
      fin
      Realizar la transformación de salida:
        $Y_1 \leftarrow X_1 \times K_1^{(9)} \mod2^{16} + 1$
        $Y_4 \leftarrow X_4 \times K_4^{(9)} \mod2^{16} + 1$
        $Y_2 \leftarrow X_3 + K_2^{(9)} \mod2^{16}$
        $Y_3 \leftarrow X_2 + K_3^{(9)} \mod2^{16}$
    fin
\end{pseudocodigo}

El descifrado se realiza con el mismo algoritmo de cifrado, pero
utilizando como entrada los datos cifrados $Y$ como entrada $M$. Se usa la
misma llave $K$; aunque las subllaves sufren una modificación al ser
generadas, pues se utiliza una tabla y se realizan las operaciones
contrarias (inverso de la adición y el inverso del producto).

Descartando los ataques a las llaves débiles, no hay un mejor ataque
publicado para el \gls{gl:idea} de 8 \glspl{gl:ronda} que el de la búsqueda
exhaustiva en el espacio de llave. Por lo que la seguridad está ligada a la
creciente debilidad de su tamaño de bloque relativamente pequeño.

\subsection{Secure And Fast Encryption Routine (SAFER)}

El cifrado \gls{gl:safer} K-64 es un cifrado por bloques de 64 bits
iterativo. Consiste en $r$ \glspl{gl:ronda} idénticas seguidas por una
transformación. Originalmente se recomendaban $6$ \glspl{gl:ronda} seguidas,
sin embargo, ahora se utiliza una generación de claves ligeramente modificada y
el uso de $8$ \glspl{gl:ronda} (máximo 10). Ambas generaciones de llaves
expanden la llave de 64 bits en $2r+1$ subllaves, cada una de 64 bits (dos por
cada ronda y una más para la transformación de salida).

Este cifrado consiste completamente en operaciones de bytes, por lo que
es adecuado para procesadores con tamaños de palabra pequeños, como los
chips de tarjetas.

%   Lo siento, pero si corto la línea de la entrada, la entrada queda a
%  la mitad y se ve muy raro.

\begin{pseudocodigo}[caption={SAFER K-64, cifrado.}, label={safer:1}]
  entrada: $r, 6\leq r \leq10$; $64-$bits de datos en claro $M = m_1 \dots m_{64}$; $ K = k_1 \dots k_{64}$.
  salida: bloque cifrado de $64-$bits $Y = (Y_1, \dots, Y_8)$.
  inicio
    Calcular las subllaves $K_1, \dots, K_{2r+1}$
    $(X_1, X_2, \dots X_8) \leftarrow (m_1 \dots m_8, m_9 \dots m_{16}, \dots, m_{57} \dots m_{64})$
    para_todo $i$ desde $1$ hasta $r$:
      a) Para $j = 1, 4, 5, 8: X_j \leftarrow X_j \oplus K_{2i-1}[j]$
        Para $j = 2, 3, 6, 7: X_j \leftarrow X_j + K_{2i-1}[j]$$mod$ $2^8$
      b) Para $j = 1, 4, 5, 8: X_j \leftarrow S$[$X_j$]
        Para $j = 2, 3, 6, 7: X_j \leftarrow S_{inversa}X_j$
      c) Para $j = 1, 4, 5, 8: X_j \leftarrow X_j + K_{2i}[j]$$mod$ $2^8$
        Para $j = 2, 3, 6, 7: X_j \leftarrow X_j \oplus K_{2i}[j]$
      d) Para $j = 1, 3, 5, 7: (X_j, X_{j+1}) \leftarrow f(X_j, X_{j+1})$.
      e) $(Y_1, Y_2 ) \leftarrow f(X_1, X_3), (Y_3, Y_4 ) \leftarrow f(X_5, X_7)$,
        $(Y_5, Y_6 ) \leftarrow f(X_2, X_4), (Y_7, Y_8 ) \leftarrow f(X_6, X_8 )$.
        Para $j$ desde 1 hasta 8: $X_j \leftarrow Y_j$
      f) $(Y_1, Y_2) \leftarrow f(X_1, X_3), (Y_3, Y_4) \leftarrow f(X_5, X_7)$,
        $(Y_5, Y_6 ) \leftarrow f(X_2, X_4), (Y_7, Y_8) \leftarrow f(X_6, X_8)$.
        Para $j$ desde 1 hasta 8: $X_j \leftarrow Y_j$.
    fin
    Para $j = 1, 4, 5, 8: Y_j \leftarrow X_j \oplus K_{2r+1}[j]$.
    Para $j = 2, 3, 6, 7: Y_j \leftarrow X_j + K_{2r+1} [j] \mod 2^8$.
  fin
\end{pseudocodigo}

Para descifrar, se utiliza la misma llave $K$ y las subllaves $K_i$
que fueron utilizadas al cifrar. Cada paso del cifrado se hace en orden
inverso, del último al primero; comenzando con una transformación de entrada
utilizando la llave $K_{2r+1}$ para deshacer la transformación de salida, se
sigue con las \glspl{gl:ronda} de descifrado utilizando las llaves de $K_{2r}$
a $K_1$, invirtiendo los pasos cada ronda.

\subsection{RC5}
Este cifrado por bloques tiene una arquitectura orientada a palabras (ya
sea $w = 16, 32, 64$bits) y tiene una descripción muy compacta adecuada
tanto para hardware como para software. Tanto la longitud $b$ de la
llave y el número de \glspl{gl:ronda} $r$ es variable; aunque se recomiendan 12
\glspl{gl:ronda} para 32 bits y 16 para cuando se tienen palabras de 64.

\begin{pseudocodigo}[caption={RC5, cifrado.}, label={rc5:1}]
  entrada:  $2w-$bits de datos en claro $M = (A, B)$; $r$;
      llave $K$ = $K$[0]$\dots K$[$b-1$]
  salida:   $2w-$bits de datos cifrados $C$.
  inicio
    Calcular $2r + 2$ subllaves $K_0, \dots, K_{2r+1}$
    $A \leftarrow A + K_0 \mod2^w, B \leftarrow B + K_1 \mod2^w$
    Para $i$ desde $1$ hasta $r$:
      $A \leftarrow ((A \oplus B) \hookleftarrow B) + K_{2i} \mod2^w$
      $B \leftarrow ((B \oplus A) \hookleftarrow A) + K_{2i+1} \mod2^w$
    fin
    Regresar $C \leftarrow (A,B)$
  fin
\end{pseudocodigo}

Para descifrar, RC5 utiliza el siguiente algoritmo.
\begin{pseudocodigo}[caption={RC5, descifrado.}, label={rc5:2}]
  entrada:  $2w-$bits de datos cifrados $C = (A, B)$; $r$;
      llave $K$ = $K$[0]$\dots K$[$b-1$]
  salida:   $2w-$bits de datos en claro $M$.
  inicio
    Calcular $2r + 2$ subllaves $K_0, \dots, K_{2r+1}$
    $A \leftarrow A + K_0 \mod2^w, B \leftarrow B + K_1 \mod2^w$
    Para $i$ desde $r$ hasta $1$:
      $B \leftarrow ((B - K_{2i+1} \mod2^w) \hookrightarrow A) \oplus A$
      $A \leftarrow ((A - K_{2i} \mod2^w) \hookrightarrow B) \oplus B$
    fin
    Regresar $M \leftarrow (A-K_0 \mod2^w, B-K_1 \mod2^w)$
  fin
\end{pseudocodigo}

