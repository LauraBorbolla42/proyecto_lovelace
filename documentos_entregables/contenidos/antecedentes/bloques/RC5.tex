\subsection{RC5}
Este cifrado por bloques tiene una arquitectura orientada a palabras (ya 
sea $w = 16, 32, 64$bits) y tiene una descripción muy compacta adecuada 
tanto para hardware como para software. Tanto ĺa longitud $b$ de la 
llave y el número de rondas $r$ es variable; aunque se recomiendan 12 
rondas para 32 bits y 16 para cuando se tienen palabras de 64.

\begin{pseudocodigo}[caption={RC5, cifrado.}, label={rc5:1}]
  entrada:	$2w-$bits de datos en claro $M = (A, B)$; $r$; 
  		llave $K$ = $K$[0]$\dots K$[$b-1$] 
  salida: 	$2w-$bits de datos cifrados $C$.
  inicio
    Calcular $2r + 2$ subllaves $K_0, \dots, K_{2r+1}$
    $A \leftarrow A + K_0 mod 2^w, B \leftarrow B + K_1 mod 2^w$
    Para $i$ desde $1$ hasta $r$:
    	$A \leftarrow ((A \oplus B) \hookleftarrow B) + K_{2i} mod 2^w$
    	$B \leftarrow ((B \oplus A) \hookleftarrow A) + K_{2i+1} mod 2^w$ 
    fin
    Regresar $C \leftarrow (A,B)$
  fin
\end{pseudocodigo}

Para descifrar, RC5 utiliza el siguiente algoritmo.
\begin{pseudocodigo}[caption={RC5, descifrado.}, label={rc5:2}]
  entrada:  $2w-$bits de datos cifrados $C = (A, B)$; $r$; 
      llave $K$ = $K$[0]$\dots K$[$b-1$] 
  salida:   $2w-$bits de datos en claro $M$.
  inicio
    Calcular $2r + 2$ subllaves $K_0, \dots, K_{2r+1}$
    $A \leftarrow A + K_0 mod 2^w, B \leftarrow B + K_1 mod 2^w$
    Para $i$ desde $r$ hasta $1$:
      $B \leftarrow ((B - K_{2i+1} mod 2^w) \hookrightarrow A) \oplus A$
      $A \leftarrow ((A - K_{2i} mod 2^w) \hookrightarrow B) \oplus B$
    fin
    Regresar $M \leftarrow (A-K_0 mod 2^w, B-K_1 mod 2^w)$
  fin
\end{pseudocodigo}
