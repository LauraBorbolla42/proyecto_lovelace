\subsection{Data Encryption Standard (DES)}

Este es, probablemente, el cifrado simétrico por bloques más conocido;
ya que en la década de los 70 estableció un precedente al ser el primer
algoritmo a nivel comercial que publicó abiertamente sus
especificaciones y detalles de implementación. Se encuentra definido
en el estándar americano \acrshort{gl:fips} 46-2.

\acrshort{gl:des} es un cifrado Feistel que procesa bloques de $n=64$ bits y
produce bloques cifrados de la misma longitud. Aunque la llave es de 64 bits,
8 son de paridad, por lo que el tamaño \textit{efectivo} de la llave es de
56 bits. Las $2^{56}$ llaves implementan, máximo, $2^{56}$ de las
$2^{64}!$ posibles \glspl{gl:biyeccion} en bloques de 64 bits.

Con la llave $K$ se generan 16 subllaves $K_i$ de 48 bits; una para cada
\gls{gl:ronda}. En cada \gls{gl:ronda} se utilizan 8 \textit{cajas-s}
(mapeos de sustitución de 6 a 4 bits). La entrada de 64 bits es dividida por la
mitad en $L_0$ y $R_0$. Cada \gls{gl:ronda} $i$ va tomando las entradas
$L_{i-1}$ y $R_{i-1}$ de la \gls{gl:ronda} anterior y produce salidas de 32
bits $L_i$ y $R_i$ mientras $1 \leq i \leq 16$ de la siguiente manera:

\begin{equation}
  \label{cifrado_des}
  \begin{aligned}
    L_i = {} & R_{i-1} \\
    R_i = {} & L_{i-1} \oplus f(R_{i-1}, K_i) \\
    & donde \quad f(R_{i-1}, K_i) = P(S(E(R_{i-1})\oplus K_i))
  \end{aligned}
\end{equation}

$E$ se encarga de expandir $R_{i-1}$ de 32 bits a 48, $P$ es una
permutación de 32 bits y $S$ son las cajas-s.

\begin{pseudocodigo}[caption={DES, cifrado.}, label={des:1}]
  entrada:  64 bits de texto en claro $M = m_1 \dots m_{64}$;
            llave de 64 bits $K = k_1 \dots k_{64}$.
  salida:   bloque de texto cifrado de 64 bits $C = c_1 \dots c_{64}$.
  inicio
    Calcular 16 subllaves $K_i$ de 48 bits partiendo de $K$.
    Obtener $(L_0, R_0)$ de la tabla de permutaciones iniciales $IP(m_1m_2\dots m_{64})$
    para_todo $i$ desde 1 hasta 16:
      L_i = R_{i-1}
      Obtener $f(R_{i-1}, K_i)$:
        a) Expandir $R_{i-1} = r_1r_2\dots r_{32}$ de 32 a 48 bits
          usando $E$: $T \leftarrow E(R_{i-1})$.
        b) $T^\prime \leftarrow T \oplus K_i$. Donde $T^\prime$ es representado
          como ocho cadenas de 6 bits cada una $(B_1, \dots, B_8)$.
        c) $T'' \leftarrow (S_1(B_1), S_2(B_2), \dots S_8(B_8))$
        d) $T''' \leftarrow P(T'')$
      R_i = L_{i-1} \oplus f(R_{i-1}, K_i)
    fin
    $b_1b_2 \dots b_{64} \leftarrow (R_{16}, L_{16})$.
    $C \leftarrow IP^{-1}(b_1b_2 \dots b_{64})$
  fin
\end{pseudocodigo}

El descifrado \acrshort{gl:des} consiste en el mismo algoritmo de cifrado,
con la misma llave $K$, pero utilizando las subllaves en orden inverso:
$K_{16}, K_{15}, \dots, K_1$.

\subsubsection{Llaves débiles}
Tomando en cuenta las siguientes definiciones

\begin{itemize}
  \item Llave débil: una llave $K$ tal que $E_K(E_K(M)) = M$ para toda
    $x$; en otras palabras, una llave débil permite que, al cifrar dos
    veces con la misma llave, se obtenga de nuevo el mensaje en claro.
  \item Llaves semidébiles: se tiene un par de llaves $K_1, K_2$ tal que
    $E_{K_1}(E_{K_2}(x)) = x$.
\end{itemize}

\acrshort{gl:des} tiene cuatro llaves débiles y seis pares de llaves
semidébiles. Las cuatro llaves débiles generan subllaves $K_i$ iguales y,
debido a que \acrshort{gl:des} es un cifrado Feistel, el cifrado es
autorreversible. O sea que al final se obtiene de nuevo el texto en claro,
pues cifrar dos veces con la misma llave regresa la entrada original.
Respecto a los pares semidébiles, el cifrado con una de las llaves del
par es equivalente al descifrado  con la otra (o viceversa).
