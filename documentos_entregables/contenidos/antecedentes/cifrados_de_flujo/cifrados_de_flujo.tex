%
% Cifrados de flujo.
% Proyecto Lovelace.
%

\section{Cifrados de flujo}

La información de esta sección (junto con las subsecciones contenidas) puede
ser encontrada con mayor detalle en \cite{menezes} y \cite{stallings}.

A diferencia de los cifrados de bloque, que trabajan sobre grupos enteros de
bits a la vez, los cifrados de flujo trabajan sobre bits individuales,
cifrándolos uno por uno. Una manera de verlos es como cifrados por bloques con
un tamaño de bloque igual a 1.

Un cifrado de flujo aplica transformaciones de acuerdo a un flujo de llave:
una secuencia de símbolos pertenecientes al espacio de llaves. El flujo de
llave puede ser generado tanto de manera aleatoria, como por un algoritmo
pseudoaleatorio que reciba a la entrada, o bien una semilla, o bien una
semilla y algunos bits del texto cifrado.

Entre las ventajas de los cifrados de flujo sobre los cifrados de bloque se
encuentra el hecho de que son más rápidos en hardware y más útiles cuando
el buffer es limitado o se necesita procesar la información al momento de
llegada. La propagación de los errores es limitada o nula, por lo que también
son más apropiados en casos en los que hay probabilidades altas de errores en
la transmisión.

Los cifrados de bloques funcionan sin ninguna clase de memoria (por sí solos);
en contraste, la función de cifrado de un cifrado de flujo puede variar
mientras se procesa el texto en claro, por lo cuál tienen un mecanismo de
memoria asociado. Otra denominación para estos cifrados es \textit{de estado},
por que la salida no depende solamente del texto en claro y de la llave, sino
que también depende del estado actual.

%
% Clasificación de algoritmos tokenizadores,
% presentación en RCI.
% Proyecto Lovelace.
%

\section{Clasificación del PCI}

\begin{frame}{Clasificación de los algoritmos tokenizadores}
  {Clasificación del PCI \cite{pci_tokens}}
  \begin{itemize}
    \item \textbf{Reversibles:} se puede regresar, a partir del token, al
      número de tarjeta original.
      \begin{itemize}
        \item \textbf{Criptográficos:} cifran la tarjeta y descifran el
          token.
        \item \textbf{No criptográficos:} utilizan una base de datos para
          guardar la relación entre números de tarjeta y tokens.
      \end{itemize}
    \item \textbf{Irreversibles:} no se puede regresar al número de tarjeta a
      partir del token.
      \begin{itemize}
        \item \textbf{Autenticables:} permiten validar cuando un token
          corresponde a un número de tarjeta dado.
        \item \textbf{No autenticables:} no se puede hacer la validación
          anterior.
      \end{itemize}
  \end{itemize}
\end{frame}

\begin{frame}{Clasificación de los algoritmos tokenizadores}
  {Clasificación propuesta}
  \begin{itemize}
    \item \textbf{Criptográficos:} ocupan primitivas criptográficas en su
      operación.
    \begin{itemize}
      \item \textbf{Reversibles:} cifran la tarjeta y descifran el
        token.
      \item \textbf{Irreversibles:} requieren una base de datos para
        guardar la relación entre números de tarjetas y tokens.
    \end{itemize}
    \item \textbf{No criptográficos:} no utilizan nada relacionado con la
        criptografía.
  \end{itemize}
\end{frame}

%
% Cifrado de flujo RC4, capítulo de antecedentes.
% Proyecto Lovelace.
%

\subsection{RC4}

RC4 es un cifrado de flujo diseñado por Ron L. Rivest en 1987 para la empresa
\gls{gl:rsa}. Es usado en varios protocolos de seguridad comunes:
\gls{gl:ssl}/\gls{gl:tls}, \gls{gl:wep} y \gls{gl:wpa}; los
dos últimos son parte del estándar \gls{gl:ieee} 802.11 para
comunicaciones \gls{gl:lan} inalámbricas. RC4 se mantenía como secreto de
compañía hasta que, en septiembre de 1994, fue filtrado de forma anónima en
Internet.

En el pseudocódigo~\ref{cod:rc4} se describe el proceso de cifrado del
algoritmo. $ S $ es un vector de estado; $ T $ es un vector temporal.

\begin{pseudocodigo}[%
    caption={Proceso de cifrado de RC4.},
    label={cod:rc4}%
  ]
    entrada: llave k; mensaje original $ m_1, m_2 \dots m_n $.
    salida:  mensaje cifrado $ c_1, c_2 \dots c_n $.
    inicio
      /* Inicialización */
      para_todo i entre 0 y 255:
        S[i] $\gets$ i
        T[i] $\gets$ K[i $\mod$ longitud_de_llave]
      fin

      /* Permutación inicial */
      j $\gets$ 0
      para_todo i entre 0 y 255:
        j = (j + S[i] + T[i]) $\mod$ 256
        intercambiar(S[i], S[j])
      fin

      /* Proceso de cifrado */
      i, j $\gets$ 0
      para_todo m:
        i $\gets$ (i + 1) $\mod$ 256
        j $\gets$ (j + S[i]) $\mod$ 256
        intercambiar(S[i], S[j])
        k $\gets$ S[(S[i] + S[j]) $\mod$ 256]
        c $\gets$ m $\oplus$ k
      fin
      regresar c
    fin
\end{pseudocodigo}

Se han hecho varias publicaciones que analizan métodos para atacar RC4, ninguna
de las cuales presenta algo práctico cuando se utiliza una llave mayor a
128 bits. Sin embargo, en~\cite{ataque_wep} se reporta un problema más serio
sobre la implementación que se hace de RC4 en el protocolo \gls{gl:wep};
este problema en particular no ha demostrado afectar a otras aplicaciones que
usan RC4.

