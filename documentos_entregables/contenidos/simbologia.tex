%
% Definición de simbología.
% Proyecto Lovelace.
%

\addcontentsline{toc}{chapter}{Simbología}
\section*{Simbología}

A continuación se describe la simbología que se utilizará a lo largo de
este documento.

\begin{table}[H]
  \caption{Simbología}\label{tab:simb}
  \begin{center}
    \begin{tabular}{c|l}
      Símbolo & Descripción \\
      \hline
      $K$ & Llave \\
      \hline
      $pk$ & Llave pública \\
      \hline
      $sk$ & Llave privada o secreta \\
      \hline
      $k_i$ & $I\acute{e}$-sima subllave \\
      \hline
      $K_I$ & Llave de entrada o \textit{key derivation key} de una función 
      de derivación de llaves \\
      \hline
      $K_O$ & Material de llaves obtenido de una función de derivación de llaves \\
      \hline
      $IV$ & Vector de inicialización \\
      \hline
      $E$ & Operación de cifrado \\
      \hline
      $E_K$ & Operación de cifrado utilizando la llave $K$ \\
      \hline
      $D$ & Operación de descifrado \\
      \hline
      $D_K$ & Operación de descifrado utilizando la llave $K$ \\
      \hline
      $h$ & Función hash \\
      \hline
      $h_k$ & Función hash que utiliza una llave $k$ para calcular el valor\\
      \hline
      $M$ & Mensaje en claro\\
      \hline
      $C$ & Mensaje cifrado\\
      \hline
      $\mathbb{Z}_n$ & Conjunto de los números enteros módulo n\\
      \hline
      $\{0,1\}^r$ & Cadena de bits de longitud $r$ \\
      \hline
      $\{0,1\}^*$ & Cadena de bits de longitud arbitraria \\
      \hline
      $\mod$ & Operación módulo \\
      \hline
      $mcd$ & Máximo común divisor \\
      \hline
      $\oplus$ & Operación $XOR$ \\
      \hline
      $\parallel$ & Operación de concatenación \\
      \hline
      $\{X\}$ & Indicación de que el uso de $X$ es opcional \\
      \hline
      $[X]$ & El entero más pequeño que es mayor o igual que $X$ \\
      \hline
      $[X]_2$ & Representación binaria del número $X$ \\
      \hline
      $\varphi$ & Función $\phi$ de Euler \\
      \hline
      $\emptyset$ & Conjunto vacío \\
      \hline
    \end{tabular}
  \end{center}
\end{table}

Es menester aclarar que un mensaje no consiste solo en letras y números;
el \textit{mensaje} se refiere al conjunto de datos que van a ser
cifrados o descifrados.
