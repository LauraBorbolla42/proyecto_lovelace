%
% Definición de entradas del glosario.
% Proyecto Lovelace.
%
% Más información en https://www.sharelatex.com/learn/Glossaries
%

\makeglossaries

%% Glosario de términos %%%%%%%%%%%%%%%%%%%%%%%%%%%%%%%%%%%%%%%%%%%%%%%%%%%%%%%

\newglossaryentry{gl:modo_de_operacion}
{
  name = modo de operación,
  plural = modos de operación,
  description = {
    Construcción que permite extender la funcionalidad de un cifrado a bloques
    para operar sobre tamaños de información arbitrarios%
  }
}

\newglossaryentry{gl:vector_de_inicializacion}
{
  name = vector de inicialización,
  plural = vectores de inicialización,
  description = {
    Cadena de bits de tamaño fijo que sirve como entrada a muchas primitivas
    criptográficas (e. g. algunos modos de operación). Generalmente se requiere
    que sea generado de forma aleatoria%
  }
}

\newglossaryentry{gl:ronda}
{
  name = ronda,
  plural = rondas,
  description = {
    Bloque compuesto por un conjunto de operaciones que es ejecutado
    múltiples veces. Las rondas son definidas por el algoritmo de cifrado.
  }
}

%% Siglas y acrónimos %%%%%%%%%%%%%%%%%%%%%%%%%%%%%%%%%%%%%%%%%%%%%%%%%%%%%%%%

\newacronym{gl:ecb}{ECB}{\textit{Electronic Codebook}}

\newacronym{gl:cbc}{CBC}{\textit{Cipher-block Chaining}}

\newacronym{gl:cfb}{CFB}{\textit{Cipher Feedback}}

\newacronym{gl:ofb}{OFB}{\textit{Output Feedback}}

\newacronym{gl:des}{DES}{\textit{Data Encryption Standard}}

\newacronym{gl:aes}{AES}{\textit{Advanced Encryption Standard}}

\newacronym{gl:feal}{FEAL}{\textit{Fast Data Encipherment Algorithm}}

\newacronym{gl:idea}{IDEA}{\textit{International Data Encryption Algorithm}}

\newacronym{gl:safer}{SAFER}{\textit{Secure And Fast Encryption Routine}}

\newacronym{gl:md4}{MD4}{\textit{Message Digest 4}}

\newacronym{gl:md5}{MD5}{\textit{Message Digest 5}}

\newacronym{gl:sha}{SHA}{\textit{Secure Hash Algorithm}}

\newacronym{gl:nist}{NIST}{
    \textit{National Institute of Standards and Technology}}

\newacronym{gl:nsa}{NSA}{\textit{National Security Agency}}
