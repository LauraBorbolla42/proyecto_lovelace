%
% Definición de entradas del glosario.
% Proyecto Lovelace.
%
% Más información en https://www.sharelatex.com/learn/Glossaries
%

\makeglossaries

\newglossaryentry{gl:modo_de_operacion}
{
  name = modo de operación,
  plural = modos de operación,
  description = {
    Construcción que permite extender la funcionalidad de un cifrado a bloques
    para operar sobre tamaños de información arbitrarios%
  }
}

\newglossaryentry{gl:vector_de_inicializacion}
{
  name = vector de inicialización,
  plural = vectores de inicialización,
  description = {
    Cadena de bits de tamaño fijo que sirve como entrada a muchas primitivas
    criptográficas (e. g. algunos modos de operación). Generalmente se requiere
    que sea generado de forma aleatoria.
  }
}

\newglossaryentry{gl:ronda}
{
  name = ronda,
  description = {
    Bloque compuesto por un conjunto de operaciones que es ejecutado
    múltiples veces. Las rondas son definidas por el algoritmo de cifrado%
  }
}

\newglossaryentry{gl:integridad_datos}
{
  name = integridad de datos,
  description = {
    Propiedad en la que los datos no han sido alterados sin autorización desde
    que fueron creados, transmitidos o almacenados por una fuente autorizada.
    Operaciones que insertan, eliminan, modifican o reordenan bits invalidan
    la integridad de los datos. La integridad de los datos incluye que los
    datos estén completos y, cuando los datos son divididos en bloques, cada
    bloque cumplir con lo mencionado anteriormente.
  }
}

\newglossaryentry{gl:autenticacion_origen}
{
  name = autenticación de origen,
  description = {
    Tipo de autenticación donde se corrobora que una entidad es la fuente
    original de la creación de un conjunto de datos en un tiempo específico.
    Por definición, la autenticación de origen incluye la integridad de datos,
    pues cuando se modifican los datos, se tiene una nueva fuente.
  }
}

\newglossaryentry{gl:biyeccion}
{
  name = biyección,
  plural = biyecciones,
  description = {
    Dicho de las funciones que son inyectivas y sobreyectivas
    al mismo tiempo; en otras palabras, que todos los elementos
    del conjunto de salida tengan una imagen distinta en el conjunto
    de llegada y a cada elemento del conjunto de llegada le corresponde
    un elemento del conjunto de salida.
  }
}

\newglossaryentry{gl:cifrado_iterativo}
{
  name = cifrado iterativo,
  plural = cifrados iterativos,
  description = {
    Cifrado de bloque que involucra la repetición secuencial de
    una función interna llamada función de \gls{gl:ronda}. Los
    parámetros incluyen en número de rondas, el tamaño de bloque y
    el tamaño de llave.
  }
}

\newglossaryentry{gl:entropia}
{
  name = entropía,
  description = {
    Definida para una función de probabilidad de distribución discreta, mide cuánta información en promedio es requerida para identificar muestras aleatorias de esa distribución.
  }
}
