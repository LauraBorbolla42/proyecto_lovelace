%
% Presentación para congreso.
% Proyecto Lovelace.
%

\documentclass{beamer}
\usepackage{array}
\usepackage{formato}
\newcommand{\Titulo}{\textbf{Un vistazo a la tokenización}}
\newcommand{\Fecha}{Puebla, 14 de octubre de 2018}
\usepackage{formato_presentaciones}

\renewcommand{\Autor}{Daniel Ayala Zamorano \\\vspace{-2mm}
  {\tiny \texttt{daz23ayala@gmail.com}} \\
  Laura Natalia Borbolla Palacios \\\vspace{-2mm}
  {\tiny \texttt{ln.borbolla.42@gmail.com}} \\
  Ricardo Quezada Figueroa \\\vspace{-2mm}
  {\tiny \texttt{qf7.ricardo@gmail.com}} \\
  Sandra Díaz Santiago \\\vspace{-2mm}
  {\tiny \texttt{sdiazs@gmail.com}}}

\usepackage[caption=false,font=footnotesize]{subfig}
\usepackage{bold-extra}
\usepackage{tikz}
\usetikzlibrary{
  arrows.meta,
  decorations.pathmorphing,
  backgrounds,
  positioning,
  calc,
  scopes,
  shapes}

\tikzset{
  font=\tiny}

\begin{document}

  {\setbeamertemplate{footline}{}
  \frame{\titlepage}}

  \begin{frame}
    \frametitle{Contenido}
    \setcounter{tocdepth}{1}
    \tableofcontents
  \end{frame}

  % Espacio entre párrafos
  \setlength{\parskip}{0.5em}

  \import{presentacion_rci/contenidos/}{problema}
  \import{presentacion_rci/contenidos/}{tokenizacion}
  \import{presentacion_rci/contenidos/}{clasificacion}
  \import{presentacion_rci/contenidos/}{reversibles}
  \import{presentacion_rci/contenidos/}{irreversibles}
  \import{presentacion_rci/contenidos/}{conclusiones}

  \begin{frame}[allowframebreaks]{Bibliografía}
    \printbibliography
  \end{frame}

  \begin{frame}{}
    \centering \Huge
    \textsc{Gracias por su atención.}
  \end{frame}

  % Espacio entre párrafos
  \setlength{\parskip}{0.0em}

  {\setbeamertemplate{footline}{}
  \frame{\titlepage}}

\end{document}
