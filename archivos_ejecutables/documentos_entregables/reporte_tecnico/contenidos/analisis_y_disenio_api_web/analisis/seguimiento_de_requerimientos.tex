%
% Seguimiento de requerimientos de tokens,
% Capítulo de análisis y diseño API web
% Proyecto Lovelace.
%

\subsection{Seguimiento de requerimientos de 
  \texorpdfstring{\glspl{gl:token}}{tokens}}
\label{sec:seguimiento_api}

A continuación se encuentran los requerimientos que caen en la clasificación
del \hipervinculo{clasificacion:sistema} mencionados en la sección
\ref{sec:requerimientos}. Se indica, para cada requerimiento, si el
requerimiento fue satisfecho o no y cómo o por qué no.

\begin{itemize}
  \item \textbf{\hipervinculo{rq_pci:productos_de_hardware}.}
    Como no se utilizan productos de hardware en el sistema tokenizador, este
    requerimiento se satisface por vacuidad.

  \item \textbf{\hipervinculo{rq_pci:productos_de_software}.}
    Ya que este requerimiento necesita de una auditoria por parte del
    \gls{gl:fips}, este requerimiento no es satisfecho.

  \item \textbf{\hipervinculo{rq_pci:deteccion_anomalias}.}
    El contador de malas acciones es el encargado de encontrar
    \textit{anomalías} en el uso del sistema tokenizador; es aumentado cuando
    se reciben peticiones con \glspl{gl:token} o \glspl{gl:pan} inválidos, o
    cuando se intenta detokenizar un \gls{gl:token} que no está asociado
    al cliente que mandó la petición (véase \hipervinculo{rn:malas_acciones}).
    Por lo tanto, este requerimiento es satisfecho.

  \item \textbf{\hipervinculo{rq_pci:guia_de_instalacion}.}
    La API tiene documentación en su sitio en línea que permite saber cómo
    comunicarse con ella y cuáles son los posibles errores; además, el código
    es libre y público (véase \hipervinculo{rq:mostrar_documentacion}).
    Por lo tanto, este requerimiento es satisfecho.

  \item \textbf{\hipervinculo{rq_pci:acceso_a_producto}.}
    Este requerimiento contiene varios subrequerimientos que se listan
    a continuación:
    \begin{itemize}
      \item \textbf{\hipervinculo{rq_pci:control_de_peticiones}.}
        Ya que la única manera de acceder al sistema tokenizador es mediante
        peticiones y en cada una se debe mandar un token de autenticación,
        este requerimiento es satisfecho (véase
        \hipervinculo{rq:tokenizar_tarjeta},
        \hipervinculo{rq:detokenizar_token},
        \hipervinculo{rq:retokenizar_token}).

      \item \textbf{\hipervinculo{rq_pci:registros_de_acceso}.}
        Ya que el sistema no matiene un registro de los eventos y las
        operaciones realizadas, este requerimiento no es satisfecho.

      \item \textbf{\hipervinculo{rq_pci:autenticacion_multifactor}.}
        Debido al alcance (y presupuesto) del proyecto, no se cuenta con
        autenticación multifactor; por lo tanto, este requerimiento no es
        satisfecho.

      \item \textbf{\hipervinculo{rq_pci:acceso_de_sistema}.}
        Los clientes se identifican en cada petición ante el sistema mediante
        sus credenciales (usuario y contraseña); el sistema se identifica
        ante el cliente mediante \acrshort{gl:tls}/\acrshort{gl:ssl}. Por lo
        tanto, este requerimiento es satisfecho.

      \item \textbf{\hipervinculo{rq_pci:acceso_administrativo}.}
        Ya que no tenemos accesos administrativos desde fuera de la consola,
        este requerimiento es satisfecho por vacuidad. El usuario tipo
        administrador no tiene permisos para utilizar el programa tokenizador.

    \end{itemize}
    Como se puede observar, este requerimiento es satisfecho parcialmente, pues
    hay subrequerimientos con los que no se cumple.

  \item \textbf{\hipervinculo{rq_pci:manejo_de_llaves}.}
    Este requerimiento contiene varios subrequerimientos que se listan a
    continuación:
    \begin{itemize}
      \item \textbf{\hipervinculo{rq_pci:ciclo_de_vida_llaves}.}
        Los criptoperiodos por defecto en el sistema son de 6 meses, lo que
        es un tiempo relativamente corto; sin embargo, el sistema no inicia
        por sí mismo un refresco de llaves, sino que es el cliente quien debe
        realizar el refresco (véase \hipervinculo{rn:criptoperiodo_llave}).
        Por lo tanto, como no es responsabilidad del sistema, este
        requerimiento es satisfecho.

      \item \textbf{\hipervinculo{rq_pci:periodo_llaves}.}
        Dependiendo de a quién se le brinde la atención, el servicio de
        tokenización puede ser muy variado, dependiendo de las necesidades de
        cada cliente. Ya que no se describe el periodo criptográfico, este
        requerimiento no es satisfecho.

      \item \textbf{\hipervinculo{rq_pci:destruccion_de_llaves}.}
        El cliente puede, en cualquier momento hacer un refresco de llaves
        (destruir las nuevas y crear unas nuevas) sin tener que alterar el
        sistema; por lo tanto, el requerimiento es satisfecho.

      \item \textbf{\hipervinculo{rq_pci:exportar_llaves_en_claro}.}
        El sistema no permite exportar llaves, así que este requerimiento
        es satisfecho.

      \item \textbf{\hipervinculo{rq_pci:entropia_generacion_llaves}.}
        Las llaves son generadas mediante un \acrshort{gl:drbg} que funciona
        con una función hash que tiene 256 bits de entropía; por lo tanto, este
        requerimiento es satisfecho.

      \item \textbf{\hipervinculo{rq_pci:llaves_de_uso_unico}.}
        Cada cliente tiene un juego de llaves; una para cada algoritmo; además,
        todas las llaves son únicas en la base de datos y no se utilizan para
        nada más. Por lo tanto este requerimiento es satisfecho.

    \end{itemize}

  \item \textbf{\hipervinculo{rq_pci:cr_almacenamiento_tokens}.}
    Para los algoritmos reversibles implementados (\acrshort{gl:ffx} y
    \acrshort{gl:bps}), el sistema no guarda los \glspl{gl:token} generados;
    por lo tanto, este requerimiento es satisfecho.

  \item \textbf{\hipervinculo{rq_pci:cr_administracion_llaves}.}
    Dado el alcance del proyecto y que este requerimiento involucra auditorías
    o validaciones de organizaciones externas, no es satisfecho.

  \item \textbf{\hipervinculo{rq_pci:nc_independencia_estadistica}.}
    Ya que el sistema tiene una sola instancia de la base de datos, este
    requerimiento es satisfecho.

  \item \textbf{\hipervinculo{rq_pci:nc_administracion_llaves}.}
    Dado el alcance del proyecto y que este requerimiento involucra auditorías
    o validaciones de organizaciones externas, no es satisfecho.


\end{itemize}
