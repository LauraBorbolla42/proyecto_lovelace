%
% Caso de uso para rechazar a un cliente.
% Capítulo de análisis y diseño de api web,
% Proyecto Lovelace.
%

\casoDeUso[cu:rechazar_cliente]
{Rechazar cliente}
{
  Permite a un usuario tipo \textbf{administrador} rechazar a un usuario tipo
  \textbf{cliente}, haciéndolo pasar del estado \textbf{en espera} a
  \textbf{rechazado} (véase \hipervinculo{rn:estados_cliente}).

  \begin{trayectoriaPrincipal}

    \item El administrador presiona el botón \textit{Rechazar} del cliente que
      desea rechazar en la interfaz \hipervinculo{iu:administracion}.

    \item El sistema muestra el mensaje \hipervinculo{msj:rechazar_cliente}.

    \item El administrador presiona \textit{Aceptar};
      [\hipervinculoLocal{ta:cancelar}].

    \item El sistema cambia el estado de dicho cliente a \textbf{rechazado}
      tomando en cuenta la regla de negocios \hipervinculo{rn:estados_cliente}.

    \item El sistema manda un correo al usuario notificándole de su
      rechazo en el sistema.

    \item El sistema refresca la interfaz \hipervinculo{iu:administracion}.

  \end{trayectoriaPrincipal}

  %%%%%%%%%%%%%%%%%%%%%%%%%%%%%%%%%%%%%%%%%%%%%%%%%%%%%%%%%%%%%%%%%%%%%%%%%%%%%%

  \begin{trayectoriaAlternativa}[ta:cancelar]
    {Cancelar operación}

    \item El administrador presiona \textit{Cancelar}.

  \end{trayectoriaAlternativa}
}
