%
% Apéndice sobre el documento.
% Reporte técnico.
% Proyecto Lovelace.
%

\capitulo[TEX]{Sobre este documento}{sec:sobre_este_documento}{%
  \epigrafe{%
    The trouble with computer science today is an obsessive concern with
    form instead of content.}{%
    Form and Content in Computer Science, 1970 \acrshort{gl:acm} Turing Lecture,
    \textsc{Marvin Minsky}.}}%
%
En este apéndice se discuten algunos de los aspectos más relevantes de la
elaboración de este documento. Primero, se presentan algunas de las dependencias
más importantes de \LaTeX; después, se presenta una lista de los comandos hechos
específicamente para este reporte y se muestran, a modo de ejemplo, algunos de
ellos.

Primero, ¿por qué \LaTeX? Una de las principales ventajas con las que se
promociona a \LaTeX~es la separación de responsabilidades: el autor de un
documento solamente debe de procuparse por el contenido, mientras que el
programa, \LaTeX, es el responsable de la forma, esto es, de la apariencia
final del documento~\cite{introduccion_latex}. En el caso de este trabajo, lo
anterior no es enteramente cierto, pues hay un alto grado de modificaciones
sobre los comportamientos por defecto (de lo contrario no existiría este
apéndice). Lo que sí es cierto es que el uso de \LaTeX~implica una separación
base entre la forma y el contenido de una trabajo y, siendo cuidadosos, esta
separación se puede mantener aún cuando se modifique el comportamiento por
defecto.

La segunda razón de peso tiene que ver con las circunstancias del producto
requerido: se trata de un documento muy extenso que requiere modificaciones
simultáneas de todo el equipo; en particular, se necesita que las
modificaciones al documento estén adentro del seguimiento del control de
versiones (dependencia \hipervinculo{dep:git}). A diferencia de un procesador
de textos \gls{gl:wysiwyg}, \LaTeX~trabaja con archivos de texto plano, lo cual
permite mantener el control de versiones tan claro como se quiera. El punto
anterior también permite importar buenas prácticas de programación en la
elaboración del documento: la modularización del código en archivos y carpetas
significativos (este documento está distribuido en alrededor de~500 archivos,
la estructura de los archivos emula a la estructura del documento); líneas
cortas (80~caracteres en este caso) para facilitar la lectura directo en
archivos fuente y para facilitar la comparación de versiones.

% Qué dramático:
La inclusión de este apéndice es una forma de reconocimiento hacia el tiempo y
esfuerzo dedicados para que este documento llegara a ser lo que es.

\section{Dependencias}

El documento depende actualmente de 138 paquetes de \LaTeX; aquí se enlistan
aquellos cuyo uso responde a un proceso de decisión, y no a una simple
necesidad.

\dependencia[dep:tikz]{Elaboración de diagramas}{Ti\textit{k}Z}{%
  \acrshort{gl:lppl} v1.3}{https://www.ctan.org/pkg/pgf}{%
  Paquete para crear gráficos. La interfaz al usuario es similar a la de
  \LaTeX: define un lenguaje para la descripción del gráfico; el producto
  final se obtiene mediante un proceso de compilado.

  Existen varias ventajas al utilizar Ti\textit{k}Z. Las dos principales son
  las mismas que las del propio \LaTeX: al trabajar con texto plano, los
  diagramas se encuentran dentro del seguimiento del control de versiones; con
  un poco de cuidado, se puede mantener separada a la forma del contenido, lo
  cuál permite crear diagramas con exactamente el mismo aspecto. Otra ventaja
  es la integración entre los diagramas y el propio documento: Ti\textit{k}Z
  produce comandos en \gls{gl:pgf} que se encuentran incrustados directamente
  en el \gls{gl:pdf} producto, lo cual implica que el texto de los diagramas es
  seleccionable (en algunos diagramas incluso hay hipervínculos, por ejemplo,
  el diagrama de estados~\ref{estados_actores}); la tipografía es
  la misma para diagramas y documento. La única desventaja está dada por el
  tiempo de aprendizaje de la sintaxis, que en comparación con un editor
  gráfico, es algo mucho más lento. A modo de ejemplo, el código
  fuente~\ref{codigo:diagrama_tikz} muestra un fragmento del archivo fuente del
  diagrama~\ref{diagrama:aes_add_round_key}}.

\codigoFuente[codigo:diagrama_tikz]{1}{38}{latex}{%
  documentos_entregables/reporte_tecnico/contenidos/marco_teorico/%
  bloques/diagramas/addRoundKey.tikz.tex}{%
  Ejemplo de diagrama en Ti\textit{k}Z;
  diagrama~\ref{diagrama:aes_add_round_key}}

\dependencia{Diagramas \acrshort{gl:uml}}{Ti\textit{k}Z-\acrshort{gl:uml}}{%
  \gls{gl:lppl}}{http://perso.ensta-paristech.fr/~kielbasi/tikzuml/index.php}{%
  Conjunto de comandos alrededor de \hipervinculo{dep:tikz} para la creación de
  diagramas en \gls{gl:uml}.

  Define una sintaxis que permite crear diagramas de Ti\textit{k}Z pensando
  solamente en el contenido, sin tener que preocuparse por el aspecto. El
  código fuente~\ref{codigo:muestra_tikz_uml} muestra un fragmento del
  diagrama de secuencia~\ref{fig:secuencia_registrar_cliente}.}

\codigoFuente[codigo:muestra_tikz_uml]{1}{31}{latex}{%
  documentos_entregables/reporte_tecnico/contenidos/analisis_y_disenio_api_web%
  /disenio/diagramas/secuencia_registrar_cliente.tikz.tex}{%
  Ejemplo de diagrama en Ti\textit{k}Z-\acrshort{gl:uml};
  diagrama~\ref{fig:secuencia_registrar_cliente}.}

\dependencia{Para distribución de código fuente}{Import}{%
  \acrshort{gl:pds}}{https://www.ctan.org/pkg/import}{%
  Paquete para importar archivos fuente de forma relativa al archivo fuente
  actual.

  Si se trabaja con documentos pequeños, la importancia de este paquete no es
  muy clara; sin embargo, al trabajar con un documento distribuido en más
  de~500 archivos, los importados relativos son indispensables. Además de que,
  si se trabajara con el comportamiento por defecto (importados absolutos al
  archivo origen) las rutas internas de los archivos más profundos serían
  ridículamente largas e inmanejables, el importado relativo también permite
  mover carpetas enteras sin tener que alterar el código interno.
  Por ejemplo, para cambiar una sección de capítulo basta con mover la carpeta
  correspondiente de lugar y cambiar la línea de llamada. En el código
  fuente~\ref{codigo:muestra_imports} se muestra un fragmento del archivo
  principal del documento en donde se hace uso de este paquete.}

\codigoFuente[codigo:muestra_imports]{27}{41}{latex}{%
  documentos_entregables/reporte_tecnico/reporte_tecnico.tex}{%
  Muestra de importados relativos.}

\dependencia{Para administración de fuentes bibliográficas}{Bib\LaTeX~y Biber}{%
  \acrshort{gl:lppl} y \acrshort{gl:pal}}{https://www.ctan.org/pkg/biber}{%
  Bib\LaTeX~es el sucesor de Bib\TeX, el gestor de fuentes bibliográficas
  por defecto de \LaTeX. Bib\LaTeX~ es la interfaz para Biber: un programa
  en Perl que se encarga de la construcción de la bibliografía.

  En el archivo fuente~\ref{codigo:muestra_fuentes} se muestran un par de las
  referencias de este documento~(\cite{DBLP:journals/rfc/rfc4648}
  y~\cite{DBLP:books/crc/AU1992})\footnotemark{}. Existen varias ventajas al
  trabajar de esta forma. Primero, al igual que lo que pasa al trabajar con
  otros paquetes y con \LaTeX~mismo, la forma y el contenido son
  independientes entre sí (véase \gls{gl:acoplamiento}); el contenido es el
  que se muestra en el fragmento de código~\ref{codigo:muestra_fuentes},
  mientras que la forma va dada por la configuración de Bib\LaTeX. Por ejemplo,
  las referencias en este trabajo se encuentran en formato \gls{gl:ieee}, sin
  embargo, en el supuesto de que se quisieran cambiar a \gls{gl:apa}, bastaría
  con dar la orden al programa, sin tener que cambiar las descripciones fuente.
  Otra ventaja es la reutilización de las fuentes a través de distintos
  documentos: el mismo archivo de referencias se usa en este documento, en las
  presentaciones de Trabajo Terminal, en el artículo sobre tokenización,
  etcétera.}

% TODO:
% ¿En nuestro reporte técnico no tenemos que citar el artículo de RCI?
% ¿Eso se va a publicar, no?

\codigoFuente[codigo:muestra_fuentes]{899}{924}{latex}{%
  documentos_entregables/referencias.bib}{%
  Muestra de fuentes bibliográficas.}

\footnotetext{
  \gls{gl:dblp} es una base de datos en línea (\url{http://dblp.org/}) enfocada
  a documentos de ciencias de la computación. Muchas de las fuentes de este
  trabajo provienen de esta herramienta.}

% Dependencias a describir:
% Listings
% Minted
% Glossaries

\section{Comandos propios}

Como parte de la elaboración de este documento se crearon los siguietes
comandos y entornos:

\begin{itemize}

  \item \textbf{\texttt{{\textbackslash}capitulo}}
    Para insertar capítulos con epígrafes.

  \item \textbf{\texttt{{\textbackslash}codigoFuente}}
    Construcción alrededor de \textit{minted} para insertar código fuente.

  \item \textbf{\texttt{{\textbackslash}dependencia}}
    Para definir una dependencia; por ejemplo, las de la sección anterior.

  \item \textbf{\texttt{{\textbackslash}epigrafe}}
    Construcción alrededor del paquete \textit{epigraph} para definir un
    formato propio.

  \item \textbf{\texttt{{\textbackslash}expresionRegular}}
    Para insertar una expresión regular desde un archivo externo.

  \item \textbf{\texttt{{\textbackslash}formatearNumero}}
    Para colocar ceros a la izquierda de los comandos que ocupan contadores.

  \item \textbf{\texttt{{\textbackslash}hipervinculo}}
    Para definir vículos referenciables con una etiqueta en los cuales el
    texto que aparece en el vínculo queda definido por el origen. La gran
    mayoría de los vínculos \textsc{en este tipo de letra} fueron creados
    con este comando.

  \item \textbf{\texttt{{\textbackslash}parrafoConListaCorta}}
    Entorno para definir un párrafo con una lista contigua relativamente corta.
    No hay saltos de línea entre el párrafo y toda la lista.

  \item \textbf{\texttt{{\textbackslash}parrafoConListaLarga}}
    Entorno para definir un párrafo con una lista contigua. No hay saltos de
    línea entre el párrafo dado y el primer elemento de la lista.

  \item \textbf{\texttt{{\textbackslash}parte}}
    Para insertar partes con epígrafes.

  \item \textbf{\texttt{{\textbackslash}portada}}
    Para definir el formato de la portada del trabajo.

  \item \textbf{\texttt{{\textbackslash}pseudocodigo}}
    Construcción alrededor de \textit{listings} para la inserción de
    pseudocódigo.

  \item \textbf{\texttt{{\textbackslash}unidadIrrompible}}
    Entorno para evitar saltos de página.

  \item Comandos de los capítulos de análisis y diseño:

    \begin{itemize}

      \item \textbf{\texttt{{\textbackslash}casoDeUso}}
        Para definir el formato de un caso de uso. Ocupa otros subcomandos,
        propios del contexto de su caso de uso.

        \begin{itemize}

          \item \textbf{\texttt{{\textbackslash}etiquetaLocal}}
            Define etiquetas con un prefijo único.

          \item \textbf{\texttt{{\textbackslash}hipervinculoLocal}}
            Para hacer referencia a una trayectoria alternativa mostrando
            el nombre definido por esta.

          \item \textbf{\texttt{{\textbackslash}refernciaLocal}}
            Para hacer referencia a un paso de la trayectoria principal.

          \item \textbf{\texttt{{\textbackslash}trayectoriaAlternativa}}
            Entorno para enlistar los pasos de una trayectoria
            alternativa.

          \item \textbf{\texttt{{\textbackslash}trayectoriaPrincipal}}
            Entorno para enlistar los pasos de la trayectoria
            principal.

        \end{itemize}

      \item \textbf{\texttt{{\textbackslash}interfazDeUsuario}}
        Para insertar el conjunto de capturas que forman la figura de una
        interfaz de usuario.

      \item \textbf{\texttt{{\textbackslash}mensaje}}
        Para insertar un mensaje desde un archivo externo.

      \item \textbf{\texttt{{\textbackslash}reglaDeNegocio}}
        Para la definición de reglas de negocio.

      \item \textbf{\texttt{{\textbackslash}requerimiento}}
        Para la definición de requerimientos. Contiene comandos para la
        definción de \textit{subrequerimientos} y \textit{subsubrequerimientos}.

    \end{itemize}

\end{itemize}

% TODO:
% Al parecer no están funcionando las referencias a los códigos.

Es mediante la creación de estos comandos como se logra la separación de forma
y contenido de la que se habló anteriormente: el comando define la forma,
mientras que el contenido queda definido por el código dependiente del comando.
Para exponer esto se muestra, en los códigos
fuente~\ref{codigo:forma_de_dependencia} y~\ref{codigo:contenido_de_dependencia}
la forma y el contenido, respectivamente, de las dependencias. Estos comandos
se utilizan en las secciones de tecnologías de los capítulos de
implementaciones, y en la sección anterior de este apéndice.

\codigoFuente[codigo:forma_de_dependencia]{1}{44}{latex}{%
  documentos_entregables/reporte_tecnico/plantillas/dependencia.sty}{%
  Comando para las dependencias.}

% LOL, tengo la oportunidad de hacer una inclusión totalmente recursiva:
\codigoFuente[codigo:contenido_de_dependencia]{64}{84}{latex}{%
  documentos_entregables/reporte_tecnico/contenidos/%
  apendices/sobre_latex/sobre_latex.tex}{%
  Uso del comando para dependencias.}

Los casos de uso son, probablemente, los comandos más complejos; principalmente
por el número de subcomandos y entornos asociados. Como ejemplo, el código
fuente~\ref{codigo:ejemplo_caso_de_uso} muestra la definición del caso de uso
\hipervinculo{cu:eliminar_cliente}. El primer argumento define la etiqueta para
referirse a este caso de uso: el código
\verb|\hipervinculo{cu:eliminar_cliente}| dará como resultado un vínculo como
el del final de la oración anterior. El siguiente argumento define el nombre
del caso de uso. El primer párrafo del tercer argumento determina la
descripción del caso de uso; después de esto van los entornos de la trayectoria
principal y las trayectorias alternativas.

\codigoFuente[codigo:ejemplo_caso_de_uso]{1}{46}{latex}{%
  documentos_entregables/reporte_tecnico/contenidos/%
  analisis_y_disenio_api_web/analisis/casos_de_uso/eliminar_cliente.tex}{%
  Ejemplo de definición de caso de uso.}
