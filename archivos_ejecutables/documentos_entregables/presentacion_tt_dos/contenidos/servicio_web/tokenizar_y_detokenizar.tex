%
% Presentación de TT2.
% Proyecto Lovelace.
%
% Prototipo de servicio web.
% CU de tokenizar y detokenizar.
%

\subsection{Tokenización y detokenización}

\begin{frame}{Tokenización y detokenización}
  {Operaciones básicas del servicio web}

  \begin{itemize}
    \item Ocupan el prototipo de generación de tokens.

    \item La aplicación sigue el modelo arquitectónico REST\footnotemark{}.

      \footnotetext{\textit{REpresentational State Transfer}}

    \item Esquema de autenticación básico: las credenciales
      viajan en el encabezado de la petición.

    \item El protocolo encargado de asegurar la confidencialidad de los
      datos es HTTPS\footnotemark{}.

      \footnotetext{\textit{HyperText Transfer Protocol} con SSL/TLS}

    \item La autoridad certificadora ocupada es \textit{Let's  Encrypt}. Para
      evitar tener que conseguir un nombre de dominio se ocupa el servicio de
      \url{xip.io}.

  \end{itemize}

  \note{
    \begin{itemize}
      \item Estos dos casos de uso, más allá de todas las trayectorias
        alternativas que implican, son muy simples: la tokenización
        recibe un número de tarjeta y el método a ocupar, regresa un token.
        La detokenización recibe el token y el método, regresa el número
        de tarjeta asociado.
      \item REST fue creado por Roy Fielding: uno de los principales autores
        de la especificación de HTTP y el creador del proyecto Apache.
      \item Hasta lo que concierne al esquema básico de autenticación, las
        credenciales viajan en texto claro (en base 64). Es el protocolo
        envolvente, HTTPS, el encargado de asegurar la confidencialidad.
      \item \textit{Let's  Encrypt} funciona con puro software libre y su
        administración es pública. Es un proyecto de la EFF.
      \item \url{xip.io} es un servidor DNS modificado. Un servidor DNS normal
        recibe un nombre de dominio y regresa la IP asociada. Este servidor DNS
        recibe el nombre del dominio, del nombre del dominio decodifica la IP
        que se quiere regresar y ya está: certificados gratis para todos.
    \end{itemize}}

\end{frame}
