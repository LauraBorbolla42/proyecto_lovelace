%
% Diagrama de arquitectura de sistema tokenizador,,
% presentación en RCI.
% Proyecto Lovelace.
%
% IMPORTANTE: está hecho para ir dentro de una presentación de beamer.
% En el mprimer marco se muestra la operación de tokenización; en el segundo
% se muestra una transacción con el entorno del banco.
%

\begin{tikzpicture}[
  entorno/.style={
    rectangle,
    draw = black,
    thin,
    inner sep = 1mm,
    text width = 15mm,
    align = center,
    minimum height = 1.0cm},
  etiqueta/.style={
    align = center
  }]

  \node[entorno]
    (usuario)
    {Usuario};
  \node[entorno]
    (tienda)
    [right = of usuario]
    {Tienda en línea};
  \node[entorno,
    fill=gray!40,
    thick]
    (sistema_tokenizador)
    [right = of tienda]
    {\textbf{Sistema tokenizador}};

  \draw[-Stealth]
    ($(usuario.north) + (0.1in, 0)$)
    --
    ($(usuario.north) + (0.1in, 0.1in)$)
    --
    node[
      etiqueta,
      above]
      {1. Tarjeta: \\ \texttt{1234567891234567}}
    ($(tienda.north) + (-0.1in, 0.1in)$)
    --
    ($(tienda.north) + (-0.1in, 0)$);

  \draw[-Stealth]
    ($(tienda.north) + (0.1in, 0)$)
    --
    ($(tienda.north) + (0.1in, 0.1in)$)
    --
    node[
      etiqueta,
      above]
      {2. Tarjeta: \\ \texttt{1234567891234567}}
    ($(sistema_tokenizador.north) + (-0.1in, 0.1in)$)
    --
    ($(sistema_tokenizador.north) + (-0.1in, 0)$);

  \draw[-Stealth]
    ($(sistema_tokenizador.south) + (-0.1in, 0)$)
    --
    ($(sistema_tokenizador.south) + (-0.1in, -0.1in)$)
    --
    node[
      etiqueta,
      below]
      {3. Token: \\ \texttt{2954634586102951}}
    ($(tienda.south) + (0.1in, -0.1in)$)
    --
    ($(tienda.south) + (0.1in, 0)$);

  \draw[-Stealth]
    ($(tienda.south) + (-0.1in, 0)$)
    --
    ($(tienda.south) + (-0.1in, -0.1in)$)
    --
    node[
      etiqueta,
      below]
      {4. Resultado de \\ operación}
    ($(usuario.south) + (0.1in, -0.1in)$)
    --
    ($(usuario.south) + (0.1in, 0)$);

\end{tikzpicture}
