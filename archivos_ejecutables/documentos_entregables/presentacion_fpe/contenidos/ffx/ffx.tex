%
% Sección de FFX, presentación de cifrados que preservan el formato.
% Proyecto Lovelace.
%

\section{FFX}

\begin{frame}{FFX}{Introducción}

  FFX (\textit{Format-preserving, Feistel-based encryption}) es un modo de
  operación para lograr FPE en espacios pequeños.

  El mecanismo general usado en FFX son las redes Feistel, aplicadas sobre
  alfabetos arbitrarios.

  La primera versión fue presentada al NIST en~\cite{ffx_1}, en noviembre de
  2009; en la segunda versión~\cite{ffx_2}, se agregó el perfil de parámetros
  FF2, para cadenas binarias.

  \note
  {
    Fue creado por Rogaway, Bellare y Spies. El último, asociado a voltage
    security, había propuesto antes FFSEM. FFX es una generalización de
    FFSEM (las redes Feistel de este último solo funcionan en alfabetos
    binarios).

    Terence Spies fue quien empezó a usar el término de cifrados que preservan
    el formato.

    \textbf{TODO:} ¿Cómo estuvo lo de las vulnerabilidades encontradas?
  }

\end{frame}

\subimport{/}{feistel}
\subimport{/}{parametros}
\subimport{/}{a10}
